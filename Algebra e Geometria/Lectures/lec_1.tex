\chapter{Introduzione}
\lecture{1}{25 Nov. 12:11}{Richiami}
\section{Insiemi}
Partiamo col dire che nel vasto spettro degli insiemi troviamo anche quelli numerici. Questi insiemi si dicono infiniti 
perché racchiudono al loro interno elementi che continuano ad incrementare o/e decrementare all'infinito. Vediamo ora i 
vari insiemi numerici che potremo incontrare nel corso:\\
\begin{definizione}[Numeri Naturali]\label{nnaturali}
  L'insieme $\mathbb{\MakeUppercase{n}}$ comprende al suo interno tutti i numeri non negativi\\
  \begin{es}
   $\mathbb{\MakeUppercase{n}}=\{1,2,3,4,5,...,+\infty\}$
  \end{es}
\end{definizione}

\begin{definizione}[Numeri Interi]\label{ninteri}
  L'insieme $\mathbb{\MakeUppercase{z}}$ comprende al suo interno tutti i numeri negativi e positivi compreso quello nullo\\
  \begin{es}
   $\mathbb{\MakeUppercase{z}}=\{0,\pm1,\pm2,\pm3,\pm4,\pm5,...,\pm\infty\}$
  \end{es}
\end{definizione}

\begin{definizione}[Numeri Razionali]\label{nrazionali}
  L'insieme $\mathbb{\MakeUppercase{q}}$ comprende al suo interno tutti i numeri interi e comprende la notazione del tipo $\frac{m}{n}, m\in\mathbb{\MakeUppercase{z}} \wedge n\in\mathbb{\MakeUppercase{n}}$\\
  \begin{es}
   $\mathbb{\MakeUppercase{q}}=\{-\frac{5}{7},0,\frac{3}{5},1.5\overline{3},1.23(\frac{111}{90}),\frac{88}{1},...\}$
  \end{es}
\end{definizione}

\begin{definizione}[Numeri Reali]\label{nreali}
  L'insieme $\mathbb{\MakeUppercase{r}}$ comprende quei numeri che possono essere rappresentati con notazione decimale senza per forza essere del tipo $\frac{m}{n}$ \\
  \begin{es}
   $\mathbb{\MakeUppercase{r}}=\{\sqrt{2},\pi,e^{4}\}$
  \end{es}
\end{definizione}
\leavevmode\\
Quindi possiamo dire che $\mathbb{\MakeUppercase{n}}\subseteq\mathbb{\MakeUppercase{z}}\subseteq\mathbb{\MakeUppercase{q}}\subseteq\mathbb{\MakeUppercase{r}}$
\subsection{N-ple, $R^n$}

Identifichiamo ora un nuovo ente [$(a,b)$] individuato da due oggetti a e b non necessariamente distinti, e dall'ordine dei due. Un buon esempio potrebbe essere quello degli scacchi dove la posizione di una casella è identificata da due valori [$(n,x)$].\footnote{\href{https://upload.wikimedia.org/wikipedia/commons/thumb/b/b6/SCD_algebraic_notation.svg/1200px-SCD_algebraic_notation.svg.png}{Coordinate Scacchiera}}
\leavevmode\\
Possiamo definire adesso un nuovo insieme che è quello di $\mathbb{R}^{2}=\mathbb{R}\times \mathbb{R}, \{(a,b):a\in \mathbb{R} \wedge b\in \mathbb{R}\}$ definito da numeri razionali.
\begin{es}
	Un insieme $\mathbb{R}^{2}$ potrebbe essere $v=(2,\sqrt{6.4}), \in \mathbb{R}^{2}$
\end{es}

\begin{nota}
	In $\mathbb{R}^{2}$ ci sono anche particolari combinazioni che prendono il nome di \textbf{Diagonale Principale} e \textbf{Diagonale Secondaria}.
	\begin{es}
		\phantom{}\\
		\begin{align*}
			v=(x,&x) \text{: Diagonale Principale (i due elementi sono uguali)}\\
			v=(x,-&x) \text{: Diagonale Secondaria (un elemento è l'opposto dell'altro)}
		\end{align*}
	\end{es}
\end{nota}

Oltre all'insieme $\mathbb{R}^{2}$ ci sono poi tutta una serie incrementale di insiemi fino ad arrivare alla ennupla $\mathbb{R}^{n}$ che ha un numero di elementi virtualmente infinito. L'insieme di $\mathbb{R}^{n}, n>2$ viene chiamato spazio euclideo mentre i suoi elementi $(x_{1},x_{2},...,x_{n}, x\in \mathbb{R})$ vengono chiamati punti o vettori.\\
Possiamo usare gli spazi $\mathbb{R}$ per rappresentare graficamente dei riferimenti. $\mathbb{R}^{1}$ ci permette di orientarci su una retta mentre $\mathbb{R}^{2}$ e $\mathbb{R}^{3}$ rispettivamente per visualizzare figure piane e solidi.

\begin{figure}[H]
	\centering
	\incfig{piano cartesiano}
	\caption[Caption]{Coordinate degli assi in $\mathbb{R}^{2}$ e $\mathbb{R}^{3}$}
	\label{fig:pianocartesiano}
\end{figure}

Adesso possiamo usare lo spazio $\mathbb{R}^{2}$ per fare un esercizio. Avremo due vettori $q=(3,5)\wedge p=(5,3)$ e vogliamo visualizzare il vettore $s=\frac{1}{2}q+2p$. Per fare ciò, si usa un metodo grafico chiamato punta-coda, dove mettiamo i nostri vettori in successione connettendo l'inizio di uno alla fine dell'altro.

\begin{figure}[H]
	\centering
	\incfig{somma vettori}
	\caption[Caption]{}
	\label{fig:sommavettori}
\end{figure}

\section{Proprietà}

\subsection{Somma}

\begin{definizione}
	$(a_{1}, a_{2})+(b_{1},b_{2})=(a_{1}+b_{1}, a_{2}+b_{2}), (a, b)\in \mathbb{R}$
\end{definizione}

La somma è un'operazione interna, dato che gli addendi e il risultato dell'operazione si trovano nello stesso insieme.

\begin{nota}
	\phantom{}\\
	\begin{description}
		\item[1.] Elemento neutro: $(a_{1}, a_{2})+(0,0)=(0,0)+(a_{1}, a_{2})=(a_{1}, a_{2}), \forall(a_{1}, a_{2}\in \mathbb{R}^{2})$
		\item[2.] Opposto: $(a_{1}, a_{2})+(-a_{1}, -a_{2})=(0, 0), \forall(a_{1}, a_{2}\in \mathbb{R}^{2})$\\
		\item[3.] P. Associativa: $\forall (a_{1}, a_{2}), (b_{1}, b_{2}), (c_{1}, c_{2}), \in \mathbb{R}^{2}$\\\\
		\phantom{texttexttextt}$[(a_{1}, a_{2})+(b_{1}, b_{2})]+(c_{1}, c_{2})=(a_{1}, a_{2})[(b_{1}, b_{2})+(c_{1}, c_{2})]$\\
		\item[4.] P. Commutativa: $\forall (a_{1}, a_{2}), (b_{1}, b_{2}), \in \mathbb{R}^{2}$\\\\
		\phantom{texttexttexttex}$(a_{1}, a_{2})+(b_{1}, b_{2})=(b_{1}, b_{2})+(a_{1}, a_{2})$
	\end{description}
\end{nota}
\leavevmode\\
Una struttura algebrica del tipo $(\mathbb{R}^{2},+)$ se gode delle precedenti proprietà da 1 a 4 viene chiamata \textbf{Gruppo Abeliano}

\subsection{Moltiplicazione di un vettore per uno scalare}

\begin{definizione}
	$(a_{1}, a_{2})\cdot(b_{1},b_{2})=(a_{1}\cdot b_{1}, a_{2}\cdot b_{2})\dagger a,b\in \mathbb{R}$
\end{definizione}
\begin{nota}
	\phantom{}\\
	\begin{description}
		\item[5.] P. distributiva: $\forall \beta\in \mathbb{R} \wedge \forall (a_{1}, a_{2}),(b_{1}, b_{2}), \in \mathbb{R}^{2}$\\\\
		\phantom{texttexttextt}$\beta[(a_{1}, a_{2})+(b_{1}, b_{2})]=\beta(a_{1}, a_{2})+\beta(b_{1}, b_{2})$\\
		\item[6.] P. distributiva: $\forall \beta,\delta\in \mathbb{R} \wedge \forall (a_{1}, a_{2}), \in \mathbb{R}^{2}$\\\\
		\phantom{texttexttextt}$(\beta+\delta)(a_{1}, a_{2})=\beta(a_{1}, a_{2})+\delta(a_{1}, a_{2})$\\
		\item[7.] P. Associativa: $\forall \beta,\delta\in \mathbb{R} \wedge \forall (a_{1}, a_{2}), \in \mathbb{R}^{2}$\\\\
		\phantom{texttexttextt}$\beta\gamma(a_{1}, a_{2})=\beta[\gamma(a_{1}, a_{2})]=(\beta\gamma)(a_{1}, a_{2})$\\
		\item[8.] Elemento neutro: $1(a_{1}, a_{2})=(a_{1}, a_{2}), \forall(a_{1}, a_{2})\in\mathbb{R}^{2}$
	\end{description}
\end{nota}







