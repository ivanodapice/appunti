\documentclass[a4paper,12pt, oneside]{book}
%\usepackage{fullpage}
\usepackage[italian]{babel}
\usepackage[utf8x]{inputenc}
\usepackage{float}
\usepackage{amssymb}
\usepackage{graphicx} 
\usepackage{amsthm}
\usepackage{graphics}
\usepackage{amsfonts}
\usepackage{amsmath}
\usepackage{amstext}
\usepackage{engrec}
\usepackage{rotating}
\usepackage[safe,extra]{tipa}
\usepackage{tikz,pgfplots}
\usetikzlibrary{positioning}
\usetikzlibrary{calc,through,backgrounds}
%\usepackage{showkeys}
\usepackage{multirow}
\usepackage{titlesec}
\usepackage{hyperref}
\usepackage{microtype}
\usepackage{enumerate}
\usepackage{braket}
\usepackage{marginnote}
\usepackage{pgfplots}
\usepackage{cancel}
\usepackage{polynom}
\usepackage{caption}
\usepackage{booktabs}
\usepackage{enumitem}
\usepackage{framed}
\usepackage{pdfpages}
\usepackage{pgfplots}
\usepackage{fancyhdr}
\fancyhead[LE,RO]{\slshape \rightmark}
\fancyhead[LO,RE]{\slshape \leftmark}
\fancyfoot[C]{\thepage}
\usepackage{stanli}


\title{Tecnica delle Costruzioni}
\author{UniNA\\\ Ivano D'Apice\\\href{https://t.me/sonoivano}{@sonoivano}}
\date{}

\pgfplotsset{compat=1.13}
\begin{document}
	\maketitle
	
	\definecolor{shadecolor}{gray}{0.80}
	\newtheorem{teorema}{Teorema}
	\newtheorem{definizione}{Definizione}
	\newtheorem{principio}{Principio}
	\newtheorem{esempio}{Esempio}
	\newtheorem{corollario}{Corollario}
	\newtheorem{lemma}{Lemma}
	\newtheorem{osservazione}{Osservazione}
	\newtheorem{nota}{Nota}
	\newtheorem{algoritmo}{Algoritmo}
	\tableofcontents
	\renewcommand{\chaptermark}[1]{%
		\markboth{\chaptername
			\ \thechapter.\ #1}{}}
	\renewcommand{\sectionmark}[1]{\markright{\thesection.\ #1}}
	
	\chapter{Introduzione}
	
	\textbf{Questi appunti sono presi a lezione. Per quanto sia stata fatta una revisione è altamente probabile (praticamente certo) che possano contenere errori, sia di stampa che di vero e proprio contenuto. Per eventuali proposte di correzione inviare una mail alla seguente casella postale. Link: } \url{ivanodapice@icloud.com}.\\
	\textbf{Grazie mille e buono studio!}
	
	\chapter{Richiami}
	
	Alcuni dei primi concetti da ricordare sono:
	
	\begin{itemize} 
		\item[$\ast$] \textbf{Corpo in Equilibrio}:\\
		Una struttura in cui la sommatoria delle forze agenti è uguale a 0.
		\item[$\ast$] \textbf{Metodi di risoluzione di strutture}\\
		Metodo delle forze, metodo degli spostamenti, eq. statica, plv.
	\end{itemize}
	
	\section{Travi continue, taglio e momento}
	
	\begin{figure}[H]
		\centering
		\begin{tikzpicture}
			% here we construct our structure
			\scaling{.45};
			%\draw[help lines,step=.45]
			(0,-.9) grid (12.6,1.81);
			\point{a}{2}{1};                                 %Punto A
			\point{b}{26}{1};                                %Punto B
			\point{c}{14}{1};                                %Punto C
			\point{d}{14.5}{2};                              %Punto D
			\point{e}{2.2}{-2};                              %Punto E
			\point{f}{25.6}{-2};                             %Punto F
			\point{g}{8}{1};								 %Punto G
			\point{h}{20}{1};								 %Punto H
			\beam{2}{a}{b};                                  %Trave A-B
			\support{1}{a}[0];                               %Cerniera A
			\support{2}{b}[0];                               %Carrello B
			%{type}{insertion point}[rotation][length or angle][load distance]
			\load{1}{c}[90][0][0];                           %Forza F Mezzeria
			\load{1}{a}[270][0][1];                          %Reazione Va
			\load{1}{a}[180][0][0];                          %Reazione Ha
			\load{1}{b}[270][0][1];                          %Reazione Vb
			\notation{1}{d}{F}[above];                       %Forza F Nomenclatura
			\notation{1}{a}{$H_a$}[above left=1mm];          %Reazione Ha Nomenclatura
			\notation{1}{e}{$V_a$}[right];                   %Reazione Va Nomenclatura
			\notation{1}{f}{$V_b$}[left];                    %Reazione Vb Nomenclatura
			\notation{1}{a}{A}[above];                       %Punto A Nomenclatura
			\notation{1}{b}{B}[above];                       %Punto B Nomenclatura
			\dimensioning{1}{a}{c}{-2}[a];					 %Dimensione a
			\dimensioning{1}{c}{b}{-2}[b];					 %Dimensione b
		\end{tikzpicture}
		\caption{Trave appoggiata-appoggiata.}
		\label{fig:traveuno}
	\end{figure}
	
	Cominciano a parlare di questa trave appoggiata-appoggiata.
	Abbiamo una forza F in mezzeria che sarà equilibrata dalle altre 2 forze Verticali $V_a$ e $V_b$. La forza orizzontale $H_a$ è nulla perché appunto è una trave continua e non ci sono lavori orizzontali.
	Se facciamo una equazione di equilibrio troveremo i seguenti dati:
	
	\phantom{.}
	
	$\Sigma V=0,‎‎‎‎‎\hfill V_a + V_b - F = 0$
	
	\phantom{.}
	
	$\Sigma H=0,‎‎‎‎‎\hfill H_a = 0$
	
	\phantom{.}
	
	$\Sigma M_a=0,‎‎‎‎‎\hfill F\cdot a - V_b\cdot l = 0$
	
	\phantom{.}
	
	Da cui possiamo ricavare $V_a=F-V_b$ e $V_b=\dfrac{F\cdot a}{a+b}$.
	Nel caso in cui $a=1$ e $b=1$ avremo $V_b=\dfrac{F\cdot 1}{2}$
	
	\begin{figure}[H]
		\centering
		\begin{tikzpicture}
			% here we construct our structure
			\scaling{.45};
			%\draw[help lines,step=.45]
			(0,-.9) grid (12.6,1.81);
			\point{a}{2}{1};               					%Punto A
			\point{b}{26}{1};             					%Punto B
			\point{c}{14}{1};              					%Punto C
			\beam{2}{a}{b};                					%Trave A-B
			\notation{1}{a}{A}[left];      					%Punto A Nomenclatura
			\notation{1}{b}{B}[right];     					%Punto B Nomenclatura
			\notation{1}{a}{$+$}[above left=5mm];      
			\notation{1}{a}{$-$}[below left=5mm];  
			\internalforces{a}{c}{-1}{-1}[0][gray]; 		%Taglio A-C
			\internalforces{c}{b}{1}{1}[0][gray];   		%Taglio C-B
		\end{tikzpicture}
		\caption{Taglio.}
		\label{fig:tagliouno}
	\end{figure}
	
	Come possiamo vedere dall'esempio [\ref{fig:tagliouno}], il diagramma del taglio della trave [\ref{fig:traveuno}] è positivo per convenzione quando mettendoci in un generico concio avremo le forze disposte in questo modo:
	
	\centering $\uparrow \Box \downarrow$
	
	In presenza di forze, allora, il diagramma presenterà un salto. In situazioni di forze concentrate (carichi) invece, avremo un andamento lineare.\phantom{..............}
	
	\begin{figure}[H]
		\centering
		\begin{tikzpicture}
			% here we construct our structure
			\scaling{.45};
			%\draw[help lines,step=.45]
			(0,-.9) grid (12.6,1.81);
			\point{a}{2}{1};               					%Punto A
			\point{b}{26}{1};              					%Punto B
			\point{c}{14}{1};              					%Punto C
			\beam{2}{a}{b};                					%Trave A-B
			\notation{1}{a}{A}[left];      					%Punto A Nomenclatura
			\notation{1}{b}{B}[right];     					%Punto B Nomenclatura
			\notation{1}{a}{$-$}[above left=5mm];      
			\notation{1}{a}{$+$}[below left=5mm];  
			\internalforces{a}{c}{0}{1}[0][gray]; 			%Taglio A-C
			\internalforces{c}{b}{1}{0}[0][gray];   		%Taglio C-B
		\end{tikzpicture}
		\caption{Momento Flettente.}
		\label{fig:momentouno}
	\end{figure}
	
	A differenza del taglio, quando abbiamo una forza il diagramma del  momento presenterà un punto angoloso o cuspide. 
	Se ad esempio avremo una forza molto piccola da non far cambiare il segno al taglio avremo solo un punto angoloso.
	Con forze grandi tali da cambiare segno invece ci saranno le condizioni per verificare una cuspide.
	Convenzionalmente avremo valori positivi per forze generanti tali  momenti sul concio infinitesimo:
	
	\centering $\circlearrowright \Box \circlearrowleft$
	
	\section{Metodo delle Forze}
	
	\begin{figure}[H]
		\centering
		\begin{tikzpicture}
			% here we construct our structure
			\scaling{.45};
			%\draw[help lines,step=.45]
			(0,-.9) grid (12.6,1.81);
			\point{a}{2}{1};                                 %Punto A
			\point{b}{26}{1};                                %Punto B
			\point{c}{14}{1};                                %Punto C
			\point{caricoin}{2}{3};                          %Punto iniziale Carico q1 
			\point{caricofin}{26}{3};                        %Punto finale Carico q1
			\beam{2}{a}{b};                                  %Trave A-B
			\support{3}{a}[-90];                              %Incastro A
			\lineload{2}{caricoin}{caricofin}[1][1];         %Carico q1
			\dimensioning{1}{a}{b}{-1}[$L_1$];					 %Dimensione a
			\notation{1}{a}{A}[above left];                  %Punto A Nomenclatura
			\notation{1}{b}{B}[above right];                 %Punto B Nomenclatura
			\notation{1}{caricofin}{$q_1$}[above right];     %Carico q1 Nomenclatura
			
		\end{tikzpicture}
		\caption{Trave a sbalzo.}
		\label{fig:sbalzouno}
	\end{figure}
	
	Iniziamo con un esempio classico. La [\ref{fig:sbalzouno}] è una trave a sbalzo con un estremo libero e un incastro. Sul tronco della trave inoltre è stato applicato un carico $q_1$. In questo caso il corpo sarà isostatico dal $3t-s=3-3=0$.
	
	Ora calcoliamo le reazioni che genera l'incastro nella trave con le equazioni della statica.
	
	\begin{figure}[H]
		\centering
		\hspace*{-1.6cm}
		\begin{tikzpicture}
			% here we construct our structure
			\scaling{.45};
			%\draw[help lines,step=.45]
			(0,-.9) grid (12.6,1.81);
			\point{a}{2}{1};                                 %Punto A
			\point{b}{26}{1};                                %Punto B
			\point{c}{14}{1};                                %Punto C
			\point{scappaaapapapa}{1}{1};
			%\point{caricoin}{2}{3};                         %Punto iniziale Carico q1 
			%\point{caricofin}{26}{3};                       %Punto finale Carico q1
			\beam{2}{a}{b};                                  %Trave A-B
			\support{3}{a}[-90];                              %Incastro A
			\load{1}{c}[90][0][0];                           %Carico q1 Forza Mezzeria
			\load{1}{a}[270][0][1];                          %Reazione Va
			\load{3}{scappaaapapapa}[90][180];                          %Reazione Ha
			%\lineload{2}{caricoin}{caricofin}[1][1];        %Carico q1
			%\dimensioning{1}{a}{b}{-1}[1];					 %Dimensione a
			%\notation{1}{a}{A}[above left];                  %Punto A Nomenclatura
			%\notation{1}{b}{B}[above right];                 %Punto B Nomenclatura
			\notation{1}{e}{$V_a$}[right];                   %Reazione Va Nomenclatura
			\notation{1}{a}{$M_a$}[left=8mm];          %Reazione Ha Nomenclatura
			\notation{1}{d}{$q_1L$}[above right];            %Carico q1l Forza Nomenclatura
			%\notation{1}{c}{a}[below=15mm];			     %Dimensione a Nomenclatura. Punto C
			%\notation{1}{caricofin}{$q_1$}[above right];    %Carico q1 Nomenclatura
			
		\end{tikzpicture}
		\caption{Reazioni.}
		\label{fig:sbalzouno1}
	\end{figure}
	
	$\Sigma V=0,‎‎‎‎‎\hfill - V_a + q_1L= 0$
	
	\phantom{.}
	
	$\Sigma M_a=0,‎‎‎‎‎\hfill -\dfrac{qL^{2}}{2} + M_a = 0$
	
	\phantom{text}
	
	Vediamo che sia $V_a$ che $M_a$ sono di calcoli banali. Cosa succede però se alla stessa trave aggiungiamo un vincolo rendendola così 1 volta iperstatica?
	
	\begin{figure}[H]
		\centering
		\begin{tikzpicture}
			% here we construct our structure
			\scaling{.45};
			%\draw[help lines,step=.45]
			(0,-.9) grid (12.6,1.81);
			\point{a}{2}{1};                                 %Punto A
			\point{b}{26}{1};                                %Punto B
			\point{c}{14}{1};                                %Punto C
			\point{scappaaapapapa}{1}{1};
			%\point{caricoin}{2}{3};                         %Punto iniziale Carico q1 
			%\point{caricofin}{26}{3};                       %Punto finale Carico q1
			\beam{2}{a}{b};                                  %Trave A-B
			\support{3}{a}[-90];                              %Incastro A
			\support{2}{b}[0];                               %Carrello B
			\load{1}{c}[90][0][0];                           %Carico q1 Forza Mezzeria
			\load{1}{a}[270][0][1];                          %Reazione Va
			\load{3}{scappaaapapapa}[90][180];                          %Reazione Ha
			\load{1}{b}[270][0][1];                          %Reazione Vb
			%\lineload{2}{caricoin}{caricofin}[1][1];        %Carico q1
			%\dimensioning{1}{a}{b}{-1}[1];					 %Dimensione a
			%\notation{1}{a}{A}[above left];                  %Punto A Nomenclatura
			%\notation{1}{b}{B}[above right];                 %Punto B Nomenclatura
			\notation{1}{e}{$V_a$}[right];                   %Reazione Va Nomenclatura
			\notation{1}{f}{$V_b$}[left];                    %Reazione Vb Nomenclatura
			\notation{1}{a}{$M_a$}[left=8mm];          %Reazione Ha Nomenclatura
			\notation{1}{d}{$q_1L$}[above right];            %Carico q1l Forza Nomenclatura
			%\notation{1}{c}{a}[below=15mm];			     %Dimensione a Nomenclatura. Punto C
			%\notation{1}{caricofin}{$q_1$}[above right];    %Carico q1 Nomenclatura
			
		\end{tikzpicture}
		\caption{Trave incastrata-appoggiata.}
		\label{fig:sbalzounopiucarrello}
	\end{figure}
	
	$\Sigma V=0,‎‎‎‎‎\hfill q_1L - R_a - R_b = 0$
	
	\phantom{.}
	
	$\Sigma M_a=0,‎‎‎‎‎\hfill -\dfrac{qL^{2}}{2} + M_a + R_bL = 0$
	
	\phantom{.}
	
	Queste di sopra sono due equazioni in tre incognite che ci daranno $\infty^{1}$ risultati. Pertanto è impossibile, o molto laborioso, trovare l'unica soluzione \textbf{congruente} al nostro sistema che ha nell'incastro condizioni di $V_a=0, \varphi_a=0$ e nel carrello $V_b=0$.
	
	\phantom{.}
	
	Tratteremo questa struttura staticamente indefinita rimuovendo un vincolo e aggiungendo una forza fittizia.
	
	\begin{figure}[H]
		\centering
		\hspace*{-1cm}
		\begin{tikzpicture}
			% here we construct our structure
			\scaling{.45};
			%\draw[help lines,step=.45]
			(0,-.9) grid (12.6,1.81);
			\point{a}{2}{1};                                 %Punto A
			\point{b}{26}{1};                                %Punto B
			\point{c}{14}{1};                                %Punto C
			\point{scappaaapapapa}{1}{1};
			%\point{caricoin}{2}{3};                         %Punto iniziale Carico q1 
			%\point{caricofin}{26}{3};                       %Punto finale Carico q1
			\beam{2}{a}{b};                                  %Trave A-B
			\support{3}{a}[-90];                              %Incastro A
			\load{1}{c}[90][0][0];                           %Carico q1 Forza Mezzeria
			\load{1}{a}[270][0][1];                          %Reazione Va
			\load{3}{scappaaapapapa}[90][180];                          %Reazione Ha
			\load{1}{b}[270][0][1];                          %Reazione Vb
			%\lineload{2}{caricoin}{caricofin}[1][1];        %Carico q1
			%\dimensioning{1}{a}{b}{-1}[1];					 %Dimensione a
			%\notation{1}{a}{A}[above left];                  %Punto A Nomenclatura
			%\notation{1}{b}{B}[above right];                 %Punto B Nomenclatura
			\notation{1}{e}{$V_a$}[right];                   %Reazione Va Nomenclatura
			\notation{1}{f}{$F_x$}[left];                    %Reazione Vb Nomenclatura
			\notation{1}{a}{$M_a$}[left=8mm];          %Reazione Ha Nomenclatura
			\notation{1}{d}{$q_1L$}[above right];            %Carico q1l Forza Nomenclatura
			%\notation{1}{c}{a}[below=15mm];			     %Dimensione a Nomenclatura. Punto C
			%\notation{1}{caricofin}{$q_1$}[above right];    %Carico q1 Nomenclatura
			
		\end{tikzpicture}
		\caption{Trave dopo soppressione del vincolo carrello.}
		\label{fig:sbalzounoforzafitt}
	\end{figure}
	
	Da questa nuova trave possiamo ricavare la reazione $V_b$ tramite equazioni di \textbf{congruenza}.
	
	\phantom{.}
	
	$[V_b=0]\hfill$
	
	$V_b = V_b(q_1)+V_b(x)=0, \Rightarrow \dfrac{qL^{4}}{8EI} - \dfrac{xL^{3}}{3EI} = 0\hfill$
	
	\phantom{.}
	
	$x=\frac{3}{8}qL\hfill$
	
	\phantom{.}
	
	$\boxtimes$ \textbf{N.B.} I risultati relativi a $V_b(q_1), V_b(x)$ sono stati ottenuti tramite tabelle di valori noti.
	
	\phantom{.}
	
	\phantom{.}
	
	il nostro nuovo sistema di equazioni sarà:
	
	\begin{equation}
		\centering
		\begin{cases}
			q_1L - R_a - R_b &= 0 \\ \\
			-\dfrac{qL^{2}}{2} + M_a &= 0 \\ \\
			\dfrac{qL^{4}}{8EI} - \dfrac{xL^{3}}{3EI} &=0\\	
		\end{cases}
	\end{equation}
	
	\phantom{.}
	
	Ora inizieremo a trattare il vero e proprio metodo delle forze. Verrà presentato mediante questa struttura:
	
	\begin{figure}[H]
		\centering
		\begin{tikzpicture}
			% here we construct our structure
			\scaling{.45};
			%\draw[help lines,step=.45]
			(0,-.9) grid (12.6,1.81);
			\point{a}{2}{1};                                 %Punto A
			\point{b}{26}{1};                                %Punto B
			\point{c}{14}{1};                                %Punto C
			\point{caricoin}{2}{3};                          %Punto iniziale Carico q1 
			\point{caricofin}{26}{3};                        %Punto finale Carico q1
			\point{caricomed}{14}{3};                        %Punto medio Carico q1
			\beam{2}{a}{b};                                  %Trave A-B
			\support{3}{a}[-90];                              %Incastro A
			\support{2}{c}[0];
			\support{2}{b}[0];
			\lineload{2}{caricoin}{caricomed}[1][1];         %Carico q1
			\lineload{2}{caricomed}{caricofin}[2][2];        %Carico q2
			\dimensioning{1}{a}{c}{-1}[$L_1$];					 %Dimensione a
			\dimensioning{1}{c}{b}{-1}[$L_2$];					 %Dimensione a
			\notation{1}{a}{A}[above left];                  %Punto A Nomenclatura
			\notation{1}{b}{C}[above right];                 %Punto B Nomenclatura
			\notation{1}{c}{B}[above];                       %Punto C Nomenclatura
			\notation{1}{caricofin}{$q_2$}[above right];     %Carico q2 Nomenclatura
			\notation{1}{caricoin}{$q_1$}[above left];       %Carico q1 Nomenclatura
			
		\end{tikzpicture}
		\caption{}
		\label{fig:traveforzaaaa}
	\end{figure}
	
	Definiamo la struttura in quanto isostatica, iperstatica o labile. Dal $3t-s$ calcoliamo come segue:
	
	$3t-s=3-5=-2$ quindi avremo una struttura 2 volte iperstatica.
	
	\begin{figure}[H]
		\centering
		\begin{tikzpicture}
			% here we construct our structure
			\scaling{.45};
			%\draw[help lines,step=.45]
			(0,-.9) grid (12.6,1.81);
			\point{a}{2}{1};                                 %Punto A
			\point{b}{26}{1};                                %Punto B
			\point{c}{14}{1};                                %Punto C
			\point{d}{13}{1};                                %Punto C
			\point{e}{15}{1};                                %Punto C
			\point{caricoin}{2}{3};                          %Punto iniziale Carico q1 
			\point{caricofin}{26}{3};                        %Punto finale Carico q1
			\point{caricomed}{14}{3};                        %Punto medio Carico q1
			\beam{2}{a}{b};                                  %Trave A-B
			\hinge{1}{a};                                    %Incastro A
			\hinge{1}{c};
			\load{3}{a}[90][180];
			\load{2}{d}[90][180];
			\load{3}{e}[270][180];
			\support{2}{c}[0];
			\support{2}{b}[0];
			\notation{1}{a}{$X_a$}[left=4mm];                  %Punto A Nomenclatura
			\notation{1}{d}{$X_{bsx}$}[above left=2mm];                       %Punto C Nomenclatura
			\notation{1}{e}{$X_{bdx}$}[above right=2mm];                       %Punto C Nomenclatura
		\end{tikzpicture}
		\caption{}
		\label{fig:traveforzacazzoculomega}
	\end{figure}
	
	Dato che il numero di volte in cui la struttura risultava iperstatica erano 2 abbiamo declassato il vincolo incastro e creato una cerniera di distacco tra le due travi. Quindi avremo due incognite iperstatiche che saranno $X_a$ e $X_b$.
	
	Equazioni di congruenza:
	\begin{equation}
		\centering
		\begin{cases}
			\varphi_a=0 \\ \\
			\varphi_{bsx}=\varphi_{bdx} \\ \\
		\end{cases}
	\end{equation}
	
	Per calcolare queste due incognite dobbiamo usare dal formulario i casi notevoli come nel precedente esercizio. In questo frangente ci aiuteremo con una rappresentazione visiva dei suddetti casi.
	
	\begin{figure}[H]
		\centering
		\begin{tikzpicture}
			% here we construct our structure
			\scaling{.45};
			%\draw[help lines,step=.45]
			(0,-.9) grid (12.6,1.81);
			\point{a}{2}{1};                                 %Punto A
			\point{b}{26}{1};                                %Punto B
			\point{c}{14}{1};                                %Punto C
			\point{d}{.6}{1};
			\beam{2}{a}{b};                                  %Trave A-B
			\support{1}{a}[0];                               %Cerniera A
			\support{1}{b}[0];                               %Carrello B
			\internalforces{a}{b}{0}{0}[1][gray];
			\load{3}{d}[90][180];
			\notation{1}{d}{$X_a$}[left=4mm];
		\end{tikzpicture}
		\caption{Caso $X_a$. Tratto AB.}
		\label{fig:appappbnesjkbdj}
	\end{figure}
	
	$ \varphi_a(X_a)=\varphi_{bsx}(X_a) \Rightarrow \dfrac{X_a L_1}{3EI}-\dfrac{X_a L_1}{6EI} $
	
	\begin{figure}[H]
		\centering
		\hspace*{1.2cm}
		\begin{tikzpicture}
			% here we construct our structure
			\scaling{.45};
			%\draw[help lines,step=.45]
			(0,-.9) grid (12.6,1.81);
			\point{a}{2}{1};                                 %Punto A
			\point{b}{26}{1};                                %Punto B
			\point{c}{14}{1};                                %Punto C
			\point{d}{.6}{1};
			\beam{2}{a}{b};                                  %Trave A-B
			\support{1}{a}[0];                               %Cerniera A
			\support{1}{b}[0];                               %Carrello B
			\internalforces{a}{b}{0}{0}[-1][gray];
			\lineload{2}{a}{b};
			\notation{1}{b}{$q_1$}[above right=2mm];
			%\load{3}{d}[90][180];
			%\notation{1}{d}{$X_a$}[left=4mm];
		\end{tikzpicture}
		\caption{Caso $q_1$. Tratto AB.}
		\label{fig:appappbnasdasdj}
	\end{figure}
	
	$ \varphi_a(q_1)=\varphi_{bsx}(q_1) \Rightarrow -\dfrac{q_1 L_1^{3}}{24EI}+\dfrac{q_1 L_1^{3}}{24EI} $
	
	\begin{figure}[H]
		\centering
		\hspace*{1.4cm}
		\begin{tikzpicture}
			% here we construct our structure
			\scaling{.45};
			%\draw[help lines,step=.45]
			(0,-.9) grid (12.6,1.81);
			\point{a}{2}{1};                                 %Punto A
			\point{b}{26}{1};                                %Punto B
			\point{c}{14}{1};                                %Punto C
			\point{d}{27.4}{1};
			\beam{2}{a}{b};                                  %Trave A-B
			\support{1}{a}[0];                               %Cerniera A
			\support{1}{b}[0];                               %Carrello B
			\internalforces{a}{b}{0}{0}[1][gray];
			\load{2}{d}[270][180];
			\notation{1}{d}{$X_b$}[right=4mm];
		\end{tikzpicture}
		\caption{Caso $X_{bsx}$. Tratto AB.}
		\label{fig:appappbneqweewwedj}
	\end{figure}
	
	$ \varphi_a(x_b)=\varphi_{bsx}(x_b) \Rightarrow \dfrac{x_b L_1}{6EI}-\dfrac{x_b L_1}{3EI} $
	
	\begin{figure}[H]
		\centering
		\begin{tikzpicture}
			% here we construct our structure
			\scaling{.45};
			%\draw[help lines,step=.45]
			(0,-.9) grid (12.6,1.81);
			\point{a}{2}{1};                                 %Punto A
			\point{b}{26}{1};                                %Punto B
			\point{c}{14}{1};                                %Punto C
			\point{d}{.6}{1};
			\beam{2}{a}{b};                                  %Trave A-B
			\support{1}{a}[0];                               %Cerniera A
			\support{1}{b}[0];                               %Carrello B
			\internalforces{a}{b}{0}{0}[1][gray];
			\load{3}{d}[90][180];
			\notation{1}{d}{$X_b$}[left=4mm];
		\end{tikzpicture}
		\caption{Caso $X_a$. Tratto BC.}
		\label{fig:appappbnsdfsdf}
	\end{figure}
	
	$ \varphi_{bdx}(x_b)=\dfrac{X_b L_2}{3EI} $
	
	\begin{figure}[H]
		\centering
		\hspace*{1.2cm}
		\begin{tikzpicture}
			% here we construct our structure
			\scaling{.45};
			%\draw[help lines,step=.45]
			(0,-.9) grid (12.6,1.81);
			\point{a}{2}{1};                                 %Punto A
			\point{b}{26}{1};                                %Punto B
			\point{c}{14}{1};                                %Punto C
			\point{d}{.6}{1};
			\beam{2}{a}{b};                                  %Trave A-B
			\support{1}{a}[0];                               %Cerniera A
			\support{1}{b}[0];                               %Carrello B
			\internalforces{a}{b}{0}{0}[-1][gray];
			\lineload{2}{a}{b};
			\notation{1}{b}{$q_2$}[above right=2mm];
			%\load{3}{d}[90][180];
			%\notation{1}{d}{$X_a$}[left=4mm];
		\end{tikzpicture}
		\caption{Caso $q_2$. Tratto BC.}
		\label{fig:appappbnasdsdfsdfdj}
	\end{figure}
	
	$ \varphi_{bdx}(q_2)=-\dfrac{q_2 L_2^{3}}{24EI} $
	
	\phantom{.}
	
	Ricapitolando tutti i casi delle nostre equazioni di congruenza arriveremo a costruire un sistema di equazioni tale da darci le soluzioni ai nostri termini incogniti.
	
	\phantom{.}
	
	$ \varphi_a=0 \Rightarrow \varphi_a(X_a)+\varphi_a(q_1)+\varphi_a(x_b) \hfill$
	
	\phantom{.}
	
	$ \varphi_{bsx}=\varphi_{bdx} \Rightarrow \varphi_{bsx}(X_a)+\varphi_{bsx}(q_1)+\varphi_{bsx}(x_b)=\varphi_{bdx}(x_b)+\varphi_{bdx}(q_2) \hfill$
	
	\begin{equation}
		\centering
		\begin{cases}
			\dfrac{X_a L_1}{3EI}-\dfrac{q_1 L_1^{3}}{24EI}+\dfrac{x_b L_1}{6EI}&=0 \\ \\
			\dfrac{X_a L_1}{6EI}+\dfrac{q_1 L_1^{3}}{24EI}-\dfrac{x_b L_1}{3EI}&=\dfrac{X_b L_2}{3EI}-\dfrac{q_2 L_2^{3}}{24EI} \\ \\
		\end{cases}
	\end{equation}
	
	\begin{equation}
		\centering
		\begin{cases}
			\dfrac{X_a L_1}{3EI}+\dfrac{x_b L_1}{6EI}&=\dfrac{q_1 L_1^{3}}{24EI} \\ \\
			\dfrac{X_a L_1}{6EI}+x_b(\dfrac{L_1}{3EI}+\dfrac{L_2}{3EI})&=\dfrac{q_1 L_1^{3}}{24EI}+\dfrac{q_2 L_2^{3}}{24EI} \\ \\
		\end{cases}
	\end{equation}
	
	\begin{equation}
		\begin{pmatrix}
			\dfrac{L_1}{3EI} & \dfrac{L_1}{6EI}\\\\ \dfrac{L_1}{6EI} & (\dfrac{L_1}{3EI}+\dfrac{L_2}{3EI})
		\end{pmatrix}
		\begin{pmatrix}
			x_a \\ x_b
		\end{pmatrix}
		=
		\begin{pmatrix}
			\dfrac{q_1 L_1^{3}}{24EI} \\\\ \dfrac{q_1 L_1^{3}}{24EI} & \dfrac{q_2 L_2^{3}}{24EI}
		\end{pmatrix}
	\end{equation}	
	
	Il sistema matriciale qui sopra riportato ha la particolarità di essere sempre simmetrico rispetto la diagonale per il termine $D_{ij}$ \textsl{(spostamento relativo al G.D.L  i-esimo associato ad un vettore unitario)}.
	
	$\boxtimes \hfill$
	
	\break 
	
	\begin{figure}[H]
		\centering
		\hspace*{1cm}
		\begin{tikzpicture}
			% here we construct our structure
			\scaling{.45};
			%\draw[help lines,step=.45]
			(0,-.9) grid (12.6,1.81);
			\point{a}{2}{1};                                
			\point{b}{26}{1};                               
			\point{c}{14}{1};                               
			\point{d}{0}{1};            
			\point{e}{28}{1};       
			\point{forza}{28}{2.5};  
			\point{carico1}{2.4}{2.5};   
			\point{carico2}{26.4}{2.5};        
			\point{caricoin1}{0}{2};   
			\point{caricofin1}{2}{2};     
			\point{caricoin2}{14}{2};        
			\point{caricofin2}{26}{2};
			\point{momento}{14.4}{1};      
			\beam{2}{d}{e};                                  
			\support{1}{a}[0];                               
			\support{1}{b}[0];                               
			\support{1}{c}[0];
			\load{3}{momento}[270][180];
			\load{1}{e}[90];
			\lineload{2}{caricoin1}{caricofin1}[1][1];        
			\lineload{2}{caricoin2}{caricofin2}[1][1];        
			\dimensioning{1}{d}{a}{-1}[$L_1$];
			\dimensioning{1}{a}{c}{-1}[$L_2$];
			\dimensioning{1}{c}{b}{-1}[$L_3$];
			\dimensioning{1}{b}{e}{-1}[$L_4$];
			\notation{1}{d}{A}[left];
			\notation{1}{a}{B}[above];
			\notation{1}{c}{C}[above];
			\notation{1}{b}{D}[above];
			\notation{1}{e}{E}[right];
			\notation{1}{forza}{$F$}[right];
			\notation{1}{carico2}{$q_2$}[];
			\notation{1}{carico1}{$q_1$}[];
			
		\end{tikzpicture}
		\caption{}
		\label{fig:traveesss1}
	\end{figure}
	
	Possiamo eliminare lo sbalzo trasportando le forze $q_1$ e $F$ nei rispettivi punti B e D.
	
	\begin{figure}[H]
		\centering
		\begin{tikzpicture}
			% here we construct our structure
			\scaling{.45};
			%\draw[help lines,step=.45]
			(0,-.9) grid (12.6,1.81);
			\point{a}{2}{1};                                 %Punto A
			\point{b}{26}{1};                                %Punto B
			\point{c}{14}{1};                                %Punto C
			\point{d}{28}{1};
			\point{e}{0}{1};
			\point{momento}{15}{1};
			\point{caricofin}{26}{3};                        %Punto finale Carico q1
			\point{caricomed}{14}{3};                        %Punto medio Carico q1
			\beam{2}{a}{b};                                  %Trave A-B
			\support{1}{a}[0];                              %Incastro A
			\support{1}{c}[0];
			\support{1}{b}[0];
			\load{2}{d}[270][180];
			\load{3}{e}[90][180];
			\load{3}{momento}[270][180];
			\lineload{2}{caricomed}{caricofin}[1][1];        %Carico q2
			\dimensioning{1}{a}{c}{-1}[$L_2$];					 %Dimensione a
			\dimensioning{1}{c}{b}{-1}[$L_3$];					 %Dimensione a
			\notation{1}{a}{A}[above left];                  %Punto A Nomenclatura
			\notation{1}{b}{C}[above right];                 %Punto B Nomenclatura
			\notation{1}{c}{B}[above];                       %Punto C Nomenclatura
			\notation{1}{caricofin}{$q_2$}[above right];     %Carico q2 Nomenclatura
			\notation{1}{d}{$FL_4$}[right=4mm];
			\notation{1}{e}{$\dfrac{q_1 L_1^{2}}{2}$}[left=4mm];
			\notation{1}{momento}{$c$}[above right=3mm];
			
		\end{tikzpicture}
		\caption{}
		\label{fig:traveforzaaaa11111}
	\end{figure}
	
	Ora creiamo una sconnessione nel punto B che verrà bilanciata da due momenti opposti.
	
	\begin{figure}[H]
		\centering
		\begin{tikzpicture}
			% here we construct our structure
			\scaling{.45};
			%\draw[help lines,step=.45]
			(0,-.9) grid (12.6,1.81);
			\point{a}{2}{1};                                 %Punto A
			\point{b}{26}{1};                                %Punto B
			\point{c}{14}{1};                                %Punto C
			\point{d}{28}{1};
			\point{e}{0}{1};
			\point{momento}{15}{1};
			\point{momento2}{16.5}{1};
			\point{momento3}{13}{1};
			\point{caricofin}{26}{3};                        %Punto finale Carico q1
			\point{caricomed}{14}{3};                        %Punto medio Carico q1
			\beam{2}{a}{b};                                  %Trave A-B
			\support{1}{a}[0];                              %Incastro A
			\support{1}{c}[0];
			\support{1}{b}[0];
			\hinge{1}{c};
			\load{2}{d}[270][180];
			\load{3}{e}[90][180];
			\load{2}{momento3}[90][180];
			\load{3}{momento}[270][180];
			\load{3}{momento2}[270][180];
			\lineload{2}{caricomed}{caricofin}[1][1];        %Carico q2
			\notation{1}{a}{A}[above left];                  %Punto A Nomenclatura
			\notation{1}{b}{C}[above right];                 %Punto B Nomenclatura
			\notation{1}{c}{B}[above];                       %Punto C Nomenclatura
			\notation{1}{caricofin}{$q_2$}[above right];     %Carico q2 Nomenclatura
			\notation{1}{d}{$FL_4$}[right=4mm];
			\notation{1}{e}{$\dfrac{q_1 L_1^{2}}{2}$}[left=4mm];
			\notation{1}{momento}{$c$}[above right=3mm];
			\notation{1}{momento2}{$x_{bdx}$}[above right=3mm];
			\notation{1}{momento3}{$x_{bsx}$}[above left=3mm];
			
		\end{tikzpicture}
		\caption{}
		\label{fig:traveforzaaaa122111}
	\end{figure}
	
	Dato che sul nodo in B c'è un momento $c$ dobbiamo stare attenti a ricavare il valore risultante sulla trave. Infatti ciò è riconducibile a un nodo triplo che si potrà visualizzare con la seguente figura:
	
	\begin{figure}[H]
		\centering
		\begin{tikzpicture}
			% here we construct our structure
			\scaling{.45};
			%\draw[help lines,step=.45]
			(0,-.9) grid (12.6,1.81);
			\point{a}{2}{1};
			\point{b}{4}{1};
			\point{c}{10}{1};
			\point{d}{14}{1};
			\point{hinge}{12}{1};
			\point{e}{20}{1};
			\point{f}{22}{1};
			\point{g}{12}{2.2};
			\beam{1}{a}{b};
			\beam{1}{c}{d};
			\beam{1}{e}{f};
			\hinge{1}{hinge};
			\load{2}{b}[270][180];
			\load{3}{c}[90][180];
			\load{2}{d}[270][180];
			\load{3}{e}[90][180];
			\load{3}{g}[0][180];
			\notation{1}{b}{$x_{bsx}$}[above right=2mm];
			\notation{1}{c}{$x_{bsx}$}[below left=2mm];
			\notation{1}{d}{$x_{bdx}+c$}[above right=2mm];
			\notation{1}{e}{$x_{bdx}+c$}[below left=2.4mm];
			\notation{1}{g}{$c$}[above right=2.4mm];
			\notation{1}{hinge}{B}[below=2mm];
		\end{tikzpicture}
		\caption{}
		\label{fig:traveforzaaaa2211}
	\end{figure}
	
	$\boxtimes$ \textbf{N.B.} Quando abbiamo un \textbf{momento sul nodo} ci sarà sempre un salto nel diagramma del momento.
	
	Ora scriviamo le equazioni di congruenza per il nuovo corpo sconnesso.
	
	\begin{figure}[H]
		\centering
		\begin{tikzpicture}
			% here we construct our structure
			\scaling{.45};
			%\draw[help lines,step=.45]
			(0,-.9) grid (12.6,1.81);
			\point{a}{2}{1};                                 %Punto A
			\point{b}{26}{1};                                %Punto B
			\point{c}{14}{1};                                %Punto C
			\point{d}{.6}{1};
			\beam{2}{a}{b};                                  %Trave A-B
			\support{1}{a}[0];                               %Cerniera A
			\support{1}{b}[0];                               %Carrello B
			\internalforces{a}{b}{0}{0}[1][gray];
			\load{3}{d}[90][180];
			\notation{1}{d}{$\dfrac{q_1 L_1^{2}}{2}$}[left=4mm];
		\end{tikzpicture}
		\caption{Caso Momento $sx$ di trasporto $q_1$. Tratto AB.}
		\label{fig:appappbnesjsedfdkbdj}
	\end{figure}
	
	$ \varphi_{bsx}(q_1)=-\dfrac{q_1 L_1^{2}}{2}\cdot \dfrac{L_2}{6EI}$
	
	\begin{figure}[H]
		\centering
		\hspace*{1.6cm}
		\begin{tikzpicture}
			% here we construct our structure
			\scaling{.45};
			%\draw[help lines,step=.45]
			(0,-.9) grid (12.6,1.81);
			\point{a}{2}{1};                                 %Punto A
			\point{b}{26}{1};                                %Punto B
			\point{c}{14}{1};                                %Punto C
			\point{d}{27.4}{1};
			\beam{2}{a}{b};                                  %Trave A-B
			\support{1}{a}[0];                               %Cerniera A
			\support{1}{b}[0];                               %Carrello B
			\internalforces{a}{b}{0}{0}[1][gray];
			\load{2}{d}[270][180];
			\notation{1}{d}{$X_{bsx}$}[right=4mm];
		\end{tikzpicture}
		\caption{Caso $X_{bsx}$. Tratto AB.}
		\label{fig:appappbneqwedfsewwedj}
	\end{figure}
	
	$ \varphi_{bsx}(x)=-\dfrac{xL_2}{3EI}$
	
	\begin{figure}[H]
		\centering
		\hspace*{1.6cm}
		\begin{tikzpicture}
			% here we construct our structure
			\scaling{.45};
			%\draw[help lines,step=.45]
			(0,-.9) grid (12.6,1.81);
			\point{a}{2}{1};                                 %Punto A
			\point{b}{26}{1};                                %Punto B
			\point{c}{14}{1};                                %Punto C
			\point{d}{.6}{1};
			\beam{2}{a}{b};                                  %Trave A-B
			\support{1}{a}[0];                               %Cerniera A
			\support{1}{b}[0];                               %Carrello B
			\internalforces{a}{b}{0}{0}[-1][gray];
			\lineload{2}{a}{b};
			\notation{1}{b}{$q_2$}[above right=2mm];
			%\load{3}{d}[90][180];
			%\notation{1}{d}{$X_a$}[left=4mm];
		\end{tikzpicture}
		\caption{Caso $q_2$. Tratto BC.}
		\label{fig:asdgrfdf}
	\end{figure}
	
	$ \varphi_{bdx}(q_2)=-\dfrac{q_2 L_3^{3}}{24EI}$
	
	\begin{figure}[H]
		\centering
		\hspace*{1.6cm}
		\begin{tikzpicture}
			% here we construct our structure
			\scaling{.45};
			%\draw[help lines,step=.45]
			(0,-.9) grid (12.6,1.81);
			\point{a}{2}{1};                                 %Punto A
			\point{b}{26}{1};                                %Punto B
			\point{c}{14}{1};                                %Punto C
			\point{d}{27.4}{1};
			\beam{2}{a}{b};                                  %Trave A-B
			\support{1}{a}[0];                               %Cerniera A
			\support{1}{b}[0];                               %Carrello B
			\internalforces{a}{b}{0}{0}[1][gray];
			\load{2}{d}[270][180];
			\notation{1}{d}{$FL_4$}[right=4mm];
		\end{tikzpicture}
		\caption{Caso $F$. Tratto BC.}
		\label{fig:asasdcd}
	\end{figure}
	
	$ \varphi_{bdx}(F)=FL_4\cdot \dfrac{L_3}{6EI}$
	
	\begin{figure}[H]
		\centering
		\begin{tikzpicture}
			% here we construct our structure
			\scaling{.45};
			%\draw[help lines,step=.45]
			(0,-.9) grid (12.6,1.81);
			\point{a}{2}{1};                                 %Punto A
			\point{b}{26}{1};                                %Punto B
			\point{c}{14}{1};                                %Punto C
			\point{d}{.6}{1};
			\beam{2}{a}{b};                                  %Trave A-B
			\support{1}{a}[0];                               %Cerniera A
			\support{1}{b}[0];                               %Carrello B
			\internalforces{a}{b}{0}{0}[1][gray];
			\load{3}{d}[90][180];
			\notation{1}{d}{$x+c$}[left=4mm];
		\end{tikzpicture}
		\caption{Caso $x+c$. Tratto BC.}
		\label{fig:dsfvcbb}
	\end{figure}
	
	$ \varphi_{bdx}(x+c)=(x+c)\cdot \dfrac{L_3}{3EI}$
	
	\phantom{.}
	
	\phantom{.}
	
	$[\varphi_{bsx}=\varphi_{bdx}] \hfill$
	
	$\varphi_{bsx}=\varphi_{bsx}(q_1)+\varphi_{bsx}(x) \hfill$
	
	$\varphi_{bdx}=\varphi_{bdx}(q_2)+\varphi_{bdx}(F)+\varphi_{bdx}(x+c) \hfill$
	
	\phantom{.}
	
	Se volessimo ad esempio trovare il valore di x, grazie alle equazioni di congruenza basterà mettere a denominatore i termini di interesse e i restanti al numeratore:
	
	\phantom{.}
	
	$x=\dfrac{-\dfrac{q_1 L_1^{2}}{2}\cdot \dfrac{L_2}{6EI}+\dfrac{q_2 L_3^{3}}{24EI}-\dfrac{FL_4L_3}{6EI}-\dfrac{cL_3}{3EI} }{-\dfrac{L_2}{3EI}+\dfrac{L_3}{3EI}}$
	
	\break
	
	$\boxtimes$ Metodo delle forze (Esempio  Numerico) $\hfill$
	
	\begin{figure}[H]
		\centering
		\begin{tikzpicture}
			% here we construct our structure
			\scaling{.45};
			%\draw[help lines,step=.45]
			(0,-.9) grid (12.6,1.81);
			\point{a}{2}{1};               				
			\point{b}{14}{1};             				
			\point{c}{26}{1};
			\point{d}{2}{-4};
			\point{e}{14}{-4};
			\point{f}{26}{-4};
			\point{g}{1.4}{3};
			\point{h}{26.6}{3};
			\point{caricounoinizio}{2}{2};
			\point{caricounofine}{14}{2};
			\point{caricodueinizio}{14}{2};
			\point{caricoduefine}{26}{2};
			\beam{2}{a}{c};
			\support{3}{a}[-90];
			\support{1}{b}[0];
			\support{1}{c}[0];
			\lineload{2}{caricounoinizio}{caricounofine}[1][1];
			\lineload{2}{caricodueinizio}{caricoduefine}[2][2];
			\dimensioning{1}{a}{b}{-1}[$L_1$];
			\dimensioning{1}{b}{c}{-1}[$L_2$];
			\notation{1}{d}{$A$}[];
			\notation{1}{e}{$B$}[];
			\notation{1}{f}{$C$}[];
			\notation{1}{g}{$q_1$}[];
			\notation{1}{h}{$q_2$}[];
		\end{tikzpicture}
		\caption{Taglio.}
		\label{fig:taglifjouno}
	\end{figure}
	
	Calcoliamo la condizione statica di questa struttura:
	
	$2t-s=2-4 \Rightarrow 2$ volte iperstatica.
	
	Dati numerici:$\hfill$
	
	$q_1=2,00kN/m \hfill$
	
	$q_2=3,00kN/m \hfill$
	
	$L_1=3,00m \hfill$
	
	$L_2=2,00m \hfill$
	
	Ora costruiamo la nostra struttura declassata
	
	\begin{figure}[H]
		\centering
		\begin{tikzpicture}
			% here we construct our structure
			\scaling{.45};
			%\draw[help lines,step=.45]
			(0,-.9) grid (12.6,1.81);
			\point{a}{2}{1};               				
			\point{b}{14}{1};             				
			\point{c}{26}{1};
			\point{d}{12}{1};
			\point{e}{16}{1};
			\point{nota1}{11}{1};
			\point{nota2}{17}{1};
			\beam{2}{a}{c};
			\support{3}{a}[-90];
			\support{1}{b}[0];
			\support{1}{c}[0];
			\hinge{1}{a};
			\hinge{1}{b};
			\load{3}{a}[270][180][.8];
			\load{2}{d}[90][180][.8];
			\load{3}{e}[270][180][.8];
			\notation{1}{a}{$X_a$}[above right=5.8mm];
			\notation{1}{nota1}{$X_bsx$}[above left=5.8mm];
			\notation{1}{nota2}{$X_bdx$}[above right=5.8mm];
		\end{tikzpicture}
		\caption{Taglio.}
		\label{fig:mddkdkfnjvkeo}
	\end{figure}
	
	Eq. di congruenza:
	
	\begin{alignat*}{7}
		\varphi_{a} & {}=0{} & & {}\Rightarrow{} & \varphi_a(q_1) & {}+{} & \varphi_a(x_a) & {}+{} & \varphi_a(x_b) & {}={} & 0 \\
		\varphi_{bsx} & {}=\varphi_{bdx} {} & & {}\Rightarrow{} & \varphi_{bsx}(q_1) & {}+{} & \varphi_{bsx}(x_a) & {}+{} & \varphi_{bsx}(x_b) & {}={} & \varphi_{bdx}(q_2) & {}+{} & \varphi_{bdx}(x_b) 
	\end{alignat*}
	
	\begin{figure}[H]
		\centering
		\begin{tikzpicture}
			% here we construct our structure
			\scaling{.45};
			%\draw[help lines,step=.45]
			(0,-.9) grid (12.6,1.81);
			\point{a}{2}{1};                                 %Punto A
			\point{b}{26}{1};                                %Punto B
			\point{c}{14}{1};                                %Punto C
			\point{d}{.6}{1};
			\beam{2}{a}{b};                                  %Trave A-B
			\support{1}{a}[0];                               %Cerniera A
			\support{1}{b}[0];                               %Carrello B
			\internalforces{a}{b}{0}{0}[-1][gray];
			\lineload{2}{a}{b};
			\notation{1}{b}{$q_1$}[above right=2mm];
			%\load{3}{d}[90][180];
			%\notation{1}{d}{$X_a$}[left=4mm];
		\end{tikzpicture}
		\caption{Tratto AB.}
		\label{fig:asddddgrfdf}
	\end{figure}
	
	$ \varphi_a(q_1)=\dfrac{q_1 L_1^{3}}{24EI},\phantom{....} \varphi_{bsx}(q_1)=-\dfrac{q_1 L_1^{3}}{24EI} $
	
	\begin{figure}[H]
		\centering
		\hspace*{-1.6cm}
		\begin{tikzpicture}
			% here we construct our structure
			\scaling{.45};
			%\draw[help lines,step=.45]
			(0,-.9) grid (12.6,1.81);
			\point{a}{2}{1};                                 %Punto A
			\point{b}{26}{1};                                %Punto B
			\point{c}{14}{1};                                %Punto C
			\point{d}{.6}{1};
			\beam{2}{a}{b};                                  %Trave A-B
			\support{1}{a}[0];                               %Cerniera A
			\support{1}{b}[0];                               %Carrello B
			\internalforces{a}{b}{0}{0}[1][gray];
			\load{3}{d}[90][180];
			\notation{1}{d}{$X_a$}[left=4mm];
		\end{tikzpicture}
		\caption{Tratto AB.}
		\label{fig:ddfdfvcsaazx}
	\end{figure}
	
	$ \varphi_a(x_a)=-\dfrac{x_a L_1}{3EI}, \phantom{....} \varphi_{bsx}(x_a)=\dfrac{x_a L_1}{6EI} $
	
	\begin{figure}[H]
		\centering
		\hspace*{.6cm}
		\begin{tikzpicture}
			% here we construct our structure
			\scaling{.45};
			%\draw[help lines,step=.45]
			(0,-.9) grid (12.6,1.81);
			\point{a}{2}{1};                                 %Punto A
			\point{b}{26}{1};                                %Punto B
			\point{c}{14}{1};                                %Punto C
			\point{d}{27.4}{1};
			\beam{2}{a}{b};                                  %Trave A-B
			\support{1}{a}[0];                               %Cerniera A
			\support{1}{b}[0];                               %Carrello B
			\internalforces{a}{b}{0}{0}[1][gray];
			\load{2}{d}[270][180];
			\notation{1}{d}{$X_b$}[right=4mm];
		\end{tikzpicture}
		\caption{Tratto AB.}
		\label{fig:asasdsfdcd}
	\end{figure}
	
	$ \varphi_a(x_b)=-\dfrac{x_b L_1}{6EI},\phantom{....} \varphi_{bsx}=\dfrac{x_b L_1}{3EI} $
	
	\begin{figure}[H]
		\centering
		\hspace*{-.2cm}
		\begin{tikzpicture}
			% here we construct our structure
			\scaling{.45};
			%\draw[help lines,step=.45]
			(0,-.9) grid (12.6,1.81);
			\point{a}{2}{1};                                 %Punto A
			\point{b}{26}{1};                                %Punto B
			\point{c}{14}{1};                                %Punto C
			\point{d}{.6}{1};
			\beam{2}{a}{b};                                  %Trave A-B
			\support{1}{a}[0];                               %Cerniera A
			\support{1}{b}[0];                               %Carrello B
			\internalforces{a}{b}{0}{0}[-1][gray];
			\lineload{2}{a}{b};
			\notation{1}{b}{$q_2$}[above right=2mm];
			%\load{3}{d}[90][180];
			%\notation{1}{d}{$X_a$}[left=4mm];
		\end{tikzpicture}
		\caption{Tratto BC.}
		\label{fig:asdfddddgrfdf}
	\end{figure}
	
	$ \varphi_{bdx}(q_2)=-\dfrac{q_2 L_2^{3}}{24EI} $
	
	\begin{figure}[H]
		\centering
		\begin{tikzpicture}
			% here we construct our structure
			\scaling{.45};
			%\draw[help lines,step=.45]
			(0,-.9) grid (12.6,1.81);
			\point{a}{2}{1};                                 %Punto A
			\point{b}{26}{1};                                %Punto B
			\point{c}{14}{1};                                %Punto C
			\point{d}{.6}{1};
			\beam{2}{a}{b};                                  %Trave A-B
			\support{1}{a}[0];                               %Cerniera A
			\support{1}{b}[0];                               %Carrello B
			\internalforces{a}{b}{0}{0}[1][gray];
			\load{3}{d}[90][180];
			\notation{1}{d}{$X_b$}[left=4mm];
		\end{tikzpicture}
		\caption{Tratto BC.}
		\label{fig:fva23edcv}
	\end{figure}
	
	$ \varphi_{bsx}(x_b)=-\dfrac{x_b L_2}{3EI} $
	
	\phantom{text}
	
	\phantom{text}
	
	\begin{equation}
		\centering
		\begin{cases}
			\dfrac{q_1 L_1^{3}}{24EI}-\dfrac{x_a L_1}{3EI}-\dfrac{x_b L_1}{6EI}&=0 \\ \\
			-\dfrac{q_1 L_1^{3}}{24EI}+\dfrac{x_a L_1}{6EI}+\dfrac{x_b L_1}{3EI}&=\dfrac{q_2 L_2^{3}}{24EI}-\dfrac{x_b L_2}{3EI}\\ \\ 
		\end{cases}
		\hfill
	\end{equation}
	
	Trasformiamo il sistema in termini noti tutti positivi:
	
	\begin{equation}
		\centering
		\begin{cases}
			\dfrac{x_a L_1}{3EI}+\dfrac{x_b L_1}{6EI}&=\dfrac{q_1 L_1^{3}}{24EI} \\ \\
			\dfrac{x_a L_1}{6EI}+\dfrac{x_b L_1}{3EI}+\dfrac{x_b L_2}{3EI}&=\dfrac{q_1 L_1^{3}}{24EI}+\dfrac{q_2 L_2^{3}}{24EI}\\ \\ 
		\end{cases}
		\hfill
	\end{equation}
	
	\begin{equation}
		\begin{pmatrix}
			\dfrac{L_1}{3EI} & \dfrac{L_1}{6EI}\\\\ \dfrac{L_1}{6EI} & (\dfrac{L_1}{3EI}+\dfrac{L_2}{3EI})
		\end{pmatrix}
		\begin{pmatrix}
			x_a \\ x_b
		\end{pmatrix}
		=
		\begin{pmatrix}
			\dfrac{q_1 L_1^{3}}{24EI} \\\\ \dfrac{q_1 L_1^{3}}{24EI} & \dfrac{q_2 L_2^{3}}{24EI}
		\end{pmatrix}
	\end{equation}	
	
	\phantom{text}
	
	\phantom{text}
	
	Ora calcoliamo le reazioni vincolari:
	
	\begin{figure}[H]
		\centering
		\begin{tikzpicture}
			% here we construct our structure
			\scaling{.45};
			%\draw[help lines,step=.45]
			(0,-.9) grid (12.6,1.81);
			\point{a}{2}{1};                                 %Punto A
			\point{b}{26}{1};                                %Punto B
			\point{c}{14}{1};                                %Punto C
			\point{d}{.6}{1};
			\point{e}{27.4}{1};
			\beam{2}{a}{b};                                  %Trave A-B
			\support{1}{a}[0];                               %Cerniera A
			\support{1}{b}[0];                               %Carrello B
			\internalforces{a}{b}{0}{0}[-1][gray];
			\load{3}{d}[90][180];
			\load{2}{e}[270][180];
			\load{1}{a}[270][0][1];                          %Reazione Va
			\load{1}{b}[270][0][1];                          %Reazione Vb
			\lineload{2}{a}{b}
			\notation{1}{b}{$q_1$}[above right=1.8mm];
			\notation{1}{d}{$X_a$}[left=4mm];
			\notation{1}{e}{$X_b$}[right=4mm];
			\notation{1}{a}{$R_a$}[below=22mm];                   %Reazione Va Nomenclatura
			\notation{1}{b}{$R_b$}[below=22mm];                    %Reazione Vb Nomenclatura
		\end{tikzpicture}
		\caption{Tratto AB.}
		\label{fig:ddfdfvcfvbccvsdsaazx}
	\end{figure}
	
	\phantom{.}
	
	$\Sigma V=0,‎‎‎‎‎\hfill R_a + R_{bsx} - q_1 L_1 = 0$
	
	\phantom{.}
	
	$\Sigma M_a=0,‎‎‎‎‎\hfill -x_a+x_{bsx}+\dfrac{q_1 L_1^{2}}{2}-R_b L_1 = 0$
	
	\phantom{.}
	
	Otteniamo $R_a=3,00kN$,\phantom{,,} $R_{bsx}=3,00kN$,\phantom{,,} $x_a=1,50kNm$,\phantom{,,} $x_b=1,50kNm$.
	
	\phantom{text}
	
	\phantom{text}
	
	\begin{figure}[H]
		\centering
		\begin{tikzpicture}
			% here we construct our structure
			\scaling{.45};
			%\draw[help lines,step=.45]
			(0,-.9) grid (12.6,1.81);
			\point{a}{2}{1};                                 %Punto A
			\point{b}{26}{1};                                %Punto B
			\point{c}{14}{1};                                %Punto C
			\point{d}{.6}{1};
			\point{e}{27.4}{1};
			\beam{2}{a}{b};                                  %Trave A-B
			\support{1}{a}[0];                               %Cerniera A
			\support{1}{b}[0];                               %Carrello B
			\internalforces{a}{b}{0}{0}[-1][gray];
			\load{3}{d}[90][180];
			\load{1}{a}[270][0][1];                          %Reazione Va
			\load{1}{b}[270][0][1];                          %Reazione Vb
			\lineload{2}{a}{b}
			\notation{1}{b}{$q_2$}[above right=1.8mm];
			\notation{1}{d}{$X_b$}[left=4mm]
			\notation{1}{a}{$R_{bdx}$}[below=22mm];                   %Reazione Va Nomenclatura
			\notation{1}{b}{$R_c$}[below=22mm];                    %Reazione Vb Nomenclatura
		\end{tikzpicture}
		\caption{Tratto AB.}
		\label{fig:dfsgbvsmdlknfvc}
	\end{figure}
	
	\phantom{.}
	
	$\Sigma V=0,‎‎‎‎‎\hfill R_{bdx}+R_c-q_2 L_2 = 0$
	
	\phantom{.}
	
	$\Sigma M_b=0,‎‎‎‎‎\hfill -x_b+\dfrac{q_2 L_2^{2}}{2}-R_c L_2 = 0$
	
	\phantom{.}
	
	Otteniamo $R_b=3,75kN$,\phantom{,,} $R_c=2,25kN$.
	
	Adesso calcoliamo analiticamente il taglio e momento della struttura:
	
	\break
	
	TRATTO \textbf{A-B}
	
	\begin{equation}
		\centering
		\begin{cases}
			T(x)&=R_a - q_1 x \\ \\
			M(x)&=-x_a+R_a x-\dfrac{q_1 x^2}{2}\\ \\ 
		\end{cases}
		\hfill
	\end{equation}
	
	TRATTO \textbf{B-C}
	
	\begin{equation}
		\centering
		\begin{cases}
			T(x)&=R_{bdx} - q_2 x \\ \\
			M(x)&=-x_b+R_{bdx} x-\dfrac{q_2 x^2}{2}\\ \\ 
		\end{cases}
		\hfill
	\end{equation}
	
	\begin{figure}[H]
		\centering
		\begin{tikzpicture}
			% here we construct our structure
			\scaling{.45};
			%\draw[help lines,step=.45]
			(0,-.9) grid (12.6,1.81);
			\point{a}{2}{1};
			\point{b}{12}{1};
			\point{c}{22}{1};
			\beam{1}{a}{c};
			\internalforces{a}{b}{-3}{3}[0][blue][0];
			\internalforces{b}{c}{-3.75}{2.75}[0][blue][0];
		\end{tikzpicture}
		\caption{Diagramma Taglio.}
		\label{fig:sdv xccxcv}
	\end{figure}
	
	\begin{figure}[H]
		\centering
		\begin{tikzpicture}
			% here we construct our structure
			\scaling{.45};
			%\draw[help lines,step=.45]
			(0,-.9) grid (12.6,1.81);
			\point{a}{2}{1};
			\point{b}{12}{1};
			\point{c}{22}{1};
			\beam{1}{a}{c};
			\internalforces{a}{b}{-1.5}{-3}[-2.8][red];
			\internalforces{b}{c}{-3}{0}[-1.3][red];
		\end{tikzpicture}
		\caption{Diagramma Momento.}
		\label{fig:hvjghv}
	\end{figure}
	
	$\boxtimes$ \textbf{N.B.} Il punto di minimo parabolico del momento si trova dove il taglio è nullo.
	
	\break
	
	\section{Metodo degli Spostamenti}
	
	\begin{tabular}{|l|l|l|}
		\hline
		\rule[-1ex]{0pt}{3.5ex}  & \textbf{INCOGNITE} & \textbf{EQUAZIONI} \\
		\hline
		\rule[-1ex]{0pt}{3.5ex} Metodo delle forze & FORZE & Equazioni di congruenza \\
		\hline
		\rule[-1ex]{0pt}{3.5ex} Metodo degli spostamenti & SPOSTAMENTI & Equazioni di equilibrio \\
		\hline
	\end{tabular}
	
	\begin{figure}[H]
		\centering
		\begin{tikzpicture}
			% here we construct our structure
			\scaling{.45};
			%\draw[help lines,step=.45]
			(0,-.9) grid (12.6,1.81);
			\point{a}{2}{1};               				
			\point{b}{14}{1};             				
			\point{c}{26}{1};
			\point{d}{2}{-4};
			\point{e}{14}{-4};
			\point{f}{26}{-4};
			\point{g}{1.4}{3};
			\point{h}{26.6}{3};
			\point{caricounoinizio}{2}{2};
			\point{caricounofine}{14}{2};
			\point{caricodueinizio}{14}{2};
			\point{caricoduefine}{26}{2};
			\beam{2}{a}{c};
			\support{3}{a}[-90];
			\support{1}{b}[0];
			\support{1}{c}[0];
			\lineload{2}{caricounoinizio}{caricounofine}[1][1];
			\lineload{2}{caricodueinizio}{caricoduefine}[2][2];
			\dimensioning{1}{a}{b}{-1}[$L_1$];
			\dimensioning{1}{b}{c}{-1}[$L_2$];
			\notation{1}{d}{$A$}[];
			\notation{1}{e}{$B$}[];
			\notation{1}{f}{$C$}[];
			\notation{1}{g}{$q_1$}[];
			\notation{1}{h}{$q_2$}[];
			\notation{1}{b}{$\bullet$}[above left];
			\notation{1}{b}{$\bullet$}[above right];
			\notation{1}{b}{$\bullet$}[below left];
			\notation{1}{b}{$\bullet$}[below right];
		\end{tikzpicture}
		\caption{}
		\label{fig:dsxgfhvcvbrdgc}
	\end{figure}
	
    $\alpha$) \textbf{FASE A NODI BLOCCATI.} $\hfill$
	
	\phantom{.}
	
	In questa fase blocchiamo tanti nodi quante volte è iperstatica la struttura. I morsetti (Se guardiamo la [\ref{fig:dsxgfhvcvbrdgc}] ce n'è uno nel punto B) potranno essere disposti solo su vincoli interni. $\hfill$
	
	\phantom{text}
	
	Il morsetto è un ulteriore vincolo fittizio che ci farà alzare di 1 volta la iperstaticità strutturale. La sua condizione cinematica è $\varphi=0.\hfill$
	
	\begin{figure}[H]
		\centering
		\begin{tikzpicture}
			\centering
			% here we construct our structure
			\scaling{.40};
			%\draw[help lines,step=.45]
			(0,-.9) grid (12.6,1.81);
			\point{a}{0}{1};                                 %Punto A
			\point{b}{24}{1};                                %Punto B
			\point{c}{12}{1};                                %Punto C
			\point{d}{-1}{1};
			\point{e}{25.4}{1};
			\point{caricoin}{0}{2};
			\point{caricofin}{24}{2};
			\beam{2}{a}{b};                                  %Trave A-B
			\support{3}{a}[-90];                               %Cerniera A
			\support{3}{b}[90];                               %Carrello B
			\load{3}{d}[90][180];
			\load{2}{e}[270][180];
			\lineload{2}{caricoin}{caricofin}
			\notation{1}{caricofin}{$q_1$}[above right=1.8mm];
			\notation{1}{d}{$\dfrac{q_1 L_1^{2}}{12}$}[left=4mm];
			\notation{1}{e}{$M_{bsx}(q_1)$}[right=4mm];
		\end{tikzpicture}
		\caption{Tratto AB.}
		\label{fig:sfdgfgvcxesrg}
	\end{figure}
	
	$M_{bsx}(q_1)=-\dfrac{q_1 L_1^{2}}{12}$
	
	\begin{figure}[H]
		\centering
		\hspace*{-2.2cm}
		\begin{tikzpicture}
			\centering
			% here we construct our structure
			\scaling{.40};
			%\draw[help lines,step=.45]
			(0,-.9) grid (12.6,1.81);
			\point{a}{0}{1};                                 %Punto A
			\point{b}{24}{1};                                %Punto B
			\point{c}{12}{1};                                %Punto C
			\point{d}{-1}{1};
			\point{e}{25.4}{1};
			\point{caricoin}{0}{2};
			\point{caricofin}{24}{2};
			\beam{2}{a}{b};                                  %Trave A-B
			\support{3}{a}[-90];                               %Cerniera A
			\support{1}{b};                               %Carrello B
			\load{3}{d}[90][180];
			\notation{1}{d}{$M_{bdx}(q_2)$}[left=4mm];
		\end{tikzpicture}
		\caption{Tratto BC.}
		\label{fig:sdefsdfxcvcvbdes}
	\end{figure}
	
	$M_{bdx}(q_2)=\dfrac{q_2 L_2^{2}}{8}$
	
	\phantom{text}
	
	$M_{bsx}(q_1), M_{bdx}(q_2)$ sono azioni applicate dal nodo per effetto del carico esterno. Azioni che il vincolo ausiliario [\textbf{::}] deve esercitare per imporre la condizione cinematica $\varphi=0. \hfill$
	
	$\boxtimes \hfill$
	
	\phantom{text}
	
	\phantom{text}
	
	\phantom{text}
	
	$\beta$) \textbf{ATTIVAZIONE DEGLI SPOSTAMENTI NODALI.} $\hfill$
	
	\begin{figure}[H]
		\centering
		\begin{tikzpicture}
			% here we construct our structure
			\scaling{.45};
			%\draw[help lines,step=.45]
			(0,-.9) grid (12.6,1.81);
			\point{a}{2}{1};               				
			\point{b}{14}{1};             				
			\point{c}{26}{1};
			\point{d}{2}{-4};
			\point{e}{14}{-4};
			\point{f}{26}{-4};
			\point{g}{1.4}{3};
			\point{h}{26.6}{3};
			\beam{2}{a}{c};
			\support{3}{a}[-90];
			\support{1}{b}[0];
			\support{1}{c}[0];
			\internalforces{a}{b}{0}{0}[-1][gray];
			\internalforces{b}{c}{0}{0}[1][gray];
			\notation{1}{b}{$\bullet$}[above left];
			\notation{1}{b}{$\bullet$}[above right];
			\notation{1}{b}{$\bullet$}[below left];
			\notation{1}{b}{$\bullet$}[below right];
		\end{tikzpicture}
		\caption{Flessione dei tronchi indotta dagli spostamenti.}
		\label{fig:sdftgerte4t342td}
	\end{figure}
	
	\begin{figure}[H]
		\centering
		\hspace*{.9cm}
		\begin{tikzpicture}
			\centering
			% here we construct our structure
			\scaling{.45};
			%\draw[help lines,step=.45]
			(0,-.9) grid (12.6,1.81);
			\point{a}{0}{1};                                 %Punto A
			\point{b}{24}{1};                                %Punto B
			\point{c}{12}{1};                                %Punto C
			\point{d}{-1}{1};
			\point{e}{25.4}{1};
			\point{caricoin}{0}{2};
			\point{caricofin}{24}{2};
			\beam{2}{a}{b};                                  %Trave A-B
			\support{3}{a}[-90];                               %Cerniera A
			\support{1}{b}[0];                               %Carrello B
			\load{3}{e}[270][180];
			\notation{1}{e}{$M_{bsx}(\varphi_b)$}[right=4mm];
		\end{tikzpicture}
		\caption{}
		\label{fig:43terdfgdfg}
	\end{figure}
	
	$M_{bsx}(\varphi_b)=\dfrac{4EI}{L_1}\varphi_b$
	
	\begin{figure}[H]
		\centering
		\hspace*{-2.5cm}
		\begin{tikzpicture}
			\centering
			% here we construct our structure
			\scaling{.45};
			%\draw[help lines,step=.45]
			(0,-.9) grid (12.6,1.81);
			\point{a}{0}{1};                                 %Punto A
			\point{b}{24}{1};                                %Punto B
			\point{c}{12}{1};                                %Punto C
			\point{d}{-1}{1};
			\point{e}{25.4}{1};
			\point{caricoin}{0}{2};
			\point{caricofin}{24}{2};
			\beam{2}{a}{b};                                  %Trave A-B
			\support{3}{a}[-90];                               %Cerniera A
			\support{1}{b}[0];                               %Carrello B
		    \load{3}{d}[90][180];
			\notation{1}{d}{$M_{bdx}(\varphi_b)$}[left=4mm];
		\end{tikzpicture}
		\caption{}
		\label{fig:wer435tret453}
	\end{figure}
	 
	 $M_{bdx}(\varphi_b)=\dfrac{3EI}{L_2}\varphi_b$
	 
	$\boxtimes \hfill$
	
	\break
	
	$\gamma$) \textbf{SCRITTURA DELL'EQUAZIONE DI EQUILIBRIO AL NODO.} 
	
	\begin{figure}[H]
		\centering
		\begin{tikzpicture}
			% here we construct our structure
			\scaling{.45};
			%\draw[help lines,step=.45]
			(0,-.9) grid (12.6,1.81);
			\point{a}{2}{1};
			\point{b}{4}{1};
			\point{c}{10}{1};
			\point{d}{14}{1};
			\point{hinge}{12}{1};
			\point{e}{20}{1};
			\point{f}{22}{1};
			\point{g}{12}{2.2};
			\beam{1}{a}{b};
			\beam{1}{c}{d};
			\beam{1}{e}{f};
			\support{1}{hinge}[0];
			\load{3}{b}[270][180];
			\load{2}{c}[90][180];
			\load{2}{d}[270][180];
			\load{3}{e}[90][180];
			\load{3}{g}[0][180];
			\notation{1}{b}{$M_{bsx}$}[above right=2mm];
			\notation{1}{c}{$-M_{bsx}$}[below left=2mm];
			\notation{1}{d}{$-M_{bdx}$}[above right=2mm];
			\notation{1}{e}{$M_{bdx}$}[below left=2.4mm];
			\notation{1}{g}{$M_{est}$}[above right=2.4mm];
			\notation{1}{hinge}{B}[below=10mm];
		\end{tikzpicture}
		\caption{}
		\label{fig:sdfdvcxw222}
	\end{figure}
	
	La presenza di $M_{est}$ non è per forza verificata, ma nel caso sia la scriveremo come azione esterna applicata al nodo
	
	\phantom{text}
	
	$ -M_{bsx}-M_{bdx}+M_{est}=0 \Rightarrow M_{bsx}+M_{bdx}=M_{est} $
	
	\phantom{text}
	
	$M_{bsx}+M_{bdx}=$ Somma delle azioni applicate dal nodo sulla struttura.
	
	\phantom{,}
	
	$M_{est}$ Azioni esterne applicate sul nodo.
	
	\phantom{text}
	
	$-\dfrac{q_1 L_1^{2}}{12}+\dfrac{4EI}{L_1}\varphi_b+\dfrac{q_2 L_2^{2}}{8}+\dfrac{3EI}{L_2}\varphi_b=0$
	
	\phantom{text}
	
	\phantom{text}
	
	$ \varphi_b=\dfrac{\dfrac{q_1 L_1^{2}}{12}-\dfrac{q_2 L_2^{2}}{8}}{\dfrac{4EI}{L_1}+\dfrac{3EI}{L_2}} $
	
	\chapter{Sicurezza strutturale}
	
	La figura dell'ingegnere ha da valutare sia il grado di stabilità strutturale con cui la struttura si trova ad affrontare carichi che verificare la sicurezza in genere e progettare ad opera d'arte.$\hfill$
	
	\phantom{,}
	
	Oggi si parla molto di consequence based engineering, ovvero il pensare alle conseguenze di un collasso strutturale. Il collasso o fallimento di una struttura è l'incapacità di garantire le proprie funzioni/prestazioni. Ad esempio verso vibrazioni (caso dei ponti), fessurazioni (calcestruzzo), flessioni (solaio).$\hfill$
	
	\phantom{,}
	
	Nella progettazione moderna, si ha come target sia gli utenti di un fabbricato che i beni in esso contenuti. Quindi vogliamo che la struttura sia \textbf{resiliente} (capace di assorbire scosse e altri danni).$\hfill$
	
	\phantom{,}
	
	Il lavoro di un ingegnere è quindi quello di progettare tenendo conto di diversi stati limite. Questi stati limite si dividono in S.L di esercizio e S.L ultimi.$\hfill$
	
	\phantom{,}
	
	Gli stai limite poi si definiscono in base al tipo di fabbricato che andiamo a progettare. Tutto ciò va anche definito in base alla finestra temporale (garantire la stabilità strutturale per 5 o 10 o ... anni).$\hfill$
	
	\phantom{,}
	
	I parametri degli S.L sono da andare a ricercare nelle \textbf{NORME TECNICHE PER LE COSTRUZIONI}, e con riferimento a queste norme, nel punto 2.4 c'è ad esempio la vita nominale di progetto.$\hfill$
	
	\phantom{,}
	
	Vengono riportate vari range di durata per tipo di costruzione, ma cosa succede ecceduti quei valori standard? Come una macchina o altri oggetti bisogna manutenere e nel caso ristrutturare ciò che è soggetto a usura. Con una buona manutenzione si può mantenere un buono stato strutturale.$\hfill$
	
	\chapter{Calcestruzzo}
	
	
	
\end{document}