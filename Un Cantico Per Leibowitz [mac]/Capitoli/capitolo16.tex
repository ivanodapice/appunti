	\chapter{\phantom{title}}

\lettrine{I}{l} vecchio eremita stava sull\textquotesingle orlo della mesa e
osservava l\textquotesingle avvicinarsi del puntino di polvere
attraverso il deserto. L\textquotesingle eremita masticava, mormorava
parole e ridacchiava silenzioso nel vento. La sua pelle grinzosa era
bruciata dal sole, e aveva assunto il colore del vecchio cuoio, la sua
barba ispida era macchiata di giallo, attorno al mento. Portava in testa
un cappellaccio e, avvolto ai fianchi, un pezzo di tela rozzamente
tessuta a mano che sembrava tela da sacco\ldots{} e quelli erano i suoi
soli indumenti, a eccezione dei sandali e di un otre di pelle di capra.

Seguì con lo sguardo il puntino di polvere fino a che quello non
attraversò il villaggio di Sanly Bowitts e si avviò per la strada che
conduceva oltre la mesa.

--- Ah! --- sbuffò l\textquotesingle eremita, mentre gli occhi gli si
accendevano. --- Il \emph{suo} impero sarà ingrandito, e non vi sarà
fine alla \emph{sua} pace; \emph{egli} siederà sopra il \emph{suo}
regno.

Improvvisamente cominciò a scendere il letto inaridito del fiume come un
gatto a tre zampe, appoggiandosi al bastone, saltando da una pietra
all\textquotesingle altra e sdrucciolando. La polvere sollevata dalla
sua rapida discesa si alzò nel vento e si sparse, in lontananza.

Ai piedi della mesa scomparve nel labirinto e sedette, in attesa. Ben
presto sentì che il cavaliere si avvicinava a un trotto moderato, e
cominciò a spingersi verso la strada per guardare tra gli arbusti. Il
cavallo apparve, oltre la curva, avvolto in un sottile pulviscolo.
L\textquotesingle eremita balzò sul sentiero e alzò le braccia.

--- Ollallà! --- gridò; e quando il cavaliere si fermò, sfrecciò avanti
per afferrare le redini e per guardare ansiosamente
l\textquotesingle uomo che stava in sella.

I suoi occhi lampeggiarono per un attimo. --- Perché ci è nato un
bambino, e ci è dato un Figlio\ldots{} --- Ma poi
l\textquotesingle espressione ansiosa sfumò in tristezza. --- Non è
\emph{Lui}! --- brontolò irritato, rivolto al cielo.

Il cavaliere aveva gettato all\textquotesingle indietro il cappuccio e
stava ridendo. L\textquotesingle eremita lo guardò irritato, per un
momento, battendo le palpebre. Poi lo riconobbe.

Oh! --- grugnì. --- Tu! Pensavo che fossi morto, ormai! Che cosa stai
facendo, qui?

---Ti ho riportato la tua fuggitiva, Benjamin --- disse don Paulo. Tirò
un guinzaglio, e la capra dalla testa azzurra avanzò trotterellando, da
dietro il cavallo. Belò e tirò la corda, non appena vide
l\textquotesingle eremita. --- E\ldots{} ho pensato di farti visita.

Quella bestia è del Poeta --- grugnì l\textquotesingle eremita. ---
L\textquotesingle ha vinta onestamente in un gioco
d\textquotesingle azzardo\ldots{} sebbene barasse come un miserabile.
Riportala a lui, e permettimi di consigliarti di non immischiarti in
queste faccende mondane che non ti riguardano. Buongiorno. --- E si
voltò verso il fiume inaridito.

--- Aspetta, Benjamin. Prendi la tua capra, o la darò a un contadino.
Non voglio che se ne vada in giro per l\textquotesingle abbazia e vada a
belare in chiesa.

--- Non è una capra --- disse di rimando l\textquotesingle eremita. ---
È la Bestia che vide il tuo profeta, ed è stata creata perché una donna
la cavalcasse. Ti consiglio di maledirla e di cacciarla nel deserto.
Noterai, comunque, che ha lo zoccolo fesso e che rumina. --- E fece di
nuovo per allontanarsi.

Il sorriso dell\textquotesingle abate si spense. --- Benjamin, hai
veramente intenzione di risalire su quella collina senza neppure dire
``salve'' a un vecchio amico?

--- Salve gridò il Vecchio Ebreo, e prosegui indignato la sua marcia.
Dopo pochi passi si fermò e si voltò indietro. --- Non è necessario che
ti mostri così offeso --- disse. --- Sono passati cinque anni da quando
ti sei disturbato a venire qui per l\textquotesingle ultima volta,
``vecchio amico''. Ah!

--- Dunque è così! --- mormorò l\textquotesingle abate. Smontò di sella
e rincorse il Vecchio Ebreo. --- Benjamin, Benjamin, avrei voluto
venire\ldots{} ma non ho potuto.

L\textquotesingle eremita si fermò. --- Ecco, Paulo, visto che sei
qui\ldots{} Improvvisamente scoppiarono a ridere e si abbracciarono.

--- Va bene, vecchio brontolone --- disse l\textquotesingle eremita.

--- Io sarei un brontolone?

--- Bene, anch\textquotesingle io sto diventando bisbetico, credo.
L\textquotesingle ultimo secolo è stato molto duro, per me.

--- Ho sentito che getti pietre contro i novizi che vengono in questa
zona per il digiuno quaresimale nel deserto. È vero? E guardò
l\textquotesingle eremita con ironico rimprovero.

--- Getto soltanto ciottoli.

--- Miserabile vecchio ebreo!

--- Suvvia, suvvia, Paulo. Uno di loro, una volta mi scambiò per un mio
lontano parente\ldots{} che si chiamava Leibowitz. Credeva che fossi
stato mandato per portargli un messaggio\ldots{} o qualcuno dei vostri
buoni a nulla lo credette. Non voglio che questo accada di nuovo, quindi
getto ciottoli contro i novizi, qualche volta. Ah! Non voglio più essere
scambiato per \emph{quel} mio parente, perché egli non appartiene più
alla mia famiglia.

Il prete assunse un\textquotesingle espressione perplessa. --- Ti
scambiò per chi? Per san Leibowitz? Suvvia, Benjamin! Stai andando
troppo oltre!

Benjamin ripeté la frase in una cantilena ironica: --- Mi scambiò per un
mio lontano parente\ldots{} che si chiamava Leibowitz, quindi adesso
getto ciottoli contro i novizi.

Don Paulo era molto perplesso. --- San Leibowitz è morto da dodici
secoli. Come poteva\ldots{} --- Si interruppe e guardò fisso il vecchio
eremita. --- Suvvia, Benjamin, non ricominciamo con quella favola. Tu
non puoi avere milleduecento\ldots{}

--- Sciocchezze! --- interruppe il Vecchio Ebreo. --- Non ho detto che
questo accadde dodici secoli fa. Fu soltanto sei secoli fa. Molto tempo
dopo la morte del tuo santo; ecco perché è stato così assurdo.
Naturalmente, i vostri novizi erano molto più devoti, a quei tempi, e
più creduli. Credo che quello si chiamasse Francis. Poveraccio. Fui io a
seppellirlo, più tardi. E dissi a quelli di Nuova Roma dove dovevano
scavare, se volevano trovarlo. Ecco in che modo avete riavuto la sua
carcassa.

L\textquotesingle abate fissò il vecchio a bocca spalancata, mentre si
dirigevano verso il pozzo, guidando il cavallo e la capra.
\emph{Francis?} si chiese. Francis. Poteva essere il venerabile Francis
Gerard dello Utah, forse\ldots{} al quale un pellegrino aveva rivelato
un tempo l\textquotesingle ubicazione dell\textquotesingle antico
rifugio nel villaggio, così affermava la storia\ldots{} ma questo era
avvenuto prima che lì sorgesse il villaggio. E circa sei secoli prima,
sì, e\ldots{} e adesso quel vecchio imbroglione sosteneva di essere
stato lui, quel pellegrino? Qualche volta si chiedeva dove mai Benjamin
avesse attinto una sufficiente conoscenza della storia
dell\textquotesingle abbazia per inventare simili fole. Forse dal Poeta.

--- Questo, naturalmente, fu durante la mia precedente carriera ---
proseguì il Vecchio Ebreo. --- E forse quell\textquotesingle errore era
comprensibile.

--- Precedente carriera?

--- Vagabondo.

--- E come puoi pretendere che io creda a queste sciocchezze?

--- \emph{Hmmmm-hmmm!} Il Poeta mi crede.

--- Senza dubbio! Il Poeta non crederebbe mai che il venerabile Francis
abbia incontrato un santo. Questo sarebbe superstizione. Il Poeta
preferirebbe credere che abbia incontrato \emph{te\ldots{}} sei secoli
fa. Una spiegazione del tutto naturale, eh?

Benjamin ridacchiò, maliziosamente. Paulo lo guardò abbassare nel pozzo
una logora tazza di corteccia, vuotarla nell\textquotesingle otre, e
riabbassarla di nuovo. L\textquotesingle acqua era torbida e brulicante
di cose vive e piuttosto incerte, come lo era il flusso dei ricordi del
Vecchio Ebreo. Ma la sua memoria era veramente incerta? O si prende
gioco di tutti noi?, si chiese il prete. A parte la sua illusione di
essere più vecchio di Matusalemme, il vecchio Benjamin Eleazar sembrava
abbastanza sano di mente, in quel suo modo bizzarro e malizioso.

--- Vuoi bere? --- offri l\textquotesingle eremita, tendendogli la
tazza. L\textquotesingle abate represse un brivido, ma accettò la tazza,
per non offenderlo, e inghiottì in un solo sorso il liquido fangoso.

--- Non sei molto schizzinoso, vero? --- disse Benjamin, osservandolo
con aria critica. --- io non la toccherei. --- E batté una mano
sull\textquotesingle otre. -\/- La tengo per gli animali.

L\textquotesingle abate si morse le labbra.

--- Sei cambiato --- disse Benjamin, continuando a osservarlo. --- Sei
pallido come il formaggio, e sciupato.

--- Sono stato malato.

--- Sembri malato anche adesso. Vieni alla mia baracca, se salire non ti
stanca troppo.

--- Mi sentirò subito meglio. Ho avuto un piccolo disturbo,
l\textquotesingle altro giorno, e il nostro medico mi ha ordinato di
riposare. Puah! Se non stesse per arrivare un ospite importante non vi
farei caso. Ma sta per arrivare, quindi mi riposo. È molto noioso
riposarsi.

Benjamin si voltò a guardarlo con un sogghigno mentre risalivano il
letto del fiume. Poi scosse la testa grigia. --- Cavalcare per dieci
miglia nel deserto è riposante?

--- Per me è un riposo. E poi volevo vederti, Benjamin.

--- Che cosa diranno gli abitanti del villaggio? --- chiese in tono
canzonatorio il Vecchio Ebreo. --- Penseranno che ci siamo riconciliati,
e questo rovinerà la reputazione di entrambi.

--- La nostra reputazione non è mai stata molto quotata sul mercato, non
è così?

--- È vero --- ammise, ma aggiunse, enigmaticamente: --- Per il momento.

--- Stai ancora aspettando, Vecchio Ebreo?

--- Certamente! --- insorse l\textquotesingle eremita.

La salita stancò l\textquotesingle abate. Per due volte si fermarono a
riposare. Prima che raggiungessero il piccolo altipiano, don Paulo fu in
preda alle vertigini e si appoggiò al magrissimo eremita. Un fuoco gli
bruciava nel petto, mettendolo in guardia contro ogni ulteriore sforzo,
ma non c\textquotesingle era traccia della stretta irosa che
l\textquotesingle aveva aggredito altre volte.

Un gregge di capre mutanti dalla testa azzurra si sparpagliò
all\textquotesingle avvicinarsi del forestiero e fuggì nel labirinto.
Stranamente, la mesa sembrava più verdeggiante del deserto che la
circondava, sebbene non vi fosse alcuna sorta visibile di umidità.

--- Da questa parte, Paulo. Alla mia magione.

La casa del Vecchio Ebreo era composta di una sola stanza, priva di
finestre e dai muri di pietra, in cui i sassi erano radi come in una
staccionata, con ampi varchi attraverso i quali poteva soffiare il
vento. Il tetto era un fragile intreccio di pali, quasi tutti contorti,
coperto da un mucchio di paglia, di arbusti e di pelli di capra. Su una
grande pietra piatta, posata su una corta colonna accanto alla porta,
c\textquotesingle era una scritta in ebraico:

\begin{center}
	{\Huge{\textcjheb{Mylhw' NynqtM	.hp}}}
\end{center}
~

Le dimensioni dell\textquotesingle insegna e il suo apparente tentativo
pubblicitario fecero sogghignare don Paulo, che domandò: --- Cosa
significa, Benjamin? Attira molti clienti, quassù?

--- Ah!\ldots{} Cosa dovrebbe significare? Dice: QUI SI RIPARANO TENDE.

Il prete sbuffò, incredulo.

--- Benissimo, dubita pure di me. Ma se non credi a ciò che è scritto
lì, non potrai credere certamente a ciò che è scritto
sull\textquotesingle altro lato dell\textquotesingle insegna.

--- Il lato verso il muro?

--- Certo, e quale se no?

La colonna era posta vicino alla soglia, così che
v\textquotesingle erano soltanto poco spazio libero fra la pietra piatta
e il muro della capanna. Paulo si piegò e strinse gli occhi per aguzzare
la vista. Gli occorse un po\textquotesingle{} di tempo per distinguere
la scritta, ma senza dubbio c\textquotesingle era qualcosa, sul tergo
della pietra, in lettere più piccole:

\begin{center}
	{\Huge{\textcjheb{K.h' hwhy wnyhl' hwhy l'd/sy .sM/s}}}
\end{center}

~

--- Non giri mai la pietra?

--- Girarla? Credi che io sia pazzo? In tempi come questi?

--- Che cosa dice, lì dietro?

--- \emph{Hmmmm-hmmm!} ---cantilenò l\textquotesingle eremita,
rifiutando di rispondere. --- Ma fatti più vicino, non puoi leggere in
quel modo.

--- C\textquotesingle è il muro che me lo impedisce.

--- Non è sempre stato così, forse?

Il prete sospirò. --- Sta bene, Benjamin, so che è ciò che hai avuto
l\textquotesingle ordine di scrivere all\textquotesingle ingresso e
sulla porta della tua casa. Ma soltanto tu potevi pensare di mettere
quella scritta a faccia in giù.

Con la faccia verso l\textquotesingle interno --- corresse
l\textquotesingle eremita.

Finché vi sono tende da riparare, in Israele\ldots{} ma non cominciamo a
punzecchiarci, finché non ti sarai riposato. Ti porterò un
po\textquotesingle{} di latte, e tu mi dirai del visitatore che ti
preoccupa.

--- C\textquotesingle è del vino nella mia borsa, se ne vuoi --- disse
l\textquotesingle abate, lasciandosi cadere, con sollievo, su un mucchio
di pelli.

--- Ma preferirei non parlare del Thon Taddeo.

--- Oh! \emph{Lui}.

--- Hai sentito parlare del Thon Taddeo? Dimmi, come mai tu riesci
sempre a conoscere tutto e tutti senza allontanarti da questa collina?

--- Io vedo e sento --- disse enigmatico l\textquotesingle eremita.

--- Dimmi, cosa ne pensi di lui?

--- Non l\textquotesingle ho mai visto. Ma credo che sarà un dolore. Un
dolore di parto, forse, ma un dolore.

--- Dolore di parto? Credi davvero che avremo un nuovo Rinascimento,
come afferma qualcuno?

--- \emph{Hmmmm-hmmm}.

--- Smettila di fare il misterioso, Vecchio Ebreo, e dimmi la tua
opinione. Devi averne una. L\textquotesingle hai sempre. Perché è così
difficile ottenere la tua confidenza? Non siamo amici, forse?

--- In certi campi, in certi campi. Ma tra me e te vi sono alcune
differenze di idee.

--- E cos\textquotesingle hanno a che vedere le nostre differenze di
idee con il Thon Taddeo e con un Rinascimento che entrambi vorremmo
vedere? Il Thon Taddeo è uno studioso secolare, e piuttosto lontano
dalle nostre differenze di idee.

Benjamin scrollò eloquentemente le spalle. --- Differenze di idee,
studiosi secolari --- echeggiò, sputando le parole come se fossero semi
di mela. --- Anch\textquotesingle io sono stato definito uno studioso
secolare, in varie epoche, da certa gente, e qualche volta sono stato
arso vivo, lapidato e impalato, per questo.

--- Ma tu non sei mai\ldots{} --- Il prete si interruppe, corrugando
severamente la fronte. Ancora quella pazzia. Benjamin lo fissava
sospettoso, e il suo sorriso era diventato freddo. ``Adesso'' pensò
l\textquotesingle abate ``mi guarda come se io fossi uno di
\emph{Loro\ldots{}} chiunque siano stati quei
\textquotesingle Loro\textquotesingle{} senza forma che lo hanno spinto
qui, in solitudine. Arso vivo, lapidato e impalato? O il suo
\textquotesingle io\textquotesingle{} significava
\textquotesingle noi\textquotesingle, come nella frase
\textquotesingle io, il mio popolo\textquotesingle?''

--- Benjamin\ldots{} io sono Paulo. Torquemada è morto. Io nacqui
settanta e più anni or sono, e presto morirò. Ti ho voluto bene,
vecchio, e quando tu mi guardi, vorrei che vedessi Paulo del Pecos e
niente altro.

Benjamin ondeggiò, per un momento. Gli occhi gli si inumidirono. ---
Qualche volta\ldots{} dimentico\ldots{}

--- E qualche volta dimentichi che Benjamin è soltanto Benjamin, e non
tutto Israele.

--- Mai --- scattò l\textquotesingle eremita, con gli occhi che
lampeggiavano di nuovo. --- Da trentadue secoli, io.., --- Si interruppe
e serrò strettamente le labbra.

--- Perché? --- sussurrò l\textquotesingle abate, quasi intimorito. ---
Perché prendi su te solo il fardello di un intero popolo e del suo
passato?

Gli occhi dell\textquotesingle eremita lampeggiarono un breve
avvertimento; ma poi deglutì, con un suono gutturale e si nascose la
faccia fra le mani. --- Tu peschi in acque cupe.

--- Perdonami.

--- Il fardello\ldots{} mi è stato imposto da altri. --- E alzò il capo,
lentamente. --- Dovrei rifiutare di portarlo?

Il prete trattenne il respiro. Per qualche tempo non si udì altro suono
che il rumore del vento. C\textquotesingle era un tocco di divinità in
quella follia!, pensò don Paulo. La comunità ebrea era molto dispersa,
in quei tempi. Forse Benjamin era sopravvissuto ai suoi figli, ed era
diventato, in qualche modo, uno sbandato. Un vecchio israelita poteva
vagabondare per anni senza incontrare altri della sua razza. Forse,
nella sua solitudine, aveva acquisito la silenziosa convinzione che egli
era l\textquotesingle ultimo, l\textquotesingle uno, il solo. Ed essendo
l\textquotesingle ultimo, aveva cessato di essere Benjamin, ed era
diventato Israele. E sul suo cuore aveva fondato la storia di cinquemila
anni, non più remota, ma divenuta la storia della sua stessa vita. Il
suo ``io'' era l\textquotesingle equivalente del ``noi'' imperiale.

``Ma anch\textquotesingle io sono un membro di una unicità'' pensò don
Paulo ``una parte d\textquotesingle una congregazione e
d\textquotesingle una comunità. Anche i miei sono stati disprezzati dal
mondo. Eppure per me la distinzione fra l\textquotesingle individuo e la
nazione è chiara. Per te, vecchio amico, in un certo senso è diventata
oscura. Un fardello imposto sulle tue spalle, da altri? E tu lo hai
accettato? Quanto deve pesare? Quanto peserebbe, per me? Vi mise sotto
le spalle e tentò di sollevarlo, per saggiarne la massa: io sono un
monaco e un sacerdote cristiano, e perciò sono responsabile davanti a
Dio delle azioni e dei gesti di ogni monaco e di ogni sacerdote che ha
respirato e ha camminato sulla terra dopo Cristo, come lo sono dei miei
stessi atti.''

Rabbrividì, cominciò a scuotere il capo.

No, no. Quel fardello spezzava la spina dorsale. Era troppo pesante
perché un uomo potesse portarlo, a eccezione di Cristo. Essere maledetto
per una fede era già un fardello abbastanza grave. Sopportare le
maledizioni era possibile, ma allora\ldots{} accettare
l\textquotesingle illogicità che stava dietro quelle maledizioni,
l\textquotesingle illogicità che chiamava un individuo a rispondere non
solo per se stesso ma anche per ogni membro della sua razza o della sua
fede, per ogni loro azione come delle sue stesse azioni? Accettare anche
questo?\ldots{} come stava tentando di fare Benjamin?

No, no.

Eppure, la Fede di don Paulo gli diceva che c\textquotesingle era quel
fardello, c\textquotesingle era sempre stato fin dal tempo di
Adamo\ldots{} e quel fardello era imposto da un malvagio che gridava
beffardamente \emph{``Uomo!''} all\textquotesingle uomo.
\emph{``Uomo!''} e chiamava ciascuno a rendere conto delle azioni di
tutti, fin dal principio; un fardello imposto a ogni generazione prima
ancora che uscisse dal grembo materno, il fardello della colpa del
peccato originale. Lascia che gli sciocchi ne discutano. Gli stessi
sciocchi accettavano con grande orgoglio \emph{l\textquotesingle altra}
eredità\ldots{} l\textquotesingle eredità d\textquotesingle una gloria,
d\textquotesingle una virtù, d\textquotesingle un trionfo,
d\textquotesingle una dignità ancestrale che li rendevano ``coraggiosi e
nobili in virtù del diritto di nascita'', senza protestare che,
personalmente, non avevano fatto nulla per meritare
quell\textquotesingle eredità, oltre a essere nati dalla razza
dell\textquotesingle Uomo. Le proteste erano riservate per il Fardello
ereditario che li rendeva ``colpevoli e fuorilegge per diritto di
nascita'', e contro quel verdetto si sforzavano di chiudersi le
orecchie. Quel fardello, in verità, era pesante. E la sua stessa Fede
gli diceva che quel fardello gli era stato tolto dalle spalle a opera di
Uno la cui immagine pendeva da una croce sugli altari, sebbene
l\textquotesingle impronta di quel fardello fosse ancora lì.
Quell\textquotesingle impronta era un giogo molto leggero, paragonato al
peso pieno della maledizione originale. Non riusciva a indursi a dirlo
al vecchio, poiché il vecchio sapeva già che lui lo credeva. Benjamin
stava cercando Un Altro. E l\textquotesingle ultimo Vecchio Ebreo sedeva
solo su di una montagna e faceva penitenza per Israele e aspettava il
Messia, e aspettava, e aspettava, e\ldots{}

--- Dio ti benedica perché sei un pazzo coraggioso. E anche un pazzo
molto saggio.

--- \emph{Hmmmm-hmmm!} Un pazzo molto saggio! --- lo scimmiottò
l\textquotesingle eremita. --- Ma tu ti specializzi sempre in paradossi
e misteri, non è vero, Paulo? Se qualcosa non può essere in
contraddizione con se stessa, allora non ti interessa neppure, non è
così? Tu devi trovare la Trinità nell\textquotesingle Unità, la vita
nella morte, la saggezza nella follia. Altrimenti basterebbe il senso
comune!

--- Sentire la responsabilità è saggezza, Benjamin. Pensare di poterla
sopportare da solo è follia.

--- Non pazzia?

--- Un po\textquotesingle, forse. Ma una pazzia coraggiosa.

--- E allora ti rivelerò un piccolo segreto. Ho sempre saputo che non
posso sopportarla, fin da quando Egli mi resuscitò. Ma stiamo parlando
della stessa cosa?

Il prete alzò le spalle. --- Tu lo chiameresti il peso di essere Eletto.
Io lo chiamerei il fardello della Colpa Originale. In ogni caso, la
responsabilità sottintesa è la stessa, sebbene noi possiamo darne
differenti versioni, e possiamo discutere con parole violente di ciò che
noi intendiamo dire con le parole e che non può essere affatto spiegato
con le parole\ldots{} poiché è qualcosa che può essere spiegato soltanto
nel silenzio mortale di un cuore.

Benjamin ridacchiò. --- Bene, sono contento di sentirtelo ammettere,
finalmente, anche se tutto ciò che dici è che in realtà tu non hai mai
detto niente.

--- Smettila di sghignazzare, reprobo!

--- Ma voi avete sempre usato parole così verbose in abile difesa della
vostra Trinità, anche se Egli non ha mai avuto bisogno di una simile
difesa, prima che Lo prendeste da me, come Unità. Eh?

Il prete arrossì, ma non disse nulla.

--- Ecco! gridò Benjamin, saltando su e giù. --- Ti ho messo voglia di
discutere, per una volta! Ah! Ma non badarci! Anch\textquotesingle io mi
servo di molte parole, però non sono mai sicuro che io e Lui vogliamo
dire la stessa cosa. Suppongo che non vi si possa biasimare: deve essere
molto più complicato con Tre che con Uno.

--- Vecchio cactus bestemmiatore! Io volevo conoscere la tua opinione
sul Thon Taddeo e su ciò che sta bollendo in pentola.

--- E perché cerchi l\textquotesingle opinione d\textquotesingle un
povero vecchio anacoreta?

--- Perché, Benjamin Eleazar bar Joshua, se tutti questi anni di attesa
per Colui. Che Non Verrà non ti hanno insegnato la saggezza, per lo meno
ti hanno reso acuto.

Il Vecchio Ebreo chiuse gli occhi, levò il volto al soffitto, e sorrise,
astutamente. --- Insultami --- disse in tono beffardo --- imprigionami,
tormentami, perseguitami\ldots{} ma sai ciò che ti dirò?

--- Dirai \emph{``Hmmmm-hmmm!''}.

--- No! Ti dirò che Egli è già qui. Una volta io l\textquotesingle ho
perfino veduto.

--- Cosa? Di chi stai parlando? Del Thon Taddeo?

--- No! Inoltre, non ci tengo a profetizzare, a meno che tu non mi dica
che cosa ti turba veramente, Paulo.

--- Bene, tutto è cominciato con la lampada di frate Kornhoer.

--- Lampada? Oh, si, il Poeta me ne ha parlato. Aveva profetizzato che
non avrebbe funzionato.

--- Il Poeta si sbagliava, come al solito. Così mi hanno detto. Non ho
assistito all\textquotesingle esperimento.

--- Allora ha funzionato? Splendido. E questo a cosa ha dato inizio?

--- Ai miei problemi. Quanto siamo vicini all\textquotesingle orlo di
qualche cosa? O quanto siamo vicino alla riva? Essenze Elettriche nel
sotterraneo. Ti rendi conto di quante cose sono cambiate in questi
ultimi due secoli?

Rapidamente, il prete espose i suoi timori, mentre
l\textquotesingle eremita riparatore di tende ascoltava pazientemente,
fino a che il sole cominciò a filtrare fra le aperture del muro
occidentale, dipingendo strisce lucenti nell\textquotesingle aria
polverosa.

--- Dalla morte dell\textquotesingle ultima civiltà, i Memorabilia sono
stati un nostro speciale dominio, Benjamin. E noi li abbiamo serbati. Ma
ora? Io penso al calzolaio che tenta di vendere scarpe in un villaggio
di calzolai.

L\textquotesingle eremita sorrise. --- Sarebbe possibile, se sa
fabbricare un tipo di scarpa speciale e superiore.

--- Temo che gli studiosi secolari stiano già cominciando ad avanzare
pretese su questo metodo.

--- E allora abbandona la fabbricazione di scarpe, prima di andare in
rovina.

--- È una possibilità --- ammise l\textquotesingle abate. --- È
spiacevole pensarvi, comunque. Per dodici secoli, noi siamo stati
un\textquotesingle isoletta in un oceano di tenebre. Conservare i
Memorabilia è stato un compito ingrato, ma anche un compito santo, noi
pensiamo. È soltanto il nostro compito mondano, ma noi siamo sempre
stati contrabbandieri di libri e memorizzatori, e non è facile credere
che questo compito sarà presto finito\ldots{} che ben presto cesserà di
essere necessario. Non posso crederlo.

--- Così cerchi di superare gli altri ``calzolai'' costruendo strani
trabiccoli nei tuoi sotterranei?

--- Devo ammettere che può sembrare\ldots{}

--- E cosa farete, poi, per mantenere un vantaggio sui secolari?
Costruirete una macchina volante? E ricreerete la \emph{Machina
	analytica}? O forse passerete sulle loro teste e vi dedicherete alla
metafisica?

--- Tu vuoi svergognarmi, Vecchio Ebreo. Sai che noi siamo per prima
cosa monaci di Cristo, e spetta agli altri fare queste cose.

--- Non volevo svergognarti. Non vedo niente di assurdo, se i monaci di
Cristo costruiscono una macchina volante; anche se sarebbe più
confacente ai loro principi costruire invece una macchina per pregare.

--- Disgraziato! Rendo un pessimo servizio al mio Ordine confidandomi
con te!

Benjamin sorrise. --- Non provo comprensione per voi. I libri che avete
serbato possono essere carichi di anni, ma furono scritti da figli del
mondo e vi saranno tolti da figli del mondo e, tanto per cominciare, non
toccava a voi occuparvi di questo.

--- Ah, adesso ti fa piacere profetizzare!

--- No, affatto. ``Presto il sole tramonterà\ldots'' è una profezia? No,
è puramente una affermazione di fede nella consistenza degli eventi.
Anche i figli del mondo sono consistenti\ldots{} così io dico che vi
prenderanno tutto ciò che potrete offrire, vi porteranno via il vostro
lavoro, e poi vi accuseranno di essere vecchi ruderi. E, alla fine, si
dimenticheranno completamente di voi. È colpa vostra. Il Libro che io vi
diedi avrebbe dovuto essere sufficiente, per voi. Ora dovrete affrontare
le conseguenze del vostro comportamento.

Aveva parlato in tono pungente, ma la sua predizione sembrava vicina, in
modo inquietante, ai timori di don Paulo. Il viso del prete assunse
un\textquotesingle espressione triste.

--- Non badare a ciò che dico --- dichiarò l\textquotesingle eremita.
--- Non mi azzarderò a fare il profeta prima di aver visto quel vostro
ordigno, o di aver dato un\textquotesingle occhiata a questo Thon
Taddeo\ldots{} che comincia a interessarmi, fra l\textquotesingle altro.
Aspetta fino a che non avrò esaminato le viscere della nuova epoca nei
minimi particolari, se pretendi di ottenere qualche consiglio da me.

--- Già, non vedrai quella lampada perché non verrai mai
all\textquotesingle abbazia.

--- È la vostra abominevole cucina che non mi va.

--- E non vedrai il Thon Taddeo perché verrà dall\textquotesingle altra
direzione. Se aspetterai, per esaminare le viscere della nuova epoca
fino a che questa sarà nata, sarà troppo tardi per profetizzare la sua
nascita.

--- Sciocchezze. Sondare l\textquotesingle utero del futuro è dannoso
per il nascituro. Aspetterò\ldots{} e poi potrò profetizzare che è nato
e che non era ciò che io attendevo.

--- Che allegra prospettiva! Dunque, che cosa stai aspettando?

--- Qualcuno che un tempo mi gridò\ldots{}

--- Gridò?

--- ``Vieni fuori!''

--- Che abominio!

--- \emph{Hmmm-hnnn!} Per dirti la verità, non sono assolutamente
convinto che Egli verrà, ma mi è stato detto di aspettare e\ldots. ---
alzò le spalle ---\ldots{} e io aspetto.

Dopo un attimo i suoi occhi ammiccanti si socchiusero fino a diventare
due fessure; si piegò in avanti, con improvvisa impazienza. --- Paulo,
fa\textquotesingle{} passare questo Thon Taddeo davanti ai piedi della
mesa.

L\textquotesingle abate si ritrasse, con orrore scherzoso. ---
Abbordatore di pellegrini! Molestatore di novizi! Ti manderò il
Poeta\ldots{} e possa egli scendere su di te e rimanervi per sempre.
Condurre il Thon davanti alla tua tana! Che oltraggio.

Benjamin tornò ad alzare le spalle. --- Benissimo. Dimentica la mia
richiesta. Ma speriamo che il Thon sia dalla nostra parte, e non con gli
altri, questa volta.

--- Gli \emph{altri}, Benjamin?

--- Manasse, Ciro, Nabucodonosor, il Faraone, Cesare, Hannegan
secondo\ldots{} è necessario che continui? Samuele ci mise in guardia
contro di loro, poi ce ne diede uno. Quando hanno qualche uomo saggio al
fianco che li consiglia, diventano più pericolosi che mai. Questo è il
solo consiglio che ti darò.

--- Bene, Benjamin, ne ho avuto abbastanza di te, adesso, per altri
cinque anni, quindi\ldots{}

--- Insultami, imprigionami, tormentami\ldots{}

--- Finiscila. Me ne vado, vecchio. È tardi.

--- Davvero? E il tuo ecclesiastico ventre è in grado di affrontare la
cavalcata?

--- Il mio stomaco\ldots? --- Don Paulo si fermò per tastarlo, e si
accorse di stare meglio di quanto non fosse mai stato in quelle ultime
settimane. --- E un disastro, naturalmente --- si lagnò. --- Cosa altro
dovrebbe essere, dopo che ho dovuto ascoltare te?

--- È vero\ldots{} \emph{El Shaddai} è misericordioso, ma è anche
giusto. --- Dio ti protegga, vecchio. Quando frate Kornhoer avrà
reinventato la macchina per volare, manderò su qualche novizio a
scagliarti addosso le pietre.

Si abbracciarono con affetto. Il Vecchio Ebreo lo guidò fino
all\textquotesingle orlo della mesa. Benjamin rimase ritto, avvolto in
uno scialle da preghiera, il cui tessuto finissimo contrastava
bizzarramente con la rozza tela da sacco che gli cingeva i lombi, mentre
l\textquotesingle abate scendeva il sentiero e si dirigeva di nuovo
verso l\textquotesingle abbazia. Don Paulo lo poteva ancora vedere,
ritto nel tramonto, la figura emaciata profilata contro il cielo in
penombra, mentre si inchinava e masticava una preghiera sopra il
deserto.

--- \emph{Memento, Domine, omnium farnulorum tuorurm} \emph{---}
sussurrò in risposta l\textquotesingle abate, poi aggiunse: --- E possa
egli rivincere l\textquotesingle occhio di vetro del Poeta a morra.
\emph{Amen}.
