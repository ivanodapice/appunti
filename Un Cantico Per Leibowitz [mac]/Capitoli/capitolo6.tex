	\chapter{\phantom{title}}

\lettrine{I}{l} pellegrino rimase un argomento di conversazione proibito,
nell\textquotesingle abbazia, ma rispetto alle reliquie e al rifugio la
proibizione fu, per necessità, gradualmente allentata\ldots{} tranne che
per il loro scopritore, il quale aveva tuttora l\textquotesingle ordine
di non discuterne, e preferibilmente di pensarvi il meno possibile.
Eppure, non poteva evitarsi di udire qualche voce, ogni tanto, e sapeva
che in uno dei laboratori dell\textquotesingle abbazia i monaci erano al
lavoro sui documenti, non soltanto sui suoi ma anche su altri che erano
stati scoperti nell\textquotesingle antica scrivania, prima che
l\textquotesingle abate ordinasse di richiudere il rifugio.

\emph{Richiuderlo!} Quella notizia sconvolse frate Francis. Il rifugio
era stato appena toccato. A parte la sua avventura, non vi erano stati
tentativi di addentrarsi ulteriormente nei segreti del rifugio, a
eccezione dell\textquotesingle apertura della scrivania che lui aveva
tentato di aprire, senza successo, prima di notare la cassetta.
\emph{Richiuderlo!} Senza tentare di scoprire ciò che poteva esservi
oltre la porta interna contrassegnata dalla scritta ``Portello Due'',
senza aver investigato l``\,`Ambiente Sigillato''. Senza neppure
rimuovere le pietre o le ossa. \emph{Richiuderlo!}
L\textquotesingle investigazione era stata interrotta bruscamente, senza
un motivo.

Poi cominciò a spargersi una voce.

\emph{``Emily aveva un dente d\textquotesingle oro, Emily aveva un dente
	d\textquotesingle oro, Emily aveva un dente d\textquotesingle oro.''}

Era verissimo, in realtà. Era una di quelle sciocchezze storiche che in
qualche modo riescono a sopravvivere a fatti importanti che qualcuno
avrebbe dovuto prendersi il disturbo di ricordare, ma che non venivano
documentate, fino a che qualche storico d\textquotesingle un monastero
era costretto a scrivere: ``Né i contenuti dei Memorabilia né alcuna
fonte archeologica finora scoperta ci hanno rivelato il nome del
dominatore che occupava il Palazzo Bianco durante la seconda metà degli
anni Sessanta, benché frate Barcus abbia sostenuto, non senza alcune
prove a sostegno, che il suo nome era\ldots''.

Eppure, era chiaramente documentato nei Memorabilia che Emily aveva
avuto un dente d\textquotesingle oro.

Non era sorprendente che il Signor Abate avesse ordinato di sigillare la
cripta. Ricordando come aveva sollevato l\textquotesingle antico teschio
e come l\textquotesingle aveva voltato verso la parete, frate Francis
cominciò improvvisamente a temere l\textquotesingle ira celeste. Emily
Leibowitz era scomparsa dalla faccia della Terra
all\textquotesingle inizio del Diluvio di Fiamma, e soltanto dopo molti
anni il suo vedovo aveva ammesso che era morta.

Si diceva che Iddio, per mettere alla prova l\textquotesingle umanità
gonfia di orgoglio come ai tempi di Noè, aveva ordinato agli uomini
saggi di quell\textquotesingle epoca, fra i quali era il beato
Leibowitz, di inventare grandi macchine belliche, quali non erano mai
state viste sulla Terra, armi tanto potenti che avrebbero potuto
contenere lo stesso fuoco dell\textquotesingle Inferno, e che Dio aveva
permesso a quei maghi di porre le armi nelle mani dei principi e di dire
a ogni principe: ``Soltanto perché i nemici sono in possesso di una
simile arma, noi abbiamo costruito questa per te, affinché essi sappiano
che tu pure la possiedi, e abbiano così timore di colpire. Fai in modo,
o mio Signore, di temerli quanto essi ora ti temono, perché nessuno
possa scatenare questo flagello che noi abbiamo forgiato''.

Ma i principi, negligendo le parole dei loro saggi, avevano pensato:
``Se io colpisco abbastanza in fretta, e in segreto, distruggerò gli
altri nel sonno, e non rimarrà nessuno per combattere. La Terra sarà
mia''.

Tale fu la follia dei principi, e ne conseguì il Diluvio di Fiamma.

In poche settimane --- qualcuno diceva in pochi giorni --- tutto era
finito, dopo il primo scatenamento del fuoco infernale. Le città erano
diventate mucchi di vetro, circondati da vaste distese di pietre
spezzate. Mentre le nazioni erano scomparse dalla Terra, il suolo era
cosparso di cadaveri d\textquotesingle uomini e di carogne di animali
domestici e di bestie d\textquotesingle ogni genere, insieme agli
uccelli dell\textquotesingle aria e a tutti gli esseri che volavano, che
nuotavano nei fiumi e strisciavano fra l\textquotesingle erba o si
annidavano nelle buche; essendosi ammalati ed essendo periti, essi
coprirono la Terra, eppure dove i demoni del Fallout coprivano la
campagna, i corpi non si corrompevano, per molto tempo, se non erano a
contatto con il suolo fertile. Le grandi nuvole della collera divina
sommersero le foreste e i campi, facendo avvizzire gli alberi e morire i
raccolti. E vi furono grandi deserti là dove un tempo
c\textquotesingle era la vita, e in quei luoghi della Terra in cui
vivevano ancora gli uomini, essi si ammalavano per colpa
dell\textquotesingle aria avvelenata, così che, mentre alcuni sfuggivano
alla morte, nessuno rimaneva intatto; e molti morirono anche in quelle
terre che le armi non avevano colpito, a causa dell\textquotesingle aria
avvelenata.

In tutte le parti del mondo, gli uomini fuggivano da un luogo
all\textquotesingle altro, e vi fu una confusione di lingue. Molta ira
si levò contro i principi e i servitori dei principi e contro i maghi
che avevano costruito le armi. Passarono gli anni, eppure la Terra non
si era purificata. Così era chiaramente documentato nei Memorabilia.

Dalla confusione delle lingue, dal mescolarsi dei resti di molte
nazioni, dalla paura, nacque l\textquotesingle odio. E
l\textquotesingle odio disse: \emph{Lapidiamo e sventriamo e bruciamo
	coloro che fecero questo. Facciamo olocausto di coloro che compirono
	questo crimine, insieme ai loro mercenari e ai loro saggi; e bruciando
	essi periscano, e con loro le loro opere, i loro nomi e persino il loro
	ricordo. Distruggiamoli tutti, e insegniamo ai nostri figli che il mondo
	è nuovo, che essi possono ignorare i fatti che avvennero prima. Facciamo
	una grande semplificazione, e allora il mondo ricomincerà.}

E così, dopo il Diluvio, il Fallout, le pestilenze, la follia, la
confusione delle lingue, il furore, cominciò la sanguinaria
Semplificazione, quando superstiti dell\textquotesingle umanità avevano
fatto a pezzi altri superstiti, uccidendo regnanti, scienziati,
condottieri, tecnici, insegnanti e ogni persona che i capi della folla
inferocita indicavano come meritevole di morire per aver contribuito a
fare della Terra ciò che era. Nulla era stato tanto odioso al cospetto
di quelle folle quanto gli uomini sapienti, dapprima perché avevano
servito i principi, ma più tardi perché essi rifiutavano di unirsi ai
massacri e tentavano di opporsi alle folle, che chiamavano ``semplicioni
assetati di sangue''.

Le folle accettarono gioiosamente quel nome e si levò il grido:
\emph{Semplicioni! Si, sì! Io sono un semplicione! Sei un semplicione,
	tu? Costruiremo una città e la chiameremo Simple Town, perché allora
	tutti i furbi bastardi che hanno provocato tutto questo saranno morti!
	Semplicioni! Andiamo! Questo gli insegnerà. Qui c\textquotesingle è
	qualcuno che non è un semplicione? Prendete il bastardo, se
	c\textquotesingle è!}

Per sfuggire al furore delle schiere dei semplicioni, i dotti superstiti
correvano a ogni rifugio che si offrisse loro. Quando la Santa Chiesa li
accolse, li vestì di abiti monacali e cercò di nasconderli nei monasteri
e nei conventi che erano rimasti in piedi e che erano stati rioccupati,
perché i religiosi erano meno disprezzati dalla folla, tranne quando la
sfidavano apertamente e accettavano il martirio. Qualche volta questo
rifugio era efficace, ma più spesso non lo era. I monasteri venivano
invasi, i documenti e i libri sacri venivano bruciati, i rifugiati
venivano catturati e impiccati o arsi, sommariamente. La Semplificazione
aveva cessato di avere un piano o uno scopo poco tempo dopo il suo
inizio, ed era diventata una insana frenesia di sterminio di massa e di
distruzione, quale può verificarsi soltanto quando sono scomparse anche
le ultime tracce dell\textquotesingle ordine sociale. La follia fu
trasmessa ai figli, ai quali veniva insegnato l\textquotesingle odio, e
manifestazioni di furore popolare si ripeterono sporadicamente anche
durante la quarta generazione dopo il Diluvio. A quei tempi, il furore
era rivolto non contro i dotti, perché non ve ne erano più, ma contro
coloro che sapevano semplicemente leggere e scrivere.

Isaac Edward Leibowitz, dopo un\textquotesingle inutile ricerca della
moglie, si era rifugiato presso i Cistercensi, e lì rimase nascosto nei
primi anni che seguirono il Diluvio. Dopo sei anni, era andato ancora
una volta alla ricerca di Emily o della sua tomba, nel lontano Sud-est.
Là si era finalmente convinto che la donna era morta, poiché in quel
luogo la morte trionfava incondizionatamente. Lì, nel deserto, fece
silenziosamente un voto. Poi ritornò ai Cistercensi, prese il loro
abito, e qualche anno dopo diventò prete. Raccolse attorno a sé alcuni
compagni e fece loro alcune quiete proposte. Passò ancora qualche anno,
e le proposte giunsero fino a ``Roma'' che non era più Roma (che, a sua
volta, non era più una città) e che continuava a spostarsi, e a
spostarsi ancora e ancora\ldots{} in meno di due decenni, dopo essere
rimasta per due millenni in un solo luogo. Dodici anni dopo la
formulazione delle proposte, padre Isaac Edward Leibowitz aveva ottenuto
dalla Santa Sede il permesso di fondare una nuova comunità di religiosi,
che prese il nome da Alberto Magno, maestro di san Tommaso, e patrono
degli uomini di scienza. La missione dell\textquotesingle Ordine, non
annunciata e dapprima soltanto vagamente definita, era quella di
preservare la storia umana per i pro-pro-pronipoti dei figli dei
semplicioni che la volevano distruggere. Il primo abito
dell\textquotesingle Ordine fu costituito da brandelli di tela di sacco,
l\textquotesingle uniforme della folla dei semplicioni. I suoi membri
erano ``contrabbandieri di libri'' o ``memorizzatori'', secondo il
compito loro affidato. I contrabbandieri di libri portavano i libri nel
deserto sudoccidentale e li seppellivano entro i barili. I memorizzatori
imparavano a memoria interi volumi di storia, delle Sacre Scritture,
della letteratura e della scienza, nel caso che qualche sfortunato
contrabbandiere di libri fosse catturato, torturato, e costretto a
rivelare il luogo in cui erano nascosti i barili. Nel frattempo, altri
membri del nuovo Ordine scoprirono un pozzo a circa tre giorni di
viaggio dal nascondiglio dei libri e cominciarono a costruirvi un
monastero. Il progetto, che mirava a salvare un piccolo residuo della
cultura umana dal resto dell\textquotesingle umanità che lo voleva
distrutto, era così iniziato.

Leibowitz, mentre stava compiendo il suo turno come contrabbandiere di
libri, fu catturato da una folla di semplicioni; un tecnico rinnegato,
che il prete si affrettò a perdonare, lo identificò non soltanto come un
uomo dotto, ma come uno specialista nel campo delle armi. Incappucciato
di tela di sacco, fu immediatamente martirizzato per strangolamento con
un cappio da carnefice annodato in modo da non spezzare il collo, e
nello stesso tempo veniva arrostito vivo\ldots{} sistemando così una
disputa sorta tra la folla sul metodo dell\textquotesingle esecuzione.

I memorizzatori erano pochi, la loro memoria limitata.

Alcuni dei barili di libri furono trovati e bruciati, e così pure molti
altri contrabbandieri di libri. Lo stesso monastero fu assalito tre
volte, prima che la follia si placasse.

Dal vasto mare della conoscenza umana, soltanto pochi barili di libri
originali e una pietosa raccolta di testi copiati a mano, trascritti a
memoria, erano rimasti in possesso dell\textquotesingle Ordine, quando
la follia era finita.

Ora, dopo sei secoli di oscurantismo, i monaci conservavano ancora
questi Memorabilia, li studiavano, li copiavano e li ricopiavano, e
attendevano pazientemente. In principio, ai tempi di Leibowitz, si era
sperato --- e si era persino ritenuto probabile --- che la quarta o la
quinta generazione avrebbe cominciato a desiderare di riavere la propria
eredità. Ma i monaci dei primi tempi non avevano pensato alla capacità
umana di ricreare un nuovo patrimonio culturale, in un paio di
generazioni, se un patrimonio antico è completamente distrutto,
ricreandolo in virtù di legislatori e di profeti, di geni odi maniaci;
per merito di un Mosè o per merito di un Hitler, o di un avo ignorante
ma tirannico, un patrimonio culturale poteva essere acquisito tra il
crepuscolo e l\textquotesingle aurora, e molti sono stati acquisiti in
tal modo. Ma la nuova ``cultura'' era un\textquotesingle eredità delle
tenebre, dei tempi in cui ``semplicione'' aveva lo stesso significato di
``cittadino'' e di ``schiavo''. I monaci attendevano. A loro nulla
importava che la conoscenza da loro salvata fosse inutile, che gran
parte di essa non fosse più, oramai, vera conoscenza, e fosse ormai
imperscrutabile per i monaci, in certi casi, quanto lo sarebbe stata per
un selvaggio analfabeta delle colline; quella conoscenza era priva di
contenuto, le discipline di cui trattava erano scomparse da lungo tempo.
Eppure, tale conoscenza aveva una struttura simbolica caratteristica, e
per lo meno era possibile osservare il gioco reciproco dei simboli.
Osservare il modo in cui un sistema di conoscenza è costruito significa
imparare un minimo di conoscenza della conoscenza; fino a che un giorno,
forse fra qualche secolo, sarebbe venuto un Integratore, e tutto sarebbe
tornato di nuovo a posto. Così, il tempo non aveva importanza. I
Memorabilia erano là, ed era loro dovere preservarli, e li avrebbero
preservati, anche se le tenebre sul mondo fossero durate altri dieci
secoli, o anche dieci millenni, perché i monaci, sebbene nati nella più
buia delle età, erano ancora gli stessi contrabbandieri di libri e gli
stessi memorizzatori del beato Leibowitz; e quando vagavano lontano
dalla loro abbazia, ciascuno dei professi dell\textquotesingle Ordine,
dallo stalliere all\textquotesingle abate, portava, come parte del loro
abito, un libro, di solito un Breviario in quei tempi, legato in una
specie di bisaccia.

Dopo che il rifugio fu richiuso, i documenti e le reliquie che ne erano
stati asportati furono studiati, uno alla volta e in modo molto
discreto, dall\textquotesingle abate. Non fu più possibile esaminarli,
chiusi com\textquotesingle erano, probabilmente, nello studio di Arkos.
Era come se fossero scomparsi, ai fini pratici. Tutto ciò che scompariva
nello studio dell\textquotesingle abate era un soggetto pericoloso per
una pubblica conversazione: diventava qualcosa di cui si poteva
sussurrare soltanto in silenziosi corridoi. Frate Francis udiva solo di
rado quei bisbigli. Alla fine si quietarono, per rivivere quando un
messaggero venuto da Nuova Roma parlò sottovoce
all\textquotesingle abate, una sera nel refettorio. Un frammento della
loro conversazione sussurrata giunse fino alle tavole vicine. I mormorii
durarono alcune settimane, dopo la partenza del messaggero, poi
tornarono a quietarsi.

Frate Francis Gerard dello Utah ritornò nel deserto,
l\textquotesingle anno seguente, e di nuovo digiunò in solitudine per
tutta la Quaresima. Ancora una volta ritornò, debole ed emaciato, e ben
presto fu chiamato alla presenza dell\textquotesingle abate Arkos, il
quale volle sapere se pretendeva di avere avuto altri colloqui con
membri delle Schiere Celesti.

--- Oh, no, Monsignor Abate. Di giorno ho visto soltanto lucertole.

--- E di notte? chiese sospettoso Arkos.

--- Soltanto lupi --- disse Francis, e aggiunse cautamente: --- Penso.

Arkos preferì non approfondire quel prudente emendamento, ma si limitò
ad accigliarsi. Il cipiglio dell\textquotesingle abate, aveva osservato
frate Francis, era la sorgente d\textquotesingle una energia radiante
che viaggiava nello spazio a velocità finita e che non era ancora bene
compresa, a eccezione del fatto che faceva avvizzire tutto ciò su cui si
posava, poiché tale oggetto era di solito un postulante o un novizio.
Francis aveva già assorbito una scarica di cinque secondi, quando gli fu
rivolta la domanda seguente.

--- E a proposito dello scorso anno?

Il novizio fece una pausa per inghiottire saliva. --- Il\ldots{}
vecchio?

--- Il vecchio.

--- Sì, don Arkos.

Cercando di allontanare dal suo tono ogni sfumatura di punto.
interrogativo, Arkos disse: --- Era davvero un vecchio.
Nient\textquotesingle altro. Adesso ne siamo sicuri.

--- Anch\textquotesingle io penso che fosse soltanto un vecchio.

Padre Arkos tese fiaccamente la mano per impugnare il righello di
quercia.

\emph{WHACK!}

\emph{--- Deo gratias!}

\emph{WHACK!}

\emph{--- Deo\ldots..}

Mentre Francis ritornava alla sua cella, l\textquotesingle abate gli
gridò dietro, nel corridoio: --- Fra l\textquotesingle altro, volevo
dire\ldots{} --- Sì, Reverendo Padre?

--- Niente voti, quest\textquotesingle anno --- disse quello
distrattamente, e scomparve nel suo studio.
