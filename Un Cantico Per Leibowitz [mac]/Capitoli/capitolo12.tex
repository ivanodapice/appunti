	{\chapterstyle{dowding} \chapter*{PARTE SECONDA\\\leavevmode\\\footnotesize{Fiat Lux}}}
		\chapter{\phantom{text}}
	
	\lettrine{M}{arcus} Apollo fu certo dell\textquotesingle imminenza della guerra nel
	momento in cui udì la terza moglie di Hannegan dire a una fantesca che
	il suo cortigiano favorito era ritornato tutto intero da una missione
	all\textquotesingle accampamento del clan di Orso Pazzo. Il fatto che
	fosse tornato vivo dall\textquotesingle accampamento dei nomadi indicava
	che si stava preparando una guerra. Infatti, la missione
	dell\textquotesingle inviato era stata di dire alle tribù delle Pianure
	che gli Stati civili avevano aderito al Patto della Sacra Sferza
	riguardante le terre disputate, e di conseguenza avrebbero compiuto dure
	rappresaglie sulle popolazioni nomadi e sulle bande di predoni per ogni
	futura incursione. Ma nessun uomo aveva mai portato notizie del genere a
	Orso Pazzo per poi ritornare vivo. Di conseguenza, concluse Apollo,
	l\textquotesingle ultimatum non era stato consegnato, e
	l\textquotesingle emissario di Hannegan si era addentrato nelle Pianure
	con un altro scopo. E quello scopo era anche troppo chiaro.
	
	Apollo si fece largo educatamente fra la piccola folla degli ospiti,
	cercando con gli occhi attenti frate Claret, e tentando di attirarne lo
	sguardo. L\textquotesingle alta figura di Apollo, nella severa tunica
	nera, con un piccolo lampo di colore alla cintura per denotare il suo
	rango, spiccava nettamente, in contrasto con il vortice caleidoscopico
	di colori nella sala del banchetto; non impiegò molto tempo ad attirare
	lo sguardo dell\textquotesingle ecclesiastico e ad accennargli di
	dirigersi verso la tavola dei rinfreschi, che ormai era ridotta a una
	distesa di briciole, di tazze unte e di pochi pezzi
	d\textquotesingle arrosto troppo cotto.
	
	Apollo pescò nella grande coppa del punch con il mestolo, osservò uno
	scarafaggio morto che galleggiava fra le spezie, e offrì pensieroso la
	prima tazza a frate Claret, quando l\textquotesingle ecclesiastico si
	avvicinò.
	
	--- Grazie, monsignore --- disse Claret, senza notare lo scarafaggio.
	--- Volevate parlarmi?
	
	--- Non appena sarà finito il ricevimento. Nel mio alloggio. Sarkal è
	ritornato vivo.
	
	--- Oh!
	
	--- Non ho mai sentito un ``oh'' più mal augurante. Ne deduco che avete
	compreso le gravi implicazioni di questo fatto.
	
	--- Certamente, monsignore. Significa che il Patto è stato una frode, da
	parte di Hannegan, e che egli intende usarlo contro\ldots{}
	
	--- Shh! Più tardi. --- Gli occhi di Apollo segnalarono
	l\textquotesingle avvicinarsi di estranei, e il frate si voltò per
	riempire di nuovo la tazza. Il suo interesse si appuntò improvvisamente
	sulla grande tazza piena di rum; così, non guardò la snella figura
	vestita di seta marezzata che, dall\textquotesingle ingresso, si
	dirigeva verso di loro a grandi passi. Apollo sorrise in modo formale e
	si inchinò all\textquotesingle uomo. La loro stretta di mano fu breve e
	notevolmente fredda.
	
	--- Bene, Thon Taddeo --- disse il prete --- la vostra presenza mi
	sorprende, Credevo che rifuggiste questi convegni festaioli. Che cosa
	può esservi di speciale in questo, per attrarre un celebre studioso come
	voi? --- E sollevò le sopracciglia, in atto di ironica perplessità.
	
	--- Naturalmente, l\textquotesingle attrazione siete voi --- disse il
	nuovo venuto, rispondendo con il sarcasmo al sarcasmo di Apollo. --- E
	siete anche l\textquotesingle unica ragione della mia presenza.
	
	--- Io? --- Apollo finse di essere sorpreso: ma
	l\textquotesingle affermazione era probabilmente vera. Il ricevimento
	nuziale di una sorella consanguinea. non era una ragione sufficiente per
	costringere il Thon Taddeo a perdersi in raffinatezze formali e ad
	abbandonare le sale claustrali del collegio.
	
	--- Per la verità, vi ho cercato tutto il giorno. Mi hanno detto che
	sareste stato qui. Altrimenti\ldots{} --- Guardò la sala del banchetto e
	sbuffò, irritato.
	
	Quello sbuffo irritato spezzò il legame di fascino che univa lo sguardo
	di frate Claret alla tazza del punch; il religioso si voltò per
	inchinarsi al Thon.
	
	--- Volete un po\textquotesingle{} di punch, Thon Taddeo? --- chiese,
	offrendogliene una tazza colma.
	
	Lo studioso l\textquotesingle accettò con un cenno e la vuotò. ---
	Volevo farvi altre domande sui documenti leibowitziani di cui abbiamo
	discusso --- disse a Marcus Apollo. --- Ho ricevuto una lettera di un
	certo Kornhoer, dell\textquotesingle abbazia. Mi assicura che là
	conservano scritti che risalgono agli ultimi anni della civiltà
	europea-americana.
	
	Anche se il fatto di essere stato lui stesso a fornire allo studioso
	quelle assicurazioni parecchi mesi prima irritò Apollo, la sua
	espressione non lo mostrò. --- Si --- disse. --- Sono assolutamente
	autentici, mi hanno detto.
	
	--- Se è così, mi sembra molto misterioso che nessuno abbia
	sentito\ldots{} ma lasciamo perdere. Kornhoer ha elencato alcuni
	documenti e testi che sarebbero conservati nell\textquotesingle abbazia,
	e li ha descritti. Se esistono, io devo andare a vederli.
	
	--- Oh?
	
	--- Sì. Se è un\textquotesingle impostura, sarebbe bene scoprirlo, e se
	non lo è, i dati potrebbero avere un valore inestimabile.
	
	Il monsignore corrugò la fronte. --- Vi assicuro che non è
	un\textquotesingle impostura --- disse, impettito.
	
	--- La lettera contiene un invito a visitare l\textquotesingle abbazia e
	a studiare i documenti. È evidente che hanno sentito parlare di me.
	
	--- Non necessariamente --- disse Apollo, incapace di resistere a
	quell\textquotesingle occasione. --- Non badano molto a chi legge i loro
	libri, purché l\textquotesingle individuo in questione si lavi le mani e
	non deturpi le loro proprietà.
	
	Lo studioso si accigliò. L\textquotesingle allusione alla possibile
	esistenza di persone letterate che non avevano mai udito il suo nome non
	gli faceva piacere.
	
	--- Benissimo, in ogni caso! --- continuò affabilmente Apollo. --- Non
	avete alcun problema. Accettate l\textquotesingle invito, andate
	all\textquotesingle abbazia, studiate le loro reliquie. Sarete
	indubbiamente il benvenuto.
	
	Lo studioso sembrò irritato di quel suggerimento. --- E dovrei viaggiare
	attraverso le Pianure, proprio quando il clan di Orso Pazzo sta\ldots{}
	--- Il Thon Taddeo si interruppe bruscamente.
	
	--- Stavate dicendo? --- fece Apollo; il suo viso non dimostrava alcuna
	speciale attenzione, anche se sulla tempia una vena cominciava a
	pulsare, mentre il suo sguardo si fissava, in attesa, sul Thon Taddeo.
	
	--- Soltanto, è un viaggio lungo e pericoloso, e io non posso concedermi
	sei mesi di assenza dal collegio. Volevo discutere la possibilità di
	mandare un drappello di guardie del podestà, bene armate, per portare
	qui i documenti da studiare.
	
	Apollo si sentì soffocare. Provò l\textquotesingle impulso puerile di
	prendere lo studioso a calci negli stinchi. --- Ho paura --- disse,
	educatamente --- che sia impossibile. Ma, in ogni caso, la questione
	esorbita dalla mia competenza, e temo di non potervi essere minimamente
	di aiuto.
	
	--- Perché no? --- domandò il Thon Taddeo. --- Non siete il Nunzio del
	Vaticano alla corte di Hannegan?
	
	--- Precisamente. Io rappresento Nuova Roma, non gli ordini monastici.
	Il governo di una abbazia è completamente nelle mani del suo abate.
	
	--- Ma, con qualche lieve pressione da parte di Nuova Roma\ldots{}
	
	L\textquotesingle impulso di prenderlo a calci negli stinchi si fece
	sentire di nuovo. --- Faremo meglio a discuterne più tardi --- disse
	seccamente monsignor Apollo. --- Questa sera nel mio studio, se vi
	piace\ldots{} --- Si girò a mezzo, e guardò dietro di sé, con aria
	interrogativa, come per chiedere ``Ebbene?''.
	
	--- Ci sarò --- disse con voce tagliente lo studioso, e si allontanò.
	
	--- Perché non gli avete detto un \emph{no} chiaro e tondo? --- sbuffò
	Claret quando furono soli, nell\textquotesingle appartamento riservato
	all\textquotesingle ambasciata, un\textquotesingle ora più tardi.
	Trasportare reliquie inestimabili attraverso un paese di banditi, in
	questi tempi? È inimmaginabile, monsignore.
	
	--- Certamente.
	
	--- E allora perché\ldots{}
	
	--- Per due ragioni. Prima, il Thon Taddeo è parente di Hannegan, e ha
	molta influenza. Dobbiamo essere cortesi con Cesare e con la sua
	schiatta, anche se non ci va a genio. Seconda, aveva cominciato a dire
	qualcosa sul clan di Orso Pazzo, e poi si è interrotto. Credo che sappia
	ciò che sta per accadere. Non ho intenzione di dedicarmi allo
	spionaggio, ma se egli ci offre spontaneamente qualche informazione,
	nulla ci impedisce di includerla nel rapporto che voi consegnerete
	personalmente a Nuova Roma.
	
	--- Io! --- Il religioso si mostrò sconvolto. --- A Nuova Roma\ldots? Ma
	che cosa\ldots{}
	
	--- Abbassate la voce --- disse il Nunzio, guardando la porta. --- Dovrò
	mandare la mia analisi della situazione a Sua Santità, e presto. Ma si
	tratta di quel genere di cose che non si osa affidare alla carta. Se la
	gente di Hannegan intercettasse un simile dispaccio, voi. e io finiremmo
	probabilmente annegati nel Fiume Rosso. Se invece se ne impadroniscono i
	nemici di Hannegan, Hannegan si sentirebbe probabilmente in diritto di
	impiccarci in pubblico come spie. Il martirio è una cosa bellissima, ma
	prima abbiamo un compito da svolgere.
	
	--- E io dovrò riferire oralmente il rapporto al Vaticano? --- mormorò
	frate Claret, che evidentemente non si rallegrava alla prospettiva di
	attraversare un territorio ostile.
	
	--- È necessario. Può darsi, dico \emph{può darsi}, che il Thon Taddeo
	ci offra un pretesto per il vostro viaggio improvviso
	all\textquotesingle abbazia di san Leibowitz o a Nuova Roma. Nel caso
	che qui, a corte, qualcuno abbia sospetti, io cercherò di stornarli.
	
	--- E quale sarà la sostanza del rapporto, monsignore?
	
	--- Che l\textquotesingle ambizione di Hannegan, unire il continente
	sotto un\textquotesingle unica dinastia, non è un sogno pazzesco come
	noi credevamo. Che il Patto della Sacra Sferza è stato probabilmente un
	inganno, da parte di Hannegan, e che egli intende servirsene per
	spingere tanto l\textquotesingle impero di Denver quanto la nazione
	Laredana in conflitto con i nomadi delle Pianure. Se le forze laredane
	fossero impegnate in combattimento con Orso Pazzo, non occorrerebbero
	molti incoraggiamenti allo Stato di Chihuahua per attaccare Laredo da
	sud. Dopotutto, c\textquotesingle è una vecchia inimicizia, tra loro.
	Hannegan, naturalmente, potrebbe poi marciare vittorioso a Rio Laredo.
	Una volta padrone di Laredo, può pensare di impadronirsi tanto di Denver
	quanto della Repubblica del Mississippi senza doversi preoccupare di una
	eventuale pugnalata alle spalle, da sud.
	
	--- Credete che Hannegan possa riuscirci, monsignore?
	
	Marcus Apollo fece per rispondere, poi richiuse lentamente la bocca. Si
	avvicinò alla finestra e guardò la città illuminata dal sole, una città
	disordinata e ampia, costruita principalmente con le macerie di
	un\textquotesingle altra epoca. Una città senza un piano regolatore
	ordinato. Era cresciuta lentamente su antiche rovine, così come un
	giorno, forse, un\textquotesingle altra città sarebbe cresciuta sulle
	rovine della città attuale.
	
	--- Non so --- rispose, sommessamente. --- In questi tempi, è difficile
	condannare un uomo se desidera unificare questo continente macellato.
	Anche se si serve di mezzi\ldots{} ma no, non è questo che intendevo.
	--- Sospirò, pesantemente. -\/--- In ogni caso, i nostri interessi non
	sono gli interessi della politica. Dobbiamo avvertire Nuova Roma di ciò
	che sta per accadere, perché la Chiesa ne sarà colpita, qualunque cosa
	avvenga. E, una volta avvertiti, forse potremo tenerci fuori dal caos.
	
	--- Lo credete davvero?
	
	--- Naturalmente no! --- disse gentilmente il prete.
	
	Il Thon Taddeo Pfardentrott arrivò allo studio di Marcus Apollo verso
	sera, e i suoi modi erano molto cambiati, rispetto al ricevimento.
	Riusciva a esibire un sorriso cordiale, e il modo in cui parlava tradiva
	una nervosa impazienza.
	
	Quell\textquotesingle uomo, pensò. Marcus, sta cercando di ottenere
	qualcosa che desidera molto, ed è persino disposto a mostrarsi gentile
	per ottenerla. Forse l\textquotesingle elenco degli antichi scritti
	fornito dai monaci dell\textquotesingle abbazia leibowitziana aveva
	impressionato il Thon più di quanto egli fosse disposto ad ammettere. Il
	Nunzio si era preparato a una disputa accanita, ma
	l\textquotesingle evidente eccitazione dello studioso faceva di lui una
	vittima troppo facile, e Apollo abbassò la guardia del duello verbale.
	
	--- Questo pomeriggio c\textquotesingle è stata una riunione della
	facoltà del collegio --- disse il Thon Taddeo, non appena furono seduti.
	Abbiamo parlato della lettera di frate Kornhoer, e
	dell\textquotesingle elenco dei documenti. --- Si interruppe, come se
	fosse incerto sull\textquotesingle approccio da scegliere. La grigia
	luce del crepuscolo che scendeva dalla grande finestra ad arco alla sua
	sinistra dava al suo viso un aspetto intenso, e i suoi grandi occhi
	grigi studiavano il prete come se lo misurassero e lo valutassero.
	
	--- Ne deduco che qualcuno si è mostrato scettico?
	
	Gli occhi grigi si abbassarono per un attimo, poi si risollevarono,
	prontamente. --- È necessario che io sia educato? --- Non disturbatevi
	--- ridacchiò Apollo.
	
	--- Si sono mostrati scettici. Forse ``increduli'' è la parola più
	adatta. Io stesso ritengo che, se tali documenti esistono, si tratta
	probabilmente di falsificazioni che risalgono a parecchi secoli
	addietro. Dubito che i monaci dell\textquotesingle abbazia, oggi, stiano
	cercando deliberatamente di perpetrare un\textquotesingle impostura.
	Naturalmente, essi crederanno che i documenti siano validi.
	
	--- È molto gentile da parte vostra assolverli così --- disse acido
	Apollo.
	
	--- Mi sono offerto di essere educato! È necessario che lo sia?
	
	--- No. Continuate.
	
	Il Thon si alzò dalla sedia e andò a sedersi alla finestra. Guardò le
	strisce di nuvole gialle che sbiadivano a occidente e batté piano una
	mano sul davanzale, mentre parlava. --- I documenti. Non importa che
	cosa ne pensiamo, la sola idea che tali documenti possano ancora
	esistere, intatti\ldots{} l\textquotesingle idea che vi sia anche la
	minima possibilità che essi esistano. bene, è un pensiero così eccitante
	che dobbiamo indagare, immediatamente.
	
	--- Benissimo --- disse Apollo, un po\textquotesingle{} divertito. --- I
	monaci vi hanno invitato. Ma ditemi: cosa trovate di così eccitante in
	quei documenti?
	
	Lo studioso gli lanciò una rapida occhiata. --- Conoscete il mio lavoro?
	
	Il monsignore esitò. Lo conosceva, ma ammettere questo
	l\textquotesingle avrebbe costretto anche ad ammettere che il nome del
	Thon Taddeo veniva pronunciato insieme a quelli dei filosofi naturali
	morti da mille anni e più, mentre il Thon aveva poco più di
	trent\textquotesingle anni. Il prete non ci teneva a riconoscere che
	quel giovane scienziato prometteva di diventare una di quelle rare
	eccezioni del genio umano che appaiono soltanto una volta ogni uno o due
	secoli per rivoluzionare un intero campo del pensiero. Tossì, quasi in
	segno di scusa.
	
	--- Devo confessare che non ho letto molto di\ldots{}
	
	--- Non importa. --- Pfardentrott accantonò la scusa con un gesto. --- È
	soprattutto un lavoro astratto e noioso per un profano. Teorie
	dell\textquotesingle essenza elettrica. Moto planetario. Corpi che si
	attraggono. Cose del genere. Ora, l\textquotesingle elenco di Kornhoer
	cita nomi come Laplace, Maxwell ed Einstein\ldots{} significano
	qualcosa, per voi?
	
	--- Non molto. La storia li menziona come filosofi naturali, non è così?
	Vissero nel periodo precedente al crollo dell\textquotesingle ultima
	civiltà. E credo che siano nominati in una delle agiologie pagane, non è
	vero?
	
	Lo studioso annuì. --- E questo è ciò che si sa di loro o delle loro
	opere. Erano fisici, secondo i nostri storici, non del tutto
	attendibili. Furono responsabili della rapida ascesa della civiltà
	europea-americana, dicono. Gli storici riferiscono solo particolari
	insignificanti. Li avevo quasi dimenticati. Ma le descrizioni fatte da
	Kornhoer dei documenti antichi che i monaci affermano di custodire sono
	descrizioni di carte che potrebbero essere state tolte da testi di
	scienze fisiche. È impossibile!
	
	--- Ma voi volete accertarlo!
	
	--- Noi dobbiamo accertarlo. Ora che il problema si presenta, vorrei non
	averne mai sentito parlare.
	
	--- Perché?
	
	Il Thon Taddeo stava guardando qualcosa, nella strada sottostante. Fece
	un cenno di richiamo al prete. --- Venite qui un momento. Vi mostrerò
	perché.
	
	Apollo girò dietro la scrivania e guardò la strada fangosa e sconnessa,
	al di là del muro che cingeva il palazzo e gli edifici del collegio,
	isolando il santuario del podestà dalla città plebea. Lo studioso stava
	indicando la figura ombrosa d\textquotesingle un contadino che guidava
	verso casa un asinello, nel crepuscolo. I piedi
	dell\textquotesingle uomo erano avvolti in tela da sacco, e il fango li
	aveva impiastricciati al punto che l\textquotesingle uomo sembrava quasi
	incapace di sollevarli. Ma avanzava, faticosamente, un passo dopo
	l\textquotesingle altro, riposando per mezzo secondo prima di sollevare
	un piede. Sembrava troppo debole per grattare via il fango.
	
	--- Vedete, non cavalca l\textquotesingle asino --- osservò il Thon
	Taddeo perché questa mattina l\textquotesingle asino era carico di
	grano. Il contadino non pensa che adesso i sacchi sono vuoti. Ciò che va
	bene al mattino va bene anche il pomeriggio.
	
	--- Lo conoscete?
	
	--- Passa anche sotto la mia finestra. Tutte le mattine e tutte le sere.
	Non lo avete mai notato?
	
	--- Ne ho notati migliaia come lui.
	
	--- Guardate. Riuscite a credere che quel bruto sia il discendente
	diretto di uomini che avrebbero inventato macchine per volare, che
	avrebbero raggiunto la luna, imbrigliato le forze della Natura,
	costruito meccanismi capaci di parlare e forse anche di pensare? Potete
	credere che uomini simili siano esistiti?
	
	Apollo taceva.
	
	--- Guardatelo! --- insistette lo studioso. --- No, adesso è troppo
	buio. Non potete vedere le piaghe della sifilide sul suo collo, né il
	modo in cui la radice del suo naso è corrosa. È affetto da paresi. Ma
	indubbiamente, fin dall\textquotesingle inizio, era un idiota.
	Analfabeta, superstizioso, pieno di istinti malvagi. Ha contagiato i
	suoi figli. Li ucciderebbe per poche monete. Li venderà, in ogni caso,
	non appena saranno abbastanza cresciuti per rendersi utili. Guardatelo,
	e ditemi se siete al cospetto della progenie di una civiltà un tempo
	potentissima. \emph{Che cosa vedete?}
	
	--- L\textquotesingle immagine di Cristo --- scattò il monsignore,
	sorpreso della propria ira improvvisa. --- Cosa pensate che io veda?
	
	Lo studioso sbuffò, irato e impaziente. ---
	L\textquotesingle incongruenza. Uomini come voi possono osservare quella
	gente da ogni finestra, e uomini come gli storici vorrebbero farci
	credere che un tempo esistessero veri uomini. Non posso accettarlo. Come
	può una grande e saggia civiltà essersi distrutta così completamente?
	
	--- Forse --- disse Apollo --- si è distrutta perché era grande e saggia
	soltanto materialmente, e null\textquotesingle altro. --- Andò ad
	accendere una lampada a sego, perché il crepuscolo svaniva rapidamente
	nella notte. Colpi esca e acciarino fino a che la scintilla non si
	comunicò all\textquotesingle esca, poi vi soffiò sopra, dolcemente.
	
	--- Forse --- disse il Thon Taddeo. --- Ma io ne dubito. --- Voi
	rifiutate tutta la storia, dunque, come mito? --- Dalla scintilla spuntò
	una fiamma.
	
	--- Non la rifiuto. Ma deve essere controllata. Chi ha scritto la vostra
	storia?
	
	--- Gli ordini monastici, naturalmente. Durante i secoli
	dell\textquotesingle oscurantismo, non v\textquotesingle era nessun
	altro per scriverla. --- Apollo trasferì la fiamma allo stoppino.
	
	--- Ecco! Ci siamo! E durante il tempo degli antipapi, quanti Ordini
	scismatici fabbricarono versioni proprie degli eventi, e gabellarono
	quelle versioni come opera di autori precedenti? Non potete sapere, non
	potete sapere \emph{veramente}. Non si può negare che su questo
	continente vi sia stata una civiltà molto più progredita della nostra
	attuale.
	
	Basta guardare alle macerie, ai metalli arrugginiti, per saperlo. Basta
	scavare sotto una striscia di sabbia e si trovano le loro strade
	dissestate. Ma dov\textquotesingle è la prova
	dell\textquotesingle esistenza delle macchine che secondo i vostri
	storici possedevano gli antichi? Dove sono i resti dei carri che si
	muovevano da soli, delle macchine volanti?
	
	--- Ora sono fusi nei vomeri e nelle zappe.
	
	--- Se sono esistiti.
	
	--- Se ne dubitate, perché disturbarvi a studiare i documenti
	leibowitziani?
	
	--- Perché un dubbio non è una negazione. Il dubbio è uno strumento
	potente, e dovrebbe essere applicato alla storia.
	
	II Nunzio sorrise, a labbra strette. --- E cosa volete che faccia a
	questo proposito, dotto Thon?
	
	Lo studioso si tese in avanti, impaziente. --- Scrivete
	all\textquotesingle abate di quel luogo. Assicuratelo che i documenti
	saranno trattati con estrema cura, e gli saranno resi non appena li
	avremo completamente esaminati per accertarne
	l\textquotesingle autenticità e ne avremo studiato il contenuto.
	
	--- E in nome di chi dovrò fornirgli questa assicurazione\ldots{} vostro
	o mio?
	
	--- In nome di Hannegan, in nome vostro e in nome mio. --- Potrò dargli
	solo la vostra parola e quella di Hannegan. Io non ho truppe al mio
	comando.
	
	Lo studioso arrossì.
	
	--- Ditemi --- aggiunse in fretta il Nunzio --- perché, a parte la
	minaccia dei banditi, insistete per esaminarli qui, invece di recarvi
	all\textquotesingle abbazia?
	
	--- La migliore ragione che potrete dare all\textquotesingle abate è che
	se dobbiamo esaminare e provare l\textquotesingle autenticità dei
	documenti all\textquotesingle abbazia, una conferma non significherebbe
	molto per gli altri studiosi secolari.
	
	--- Volete dire che i vostri colleghi potrebbero pensare che i monaci vi
	abbiano raggirato?
	
	--- Uhm\ldots{} è possibile. Ma è importante anche questo: se i
	documenti verranno portati qui, potranno essere esaminati da chiunque,
	nel collegio. E ogni Thon che venisse qui in visita potrebbe osservarli.
	Ma non possiamo trasferire l\textquotesingle intero collegio nel deserto
	del Sud-ovest per sei mesi.
	
	--- Capisco.
	
	--- Manderete la richiesta all\textquotesingle abbazia?
	
	--- Sì.
	
	Il Thon Taddeo si mostrò sorpreso.
	
	--- Ma sarà la vostra richiesta, non la mia. Ed è giusto che vi dica che
	non credo che don Paulo, l\textquotesingle abate, accetterà.
	
	Il Thon, tuttavia, sembrò soddisfatto. Quando se ne fu andato, il Nunzio
	chiamò il suo assistente. --- Partirete per Nuova Roma domani --- gli
	disse.
	
	--- Passando dall\textquotesingle Abbazia di Leibowitz?
	
	--- No, passate dall\textquotesingle abbazia al ritorno. Il rapporto per
	Nuova Roma è urgente.
	
	--- Sì, monsignore.
	
	--- All\textquotesingle abbazia, dite a don Paulo che la regina di Saba
	aspetta che Salomone venga a lei. Portando doni. Poi farete bene a
	coprirvi le orecchie. E quando don Paulo avrà finito di esplodere,
	affrettatevi a ritornare, in modo che possa dire di no al Thon Taddeo.
	