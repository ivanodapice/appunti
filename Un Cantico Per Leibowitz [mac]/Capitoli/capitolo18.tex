	\chapter{\phantom{title}}

\lettrine{O}{ra} avvenne come nel tempo di Giobbe\ldots{} --- cominciò il frate
lettore dal leggio del refettorio:

Quando i figliuoli di Dio vennero a presentarsi al cospetto del Signore,
anche Satana era presente tra essi.

E il Signore disse a lui: ``Donde vieni tu, o Satana?''.

E Satana rispondendo disse, come d\textquotesingle antico: ``Io ho
percorso la terra, e ho camminato su di essa''.

E il Signore disse a lui: ``Hai tu considerato quel semplice e retto
principe, il mio servitore Name, che odia il male e ama la pace?''.

E Satana rispondendo disse: ``Forse che Name teme Iddio invano?
Imperocché non hai Tu benedetto la sua terra con grande ricchezza e non
Io hai Tu fatto possente fra le nazioni? Ma distendi la Tua mano e
decresci ciò che egli ha, e lascia che il suo nemico sia rafforzato: poi
vedrai se non ti bestemmierà sul Tuo viso''.

E il Signore disse a Satana: ``Guarda ciò che possiede, e diminuiscilo.
Provvedi tu a questo''.

E Satana si allontanò dalla presenza di Dio e ritornò nel mondo.

Ora il principe Name non era il santo Giobbe, perché quando la sua terra
fu afflitta da pene e quando il suo popolo fu meno ricco, quando egli
vide il suo nemico diventare più potente, divenne timoroso e cessò di
fidare in Dio, pensando fra sé: ``Io devo colpire prima che il nemico mi
sopraffaccia senza impugnare la spada''.

--- ``E così fu in quei giorni''\ldots{} --- disse il frate lettore:

\ldots{} che i principi della Terra avevano indurito i loro cuori contro
la Legge del Signore, e al loro orgoglio non era fine. E ciascheduno di
essi pensava entro di sé che era cosa molto migliore essere distrutto
che permettere alla volontà di altri principi di prevalere sulla sua.
Perché i potenti della Terra contendevano fra loro per il supremo potere
sopra ogni cosa: per inganno, violenza e tradimento essi cercavano di
dominare, e temevano grandemente la guerra e ne tremavano; imperocché il
Signore Iddio aveva permesso che gli uomini sapienti di quei tempi
imparassero i modi per cui il mondo medesimo poteva essere distrutto, e
nelle loro mani era affidata la spada dell\textquotesingle Arcangelo con
la quale Lucifero era stato abbattuto, e per cui gli uomini e i principi
potessero temere Iddio e umiliarsi davanti
all\textquotesingle Onnipotente. Ma essi non si erano umiliati.

E Satana parlò a un principe, e disse: ``Non temere di, usare la spada,
perché gli uomini sapienti ti hanno ingannato dicendoti che il mondo ne
sarebbe distrutto. Non ascoltare il consiglio dei deboli, imperocché
essi grandemente ti temono, e servono ai tuoi nemici fermando la tua
mano contro di quelli. Colpisci, e sappi che tu sarai il sovrano di
tutto''.

E il principe ascoltò la parola di Satana, e chiamò a sé tutti gli
uomini sapienti di quel reame e comandò loro di dargli consiglio dei
modi in cui il nemico poteva essere distrutto senza attirare la collera
sopra il suo regno. Ma molti degli uomini sapienti dissero: ``Signore,
questo non è possibile, imperocché anche i tuoi nemici possiedono la
spada che noi ti abbiamo data, e la terribilità di essa è come la fiamma
dell\textquotesingle Inferno, è come il furore della stella-sole, alla
quale un giorno fu accesa''.

``E allora tu me ne foggerai un\textquotesingle altra che sia ancora
sette volte più ardente dell\textquotesingle Inferno stesso'' comandò il
principe, la cui arroganza era giunta a superare quella di Faraone.

E molti degli uomini sapienti dissero: ``No, Signore, non chiedere a noi
tale cosa: imperocché persino il fumo di un tale fuoco, se noi dovessimo
accenderlo per te, sarebbe la causa perché molti periscano''.

Ora il principe era adirato delle loro risposte, e sospettava che essi
lo tradissero, e mandò le sue spie fra essi per tentarli e per
provocarli: e per questo gli uomini sapienti divennero timorosi. Alcuni
fra essi mutarono le loro risposte, perché l\textquotesingle ira del
principe non si scatenasse sopra di loro. Tre volte egli li interrogò, e
tre volte essi risposero: ``No, o Signore, anche il tuo stesso popolo
perirà, se tu farai questo''. Ma uno dei magi era simile a Giuda
Iscariota, e la sua testimonianza era artefatta, e avendo tradito i suoi
fratelli, menti a tutto il popolo, consigliandolo a non temere il demone
Fallout. Il principe ascoltò questo falso sapiente, il cui nome era
Blackeneth, e causò che le spie accusassero molti dei magi al cospetto
del popolo. Essendo intimoriti i meno saggi fra i magi consigliarono il
principe secondo il suo piacere, dicendo: ``Le armi possono essere
usate; bada soltanto a che non si ecceda tale e tale limite, o tutti
periremo sicuramente''.

E il principe cancellò le città dei suoi nemici con il nuovo fuoco, e
per tre giorni e tre notti le sue grandi catapulte e i suoi uccelli di
metallo fecero piovere l\textquotesingle ira sopra di esse. Su ciascuna
di quelle città apparve un sole, ed era più ardente del sole nei cieli,
e immediatamente quella città avvizziva e si scioglieva come cera sotto
una torcia, e le genti si fermavano nelle vie e le loro pelli fumavano
ed essi divenivano come fagotti gettati sulle braci. E quando la furia
del sole era svanita, la città era in fiamme; e un grande tuono scendeva
dal cielo, come il grande ariete PIK-A-DON, per distruggerla
completamente. Fumi velenosi ricadevano sopra tutta la terra, e la terra
splendeva la notte dei residui del fuoco e della maledizione dei residui
del fuoco che causava scrofola sulla pelle e faceva cadere i capelli e
morire il sangue nelle vene.

E un grande fetore si levò dalla terra fino al cielo. Come Sodoma e
Gomorra erano la Terra e le sue rovine, persino nelle terre di quel
principe, imperocché i suoi nemici non avevano frenato la loro vendetta,
mandando fuoco a loro volta per inabissare le sue città. Il fetore di
quella carneficina era grandemente offensivo al Signore, Che parlò al
principe Name dicendo:
\leavevmode\\
\begin{center}
	{\large{``CHE OFFERTA BRUCIANTE E MAI QUESTA CHE TU HAI
			PREPARATO DAVANTI A ME? QUALE E L\textquotesingle AROMA CHE SI LEVA DAL
			LUOGO DELL\textquotesingle OLOCAUSTO? MI HAI TU FATTO OLOCAUSTO DI
			PECORE O DI CAPRE, O HAI OFFERTO UN VITELLO A DIO?''.}}
\end{center}
\leavevmode\\
Ma il principe non rispose, e Dio disse: 
\leavevmode\\
\begin{center}
	{\large{``TU MI HAI FATTO OLOCAUSTO DEI MIEI FIGLI''.}}
\end{center}
\leavevmode\\
E il Signore lo fece morire insieme a Blackeneth, il traditore, e vi fu
pestilenza sulla Terra, e la follia fu sopra l\textquotesingle umanità,
che lapidò i sapienti insieme con i potenti, quanti ne rimanevano.

Ma v\textquotesingle era in quel tempo un uomo il cui nome era
Leibowitz, che, nella sua giovinezza, come
sant\textquotesingle Agostino, aveva amato la saggezza del mondo più
della saggezza di Dio. Ma ora vedendo che la grande sapienza, benché
buona, non aveva salvato il mondo, si rivolse al Signore in penitenza,
gridando\ldots{}

L\textquotesingle abate batté seccamente sulla tavola, il monaco che
stava leggendo l\textquotesingle antico racconto tacque immediatamente.

--- E questo è il solo resoconto di cui disponete? --- chiese il Thon
Taddeo, sorridendo all\textquotesingle abate, a labbra strette.

--- Oh, vi sono parecchie versioni. Differiscono in particolari
trascurabili. Nessuna dice con certezza quale nazione lanciò il primo
attacco\ldots{} non che questo abbia più importanza, ormai. Il testo che
il frate lettore stava leggendo fu scritto pochi decenni dopo la morte
di san Leibowitz\ldots{} fu probabilmente uno dei primi resoconti, dopo
che fu di nuovo possibile scrivere ancora, senza correre rischi.
L\textquotesingle autore fu un giovane monaco che non aveva vissuto la
distruzione: raccolse le notizie di seconda mano, dai seguaci di san
Leibowitz, i primi contrabbandieri di libri e memorizzatori, e avevano
una predilezione per l\textquotesingle imitazione della Scrittura. Io
dubito che un solo resoconto completo e accurato del Diluvio di Fiamma
esista da qualche parte. Una volta scatenato, a quanto pare fu troppo
immenso perché una singola persona potesse vederne
l\textquotesingle intero quadro.

--- In quale terra viveva questo Principe chiamato Name, e
quell\textquotesingle uomo, Blackeneth?

L\textquotesingle abate Paulo scosse il capo. --- Neppure
l\textquotesingle autore di quel racconto ne era certo. Ne abbiamo
confrontati molti, abbastanza per sapere che persino alcuni dei
governanti meno importanti di quel tempo avevano messo le mani su quelle
armi, prima della catastrofe. La situazione che egli descrisse dovette
presentarsi probabilmente in più di una nazione. I Name e i Blackeneth
erano probabilmente legioni.

Naturalmente, anch\textquotesingle io ho udito simili leggende. È
evidente che accadde qualcosa di terribile --- dichiarò il Thon; e poi,
bruscamente: --- Ma quando posso cominciare a esaminare\ldots{} come li
chiamate?

--- I Memorabilia.

--- Naturalmente. Il Thon sospirò e sorrise distrattamente
all\textquotesingle immagine del santo. --- Domani sarebbe troppo
presto?

Potete cominciare anche subito, se volete --- disse
l\textquotesingle abate. --- Dovete sentirvi libero di andare e venire a
vostro piacere.

I sotterranei erano fiocamente illuminati dalle candele, e solo pochi
monaci-studiosi vestiti di scuro si muovevano negli stalli. Frate
Armbruster studiava di malumore i suoi documenti, alla luce
d\textquotesingle una lampada, nel suo cubicolo ai piedi della scala di
pietra, e un\textquotesingle altra lampada ardeva nella sezione della
Teologia Morale, dove un monaco se ne stava chino su un antico
manoscritto. Era passata l\textquotesingle Ora Prima, e quasi tutti
lavoravano nell\textquotesingle abbazia: in cucina, nella scuola, in
giardino, nella stalla e negli uffici, lasciando la libreria quasi
deserta fino al pomeriggio avanzato, all\textquotesingle ora della
\emph{lectio divina}. Quella mattina, tuttavia, i sotterranei erano
relativamente affollati.

Tre monaci se ne stavano nell\textquotesingle ombra, dietro la nuova
macchina. Tenevano le mani affondate nelle maniche e osservavano un
quarto monaco ritto ai piedi della scala. Il quarto monaco fissava
pazientemente un quinto monaco che se ne stava sul pianerottolo e
sorvegliava l\textquotesingle ingresso della scala.

Frate Kornhoer si era preoccupato del suo apparecchio come un genitore
ansioso, ma quando non era più riuscito a trovare fili da annodare e
modifiche da effettuare, si era ritirato nell\textquotesingle alcova
della Teologia Naturale a leggere e ad aspettare. Sarebbe stato
ammissibile che impartisse un sommario di minuziose istruzioni alla sua
piccola squadra, ma aveva preferito mantenere il silenzio, e se qualche
pensiero del momento che si avvicinava gli attraversava la mente, mentre
aspettava, l\textquotesingle espressione del monaco inventore non lo
lasciava capire. Poiché l\textquotesingle abate non si era preso il
disturbo di assistere a una dimostrazione dei funzionamento della
macchina, frate Kornhoer non dava segno di aspettarsi applausi, e aveva
perfino superato la tendenza a guardare don Paulo con aria di
rimprovero.

Un lieve sibilo, proveniente dalla scala, mise di nuovo in allarme il
sotterraneo, sebbene vi fossero stati già parecchi falsi allarmi. Era
evidente che nessuno aveva informato l\textquotesingle illustre Thon che
una meravigliosa invenzione attendeva la sua ispezione nel sotterraneo.
Era evidente che se anche qualcuno gliene aveva parlato, ne aveva
minimizzato l\textquotesingle importanza. Era ovvio che il Padre Abate
aveva provveduto a smorzare l\textquotesingle entusiasmo di tutti.
Questo era il silenzioso significato delle occhiate che si scambiavano
tra loro, mentre aspettavano.

Questa volta il sibilo di avvertimento non era venuto invano. Il monaco
che stava di guardia in cima alla scala si voltò solennemente e si
inchinò verso il quinto monaco, sul pianerottolo più in basso.

--- \emph{In principio Deus} --- disse sottovoce.

Il quinto monaco si voltò e si inchinò verso il quarto, che stava ai
piedi delle scale --- \emph{Coelum et terram creavit} --- mormorò a sua
volta.

Il quarto monaco si voltò verso i tre che attendevano dietro la
macchina. --- \emph{Vacuus autem erat mundus} --- annunciò.

--- \emph{Cum tenebris in superficie profundorum} --- fece coro il
gruppo.

--- \emph{Ortus est Dei Spiritus supra aquas} --- esclamò frate
Kornhoer, rimettendo il libro nello scaffale, con un tintinnio di
catene.

--- \emph{Gratias Creatori Spiritui} --- rispose
l\textquotesingle intera squadra.

--- \emph{Dixique Deus: ``FIAT LUX''} - disse
l\textquotesingle inventore, in tono di comando.

Le sentinelle sulla scala scesero, per prendere i loro posti. Quattro
monaci azionarono la ruota. Il quinto monaco rimase ritto accanto alla
dinamo. Il sesto salì sulla scaletta e sedette
sull\textquotesingle ultimo gradino, urtando con il capo contro la
volta. Si calò sul viso una maschera di pergamena oleosa annerita con il
fumo per proteggersi gli occhi, poi armeggiò con la lampada e la
relativa vite, mentre frate Kornhoer lo sorvegliava nervosamente dal
basso.

--- \emph{Et lux ergo facto est} --- disse, quando ebbe trovato la vite.
--- \emph{Lucem esse bonam Deus vidit} --- gridò
l\textquotesingle inventore al quinto monaco.

Il quinto monaco si curvò sulla dinamo, con una candela, per dare
un\textquotesingle ultima occhiata alle spazzole di contatto.

--- \emph{Et secrevit lucem a tenebris} --- disse finalmente,
continuando a recitare la Scrittura.

--- \emph{Lucem appellavit ``diem''} --- fece coro la squadra che
azionava la dinamo --- \emph{et tenebras ``noctes''}. --- Poi
appoggiarono le spalle ai raggi del tornichetto.

Le assi scricchiolarono e gemettero. La dinamo, fatta di ruote di carro,
cominciò a girare, e il suo basso ronzio diventò un grugnito e poi un
gemito, mentre i monaci si sforzavano brontolando. Il guardiano della
dinamo osservava ansioso, mentre i raggi si confondevano, nella
velocità, e diventavano una specie di pellicola.

--- \emph{Vespere occaso} --- cominciò, poi si interruppe per leccarsi
due dita, che accostò ai contatti. Scoccò una scintilla.

--- \emph{Lucifer!} --- gridò, balzando indietro, poi finì, in tono più
calmo: --- \emph{ortus est et primo die}.

--- CONTATTO! --- disse frate Kornhoer, mentre don Paulo, il Thon Taddeo
e il suo segretario scendevano le scale.

Il monaco sulla scala colpì l\textquotesingle arco. Un netto
\emph{spffft!\ldots{}} e una luce accecante inondò il sotterraneo con
uno splendore che non era mai stato visto in dodici secoli.

Il gruppo si fermò sulla scala. Il Thon Taddeo boccheggiò una
imprecazione nella sua lingua natia, poi indietreggiò
d\textquotesingle un gradino. L\textquotesingle abate, che non aveva
assistito alla prova dell\textquotesingle ordigno e non aveva dato
credito alle stravaganti affermazioni al riguardo, impallidì e si
interruppe a metà d\textquotesingle una frase. Il segretario rimase
immobilizzato per il panico, poi all\textquotesingle improvviso fuggì,
urlando ``Al fuoco!''.

L\textquotesingle abate si fece il segno della croce. --- Non sapevo!
--- sussurrò.

Lo studioso, che aveva superato il primo trauma del bagliore, sondò il
sotterraneo con lo sguardo, notò la dinamo e i monaci che faticavano ai
suoi raggi. I suoi occhi seguirono i fili avvolti nella stoffa, notarono
il monaco sulla scala, misurarono il significato della dinamo e il
monaco che aspettava, a occhi bassi, ai piedi della scala.

--- Incredibile! --- sussurrò.

Il monaco ai piedi della scala si inchinò, umilmente. Il bagliore
biancazzurro gettava ombre nettissime nella stanza, e le fiamme delle
candele diventavano confusi fuochi fatui nella marea luminosa.

--- Splende come mille torce! --- ansimò lo studioso. Deve essere un
antico\ldots{} ma no! È impensabile!

Continuò a scendere le scale come in trance. Si fermò accanto a frate
Kornhoer e lo fissò, incuriosito, per un momento, poi posò i piedi sul
pavimento del sotterraneo. Senza toccare nulla, senza chiedere nulla e
guardando tutto, girò attorno al macchinario, ispezionò la dinamo, i
fili, la stessa lampada.

--- Non sembra possibile, ma\ldots{}

L\textquotesingle abate si era ripreso: scese le scale. --- Siete
dispensato dal silenzio! --- sussurrò a frate Kornhoer. --- Parlategli.
Io sono\ldots{} un po\textquotesingle{} stordito.

Il monaco si illuminò. --- Vi piace, Monsignor Abate?

--- È tremendo --- gemette don Paulo.

Il viso dell\textquotesingle inventore si rattristò.

--- È un modo scandaloso di trattare un ospite! Quella luce ha
spaventato a morte l\textquotesingle assistente del Thon. Ne sono
mortificato!

--- In effetti, è una luce piuttosto brillante.

--- E infernale! Andate a parlare al Thon, mentre io cerco il modo
migliore per scusarmi.

Ma lo studioso doveva avere tratto un giudizio, sulla base delle sue
osservazioni, perché avanzò a passo rapido verso di loro. Il suo viso
era teso, i suoi modi sbrigativi.

--- Una lampada a elettricità --- disse. --- Come siete riusciti a
tenerla nascosta per tutti questi secoli? Dopo tutti gli anni spesi nel
tentativo di arrivare a una teoria del\ldots{} --- Sembrò quasi
soffocato, e parve lottare per riacquistare
l\textquotesingle autocontrollo, come se fosse stato vittima di un
mostruoso scherzo. --- \emph{Perché l\textquotesingle avete nascosta?}
Vi è qualche significato religioso\ldots{} E che cosa\ldots{} --- La
confusione più completa lo costrinse a interrompersi. Scosse il capo e
si guardò intorno, come se cercasse una via d\textquotesingle uscita.

--- Voi avete frainteso --- disse debolmente l\textquotesingle abate,
afferrando per il braccio frate Kornhoer. --- Per amor di Dio, fratello,
spiegatevi!

Ma non c\textquotesingle era un balsamo capace di lenire un affronto
all\textquotesingle orgoglio professionale\ldots{} né allora, né in
alcuna altra età.
