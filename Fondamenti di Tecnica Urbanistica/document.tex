\documentclass[a4paper,12pt, oneside]{book}
\pdfminorversion=5 
\pdfcompresslevel=9
\pdfobjcompresslevel=2
\pdfminorversion=5
% \usepackage{fullpage}
\usepackage[italian]{babel}
\usepackage[utf8]{inputenc}
\usepackage{amssymb}
\usepackage{graphicx}
\usepackage[font=small,labelfont=bf]{caption}
\usepackage{csquotes}
\usepackage{amsthm}
\usepackage{graphics}
\usepackage{amsfonts}
\usepackage{amsmath}
\usepackage{amstext}
\usepackage{engrec}
\usepackage{rotating}
\usepackage[safe,extra]{tipa}
\usepackage{multirow}
\usepackage{hyperref}
\usepackage{microtype}
\usepackage{enumerate}
\usepackage{braket}
\usepackage{marginnote}
\usepackage{pgfplots}
\usepackage{cancel}
\usepackage{polynom}
\usepackage{booktabs}
\usepackage{enumitem}
\usepackage{framed}
\usepackage{pdfpages}
\usepackage{pgfplots}
\usepackage{fancyhdr}
\pagestyle{fancy}
\fancyhead[LE,RO]{\slshape \rightmark}
\fancyhead[LO,RE]{\slshape \leftmark} 
\fancyfoot[C]{\thepage}
\definecolor{iceberg}{rgb}{0.44, 0.65, 0.82}
\usepackage{multicol}

\title{\textbf{Appunti Del Corso}\\ \textbf{Di}\\ \textbf{Fondamenti Di Tecnica Urbanistica}}
\author{Prof. G. Mazzeo\\\\ A.A. 2021/2022}
\date{8 Marzo 2020}

\pgfplotsset{compat=1.13}
\begin{document}
\maketitle
	
\definecolor{shadecolor}{gray}{0.80}
	
\newtheorem{teorema}{Teorema}
\newtheorem{definizione}{Definizione}
\newtheorem{esempio}{Esempio}
\newtheorem{corollario}{Corollario}
\newtheorem{lemma}{Lemma}
\newtheorem{osservazione}{Osservazione}
\newtheorem{nota}{Nota}
\newtheorem{esercizio}{Esercizio}
\tableofcontents
\renewcommand{\chaptermark}[1]{%
	\markboth{\chaptername
		\ \thechapter.\ #1}{}}
	\renewcommand{\sectionmark}[1]{\markright{\thesection.\ #1}}

\chapter{Introduzione al Corso}
\textbf{Questi appunti sono presi a lezione con il
docente \textit{Giuseppe Mazzeo}. Per quanto sia stata fatta una revisione è altamente
probabile (praticamente certo) che possano contenere errori, sia di stampa che
di vero e proprio contenuto.\\Per eventuali proposte di correzioni : \href{mailto:ivanodapice@hotmail.com}{ivanodapice@hotmail.com}.\\
Per quanto riguarda invece informazioni utili relative a come contattare il professore, qui sotto sono riportate le sue due mail principali e in più quella relativa agli elaborati del corso.
 \begin{list}{\labelitemi}{\leftmargin=1em}
   \item \texttt{Mail:} \href{mailto:gimazzeo@unina.it}{gimazzeo@unina.it}. \\ 
   \hspace{\labelwidth}\phantom{\texttt{Mail:} }\href{mailto:giuseppe.mazzeo@ismed.cnr.it}{giuseppe.mazzeo@ismed.cnr.it}. \\
   \hspace{\labelwidth}\phantom{\texttt{Mail:} }\href{mailto:corsotu2@libero.it}{corsotu2@libero.it}.
 \end{list}\leavevmode\\   
\textbf{Grazie mille e buono studio!}}

\section{Urbanistica}
L'urbanistica è una materia che definisce scelte di pianificazione territoriale (Urbano e Rurale); Gestione di trasformazioni di tipo territoriale come espansioni urbane; Tecniche e Strumenti per la redazione di piani ed infine valutazione di fattori di rischio e sostenibilità (quale consumo, risorse rinnovabili ecc).\\
Uno dei più grandi urbanisti italiani è \textit{Giovanni Astengo}, il quale affermava che l'urbanistica è la scienza che studia i fenomeni urbani tenendo conto dello sviluppo storico, l'interpretazione e l'adattamento funzionale di aggregati urbani già esistenti.\newline
Nel corso del tempo l'importanza di espandersi e creare nuovi insediamenti urbani era molto maggiore di andare ad agire su quelli già esistenti. Guardando invece ad oggi notiamo che c'è stato un punto intorno al dopoguerra dove grazie all'incredibile stock edilizio, i volumi sono diventati sovrabbondanti. Non c'è un reale bisogno di espansione come un tempo e quindi l'interesse globale si è spostato sul curare gli insediamenti urbani già esistenti.\newline
Dalle parole di un altro noto urbanista quale \textit{Edoardo Salzano} notiamo come le nostre scelte che influiranno sulle trasformazioni territoriali dovranno essere tra loro coerenti e flessibili. Il piano urbanistico infatti rimane nel  tempo, quindi adottare soluzioni rigide per il nostro periodo potrebbe essere controproducente in un prossimo futuro.
\leavevmode\\
 \begin{center}
   \includegraphics[width=0.7\linewidth]{"Immagini/2019-World-Population-Growth-1700-2100"}
   \captionof{figure}{crescita della popolazione dal 1700 ad oggi e proiezione al 2100}
 \end{center}
\leavevmode\\
Le città nel corso degli ultimi decenni hanno subito trasformazioni notevoli.\\
Perché ci interessiamo della trasformazione del territorio? Ovviamente ce ne occupiamo per cercare di migliorare la vita delle persone, quindi dobbiamo pensare a quali fenomeni si stanno verificando a livello globale relativamente all'andamento della popolazione.\\
Se consideriamo l'andamento della popolazione dal 1700 ad oggi notiamo che negli anni '60 abbiamo avuto un notevole aumento nel numero della popolazione. Per passare poi da 3 milioni di individui a 5 milioni ci sono voluti più o meno 50 anni, il che è notevole pensando alla crescita precedente.\\
Tutto ciò è una conseguenza del miglioramento generale della vita. Il commercio è cresciuto e sono state introdotte nuove tecnologie.\\
Nella figura [\ref{fig:populationgrowth}] troviamo una stima di come la popolazione crescerà nel futuro, con previsioni diverse a seconda dei vari scenari ipotizzati. Queste ipotesi possono essere più o meno valide essendo solo stime, ma capiamo come nei casi più estremi questa crescita possa diventare esponenzialmente paurosa; come si nota dalle quasi 17 miliardi di persone previste per il 2100.
\leavevmode\\
\subsection{Cambiamento Demografico}
 \begin{center}
  \begin{minipage}{0.48\linewidth}
    \includegraphics[width=\linewidth]{"Immagini/future-timeline-2050-2100"}
    \captionof{figure}{previsioni stimate riguardo l'accrescimento della popolazione al 2100}
    \label{fig:populationgrowth}
  \end{minipage}%
\hfill
  \begin{minipage}{0.46\linewidth}
    \includegraphics[width=\linewidth]{"Immagini/countries_growth_1950_2100"}
    \captionof{figure}{distribuzione della popolazione stimata al 2100}
    \label{fig:countriesgrowth}
  \end{minipage}
 \end{center}
\leavevmode\\
Quando diciamo che la popolazione aumenta perché la qualità media della vita aumenta non intendiamo di certo per tutti. Ci sono punti del pianeta che hanno una qualità di vita nettamente superiore ad altri. Inoltre, questa condizione è strettamente legata alla fertilità generale.\\
Vediamo dal grafico [\ref{fig:countriesgrowth}] che dal 1950 al 2010 l'Europa ha subito un cambiamento del'11\%   (22\% - 11\%), il che indica che altre zone del pianeta hanno avuto un aumento considerevole.
\leavevmode\\
 \begin{center}
   \includegraphics[width=0.7\linewidth]{"Immagini/pyramid-demographic"}
   \captionof{figure}{esempi di piramidi demografiche}
 \end{center}
\leavevmode\\
Un'altro importante elemento da considerare è che nei paesi più ricchi la popolazione cambiando qualità di vita cambia anche le proprie statistiche demografiche.\\
Se consideriamo ad esempio L'Italia dal secondo dopoguerra ad oggi, la piramide dell'età nel secondo dopoguerra aveva un aspetto uniforme. Oggi queste piramidi hanno invece un andamento a bottiglia o fuso, ciò implica che nascono meno bambino ed essendo l'aspettativa di vita più alta troveremo un numero più ampio di persone anziane.\\
Tutto ciò porta una serie di problemi, quali la minoranza di bambini e persone in età lavorativa che potrebbero portare problemi riguardo il sostegno delle fasce di età più avanzate.\\
\leavevmode\\
 \begin{center}
   \includegraphics[width=0.7\linewidth]{"Immagini/popolazione urbana-rurale"}
   \captionof{figure}{comparativa tra popolazione urbana e rurale negli anni}
 \end{center}
\leavevmode\\
Questo grafico è di grande interesse perché ci dice qual è la ripartizione a livello globale tra la popolazione abitativa delle zone urbane da quelle rurali.\\
A partire dal 1975 c'è stato il sorpasso delle zone urbane su quelle rurali, inoltre da allora è sempre più crescente. Dal grafico vediamo infatti che nel 2010 il numero di abitanti per zona urbana era quasi 7 volte superiore a quello delle zone rurali. Ovviamente questo pone questioni per quanto riguarda la sostenibilità dei sistemi urbani.
\leavevmode\\
 \begin{center}
   \includegraphics[width=0.7\linewidth]{"Immagini/worldurbanpop"}
   \captionof{figure}{percentuale di popolazione urbana rispetto ad ogni continente}
 \end{center}
\leavevmode\\
Abbiamo detto che le popolazioni che vivono con una maggiore ricchezza sono anche quelle con meno fertilità, tali popolazioni sono statisticamente più incentrate sulla vita urbana.\\
Se consideriamo il Nord America vediamo che entro il 2050 si stima raggiunga una percentuale di urbanizzazione pari al 90\textdiscount. I paesi meno sviluppati come quelli dell'Africa però hanno lo stesso tipo di andamento. Tenendo conto proprio dell'Africa, nel 2010 vediamo una percentuale pari al 40\textdiscount, con una previsione di crescita fino al $\sim$62-63\% entro il 2050.
\leavevmode\\
 \begin{center}
  \begin{minipage}{0.48\linewidth}
    \includegraphics[width=\linewidth]{"Immagini/Megacities"}
    \captionof{figure}{proiezione megacities nel 2025}
    \label{fig:megacities}
  \end{minipage}%
\hfill
  \begin{minipage}{0.46\linewidth}
    \includegraphics[width=\linewidth]{"Immagini/alpha2004c"}
    \captionof{figure}{gerarchie urbane a livello globale}
    \label{fig:alphacities}
  \end{minipage}
\end{center}
\leavevmode\\
Nelle due immagini sopracitate, la prima ci mostra che le mega città dei prossimi anni saranno concentrate prevalentemente nelle regioni asiatiche, mentre la gerarchia delle città più potenti in quanto a economia globale rimane comunque incentrata ad ovest.\\
Da queste grandi espansioni abbiamo capito quindi che il fenomeno urbano ha raggiunto rilevanza planetaria. L'espansione delle strutture però avviene in molti paesi senza un reale controllo o attenzione ai fattori ambientali (Solo in occidente si cerca di regolamentare in modo efficace).\\
Le città quindi non sono uguali tra loro per dimensione, consapevolezza ambientale e soddisfacimento dei diritti primari. Per questo dobbiamo cercare di applicare il più possibile politiche di sostenibilità alle città.\\
\chapter{Origine del Fenomeno Urbano}
Nell'800 avviene un fenomeno che cambia completamente il modo di produrre i beni, e quindi le strutture sociologiche.\\
Nasce per l'evoluzione dei processi produttivi, quindi la rivoluzione industriale ha come fenomeno lo sviluppo delle città.\\
Questo processo di crescita delle città è durato per quasi più di 200 anni, infatti da poco le città hanno smesso di crescere in modo sostenuto. Dagli anni '80 invece abbiamo addirittura riscontrato un rallentamento di tali processi di crescita causati dal maggiore interessamento all'ecosistema globale.\\
Londra nell'arco di 400 anni (1300-1700) è cresciuta di 7 volte, ma con percentuali di superficie di partenza molto piccole. Per raddoppiare ulteriormente la città ce ne vogliono altri 100. Via discorrendo l'espansione territoriale accelera sempre maggiormente, tanto che dal 1944 al 2008 passa da avere un 58\% di superficie urbana al 100\%.\\
Se contiamo però la popolazione abitativa della città, con la crescita della superficie urbanizzata non c'è un uguale accrescimento della popolazione; alcune persone da un certo punto nel tempo vanno a vivere in aree limitrofe.\\
\section{Rivoluzione Industriale}
La rivoluzione industriale, come già detto in precedenza, è un fattore importante per la nascita dell'urbanizzazione.\\
È stato un fenomeno rivoluzionario che ha cambiato la storia dell'epoca. Se prima un artigiano poteva produrre pochi prodotti al giorno, con le macchine si sono raggiunti livelli di produzione di massa.\\
Massa che possiamo pensare anche come persone che vanno a lavorare per queste nuove fabbriche diventando appunto operai, e proprio queste fabbriche daranno il via a una espansione industriale incredibile, per produrre sempre di più.\\
Con la nascita dell'industrializzazione però nascono anche nuove classi sociali e lotte interne per salvaguardare i diritti degli operai.\\
Questo lato industriale ha luogo anche per il grande sviluppo scientifico dell'epoca. L'ingegneria prende forma così come anche lo sviluppo medico (scoperta dei vaccini) e di altre scienze, questo comporta la nascita di strutture in elevazione e lunghezza, nuovi mezzi di spostamento quali i primi treni e una idea di strutture fognarie atte anche a combattere le malattie con l'aiuto della già citata rivoluzione medica.\\
\chapter{Pianificazione}
Il grande sviluppo industriale, portando un miglioramento generale della vita, attira anche molte più persone verso una città industrializzata. Questo fenomeno di richiesta abitativa porta a una formulazione di piani urbanistici per permettere un collocamento dimensionato alla richiesta.\\
Questo sviluppo possiamo considerarlo in 4 fasi principali : \\
 \begin{itemize}
   \item L'utopia
   \item Il piano
   \item Le tecniche di pianificazione
   \item I modelli urbani e territoriali
 \end{itemize}
\leavevmode\\
Abbiamo una fase di analisi di predisposizione di elemento utopico, quindi risolvere varie problematiche con modalità di tipo utopistico.\\
C'è la modalità tecnica e quindi quella più ingegneristica, ovvero la costruzione del piano, da cui poi deriveranno le tecniche per la costruzione del piano stesso e lo sviluppo di modelli.\\
Modelli urbani o territoriali che potessero essere usati appunto per la costruzione del piano.\\
Questi 4 capisaldi sono quelli su cui si basa lo sviluppo complessivo dell'urbanistica.\\
\section{Utopia}
Il principale responsabile dei processi di urbanizzazione nell'800 è l'industrializzazione. Di solito quest'era industriale è vista in modo molto negativo (Fabbriche senza giornate lavorative, sfruttamento dei dipendenti). Per questo poi nasce il movimento operaio e i sindacata; per difendere i diritti degli operai.\\
L'utopia però esiste nella cerchia degli imprenditori illuminati, infatti anziché impiantare le fabbriche sulla normale idea di fabbrica produttiva cercano di realizzare impianti che siano più sicuri e con un ambiente lavorativo migliore.
\leavevmode\\
 \begin{center}
   \includegraphics[width=0.6\linewidth]{"immagini/Salina de Chaux"}
\\
   \includegraphics[width=0.6\linewidth]{"immagini/Salina de Chaux top"}
   \captionof{figure}{Le saline di Chaux, 1804}
 \end{center}
\leavevmode\\
Questo impianto ideato da Claude-Nicolas Ledoux contiene al suo interno nono solo l'impianto produttivo, ma anche altre attrezzature quali residenze per operai e altri spazi di svago.\\
Tutto ciò in ordire di creare per gli operai un ambiente lavorativo più sereno.\\
 \begin{center}
   \includegraphics[width=0.6\linewidth]{"immagini/Falansterio"}
   \captionof{figure}{Falansterio, inizi del XIX secolo}
 \end{center}
Un altro esempio classico è quello del Falansterio, idea nata da Charles Fourier che ipotizza questo edificio con al suo interno compartimenti per produzione, svago e vita giornaliera dei dipendenti.\\
    Queste strutture utopiche hanno creato realtà molto interessanti come New Lanark in Inghilterra. Fabbrica di tessuti di Robert Owen che come caratteristica principale ha sempre i due lati di produzione e residenza.\\
    Tutto questo serviva anche per ottenere controllo non solo sulla vita produttiva della fabbrica, ma anche su quella degli operai che, come in una bolla, conoscevano solo quella piccola comunità che si veniva a creare tra operai della stessa compagnia.\\
    New Lanark invece è un esempio interessante perché al suo fallimento i dipendenti assunsero i controlli di tutto il sistema e quindi avviarono una gestione cooperativa.\\
    I sistemi quindi assumono due tipi di gestione, quella classica dell'imprenditore e quella cooperativa che nasce dalla comunità operaia.\\
    \section{Il Piano Urbanistico}
    Dalla prima visione utopistica degli imprenditori, come nazione si cerca di trovare un modo per governare lo sviluppo delle città, o meglio regolamentare questi processi senza lasciare tutto in mano agli imprenditori.\\
    Da questa esigenza si afferma il piano urbanistico come strumento di regolazione della espansione delle città.\\
    \begin{center}
    	\includegraphics[width=0.6\linewidth]{"immagini/london 1829"}
    	\captionof{figure}{Piano per Londra, 1829}
    	\label{fig:londra1829}
    \end{center}
    John Claudius Loudon è stato probabilmente l'ideatore del primo piano urbanistico relativo a una città [\ref{fig:londra1829}].\\
    La città di Londra, fino ad allora si era sviluppato senza alcun piano per attuare risposte a problemi quotidiani quali ai tempi potevano essere ad esempio epidemie. Infatti nel periodo dell'800 è stata sì una grossa città, ma che nel tempo ha subito varie problematiche relative alla condizione di vita dei cittadini.\\
    Proprio per questo è stata una città che ha sviluppato per prima vari metodi risolutivi per arginare queste piaghe che affliggevano i popolani.\\
    Nel 1829 si è ideato questo piano urbanistico ad anelli concentrici. Il punto centrale della città è la cattedrale che per un cerco di quasi un miglio forma la prima "fascia" che racchiude il palazzo imperiale e le altri importanti strutture cittadine.\\
    A partire dalla striscia poco più distante dalla cattedrale c'è una zona verde che porta alla seconda fascia urbana in cui vengono realizzate le residenze.\\
    Dopo le residenze abbiamo ancora una zona verde e poi le fabbriche al centro più esterno.\\
    L'obbiettivo di questo piano dunque è che qualunque residenza si allontana non più di mezzo miglio da una fascia verde; avere a disposizione spazi non urbanizzati e aumentare la salubrità della stessa struttura urbana.\\
    \begin{center}
    	\includegraphics[width=0.6\linewidth]{"immagini/carta historica barcelona 1"}
    	\captionof{figure}{Barcellona, 150 a.C.}
    	\label{fig:barcelona1}
    \end{center}
    \leavevmode\\
    Un secondo esempio di piano è quello di Barcellona.\\
    Storicamente è sempre stata una città indipendente e non si è mai unita al regno spagnolo, nel corso del tempo viene conquistata dai romana e ne diventa uno dei loro primi possedimenti.\\
    L'impero sviluppa la città di Barcellona e viene costruita come se fosse appunto una città di fondazione romana; Cargo decumano e 4 quadranti su struttura a scacchiera.\\
    Nel 550 d.C. c'è la crisi dell'impero romano e la città viene conquistata dai visigoti.\\
    \begin{center}
    	\includegraphics[width=0.6\linewidth]{"immagini/carta historica barcelona 2"}
    	\captionof{figure}{Barcellona, 550 d.C.}
    	\label{fig:barcelona2}
    \end{center}
    \leavevmode\\
    Dal 1750 viene conquistata dagli spagnoli e costruiscono una piazzaforte per tenere sotto controllo la città, inoltre impongono il divieto di costruzione all'interno di una fascia di 10-15km intorno alle mura cittadine.\\
    Barcellona quindi diventa densamente popolata all'interno della fascia proprio per questa impossibilità di estendersi al di fuori di essa.\\
    \begin{center}
    	\includegraphics[width=0.6\linewidth]{"immagini/carta historica barcelona 3"}
    	\captionof{figure}{Barcellona, 1750}
    	\label{fig:barcelona3}
    \end{center}
    \leavevmode\\
    Nel 1855 viene eletto un governo progressista e abolisce la normativa anti-espansionista così che Barcellona possa utilizzarla per lo sviluppo.\\
    \begin{center}
    	\includegraphics[width=0.6\linewidth]{"immagini/carta historica barcelona 4"}
    	\captionof{figure}{Barcellona, 1855}
    	\label{fig:barcelona4}
    \end{center}
    \leavevmode\\
    Viene dato l'incarico di ideare un piano urbanistico a Ildefons Cerdà che costruisce una struttura dove il territorio di Barcellona espanso al di fuori delle mura è completamente diverso da quello storico.\\
    La struttura urbana che idealizza Cerdà è a maglia quadrata che viene interrotta da una serie di trasversali.\\
    Il vantaggio di questa struttura urbana è che Barcellona è situata su una pianura che ci favorisce questo tipo di soluzione.\\
    \begin{center}
    	\includegraphics[width=0.6\linewidth]{"immagini/ildefonscerdaspain"}
    	\includegraphics[width=0.6\linewidth]{"immagini/CruillesCerda"}
    	\captionof{figure}{Piano urbanistico di Barcellona ideato da Ildefons Cerdà}
    	\label{fig:cerda}
    \end{center}
    \leavevmode\\
    Barcellona si fonda quindi su una maglia quadrata che ha un modello organizzativo a isolati.\\
    Gli isolati si basano sulla costruzione di un quadrato di determinata dimensione. Con l'insieme di 100 isolati si forma una struttura di quartiere, e ciascun distretto avrà un isolato utilizzato per una qualsiasi tipologia di struttura o adibito al verde.\\
    Ogni distretto quindi sarà isolato e funzionale di per sè.\\
    \begin{center}
    	\includegraphics[width=0.6\linewidth]{"immagini/Barcelona-map-1890"}
    	\captionof{figure}{Mappa di Barcellona, 1890}
    	\label{fig:barcelona-map-1890}
    \end{center}
    \leavevmode\\
    Nell'arco di 50 anni il piano viene impostato e la città viene disegnata sul terreno.\\
    Il centro storico viene preservato mentre l'ambito della piazzaforte viene demolito per fare spazio a un giardino, i quartieri invece al di fuori delle mura iniziano ad essere sviluppati.\\
    \begin{center}
    	\includegraphics[width=0.6\linewidth]{"immagini/barcelona space components"}
    	\captionof{figure}{Isolati di Barcellona}
    	\label{fig:barcelona space components}
    \end{center}
    \leavevmode\\
    Quando Cerdà fece una prima collocazione di isolati, questi erano molto ampi perché al tempo ovviamente la popolazione di Barcellona era molto meno densa del giorno d'oggi.\\
    Troviamo nel primo piano di Cerdà una superficie edificabile pari al 28\%; una superficie viaria del 30\% e una superficie a verde del 42\%.\\
    Queste percentuali adesso sono completamente state stravolte, ma la disposizione delle aree è comunque rimasta molto simile.\\
    La prima misura su cui si basa il piano di Cerdà è la larghezza della strada, pari a 10m con 5m di marciapiede ad ambo i lati.\\
    \begin{center}
    	\includegraphics[width=0.6\linewidth]{"immagini/incrocio ottogonale"}
    	\captionof{figure}{Esempio di incrocio ottogonale}
    	\label{fig:incrocio ottogonale}
    \end{center}
    \leavevmode\\
    La seconda misura ideata è quella relativa all'incrocio tra 4 isolati, che non crea una croce ma bensì un ottagono. Da qui si ricava l'apotema dell'ottagono, ideato pari a 24m con la formula $B=A/2*(1+\sqrt{2})$\\
    Da queste due misure Cerdà ricava poi tutte le restanti, quali l'altezza degli edifici (20m); la profondità edificabile (24m); la larghezza del cortile (68m) e la dimensione dell'isolato (116m).\\
    Nel tempo, col crescente numero della popolazione a Barcellona, Cerdà introduce dei blocchi ad "U" che possono essere visti nella seconda fascia della figura [\ref{fig:barcelona space components}].\\
    Questo sistema aumenta del 33\% le volumetrie. Questi isolati poi vengono disposti in modo da avere strade con negozi o abitazioni ad ambo i lati, e strade più isolate dalla vita cittadina con molta presenza di aree verdi.\\
    Sorgono anche ordinanze comunali che incrementano di volta in volta la potenzialità di occupazione all'interno degli isolati.\\
    Con l'ordinanza edile del 1859-1889 si ha un aumento dell'occupazione massima al 50\% e la possibilità di costruire 4 piani (compreso il piano terra).\\
    Tra il 1890 e il 1932 vengono emanate nuove ordinanze comunali e l'occupazione massima passa al 75\% dell'isolato. L'altezza è pari al piano terra + 7 piani + attico e nello spazio al centro è possibile creare un edificio interno di 5m di altezza edito a magazzino o altri servizi.\\
    Altre ordinanze ci furono tra il 1933 e il 1975 con 73\% di occupazione massima, edificio interno di 5m + interrato ed edifici interni con interrati; piano terra; 7 piani; attico e superattico.\\
    A partire dal '76 invece si è cercato di ridurre tutte le specifiche degli isolati perché si raggiunse una densità edificata troppo grande. L'occupazione massima degli isolati scese al 70\% e si tolse la possibilità di avere attici e superattici. Gli interrati invece ebbero un cambiamento in positivo riguardo lo spazio occupato, infatti ci fu la possibilità di costruire doppi piani interrati.\\
    \begin{center}
    	\includegraphics[width=1.0\linewidth]{"immagini/isolati speciali barcellona"}
    	\captionof{figure}{Isolati speciali}
    	\label{fig:isolati speciali barcellona}
    \end{center}
    \leavevmode\\
    Cerdà ha anche ideato isolati con specifici casi d'uso, qui sotto la legenda relativa alla figura [\ref{fig:isolati speciali barcellona}]\\
    \begin{description}
    	\item[A.    ] \phantom{a|}Isolato doppio (Università, impianti industriali)
    	\item[B.    ] \phantom{a|}Edificio a croce di sant'Andrea
    	\item[C.    ] \phantom{a|}Decomposizione con paesaggi
    	\item[D.    ] \phantom{a|}Edifici nella banda centrale
    	\item[E, F.]  \phantom{..}Edificio su mezzo isolato
    	\item[G.   ] \phantom{a|}Chaflan (Edifici smussati con accesso su incrocio)
    \end{description}
    \section{Le Tecniche di Pianificazione}
    Il piano di Cerdà diventa anche l'occasione per l'approfondimento sulle tecniche di pianificazione.\\
    Si iniziano a scrivere manuali su questa nuova materia che è l'urbanistica, e approfondimenti sulle strutture urbane e la loro modalità di costruzione.\\
    Cerdà scrive nel 1876 un libro dal titolo "teoria general de la urbanizacion", che è la trasformazione del piano di Barcellona in teoria.\\ \\
    In questo libro Cerdà scrive : 
    \begin{displayquote}
    	Questi sono i motivi filologici che mi hanno indotto ad adottare il termine urbanizzazione. Tale termine indica l’insieme degli atti che tendono a creare un raggruppamento di costruzioni e a regolarizzare il loro funzionamento, così come designa l’insieme dei principi, dottrine e regole che si devono applicare perché le costruzioni e il loro raggruppamento, invece di reprimere, indebolire e corrompere le facoltà fisiche, morali e intellettuali dell’uomo che vive in una società, contribuiscano a favorire il suo sviluppo e ad accrescere il benessere sia individuale che pubblico.
   \end{displayquote}
   \leavevmode\\
   Questa frase ci fa capire il ruolo che ha l'urbanistica nella nostra vita e la propria importanza.\\
   Ci sono anche vari altri manuali scritti nel corso del tempo come ad esempio \textit{Stadt-Erweiterungen in Technischer Baupolizeilicher Und Wirthschaftlicher Beziehung} (1867) e \textit{Der Stadtebau} (1889).\\
   Come struttura urbana abbiamo già visto ad esempio la maglia quadratica di Cerdà, ce ne sono ovviamente anche altre come l'ipotizzazione di città lineari, ovvero città costruite lungo un asse centrale.\\
   Questa idea di Arturo Soria y Mata vedeva questo asse centrale organizzato con una strada passante proprio per quest'ultimo e poi degli spazi laterali a verde. Due corsie per il passaggio di tram elettrici e dei quartieri residenziali a destra e sinistra.\\
   \begin{center}
   	\includegraphics[width=0.8\linewidth]{"immagini/ciudad-lineal"}
   	\captionof{figure}{Ciudad Lineal, 1883}
   	\label{fig:Ciudad Lineal}
   \end{center}
   \leavevmode\\
   Questo modello non è stato molto utilizzato perché ha vari tipi di svantaggi, ad esempio l'asse centrale per città molto grandi presenta intasamenti considerevoli per la circolazione della mobilità.\\ \\
   Un altro modello di interesse è quello delle garden cities, messo appunto da Ebenezer Howard.\\
   L'ipotesi di Howard è costruire una città con un numero di abitanti stabilito (32000, di cui 2000 destinati alle aree agricole), inoltre con una superficie di 2400 ettari; una aerea agricola di 2000 ettari e una area urbana di 400 ettari.\\
   \begin{center}
   	\includegraphics[width=0.8\linewidth]{"immagini/garden cities"}
   	\captionof{figure}{Modello delle Garden Cities}
   	\label{fig:garden cities}
   \end{center}
   \leavevmode\\
  \chapter{Il Fenomeno Urbano in Italia}
   Quello che abbiamo visto essere successo in paesi come Inghilterra e Spagna è successo anche per l'Italia. Partiamo da una situazione storica completamente diversa, infatti nel 1860 non esiste uno stato unitario.\\
   Dopo la caduta dell'impero romano infatti l'Italia è sempre stata suddivisa in stati; repubbliche; principati; regni di dominazione straniera.\\
   Questo ha condizionato il processo di sviluppo economico, che si concentra in particolare nei territori del nord Italia. Il Sud Italia a quel tempo è governato dal regno delle due Sicilie e il centro dalla chiesa e vari ducati.\\
   \begin{center}
   	\begin{minipage}{0.35\linewidth}
   		\includegraphics[width=\linewidth]{"Immagini/italia_1820-1848"}
   		\captionof{figure}{Situazione geopolitica dell'Italia tra il 1820 e il 1848}
   		\label{fig:italia_1820-1848}
   	\end{minipage}%
   	\hfill
   	\begin{minipage}{0.46\linewidth}
   		\includegraphics[width=\linewidth]{"Immagini/Italia_ferrovie_1861.03.17"}
   		\captionof{figure}{Reti ferroviarie in Italia nel 1861}
   		\label{fig:Italia_ferrovie_1861.03.17}
   	\end{minipage}
   \end{center}
   \leavevmode\\
  Queste due immagini sono importanti perché vediamo che ogni Stato costruisce la propria rete ferroviaria nel 1861 e spesso non c'erano connessioni tra di loro.\\
  Gli unici due Stati connessi erano il regno di Sardegna e il regno Lombardo-Veneto, con alcuni prolungamenti fino a Bologna.\\
  La velocità delle comunicazioni date dalle ferrovie quindi in Italia non esisteva proprio. L'unificazione avviene dal 1861 che poi porterà al verificarsi di una serie di problematiche che si protrarranno nei decenni successivi.\\
  \section{Napoli}
  Napoli subì lo stesso trattamento di Barcellona. Per molti anni la città fu costretta a restare confinata nelle proprie mura.\\
  A quell'epoca poi Napoli era una delle città più popolose del mondo, ed il fatto di avere un così limitato territorio con scarse dotazioni infrastrutturali come reti idriche e fognarie ha provocato nel corso dell'800 una serie di epidemie.\\
  Nel 1884 ad esempio ci fu un epidemia di colera che portò a un numero di 5000 morti tra i cittadini napoletani. Questa epidemia diventa il movente di un piano di risanamento molto importante, dettato addirittura da leggi nazionali.\\
  La prima direttrice è il risanamento della zona relativa a corso Umberto, che era la zona più bassa della città in cui arrivavano gli scarichi. Per questo fu una delle zone più colpite dal colera.\\
  Altra zona di interesse fu quella di Satn'Antonio che arrivava fino all'albergo dei poveri.\\
  La seconda tipologia di interventi che viene messa in campo è quella di sviluppo o espansione urbana.\\
  \begin{center}
  	\includegraphics[width=0.6\linewidth]{"immagini/città di napoli 1885"}
  	\captionof{figure}{Città di napoli, 1885}
  	\label{fig:città di napoli 1885}
  \end{center}
  \leavevmode\\
  Queste zone disegnate in arancione sono proprio le zone che subirono l'espansione della pianificazione urbanistica sulla base del risanamento territoriale.\\
  \begin{center}
  	\includegraphics[width=0.6\linewidth]{"immagini/piano regolatore napoli 1885"}
  	\captionof{figure}{Piano regolatore di Napoli, 1885}
  	\label{fig:piano regolatore napoli 1885}
  \end{center}
  \leavevmode\\
  Il piano di risanamento ebbe luogo dalla costruzione di un nuovo tessuto urbano impostato su quello precedente. A sinistra vediamo la zona dell'albergo dei poveri e a destra la zona di piazza della Borsa.\\
  In rosso vediamo il vecchio tessuto urbanistico, che si evince essere molto stretto, chiuso e con poca illuminazione.\\
  Questo piano messo in piedi nel 1885 dovette far conto a finanziamenti sia privati che pubblici. Il comune di Napoli espropriava suoli del vecchio edificato che poi veniva usato dalle agenzie edilizie per costruire la nuova rete urbana.\\
  Dopo 15 anni, corso Umberto I non era ancora del tutto costruito, questo per colpa appunto di ritardi dovuti ai fondi.\\
  Nel corso del 900 fu necessario regolarizzare la quota del suolo, perché alcuni tratti nella conformazione precedente del suolo non erano regolari e dovettero essere rastremati per permettere l'edificazione delle nuove strade della rete urbana.\\
  Tra le ultime realizzazioni di questo piano ci fu piazza Matteotti.\\
  Nel 1939 si decide di impostare un piano regolatore generale che partiva su due presupposti : \\
   \begin{description}
  	\item[1.] Inglobamento di centri di periferia esterni al comune di Napoli \\(Es. San Giovanni Barra)
  	\item[2.] Serie di sviluppi in tutte le zone e quartieri all'interno di una maglia regolare circondata da parchi e terreni agricoli.
  \end{description}
  \section{Firenze}
  La città di Firenze, come Napoli, è capitale di un gran ducato che è l'erede dello stato fondato dalla famiglia  dei Medici.\\ La cosa importante è che il periodo d'oro di Firenze finisce quando entra nel regno d'Italia nel 1860.\\ Finito questo periodo d'oro la città si è fermata e mai più ebbe uno sviluppo tanto forte.\\
  La capitale del regno d'Italia, essendo situata a Torino è in una posizione decentrale rispetto a tutta l'area regnante. Si decide dunque di spostare la capitale a Firenze nel 1865.\\
  Qui si sposta quindi ogni sorta di luogo ministeriale (uffici, ministeri, consolati ecc...). Tutto questo porta a un raddoppiamento della popolazione per questa città.\\ Nel 1865 viene dunque impostato un piano, chiamato Piano Poggi. Viene impostata una maglia regolare che circonda il sistema urbano esistente, raddoppiando il numero di abitazioni e servizi.\\ Questo avviene soprattutto nella parte Est e nel lato occidentale.\\
  \begin{center}
  	\includegraphics[width=0.6\linewidth]{"immagini/Piano_Poggi_(Firenze,_1865)_-_1"}
  	\captionof{figure}{Piano regolatore di Firenze, 1865}
  	\label{fig:Piano Poggi (Firenze, 1865)}
  \end{center}
  \leavevmode\\ 
  Nel 1877 Firenze viene dotata di una nuova linea ferroviaria che collega la città sia col Nord che col Sud, e trova sicuramente uno sviluppo caotico nell'arco di questo periodo.\\ In tutto questo caos però troviamo l'attuazione di parchi magnifici come l'esempio di Piazzale Michelangelo.\\ Tutto ciò cambia anche la paesaggistica dell'intera città, infatti fino ad allora non c'erano luoghi dove poter vedere Firenze se non dal suo interno.\\
  Nel 1870 però la città di Firenze viene da un giorno all'altro a perdere il suo status di capitale. Questo perché essendo stata catturata Roma si è subito fatta capitale del regno d'Italia.\\
  Dopo questo avvenimento non era più necessaria a Firenze tutta quella volumetria, quindi dopo il 1870 la città di Firenze cade in una crisi finanziaria non da poco.\\ La quasi totalità di questi avanzamenti territoriali ed edilizi infatti era per la maggior parte finanziata da privati.
  \begin{center}
  	\includegraphics[width=0.6\linewidth]{"immagini/firenze dopo la caduta della capitale"}
  	\captionof{figure}{firenze dopo la caduta della capitale}
  	\label{fig:firenze dopo la caduta della capitale}
  \end{center}
  \leavevmode\\ 
  Da questa immagine infatti possiamo vedere come proprio dei lotti di terreno alle estremità di quella che allora era la città di Firenze siano vuoti al loro interno.\\  Per circa 20-25 anni Firenze subirà quindi una crisi economica molto forte, che poi comunque riusciranno a riprendersi con una nuova fase di sviluppo della città con insediamenti di attività e imprese.\\
  \begin{center}
  	\includegraphics[width=0.6\linewidth]{"immagini/risanamento mercato vecchio"}
  	\captionof{figure}{Risanamento della zona del Mercato Vecchio tra piazza del Duomo e piazza della Signoria}
  	\label{fig:risanamento mercato vecchio}
  \end{center}
  \leavevmode\\
  L'operazione più importante fatta a Firenze è il risanamento del Mercato Vecchio.\\ L'edilizia venne demolita, tranne alcuni edifici di storiche famiglie fiorentine, e venne costruito al posto di questi edifici un nuovo sistema di isolati basati su maglia quadrata.\\ Piazza della Repubblica divenne il centro ottocentesco della comunità borghese fiorentina.\\
  \begin{center}
  	\begin{minipage}{0.40\linewidth}
  		\includegraphics[width=\linewidth]{"Immagini/colonna dell'abbondanza prima"}
  		\label{fig:colonna dell'abbondanza prima}
  	\end{minipage}%
  	\hfill
  	\begin{minipage}{0.42\linewidth}
  		\includegraphics[width=\linewidth]{"Immagini/colonna dell'abbondanza dopo"}
  		\label{fig:colonna dell'abbondanza dopo}
  	\end{minipage}
  \captionof{figure}{Colonna dell'abbondanza prima e dopo} 
  \end{center}
  \leavevmode\\
  Gli edifici vengono demoliti e sorge piazza della Repubblica con la colonna dell'abbondanza al centro.\\
  \section{Milano}
  Anche Milano come le altre città ha subito forti trasformazioni nel corso dell'800.\\ In più Milano diventa capitale economica non solo del precedente regno Asburgico ma poi successivamente di tutto il regno d'Italia.\\
  Piazza Duomo era una zona isolata con una serie di edifici intorno, oltretutto non era per niente regolare; alla destra del Duomo c'era il palazzo reale e a sinistra edifici che continuavano fino a Piazza Marina.\\
  Si decide dopo l'unità di intervenire su quest'area e di creare il nuovo salotto borghese che ha come quinta la facciata del duomo di Milano.\\
  Vengono fatti degli interventi quindi su Piazza del Duomo e l'isolamento dello stesso sulla sua parte retrostante, in più si erano predisposti interventi su tutta Piazza Marina.\\ Tra gli edifici nella zona a destra del Duomo viene collocata la Galleria che collega tra loro 4 isolati.\\
  Molte città nell'800 hanno subito questo fato, ovvero buttare giù edifici e stravolgere i piani già presenti per costruire aree che dessero lustro alle nuove zone urbane.\\
  \begin{center}
  	\includegraphics[width=0.6\linewidth]{"immagini/milano piantina"}
  	\captionof{figure}{piantina di Milano con comuni di espansione}
  	\label{fig:milano piantina}
  \end{center}
  \leavevmode\\
  Milano, che vediamo nella figura soprastante, subì una espansione del territorio molto vasta che ha portato a un inglobamento di vari comuni vicini proprio come successe a Napoli.\\
  Nel 1884 viene impostato il primo piano regolatore della città (Piano Beruto) che trova con la sua attuazione la demolizione dei bastioni della Milano storica per permettere l'espansione verso i comuni.\\
  \begin{center}
  	\includegraphics[width=0.6\linewidth]{"immagini/Milano_-_Piano_Beruto_(bozza)"}
  	\captionof{figure}{città di Milano vista dal Piano Beruto}
  	\label{fig:Milano Piano Beruto}
  \end{center}
  \leavevmode\\
  Vediamo da questo piano che la prima stazione ferroviaria di Milano è passante, ovvero la rete entra da una parte ed esce dall'altra. I treni quindi arrivano alla stazione, si fermano e ripartono.\\
  Questo crea una serie di problematiche in quanto ci sono due pezzi di città separati da questa rete ferroviaria.\\
  Nel 1920 le industrie milanesi beneficiano della prima guerra mondiale in quanto venivano commissionate dall'esercito varie tipologie di materiali o servizi.\\ Per questi prodotti si creano anche apposite aree espositive, e quindi Piazza d'Armi viene riutilizzata come Fiera.\\
  Le piazze d'Armi servivano all'epoca per far esercitare i militari nelle proprie mansioni.\\
  La fiera di Milano ogni anno teneva la fiera campionaria che era una delle più importanti in Europa. Questo comporta il fatto che l'ambito centrale di Piazza d'Armi diventa una posizione sfavorevole.\\
  Ogni qualvolta che c'è una fiera, essendo così centrale si creano ingorghi per il grande pubblico, smog ecc.\\
  Nel 2000 si decide di trasferire la fiera di Milano a Nord-Est, che poi avrà ampliamento come con la zona dell'Expo.\\
  La stazione ferroviaria di Milano subisce grandi cambiamenti a partire dal 1912, quindi smette di essere stazione passante e diventa stazione di testa.\\
  Questo significa che le reti arrivano nella stazione e per uscirne devono ritornare proprio in questa stazione.\\
  I 2 elementi di sviluppo (Reti e Piazze) rientrano nel nuovo piano impostato per la città di Milano che è Piano Masera. Viene impiantata una rete più esterna e costruita una maglia regolare con edifici da 4 o 6 piani; tipologie diversificate con uso della schiera a due piani; tessuti chiusi con edifici allineati a filo strada e un disegno dello spazio pubblico (piani e strade).\\
  \begin{center}
  	\includegraphics[width=0.6\linewidth]{"immagini/espansione milano 3"}
  	\captionof{figure}{seconda espansione della città di Milano verso i comuni limitrofi}
  	\label{fig:espansione milano 3}
  \end{center}
  \leavevmode\\
  Con questa nuova espansione territoriale c'è bisogno di un nuovo piano che verrà ideato da Albertini nel 1934.\\
  Per quanto riguarda la parte centrale c'è stata la regolarizzazione di Piazza Duomo, e nel corso del fascismo invece vengono fatti interventi sulla parte destra. Qui vengono demoliti tutti gli edifici alle spalle del palazzo reale e viene creata Piazza San Babila.\\
  Gli unici edifici che restano sono quelli storici, come chiese presenti nella precedente struttura urbana.\\
  \section{Roma}
  Il caso di Roma, come tutti gli altri presenta delle singolarità. 
  \begin{center}
  	\includegraphics[width=0.6\linewidth]{"immagini/Nuova-pianta-di-Roma-by-GB-Nolli-1748"}
  	\captionof{figure}{Carta di Roma, Nolli, 1748}
  	\label{fig:roma cartina}
  \end{center}
  \leavevmode\\
  Riconosciamo il Colosseo, piazza del popolo, piazza navona e altri quartieri del trastevere. Se notiamo bene c'è una linea nera che rappresenta le mura Aureliane.\\
  Quando Roma era a capo dell'impero quindi, si estendeva fino a quelle mura se non oltre. Quello che successe con la caduta dell'impero fu un rattrappimento della città fino al suo centro.\\
  All'interno delle mura quasi il 70\% del territorio era a verde. Il Colosseo infatti è circondato dal nulla.\\
  \begin{center}
  	\includegraphics[width=0.6\linewidth]{"immagini/piano regolare roma 1873"}
  	\captionof{figure}{Piano regolare di Roma, 1873}
  	\label{fig:piano regolare roma 1873}
  \end{center}
  \leavevmode\\
  Quando Roma divenne capitale, ci fu ancora il bisogno di trasferire tutti gli uffici del Re da Firenze. Quindi per trovare un posto ai dipendenti, alle strutture eccetera si dette luogo a un piano regolare edito da Paolo Viviani di espansione territoriale.\\
  I tratti neri sono gli interventi di risanazione all'interno del territorio o tessuto storico già presente, mentre quelli rossi sono zone di espansione.\\
  Nel 1909 c'è un altro piano, ovvero quello Sanjust che non si sviluppa più a maglia quadrata, ma per alcune parti del territorio lo fa con una rete stradale organica.\\
  Villa Ada ad esempio è un grande parco annesso a una propietà principesca che verrà lottizzata per poi essere venduta dai ricchi borghesi dell'epoca.\\\\
  \begin{center}
  	\includegraphics[width=0.6\linewidth]{"immagini/Demolitions-in-Via-dei-Fori-Imperiali-Rome-1932"}
  	\captionof{figure}{Demolizione in vista dei Fori Imperiali, Roma, 1932}
  	\label{fig:Demolitions-in-Via-dei-Fori-Imperiali-Rome-1932}
  \end{center}
  \leavevmode\\
  Altro piano di espansione c'è in piena epoca fascista, esso comprende interventi di risanamento urbano molto forti.\\
  Il primo è quello dei Fori Imperiali del 1924. L'obbiettivo è formare una strada che colleghi Piazza Venezia con il Colosseo dove marceranno i gloriosi soldati fascisti.\\
  Per realizzare tutta questa operazione però vengono demoliti quartieri popolari e demoliti moltissimi resti della civiltà romana (Repubblicana e Imperiale).\\
  La stessa cosa successe attorno al Colosseo come l'isolamento dell'arco di Costantino che prima era invece circondato da edifici e monumenti.\\\\
  \begin{center}
  	\includegraphics[width=0.6\linewidth]{"immagini/via della conciliazione"}
  	\captionof{figure}{Via della Conciliazione, Roma}
  	\label{fig:via della conciliazione}
  \end{center}
  \leavevmode\\
  Il secondo intervento fu quello che rese possibile la realizzazione di Via della Conciliazione, e quindi il collegamento di Piazza San Pietro con Castel Sant'Angelo.\\
  Questo venne fatto dopo la firma dei patti Lateranensi e quindi la pace firmata da Mussolini tra il regno d'Italia e il Vaticano. Questo anche per ingraziarsi ancora di più le gerarchie vaticane.\\
  Roma nel 1938 ebbe l'incarico di inaugurare l'esposizione Universale del 1942, venne fatto il progetto per un nuovo quartiere che collegasse la città di Roma al mare.\\
  L'esposizione ovviamente non venne più fatta a causa della guerra, così come anche il quartiere non venne mai realizzato.\\
  \begin{center}
  	\includegraphics[width=0.6\linewidth]{"immagini/esposizione universale di roma"}
  	\captionof{figure}{Progetto per l'esposizione universale di Roma}
  	\label{fig:esposizione universale di roma}
  \end{center}
  \leavevmode\\   
  \section{Fabbrica e città}
  Abbiamo già visto che una delle spinte verso la costruzione della pianificazione e dell'urbanistica è stata quella proveniente da un gruppo di in imprenditori illuminati che hanno creato impianti industriali molto attenti alla qualità della vita per operai e impiegati. Impianti quindi circondati dal verde, dai servizi e da abitazioni.\\
  Crispi D'Adda è un villaggio ottocentesco messo insieme proprio dalla famiglia Crispi. Questi imprenditori costruirono 
  terreno sia la fabbrica che il villaggio per gli operai.\\ 
  Si ispiravano ad una dottrina di carattere religioso, e per ogni dipendente fu costruita una villetta con giardino che creavano una comunità con ogni sorta di servizi (ospedali, scuole, ecc).\\
  Un secondo esempio di villaggio operaio è quello di Nuova Schio, una città del veneto.\\ Abbiamo a che fare con un lanificio realizzato con un disegno di tipo organico fatto di villette e strade circolari.\\
  Gli edifici vennero realizzati con tecniche innovative così come gli impianti. Usavano una nuova materia che era il cemento armato.\\
  Un terzo esempio è quello piemontese del Villaggio Leumann, che serviva da fabbrica meccanica.\\
  \chapter{Modelli Urbani e Territoriali}
  Un modello è una rappresentazione di un fenomeno reale, a cui viene associato il termine "matematico".\\
  A questo modello vengono assegnate specifiche equazioni che descrivono in modo semplificativo i fenomeni, o le relazioni di un problema.\\
  Quindi un modello è una rappresentazione ideale semplificata della realtà.\\
  I modelli urbani sono di due tipi : \\
    \begin{multicols}{2}
  	  \begin{description}
  		\item [1.] Modelli Morfologici
  		\item [  -] Si incentrano sulle forme fisiche con le quali si costruisce e si evolve la città. 
  		\item [  -] Definiscono la tipologia e l'organizzazione geometrica dei pieni e dei vuoti.
  		\item [  -] Descrivono l'evoluzione fisica di una città
  		\item [  -] Possono essere riproposti nel tempo.
  		\item [2.] Modelli Teorici
  		\item [   -] Ipotizzano e spiegano le traiettorie di evoluzione di un sistema urbano. 
  		\item [   -] Si basano sull'analisi delle forze che guidano lo sviluppo delle città (forze economiche, sociali, politiche, ambientali, teologiche, storiche)
  		\item [   -] I modelli teorici non si occupano della forma della forma geometrica e fisica della città.
  	  \end{description}
    \end{multicols}
  \leavevmode\\
  \section{Modelli Morfologici}
  I modelli morfologici che vedremo sono di due tipi :
  \begin{description}
  	\item [1.] Modelli a \textbf{maglia spontanea}.
  	\item [2.] Modelli a \textbf{maglia pianificata}.
  \end{description}
  La maglia spontanea è quella tipica delle costruzioni mediterranee senza nessun tipo di ordine predefinito. Le maglie pianificate invece sono costruite sulla base di un disegno.\\
  Una città può essere caratterizzata dalla presenza di più maglie urbane (spontanee o pianificate).\\
  La forma della città è influenzata dalla morfologia del territorio, quindi se operiamo su un territorio pianeggiante, collinare o altro.\\
  Altra grande influenza sono le reti di comunicazione, come ad esempio la ferrovia nella città di Milano e tutte le sue evoluzioni.\\
  I modelli a maglia spontanea sono fondamentalmente 7, che riconosciamo con questi nominativi : \\
  \begin{multicols}{2}
   \begin{description}
  	\item [1.] A scacchiera
  	\item [2.] A diagonali
  	\item [3.] Radiale
  	\item [4.] A ventaglio
  	\item [5.] Lineare
  	\item [6.] A ramificazione
  	\item [7.] Altri modelli locali
   \end{description}
  \end{multicols}
  \leavevmode\\
  Le città a maglia spontanea si sono sviluppate nel corso dei secoli, soprattutto nell'area mediterranea. Le loro caratteristiche vengono date dalla forma del territorio e dagli eventi storici che le hanno caratterizzate.\\
  Ad esempio le città della Mesopotamia sono a maglia spontanea, ma anche moltissime città Italiane.\\
  Centri che sono caratterizzati da mura, o da influenze grandi come quelle clericali.\\
   \begin{center}
  	\includegraphics[width=0.6\linewidth]{"immagini/bari_vecchia"}
  	\captionof{figure}{Mappa relativa a Bari Vecchia}
  	\label{fig:bari_vecchia}
  \end{center}
  \leavevmode\\
  Notiamo infatti che Bari Vecchia ha una maglia molto irregolare, con edifici accalcati tutt'intorno, stradine molto strette e strutture molto facili da difendere in caso di attacchi esterni.\\
  Il modello a scacchiera è invece una struttura tipica delle città di fondazione, e deriva spesso da territori di conquista.\\ I lotti vengono suddivisi in modo da formare una scacchiera.\\
  È un modello che ha la massima efficacia sui suoli pianeggianti, e favorisce le percorrenze diagonali.\\
  Crea strutture urbane che possono diventare monotone a larga scala e non individua un nodo centrale a meno che non venga costruito appositamente.\\
  \begin{center}
  	\begin{minipage}{0.40\linewidth}
  		\includegraphics[width=\linewidth]{"Immagini/Amarna-WorkersVillage-U-90"}
  		\label{fig:Amarna-WorkersVillage-U-90}
  	\end{minipage}%
  	\hfill
  	\begin{minipage}{0.42\linewidth}
  		\includegraphics[width=\linewidth]{"Immagini/maglia rettangolare"}
  		\label{fig:maglia rettangolare}
  	\end{minipage}
  	\captionof{figure}{Maglie rettangolari, a sinistra Tel El Amarna, XIV sec. a.C.} 
  \end{center}
  \leavevmode\\
  La maglia a scacchiera è stata utilizzata anche per alcune città negli anni '60 e questo era il tipo di strutture.\\
    \begin{center}
  	\includegraphics[width=0.5\linewidth]{"immagini/La-citta-triangolare-Il-piano-de-lEnfant-per-Washington-1791"}
  	\captionof{figure}{Esempio di modello urbano a diagonale, Piano de l'Enfant, Washington 1791}
  	\label{fig:La-citta-triangolare-Il-piano-de-lEnfant-per-Washington-1791}
  \end{center}
  \leavevmode\\
  Il modello a giagonale è il secondo modello pianificato. È una maglia quadrata con la presenza di alcune diagonali, quindi di sua derivazione.\\ Vi è un uso meno ottimale del terreno per una presenza di lotti triangolari e punti di incrocio molto complessi.\\ Le diagonali possono essere uno sviluppo successivo alla quadratica e possono creare strutture a stella come il Place de l'Etoile a Parigi.\\
   \begin{center}
  	\includegraphics[width=0.6\linewidth]{"immagini/palmanova_mappa"}
  	\captionof{figure}{Esempio di maglia radiale, Palmanova, Friuli}
  	\label{fig:palmanova_mappa}
  \end{center}
  \leavevmode\\
  Il modello radiale è quello più perfetto dal punto di vista architettonico e filosofico.\\ Vediamo dalla città di Palmanova un centro ben definito con una raggiera tutt'intorno e delle mura da scudo.\\ È ottimo per le piccole città, ma con una grande area territoriale può risultare di attuativa troppo complessa.\\
  Le strade radiali e quelle anulari possono essere divise per funzione e gerarchia e in genere si definisce un polo centrale immediatamente riconoscibile.\\
  Altre città a maglie radiali sono ad esempio Grammichele in Sicilia, fondata nel 1693 dopo un terremoto.\\
  \begin{center}
  	\includegraphics[width=0.6\linewidth]{"immagini/karlsruhe 1739"}
  	\captionof{figure}{Esempio di maglia a ventaglio, karlsruhe, 1739}
  	\label{fig:karlsruhe 1739}
  \end{center}
  \leavevmode\\
  Il modello a ventaglio è un modello radiale parziale, il fulcro può essere un elemento architettonico come nel caso di Karlsruhe, o ad esempio un ponte come nella località di Dresda.\\
  Anch'esso è stato utilizzato negli anni '60, per esempio Copenaghen in quegli anni era a modello di ventaglio.\\
  \begin{center}
  	\includegraphics[width=0.5\linewidth]{"immagini/madrid sorya y mata"}
  	\captionof{figure}{Esempio di maglia lineare, Ciudad Lineal, Madrid}
  	\label{fig:madrid sorya y mata}
  \end{center}
  \leavevmode\\
  Abbiamo visto questo tipo di modello nella Ciudad Lineal di Sorya y Mata [\ref{fig:Ciudad Lineal}]. La struttura si basa su un asse centrale e residenze e servizi ai lati, percorsi longitudinali molto complessi a seconda della dimensione e percorsi trasversali semplici.\\
  \begin{center}
  	\includegraphics[width=0.6\linewidth]{"immagini/modello a ramificazione"}
  	\captionof{figure}{Esempio di modello a ramificazione}
  	\label{fig:modello a ramificazione}
  \end{center}
  \leavevmode\\
  Il modello a ramificazione ha un asse di percorrenza urbano al partire dal quale si crea una derivazione e lungo il quale ci sono strutture urbane.\\ Sono i modelli che caratterizzano i parchi urbani oppure le gated communities.\\ Sono tipologie urbane molto sviluppate, ed appunto sono nate negli ultimi decenni.\\
  \begin{center}
  	\includegraphics[width=0.6\linewidth]{"immagini/dubai mappa"}
  	\captionof{figure}{Esempio di modello locale, Dubai}
  	\label{fig:dubai mappa}
  \end{center}
  \leavevmode\\
  Sono modelli particolari che possono essere trovati solo dove sono stati ideati e realizzati, irripetibili. Ad esempio le due palme a Dubai che sono state create sul mare a formare nuovi suoli edificabili.\\
  \section{Modelli Teorici}
  Questi modelli si occupano dei comportamenti che hanno generato la struttura urbana, e allo stesso modo quelli che hanno generato le strutture territoriali.\\
  I comportamenti quindi derivano da elementi che li generano, che noi chiamiamo forze.\\
  Queste forze possono essere di diverso tipo : \\
  Le forze economiche possono essere tra quelle più importanti, ad esempio se una comunità ha molta forza economica può svilupparsi meglio in un luogo godendo di materie prime o suoli con molta risposta agricola.\\
  Possono esistere poi forze politiche, sociali, tecnologiche e molte altre.\\
  I modelli teorici individuano delle traiettorie evolutive nel sistema urbano, e ne ipotizzano altre per il futuro. Individuano ciò che è successo nel passato e a partire da quei dati ipotizzano un riscontro futuro.\\
  \chapter{Le Città Nel Tempo}
  \section{Città Pre-Industriali}
  \begin{center}
  	\includegraphics[width=0.6\linewidth]{"immagini/modelli urbani teorici"}
  	\captionof{figure}{Modelli urbani teorici}
  	\label{fig:modelli urbani teorici}
  \end{center}
  \leavevmode\\
  Nel passato, fino al periodo dell'industrializzazione la città poteva essere in maniera molto precisa per quasi ogni sistema territoriale.\\ 
  Quindi ogni città era quasi sempre strutturata allo stesso modo, con un centro in cui c'erano gli edifici religiosi o amministrativi e una prima fascia con edifici per ceti più nobili. Verso l'esterno erano situati spazi per i ceti più poveri o anche mura protettive.\\
  \begin{center}
  	\includegraphics[width=0.6\linewidth]{"immagini/città settoriale"}
  	\captionof{figure}{Modello di città settoriale}
  	\label{fig:città settoriale}
  \end{center}
  \leavevmode\\
  Con l'industrializzazione nasce quella che possiamo chiamare città settoriale.\\
  Settoriale perché quella che è stata la struttura delle città per secoli viene spezzata, si creano settori in cui sono presenti attività e settori specifici. In particolare nascono le zone industriali dove vengono situate le fabbriche.\\
  Queste necessitano spazio e a volte per averne vengono abbattute le precedenti mura o si viene meno alle vecchie forme compatte cittadine.\\
  Ovviamente le industrie hanno bisogno di operai, quindi le fasce dove sono posizionati i ceti popolari tendono ad estendersi in modo maggiore rispetto alle altre. Si estendono in periferia a contatto con quelle industriali.\\
  I ceti benestanti restano all'interno della città, ma spesso tendono ad uscire dalla città stessa e al contempo lontano dalle zone industriali. Questo fa si che i centri urbani si spopolino e diventino spazi marginali.\\
  \begin{center}
  	\includegraphics[width=0.6\linewidth]{"immagini/La città polarizzata"}
  	\captionof{figure}{Modello di città polarizzata}
  	\label{fig:La città polarizzata}
  \end{center}
  \leavevmode\\
  L'evoluzione della città comporta un incremento notevole della popolazione che abita in strutture urbane. Questo avviene in particolar modo dopo la seconda guerra mondiale.\\
  Le aree produttive diventano sempre più rilevanti e quindi c'è bisogno di più operai, addetti ai lavori ecc.\\
  Si crea quindi un fenomeno di marginalizzazione dei centri storici e urbani, questo perché gli abitanti dei ceti medi e nobiliari cercano sempre più di uscire da questa struttura urbana che sta diventando caotica.\\
  Cominciano a crearsi altri poli di una certa importanza, ad esempio i centri commerciali, i business park e le aree per il tempo libero. Ad oggi queste funzioni urbane sono normali, ma nascono e si sviluppano proprio in quegli anni dopo la seconda guerra mondiale.\\
  In questo periodo si sviluppa anche l'edilizia sociale che è sviluppata grazie a fondi pubblici.\\
  Tra i centri che vengono sviluppati come edilizia sociale si sviluppano anche aree marginali di indirizzo periferico. Questo tipo di polarizzazione quindi tende ad estendere i circuiti urbani.\\
  Troviamo anche sviluppi di comunicazione a traverso di assi stradali o vie aeree. L'espansione dell'uso dell'automobile anche è un grande fattore da considerare in questo sviluppo comunicativo.\\
  \begin{center}
  	\includegraphics[width=0.6\linewidth]{"immagini/città frammentata"}
  	\captionof{figure}{Modello di città frammentata}
  	\label{fig:città frammentata}
  \end{center}
  \leavevmode\\
  Oggi questo tipo di modello cittadino si è andato ulteriormente evolvendo, e quindi la città da polarizzata può essere oggi definita come città frammentata.\\
  Individuiamo alcuni elementi che individuano questa frammentazione, ad esempio le aree che una volta erano quelle industriali oggi hanno perso la loro vecchia posizione all'interno dei modelli urbani.\\
  Quelle zone industriali sono state riutilizzate e messe a nuovo, ci troviamo nel caso degli ambiti di riqualificazione urbana.\\
  Si sviluppano aree atte al commercio o alle aree produttive, le new-towns o città esterne alla struttura urbana. Le aree produttive subiscono processi di delocalizzazione o abbandono, ma all'esterno della città non è detto che non nascano nuove zone produttive come quelle a produzione del sistema quaternario, farmaceutiche e altro.\\
  Troviamo la presenza degli ambiti urbani interni, o gated communities di cui già abbiamo parlato [\ref{fig:modello a ramificazione}].\\
  \begin{center}
  	\includegraphics[width=0.6\linewidth]{"immagini/konx and pich 2000"}
  	\captionof{figure}{La città industriale classica}
  	\label{fig:konx and pich 20000}
  \end{center}
  \leavevmode\\
  In questo caso, Konx e Pinch parlano di una città ottocentesca che deriva anch'essa dalla struttura compatta pre-ottocentesca.\\
  Notiamo una parte centrale o meglio direzionale di uffici pubblici, centri religiosi e privati.\\
  Un'altra parte ancora centrale ma in questo caso occupata da strutture industriali, più all'esterno la comunità della working-class e sempre più esternamente il cerchio della classe media.\\
  \begin{center}
  	\includegraphics[width=0.6\linewidth]{"immagini/konx and pinch post industrial"}
  	\captionof{figure}{La città post-industriale}
  	\label{fig:konx and pinch post industrial}
  \end{center}
  \leavevmode\\
  La città industriale moderna si è polarizzata, per cui resta il central business district centrale ma restano degli elementi come le aree industriali che dopo il fenomeno di de-industrializzazione diventano aree inutilizzate e quindi problematiche dal punto di vista gestionale.\\
  La città centrale attorno al central business resta quasi uguale, mentre i sobborghi si espandono sempre di più. All'esterno di questo sistema si sono creati appunto dei poli, quindi le aree produttive si sono decentralizzate verso la fascia più esterna dove si sono create varie micro aree urbane e industriali.\\
  Ci sono poi assi costruite da sistemi autostradali o ferroviari che ovviamente si sviluppano grazie sia al bisogno di merci o comunque comunicazione tra i vari sistemi e sia per colpa proprio della decentralizzazione industriale che porta gli operai a intraprendere uno spostamento pendolare verso le nuove zone esterne.\\
  \begin{center}
  	\includegraphics[width=0.6\linewidth]{"immagini/konx and pinch post industrial +"}
  	\captionof{figure}{La città post-industriale espansa}
  	\label{fig:konx and pinch post industrial+}
  \end{center}
  \leavevmode\\ 
  La città post industriale diventa ancora più polarizzata con tendenze alla frammentazione. I centri urbani diventano luogo di architetture moderne grazie al fatto che il prezzo dei vari edifici cresce di valore.\\
  Quindi le aree prima lasciate a se stesse, con la riqualificazione prendono nuova vita. Le inner cities e i sobborghi rimangono praticamente simili, ma c'è la nuova presenza delle gated communities, zone chiuse dal resto degli altri sistemi e grandi centri di shopping situati esternamente alle varie zone urbane.\\
  In questo nuovo modello di città c'è anche la nascita del fenomeno della gentrificazione, ovvero il passaggio di un ceto più povero che lascia spazio allo stabilimento di abitanti più ricchi.\\
  \section{Modelli di Centro Urbano}
  \begin{center}
  	\includegraphics[width=0.6\linewidth]{"immagini/Concentric-zone-theory-Source-Burgess-1925"}
  	\captionof{figure}{Teoria delle zone concentriche, Burgess, 1925}
  	\label{fig:Concentric-zone-theory}
  \end{center}
  \leavevmode\\ 
  Siamo nel 1900 con le città in piena espansione, quindi sorgono anche studi sulla lettura di questa evoluzione.\\
  Burgess mette in piedi questo modello delle aree urbane ipotizzando appunto che l'evoluzione delle città nel corso di questa grande espansione industriale, sia una espansione che avviene sulla base di una struttura ad anelli.\\
  Burgess afferma che la città può essere letta attraverso questo modello a cerchi concentrici in cui la parte centrale è la parte in cui si localizzano le attività terziarie (pubbliche o private). Un primo anello attorno al CBD è l'area di transizione dove sono situate attività terziarie di minore entità. L'anello subito dopo è dove risiedono i lavoratori industriali e che individuano il ceto a basso/medio reddito. Il quarto anello è rappresentato sempre da un ceto, ma superiore a quello dei dipendenti di fabbrica o anche ambiti urbani esclusivi con restrizioni. L'ultimo cerchio invece racchiude le aree suburbane o satellite, poste a una distanza in modo da raggiungere in 30-60 minuti la zona centrale del CBD.\\
  \begin{center}
  	\includegraphics[width=0.8\linewidth]{"immagini/transect urban"}
  	\captionof{figure}{Modello transect urban}
  	\label{fig:transect urban}
  \end{center}
  \leavevmode\\ 
  Il transetto è un modello che deriva dall'ecologia e biologia. Prendiamo un sistema naturale, lo suddiviamo in tante parti e poi ne analizziamo ognuna di esse individuando gli elementi di passaggio tra una e l'altra.\\
  Applicando questo modello alla struttura urbana si ha che il modello è costruito sulla base della lettura specifica delle densità volumetriche presenti in città. Il modello individua ben 7 zone con 6 di esse rurali o urbane e 1 speciale. Passando da T1 a T6 abbiamo una densità via via maggiore.\\
  La zona speciale è destinata ai grandi impianti, come aeroporti, campus universitari, ferrovie e servizi simili.\\ \\
  Un altro modello di grande interesse è quello dell'Urban life cycle model. È un modello messo a punto da Van den Berg nel 1982 che analizza l'evoluzione di una città suddividendo la stessa in due aree specifiche (centro e periferia).\\
  Van der Berg dice che queste due aree subiscono un fenomeno evolutivo nel tempo che è continuo, e che vanno in una direzione di aumento espansionistico o di diminuzione.\\ \\
  Abbiamo diverse ipotesi :
  \begin{description}
  	\item [-] Le innovazioni favoriscono l'evoluzione della città
  	\item [-] L'evoluzione è graduale nel tempo
  	\item [-] L'evoluzione provoca sostituzione delle attività (ad es. attività del settore secondario diventano terziarie o quaternarie)
  \end{description}
  \leavevmode\\ 
  Quindi ogni volta che c'è un innovazione tecnologica o produttiva, questa impatta positivamente sulla espanzione urbana. Quando invece il periodo non presenta simili evoluzioni (è statico) le città anch'esse diventano statiche.\\
  La seconda ipotesi in pratica dà lo stesso risultato della prima, ma senza movimenti bruschi a meno che non ci siano cataclismi antropici che creino questi fenomeni.\\
  Nell'ultima ipotesi possiamo considerare ad esempio l'inizio dell'industrializzazione quando si inventa la macchina a vapore che viene poi adibita alla produzione di beni di qualunque tipo, come tessuti, automobili ecc. Questo provoca un passaggio da una attività artigianale a una di tipo industriale. Ciò comporta il bisogno di avere più persone addette e quindi un impatto positivo sulla produzione con conseguente evoluzione.\\
  \begin{center}
  	\includegraphics[width=0.8\linewidth]{"immagini/van der berg stadi urbanizzazione"}
  	\captionof{figure}{Evoluzione ciclica delle città secondo Van der Berg}
  	\label{fig:van der berg stadi urbanizzazione}
  \end{center}
  \leavevmode\\ 
  La città per Van der Berg si evolve in 4 grandi stadi, e questi sono ciclici, ovvero si sviluppano nel corso del tempo più o meno allo stesso modo. Abbiamo una prima fase di urbanizzazione dove può esserci una rivoluzione tecnica e la città subisce un grande accrescimento nella parte centrale.\\
  La popolazione poi non può più restare nell'area territoriale precedente e si ha una espansione urbana. Questo però crea problemi sociali ed economici per cui in alcuni luoghi della città vengono a crearsi diseconomie sociali.\\
  Alcuni punti della città diventano critici e si può avere una fase di deurbanizzazione. Quando tutta la città va in crisi e perde forze economiche a un certo punto reagisce con una riurbanizzazione.\\
  \begin{center}
  	\includegraphics[width=0.6\linewidth]{"immagini/stadi urbanizzazione grafico"}
  	\captionof{figure}{Evoluzione ciclica delle città, grafico spaziale}
  	\label{fig:stadi urbanizzazione grafico}
  \end{center}
  \leavevmode\\
  Vediamo una parte centrale dove c'è soltanto la città pre-industriale separata da altre città tramite una zona verde o agricola, la fase di urbanizzazione e quindi l'espansione dalla città compatta all'esterna ed infine l'agglomeramento con le altre città.\\
  \section{Modelli dinamici di continuità}
  Tre studiosi (Batty, Barros, Sinesio) nel 2005 hanno approfondito ancora una volta le modalità di evoluzione di una struttura urbana e hanno ipotizzato che le strutture si evolvono seguendo modelli dinamici, che possono essere di continuità o trasformazione.\\
  La continuità è un sistema urbano che nel corso del tempo si è evoluto in un certo modo e che a partire dal tempo t=0 si evolverà nel futuro allo stesso modo.\\
  Il modello di trasformazione invece è un modello in cui a un certo punto si crea una discontinuità all'interno della città. Il punto di discontinuità provoca una trasformazione nell'evoluzione del modello stesso.\\
  Questi modelli di continuità e trasformazione vengono chiamati in un modo molto specifico che vedremo successivamente.\\
  \begin{center}
  	\includegraphics[width=0.6\linewidth]{"immagini/Modello urbano a sviluppo cellulare"}
  	\captionof{figure}{Modello urbano a sviluppo cellulare}
  	\label{fig:modello urbano a sviluppo cellulare}
  \end{center}
  \leavevmode\\
  I modelli di continuità sono modelli analizzati nel passato, ad esempio nel corso del rinascimento si parlava molto delle città nuove come città venutesi a fondare su modelli geometrici regolari come quella di Palmanova [\ref{fig:palmanova_mappa}] che derivava da un concetto radiocentrico di perfezione idealistica.\\
  Medioevo e rinascimento quindi rientrano tra i periodi in cui gli studiosi avevano ipotizzato la possibilità di creare strutture urbane usando uno stesso modello per tutte.\\
  Le città idealistiche del rinascimento erano spesso modellati nella loro forma come strutture basate su una utopia urbana, che si fondava sui principi dell'architettura classica (ordine, simmetria, perfezione).\\
  Questo si riproponeva anche negli studi come quello per la città del sole di Tommaso Campanella, o in generale di modelli ideali. Spesso si utilizzavano celle elementari che messe una a fianco all'altra sviluppavano la città in un modo perfetto e regolare.\\
  \begin{center}
  	\includegraphics[width=0.6\linewidth]{"immagini/modello urbano a sviluppo circolare"}
  	\captionof{figure}{Modello urbano a sviluppo circolare}
  	\label{fig:modello urbano a sviluppo circolare}
  \end{center}
  \leavevmode\\
  Il principio di crescita modulare costante è anche quello che spesso si può trovare nell'accrescimento di elementi presenti in natura (flora e fauna). Strutture che tendenzialmente si accrescono usando una struttura circolare.\\
  Questo tipo di modello di continuità si può leggere nello sviluppo di alcuni centri urbani, soprattutto quelli situati in zone con grandi pianure.\\
  I tre studiosi citati prima fanno l'esempio di Las Vegas per associare la crescita di una città reale ai modelli citati prima.\\
  \begin{center}
  	\includegraphics[width=0.8\linewidth]{"immagini/The-Growth-of-Las-Vegas-from-1907-to-1995-from-Acevedo-et-al-1997"}
  	\captionof{figure}{Crescita territoriale di Las Vegas dal 1907 al 1995}
  	\label{fig:The-Growth-of-Las-Vegas-from-1907-to-1995-from-Acevedo-et-al-1997}
  \end{center}
  \leavevmode\\
  Nel 1907 non esisteva Las Vegas, quando negli anni 20 poi il Nevada decide di privatizzare il gioco d'azzardo. Arrivano fondi e si decide di sviluppare questa città proprio sull'economia del turismo e gioco d'azzardo.\\ \\
  Si ha la trasformazione del modello di crescita quando a un certo punto si crea una singolarità che modifica la traiettoria di evoluzione di un centro urbano.\\
  Il caso classico è quello della trasformazione di una città in un'area metropolitana. Il principio di crescita modulare ad un certo punto può modificarsi, ad esempio il centro città viene circondato da centri esterni o nuovi agglomerati che creano una distorsione del territorio con n centri.\\
  Si crea una struttura non più urbana ma metropolitana. Ciò significa che le connessioni non sono più monodirezionali dal centro verso l'esterno, ma possono verificarsi tra di loro bypassando il centro principale.\\
  La capacità di creare connessioni e moltiplicarle trasforma il modello classico urbano in uno metropolitano.\\
  Si stanno analizzando molto in questo periodo i modelli di deurbanizzazione controllata. La città è una struttura urbana di grande impatto ambientale e quindi deve essere sempre più controllata sulle sue emissioni nocive per l'ambiente o eccessiva distruzione naturale.\\
  Molti pensano che la sopravvivenza della città deriva dalla riduzione e dalla riorganizzazione dello spazio urbano, ad esempio uno dei modelli di cui si parla è una città che si scompone in altre piccole città con una zona verde tra ogni nuovo nucleo.\\
  \begin{center}
  	\begin{minipage}{0.46\linewidth}
  		\includegraphics[width=\linewidth]{"Immagini/modello di segregazione 2"}
  		\label{fig:modello di segregazione 2}
  	\end{minipage}%
  	\hfill
  	\begin{minipage}{0.46\linewidth}
  		\includegraphics[width=\linewidth]{"Immagini/modello di segregazione 1"}
  		\label{fig:modello di segregazione 1}
  	\end{minipage}
  \captionof{figure}{Modelli di segregazione, a sinistra uno chiuso, a destra uno aperto}
  \end{center}
  \leavevmode\\
  Altri modelli di trasformazione sono quelli di segregazione, che derivano da fenomeni di tipo sociale. Un esempio che normalmente si fa è quello dello stadio.\\
  In una realtà normale non esistono problemi tra le due tifoserie e si possono mischiare tranquillamente i posti delle due rispettive squadre. Questa è l'ipotesi di una società moderna aperta, multiculturale, libera ecc.\\
  Questo però spesso non avviene e l'unico modo per avere una parvenza di ciò è separare le due squadre come nel caso del calcio con un cordone di sicurezza che può essere fisico o virtuale.\\
  Questo è un elemento negativo che può influenzare un modello di crescita cittadino.
  \chapter{Modelli di Tipo Teorico}
  I modelli teorici non si occupano della forma geometrica e fisica del territorio, ossia della sua organizzazione spaziale. Si basano sull'analisi delle forze che guidano lo sviluppo delle città (forze economiche, sociali ecc). Individuano le traiettorie di evoluzione di un sistema territoriale nel passato e ipotizzano le stesse per il futuro.\\
  \section{Modello di localizzazione}
  \begin{center}
  	\includegraphics[width=0.8\linewidth]{"immagini/von_thunen_model"}
  	\captionof{figure}{Modello Von Thünen, 1826}
  	\label{fig:von_thunen_model}
  \end{center}
  \leavevmode\\
  Questo modello degli inizi dell'800 ha avuto una notevole importanza nel campo sia economico che di insediamenti territoriali. È un modello che ancora oggi è la base per l'evoluzione e lo studio di altri modelli econometrici.\\
  L'analisi partiva da casi studio in Germania, e Von Thünen analizza e spiega un sistema produttivo di questo genere. L'ipotesi da lui messa in campo è : " Data una città e il centro di consumo, per terminare questi beni agricoli abbiamo bisogno di un luogo di produzione. Il luogo di produzione è situato agli estremi della città, e ci deve essere un comune addendo tra la produzione e il luogo di vendita. Questo viene dato dalla distanza tra i due, che è un indicatore da cui viene definita tutta una serie di indicatori successivi. L'ipotesi fondamentale alla base del modello quindi è che i costi di trasporto dei beni agricoli verso un mercato quale può essere la città, si differenziano a seconda della località della produzione. Se un luogo di produzione è più lontano i costi di trasporto saranno ovviamente maggiori e provoca una differenziazione dell'uso del territorio intorno alla città. "\\
  Se consideriamo il punto centrale di vendita e un determinato prodotto x, a seconda del luogo da cui questo prodotto è stato fabbricato o raccolto avremo un guadagno diverso.\\
  Posto P il profitto che deriva dalla vendita di un bene, questo profitto è dato da un equazione che racchiude 3 termini : 
  \begin{description}
  	\item [-] Pm = Prezzo di mercato
  	\item [-] Cp = Costo di produzione
  	\item [-] Ct = Costo di trasporto
  \end{description}
  \leavevmode\\ 
  Questa equazione quindi avrà suddetta forma : $P = Pm - Cp - Ct$\\
  Il modello di Von Thünen quindi aveva come massima che dato Pm e Cp costanti, vi è una distanza che ci fa azzerare i profitti di vendita.\\
  In definitiva si ha che nella cerchia di distanza molto bassa tra il punto di raccolta e vendita possono essere tenuti in considerazione i prodotti deperibili, più ci si allontana verso l'esterno e più le produzioni possono essere a minor valore aggiunto.\\
  Dal punto di vista territoriale, se consideriamo un territorio vicino a una città possiamo considerare un caso 1 di perfetta isotropia. In pratica ciò vuol dire che ovunque noi ci posizioniamo sul territorio, esso avrà le medesime caratteristiche.\\
  Questo significa che in un sistema del genere tutti gli usi agricoli massimizzano la loro produttività, e le diverse tipologie di produzione si localizzano in funzione del solo profitto.\\
  Nella situazione reale però, non abbiamo un territorio del genere su cui operare, quindi ci sono linee preferenziali, di discontinuità o che creano separazioni. Ad esempio se c'è un fiume navigabile, il costo di trasporto è minore di quello via terra.\\
  Il terreno non è uguale dal punto di vista della fertilità, e ciò crea le possibilità di produzione per un determinato prodotto rispetto a un altro.\\
  Ci possono essere anche altri mercati che interferiscono con la produzione.\\
  \section{Modello di gravitazione}
  È pur sempre un modello territoriale che si basa però sulla legge gravitazionale, definito modello analogico (che viene applicato in un determinato campo, ma che deriva da altre leggi esterne).\\
  Questo modello definisce la distribuzione dei centri urbani nel territorio.\\
  Gravitazionale infatti perché si usa ala legge di Reilly, data dalla seguente equazione, $Fij=K*\frac{Pi*Pj}{f(Dij)}$\\
  Al numeratore ci sono gli indicatori di massa urbana (centro i e j). All'interno del denominatore invece c'è una espressione che è funzione tra la distanza dei due indicatori.\\
  Tutto questo viene moltiplicato per una costante specifica, tarata per ogni tipo  di operazione da compiere.\\
  L'indicatore di massa in genere per questo tipo di legge è quello dei pianeti che crea tra loro attrazione, in questo caso parliamo di massa urbana. Questa può essere qualsiasi elemento o caratteristica che caratterizza la città e che è capace di creare attrazione.\\
  Ad esempio la massa urbana può essere il numero di abitanti (100 mila contro 100). Possono essere i posti di lavoro con conseguente flusso veicolare, la dotazione di attrezzature e di servizi.\\
  Per quanto riguarda la distanza al denominatore, essa può essere vista sia in termini fisici che per esempio il costo di trasporto da un punto A a un punto B.\\
  Questo modello, come tutti gli altri del resto, deriva da una semplificazione della vita e per questo presenta degli elementi critici.\\
  Esso non prende considerazione dei fenomeni obbligati, ovvero se un soggetto è obbligato inequivocabilmente a restare su un determinato centro urbano. Gli effetti urbanistici possono anche derivare da comportamenti sociale, come tra lo scegliere due scuole si può fare la scelta di voler entrare in quella più costosa per le varie aree migliori o insegnanti più bravi.\\
  Inoltre non c'è la distinzione per l'influenza di centri terzi nello stesso territorio, e soprattutto la funzione di deterrenza $f(Dij)$ non è determinabile in modo univoco. Essa dipende da più fattori come costo, tempo, distanza ecc.\\
  \section{Modello gerarchico}
  Altro modello interessate, ideato da Christaller nel 1993 si basa sul fatto che quando si ha una città, normalmente produce beni e servizi in numero maggiore di quello che realmente serve ai cittadini. In questo caso si crea una posizione dominante, il che significa che la gerarchia della città aumenta di fascia. Si creano quindi località principali e secondarie, che si collocano all'interno di una struttura che si basa su costruzione ad esagoni.\\
  Questo comporta che lo spazio territoriale sia isotropo ed omogeneo, che il mercato dei beni sia statico e in equilibrio e che i servizi ed attrezzature sul territorio abbiano lo stesso grado di funzionalità e tecnologie.\\
  Per rendere ancora più semplice il modello, Christaller ipotizza che non vi siano più sovrapposizioni tra i centri urbani, questo ci porta alla già citata distribuzione esagonale con centri urbani su gerarchie differenti.\\
  \begin{center}
  	\includegraphics[width=0.6\linewidth]{"immagini/Christaller modello a esagono"}
  	\captionof{figure}{Modello esagonale ideato da Christaller}
  	\label{fig:Christaller modello a esagono}
  \end{center}
Esistono problematiche notevoli di applicazione relativo a questo modello, dati dalla presenza degli elementi critici come il territorio anisotropo; le reti di comunicazione discontinue; i punti dello spazio con diversa accesibilità e i mercati in competizione tra loro.\\
\chapter{Evoluzione Della Normativa}
In questa sezione del corso tratteremo dei piani territoriali ed urbanistici.\\
Questi sono il risultato di un processo che presenta momenti diversi.\\
I piani sono documenti tecnici che però incidono sul godimento della proprietà, che è un diritto tutelato dalal costituzione.\\
Per questo motivo i piani sono documenti normatici; il corpus che regola i piani è molto importante e deve avere il suo riconoscimento.\\
La normativa che noi abbiamo oggi si è stratificata nel corso del tempo. Sono norme emanate in periodi diversi e che messe insieme definiscono il corpus delle norme in campo urbanistico.\\
L'evoluzione può essere suddivisa in vari periodi. 
\subsection{Prima del 1939}
Questa è una data fondamentale, prima di essa non esistevano leggi specifiche nel campo urbanistico, bensì delle norme importanti nel settore degli espropri.\\
Questa legge (n. 2359) infatti è stata operativa dal 1865 fino agli inizi del 2000.\\
Venivano previsti strumenti volontari di pianificazione, ovvero per interventi all'interno della città o interventi all'esterno delle strutture urbane.\\
Questi due strumenti si applicavano alle città al di sopra di 10.000 abitanti, ma restavano pur sempre volontari nell'attuazione.\\
\subsection{1939-1978}
Nel 1939 vengono emanate due leggi importanti nel campo della tutela dei beni ambientali e monumentali, mentre nel 1942 
venne emanata una legge (l. 1150) sull'urbanistica nazionale e subito dopo la guerra vi è la fase di utilizzazione di queste normative.\\
Ovviamente furono emanate in un periodo storico caratteristico e negli anni '60 ci fu il bisogno di cambiare e riformare alcuni punti della legge.\\
Molteplici tentativi furono proposti per cambiare questa legge del '42 ma non ebbero mai luce.\\
\subsection{Dal 1985}
Le norme di questo periodo intervengono, in particolare, sulla tutela ambientale e riforme delle autonomie locali.\\
In seguito poi nasceranno le varie regioni, ma per ora tutto operava sul piano nazionale.\\
\subsection{Dal 2001}
In questo ultimo periodo viene approvata una riforma costituzionale che varia il titolo quinto.\\
Nella costituzione viene introdotto il concetto di governo del territorio. Un contenitore in cui vengono racchiuse tutte le materie che incidono sulle trasformazioni territoriali.\\
A partire dal 2001 quindi c'è una evoluzione molto forte delle normative.\\
\\
\\
Nel corso degli anni evolve la normativa, ma evolvono anche i soggetti che possono emanare norme in campo urbanistico.\\
Ad esempio fino al 1970, gli unici soggetti titolari ad emanare leggi in campo urbanistico era il parlamento e il ministero dei lavori pubblici. Lo stato emanava le normative, e il ministero le circolari esplicative.\\
Il ministero dei lavori pubblici inoltre operava utilizzando un braccio operativo, che era quello dei provveditorati regionali alle opere pubbliche.\\
Dal 1970 quindi vengono a formarsi normative diverse per le 21 entità regionali (Trentino e Alto Adige contano per due) che hanno però poteri di livello regionali.\\
\chapter{Pianificazione}
\section{Tipologie di Piani}
Quando parliamo di strumenti di pianificazione sono necessarie una serie di definizioni (\textbf{che vengono chieste spesso all'esame}).\\
\subsection{Piani Generali}
La prima tipologia definisce una serie di normative che comprendono il significato fondamentale della pianificazione generale.\\
Sono quindi strumenti di pianificazione in cui esiste un soggetto che pianifica (ente pubblico), il quale, per l'ambito di propria competenza costruisce un piano per disciplinare l'uso e la tutela del territorio.\\
Si chiama generale perché all'interno del piano possono essere comprese indicazioni in settori più estesi.\\
Possiamo trovare indicazioni relative alla residenza, al sistema commerciale, mobilità, tutela del verde e a quella ambientale. Quindi i settori compresi nel piano sono molteplici, per questo generale.\\
Questi piani prendono nome di :
 \begin{description}
   \item [-] Piano Territoriale Regionale
   \item [-] Piano Territoriale di Coordinamento Provinciale
   \item [-] Piano Territoriale Metropolitano
   \item [-] Piani Urbanistici Comunali
   \item [-] ...
 \end{description}
\leavevmode\\ 
\subsection{Piani Settoriali}
Anche in questo caso abbiamo a che fare con strumenti di piano, o meglio soggetti che possono essere gli enti pubblici territoriali oppure altri soggetti, che sono esterni a quelli territoriali.\\
Questi sono individuati dalla normativa che sono preposti alla tutela di specifici interessi.\\
Le due categorie di soggetti redigono un piano, quindi dettano la disciplina di tutela e uso del territorio relativamente ad un determinato settore. Detto in parole semplici relative al proprio settore.
Questi strumenti sono :
\begin{description}
	\item [-] Piano Paesaggistico\\
	che si occupa di un settore specifico, ovvero quello del paesaggio
	\item [-] Piano di Bacino\\
	tutela e difesa del suolo
	\item [-] Piano del Parco\\
	parchi nazionali e regionali
	\item [-] Piano dei Trasporti\\
	settore specifico della mobilità
	\item [-] ...
\end{description}
\leavevmode\\ 
\\
\\
Tornando alle definizioni, i piani generali vengono emanati da enti territoriali (Regione, Provincia e Comune) mentre i piani settoriali possono essere redatti in parte da enti pubblici (Ad esempio il Piano Paesaggistico viene redatto dalla Regione) e in parte da altri soggetto. Il Piano di Bacino ad esempio, viene redatto da una autorità di bacino che è sempre un ente pubblico ma non elettivo.\\
\subsection{Piani Misti}
Esiste una terza tipologia di Piani che racchiude degli strumenti di pianificazione in cui possono essere presenti indicazioni che fanno parte di un piano generale e indicazioni che fanno parte di un piano settoriale.\\
Ad esempio, se noi consideriamo un piano territoriale di coordinamento provinciale a valenza paesistica, abbiamo a che fare con uno strumento che è sia generale e sia settoriale (generale perché appunto fa parte del coordinamento provinciale, settoriale perché è un piano paesistico).\\
Quindi sono strumenti che sono di grande interesse, anche se spesso non vengono realizzati. Poiché ciascuno di questi strumenti sta a capo di un ente specifico, bisogna mettere d'accordo due enti.\\
\section{Livelli Della Pianificazione}
Ovviamente questi livelli si intersecano con le tipologie di piani, e danno luogo ad una griglia che vediamo a seguire :
\\
 \begin{center}
   \includegraphics[width=0.8\linewidth]{"immagini/livelli di pianificazione"}
   \captionof{figure}{Griglia dei livelli di pianificazione}
   \label{fig:livelli di pianificazione}
 \end{center}
\leavevmode\\
Dal punto di vista territoriale consideriamo 3 ambiti, ovvero il \textbf{Livello Territoriale} (o area vasta); \textbf{Livello Comunale} e \textbf{Livello di Ambito}.\\
Agli estremi troviamo la classificazione tipologica e i livelli. I due si incrociano nel modo che vediamo in figura.\\
Ad esempio nel campo dei piani generali il livello territoriale ha come strumenti quelli del piano regionale, di coordinamento provinciale e del piano territoriale.\\
Notiamo inoltre che i piani generali sono presenti per tutti e 3 i livelli di pianificazione, mentre i piani settoriali sono specifici per il livello territoriale e comunale ed infine i piani misti sono collegati solo al livello territoriale.\\
Le relazioni tra uno strumento e l'altro sono fondamentali, è evidente che se eseguiamo un piano attuativo, questo non può nascere senza una indicazione da livello superiore. Infatti i piani attuativi vengono chiamati in questo modo perché attuano delle previsioni contenute all'interno del piano urbanistico comunale.\\
La stessa cosa succede per il piano urbanistico comunale, che racchiude previsioni all'interno dei piani del livello territoriale.\\
Esiste quindi un rapporto di prevalenza gerarchica tra gli strumenti di pianificazione, che deriva dalla normativa della legge urbanistica nazionale (L. 1150/1942). Questa legge afferma che il livello gerarchico territoriale detta norme per il livello gerarchico comunale, il quale detta norme che devono essere rispettate dal livello di ambito (sub-comunale).\\
Inoltre, esistono anche gerarchie tra tipologie di piani. Ad esempio i piani di livello generale sono prevalenti rispetto a quelli di livello settoriale.\\
Per esempio se facessimo un piano dei trasporti regionale, questo deve vedere cosa dice il piano territoriale di coordinamento regionale relativamente al settore dei trasporti.\\
Allo stesso modo per il piano di risanamento acustico a livello comunale, questo piano deriva le sue indicazioni di risanamento dalla zonizzazione del risanamento del piano urbanistico comunale.\\
Nel corso del tempo, una delle critiche più importanti che si è data alla normativa del 1942 è appunto il fatto che questa prevalenza gerarchica fosse troppo rigida. Per questo negli anni i rapporti tra i piani si sono evoluti.\\
In prima presenza troviamo il cambiamento riguardante la prevalenza normativa e giuridica. Data la legge costituzionale avvenuta dopo la seconda guerra mondiale e quindi dal '48, c'è stato l'obbligo costituzionale di salvaguardia e tutela del paesaggio e dell'ambiente.\\
Da ciò è derivato che tutti i piani di livello di tipo ambientale (Paesaggistici; Di Bacino; Parco) hanno assunto una valenza di tipo costituzionale.\\
Queste norme quindi sia dal punto di vista giuridico che normativo sono oggi prevalenti rispetto agli strumenti di pianificazione generale.\\
Quindi quando andiamo a fare un piano territoriale dobbiamo prima controllare cosa dicono i piani di bacino, di parco e paesaggistici e dobbiamo rispettarne le indicazioni.\\
Con l'avvento della normativa Regionale, si è cercato di stemperare la prevalenza gerarchica esistente tra i diversi livelli di pianificazione. All'interno della normativa quindi si è sviluppato un concetto di co-pianificazione.\\
Questo concetto ci dice che quando un livello territoriale deve dirigere il suo strumento di pianificazione, poiché questo strumento influisce sul livello territoriale sottostante è opportuno che il piano di livello superiore venga redatto tenendo conto delle indicazioni che provengono dai livelli territoriali sottostanti.\\
Quindi co-pianificazione significa redigere un piano regionale in accordo con le province; città metropolitane; comuni.\\
Questo fa sì che il piano regionale che viene redatto abbia al suo interno giù tutte le indicazioni che possono essere recepite dai livelli inferiori perché presi su base di accordo.\\
\section{Principali Norme Nazionali - 1}
Queste normative possono essere raggruppate in quelle relative al paesaggio ed ambiente e a quelle di urbanistica.\\
\begin{description}
	\item [1939-Legge 1089] Tutela delle cose di interesse artistico e storico
	\item [1939-Legge 1497] Protezione delle bellezze naturali
\\ Queste leggi del '39 entrano poi successivamente nella costituzione come beni tutelati per principio dalla legge costituzionale

	\item [1942-Legge 1150] Legge Urbanistica
	\item [1962-Legge  167] Disposizioni per favorire l'acquisizione di aree (...) per l'edilizia economica e popolare
	\item [1967-Legge  765] Modifiche ed integrazioni alla legge urbanistica del 17 agosto 1942, n. 1150
	\item [1968-Legge 1444] Limiti inderogabili di densità edilizia, di distanza fra i fabbricati e rapporti massimi tra spazi destinati agli insediamenti residenziali e produttivi e spazi pubblici o riservati alle attività collettive, al verde pubblico o ai parcheggi da osservare ai fini della formazione dei nuovi strumenti urbanistici o della revisione di quelli esistenti, ai sensi dell'art. 17 della legge del 6 agosto 1967, n. 765
	\item [1971-Legge  865] Programmi e coordinamento per l'edilizia residenziale pubblica ... (Art. 27 - Piano degli Insediamenti Produttivi)
	\item [1978-Legge  457] Norme per l'edilizia residenziale (Art. 28 - Piano di Recupero)
\end{description}
\leavevmode\\  
Nel 1962 viene emanata una normativa specifica per l'edilizia sociale, che darà luogo ai piani per i quartieri popolari.\\
Nel '67 viene emanata questa legge ponte 765 che modifica e integra la legge del '42.\\
Nel '68 viene emanato un decreto inter-ministeriale che è una normativa fondamentale perché introduce gli standard urbanistici, obblighi non previsti precedentemente.\\
Successivamente altre due normative importanti sono le leggi 865 e 457 che prevedono nuovi piani di insediamenti produttivi e di recupero.\\
\section{Principali Norme Nazionali - 2}
Le normative sui beni e storico vengono introdotte tutte all'interno del codice culturale del paesaggio, ovvero nel DLgs 42.
\begin{description}
	\item [1985-Legge 431] (PAESAGGIO) Disposizioni urgenti per la tutela delle zone di particolare interesse ambientale (Lago Galasso)
	\item [1989-Legge 183] (AMBIENTE) Norme per il riassetto organizzativo e funzionale della difesa del suolo
	\item [1990-Legge 142] (URBANISTICA) Ordinamento delle autonomie locali (Oggi nel DLgs 267/2000 - Testo unico sull'ordinamento delle autonomie locali)
	\item [1991-Legge 349] (AMBIENTE) Legge quadro sulle aree protette
	\item [1997-Legge\phantom{a}59] (URBANISTICA) Delega del governo per il conferimento di funzioni e compiti alle regioni ed enti locali, per la riforma della pubblica amministrazione e per la semplificazione amministrativa (Legge Bersanini I)
	\item [2001-L.C\phantom{aaaa}3] (URBANISTICA) Modifiche al titolo V della parte seconda della Costituzione
	\item [2001-DPR\phantom{a}380] (EDILIZIA) Testo unico delle disposizioni legislative e regolamentari in materia edilizia
	\item [2004-DLgs\phantom{*4}42] (PAESAGGIO) Codice dei beni culturali e del paesaggio
	\item [2006-DLgs 152] (AMBIENTE) Norme in materia ambientale
\end{description}
\leavevmode\\ 
La costituzione prevedeva già nel 1948 la costruzione delle regioni, e nel 1970 che fu l'anno delle prime elezioni regionali, ci fu l'inizio della legislazione regionale in materia urbanistica.\\
La costituzione prevedeva che lo stato continuasse nella sua azione di attuazione delle norme legislative, e che alle regioni venissero date una serie di competenze nell'atto dell'attuazione. Praticamente le regioni dovevano prendere le normative razionali e organizzare l'applicazione di queste normative a livello regionale.\\
L'urbanistica è stata una delle materie che è stata trasferita alle regioni in quanto a gestione amministrativa. Fino al 1970 tutti i piani regolatori dovevano essere approvati a Roma.\\
Il passaggio delle competenze amministrative alle regioni modifica questa cosa fondamentale, a partire dal '70 i piani non vengono più mandati a Roma ma nel rispettivo capoluogo.\\
Nel '70 poi ci sono anche le prime elezioni dei Consigli Regionali. Vengono emanate una serie di norme per il trasferimento alle regioni di funzioni amministrative in campo urbanistico (DPR 8/9172 - DPR 616/1977).
\begin{displayquote}
	\textbf{Art. 79. Materia del trasferimento}
	
	1. Sono trasferite alle regioni le funzioni amministrative dello Stato e degli enti pubblici di cui all'art 1 nelle materie "urbanistica, tranvie e linee automobilistiche di interesse regionale", "viabilità, acquedotti e lavori pubblici di interesse regionale", "navigazione e porti lacuali", "caccia", "pesca nelle acque interne", come attinenti all'assetto ed utilizzazione del rispettivo territorio.
\end{displayquote}
Funzioni amministrative qui ci fa capire che le regioni devono comunque tener conto della normativa nazionale nel campo dell'urbanistica, quindi possono soltanto emanare norme che attuano la normativa nazionale.
\begin{displayquote}
\textbf{Art. 80. Urbanistica}

1. Le funzioni amministrative relative alla materia urbanistica concernono la disciplina dell'uso del territorio comprensiva di tutti gli aspetti conoscitivi, normativi e gestionali riguardanti le operazioni di salvaguardia e di trasformazione del suolo nonché la protezione dell'ambiente.
\end{displayquote}
Le regioni, in questo ventennio quindi, emanano una serie di norme che servono ad organizzare la macchina amministrativa regionale.\\
Si utilizza tutta la struttura della legge 1150 con le diverse funzioni organizzative regionali.\\
In autonomia però le regioni iniziano a fare azioni che via via aumentano proprio la stessa autonomia, ad esempio iniziano a fare la pianificazione territoriali regionali che lo stato non fu mai in grado di compiere.\\
Tutte le norme regionali mettono al centro dell'urbanistica la pianificazione comunale, quindi i piani regolatori comunali.\\
In più, inseriscono all'interno delle norme una serie di indicazioni relative alla tutela dei beni storico-architettonici e delle risorse naturali.\\
Il secondo periodo parte dal 1990 fino ad arrivare al 2001. In questo tempo le regioni hanno iniziato a lavorare in maniera più autonoma, e hanno continuato a vedere negli anni le diverse riforme urbanistiche nazionali che non andavano in porto e quindi hanno deciso di cambiare la legislazione urbanistica.\\
Quindi, queste regioni hanno visto che non si riusciva a fare la riforma della legge urbanistica nazionale e hanno pensato di prendere i principi di riforma per applicarli nella loro normativa regionale.\\
Nel quadro del 1150 da allora si è inserito una serie di innovazioni con conseguente evoluzione legislativa.\\
Il piano regolatore generale ad esempio è stato suddiviso in due parti, ovvero prescrizioni di tipo strutturale e prescrizioni di tipo operativo.\\
C'è stata una forte attenzione agli aspetti ambientali e alla compatibilità dei singoli interventi quanto anche della sostenibilità complessiva del piano.\\
Previsione di meccanismi che incrementano l'equità del piano, perché per dire, una persona può giovare dal fatto che la terra diventi edificabile, mentre un'altra potrebbe vederselo espropriato per la costruzione di un opera pubblica.\\
Allora affinché il piano diventi equo si idearono meccanismi compensativi o perequativi.\\
Sono stati introdotti tutti gli strumenti per l'accelerazione delle procedure e l'ampliazione dei principi di co-pianificazione e sussidiarietà.\\
Il terzo periodo parte dal 2001, quindi dalla riforma costituzionale. Una cosa fondamentale delle normative che si portano dagli anni precedenti è che esistono dei compiti di rilievo nazionale e dei compiti di rilievo locale. Quindi normative di cui si occupa lo Stato ed altre che possono essere trasferite alle regioni.\\
Con l'\textbf{Art. 117} nasce la differenziazione tra la legislazione esclusiva e concorrente. Esclusiva nel senso che solo lo stato può emanare leggi nelle materie considerate, a meno di deleghe a regioni.\\
Lo stato ha legislazione esclusiva nella tutela dell'ambiente, dell'ecosistema, dei beni culturali e ..., mentre materie di legislazione concorrente sono quelle che possono essere anche emanate dalle regioni, salvo che per la determinazione dei principi fondamentali, riservata alla legislazione dello Stato.\\
Lo stato ad esempio nel campo del governo del territorio emana attraverso il parlamento una riforma della legge urbanistica, questa diventa una legge di principi che viene applicata a livello regionale dalle singole regioni in quanto le regioni hanno podestà concorrente nel campo del governo del territorio.\\






%
%
%
%
\end{document}