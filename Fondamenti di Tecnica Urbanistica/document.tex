\documentclass[a4paper,12pt, oneside]{book}

% \usepackage{fullpage}
\usepackage[italian]{babel}
\usepackage[utf8]{inputenc}
\usepackage{amssymb}
\usepackage{graphicx}
\usepackage[font=small,labelfont=bf]{caption}
\usepackage{csquotes}
\usepackage{amsthm}
\usepackage{graphics}
\usepackage{amsfonts}
\usepackage{amsmath}
\usepackage{amstext}
\usepackage{engrec}
\usepackage{rotating}
\usepackage[safe,extra]{tipa}
\usepackage{multirow}
\usepackage{hyperref}
\usepackage{microtype}
\usepackage{enumerate}
\usepackage{braket}
\usepackage{marginnote}
\usepackage{pgfplots}
\usepackage{cancel}
\usepackage{polynom}
\usepackage{booktabs}
\usepackage{enumitem}
\usepackage{framed}
\usepackage{pdfpages}
\usepackage{pgfplots}
\usepackage{fancyhdr}
\pagestyle{fancy}
\fancyhead[LE,RO]{\slshape \rightmark}
\fancyhead[LO,RE]{\slshape \leftmark}
\fancyfoot[C]{\thepage}



\title{\textbf{Appunti Del Corso}\\ \textbf{Di}\\ \textbf{Fondamenti Di Tecnica Urbanistica}}
\author{Prof. G. Mazzeo\\ \\ A.A. 2021/2022}
\date{8 Marzo 2020}

\pgfplotsset{compat=1.13}
\begin{document}
	\maketitle
	
	\definecolor{shadecolor}{gray}{0.80}
	
	\newtheorem{teorema}{Teorema}
	\newtheorem{definizione}{Definizione}
	\newtheorem{esempio}{Esempio}
	\newtheorem{corollario}{Corollario}
	\newtheorem{lemma}{Lemma}
	\newtheorem{osservazione}{Osservazione}
	\newtheorem{nota}{Nota}
	\tableofcontents
	\renewcommand{\chaptermark}[1]{%
		\markboth{\chaptername
			\ \thechapter.\ #1}{}}
	\renewcommand{\sectionmark}[1]{\markright{\thesection.\ #1}}
	\chapter{Introduzione al Corso}
	\textbf{Questi appunti sono presi a lezione con il
		docente \textit{Giuseppe Mazzeo}. Per quanto sia stata fatta una revisione è altamente
		probabile (praticamente certo) che possano contenere errori, sia di stampa che
		di vero e proprio contenuto.\\Per eventuali proposte di correzioni : \href{mailto:ivanodapice@hotmail.com}{ivanodapice@hotmail.com}.\\
	    Per quanto riguarda invece informazioni utili relative a come contattare il professore, qui sotto sono riportate le sue due mail principali e in più quella relativa agli elaborati del corso.
	     \begin{list}{\labelitemi}{\leftmargin=1em}
	    	\item \texttt{Mail:} \href{mailto:gimazzeo@unina.it}{gimazzeo@unina.it}. \\ 
	    	\hspace{\labelwidth}\phantom{\texttt{Mail:} }\href{mailto:giuseppe.mazzeo@ismed.cnr.it}{giuseppe.mazzeo@ismed.cnr.it}. \\
	    	\hspace{\labelwidth}\phantom{\texttt{Mail:} }\href{mailto:corsotu2@libero.it}{corsotu2@libero.it}.
	    \end{list}\leavevmode\\   
    \textbf{Grazie mille e buono studio!}}

	\section{Urbanistica}
	L'urbanistica è una materia che definisce scelte di pianificazione territoriale (Urbano e Rurale); Gestione di trasformazioni di tipo territoriale come espansioni urbane; Tecniche e Strumenti per la redazione di piani ed infine valutazione di fattori di rischio e sostenibilità (quale consumo, risorse rinnovabili ecc).\\
	Uno dei più grandi urbanisti italiani è \textit{Giovanni Astengo}, il quale affermava che l'urbanistica è la scienza che studia i fenomeni urbani tenendo conto dello sviluppo storico, l'interpretazione e l'adattamento funzionale di aggregati urbani già esistenti.\newline
	Nel corso del tempo l'importanza di espandersi e creare nuovi insediamenti urbani era molto maggiore di andare ad agire su quelli già esistenti. Guardando invece ad oggi notiamo che c'è stato un punto intorno al dopoguerra dove grazie all'incredibile stock edilizio, i volumi sono diventati sovrabbondanti. Non c'è un reale bisogno di espansione come un tempo e quindi l'interesse globale si è spostato sul curare gli insediamenti urbani già esistenti.\newline
	Dalle parole di un altro noto urbanista quale \textit{Edoardo Salzano} notiamo come le nostre scelte che influiranno sulle trasformazioni territoriali dovranno essere tra loro coerenti e flessibili. Il piano urbanistico infatti rimane nel  tempo, quindi adottare soluzioni rigide per il nostro periodo potrebbe essere controproducente in un prossimo futuro.
	\leavevmode\\
	\begin{center}
		\includegraphics[width=0.7\linewidth]{"Immagini/2019-World-Population-Growth-1700-2100"}
		\captionof{figure}{crescita della popolazione dal 1700 ad oggi e proiezione al 2100}
	\end{center}
    \leavevmode\\
    Le città nel corso degli ultimi decenni hanno subito trasformazioni notevoli.\\
    Perché ci interessiamo della trasformazione del territorio? Ovviamente ce ne occupiamo per cercare di migliorare la vita delle persone, quindi dobbiamo pensare a quali fenomeni si stanno verificando a livello globale relativamente all'andamento della popolazione.\\
    Se consideriamo l'andamento della popolazione dal 1700 ad oggi notiamo che negli anni '60 abbiamo avuto un notevole aumento nel numero della popolazione. Per passare poi da 3 milioni di individui a 5 milioni ci sono voluti più o meno 50 anni, il che è notevole pensando alla crescita precedente.\\
    Tutto ciò è una conseguenza del miglioramento generale della vita. Il commercio è cresciuto e sono state introdotte nuove tecnologie.\\
    Nella figura [\ref{fig:populationgrowth}] troviamo una stima di come la popolazione crescerà nel futuro, con previsioni diverse a seconda dei vari scenari ipotizzati. Queste ipotesi possono essere più o meno valide essendo solo stime, ma capiamo come nei casi più estremi questa crescita possa diventare esponenzialmente paurosa; come si nota dalle quasi 17 miliardi di persone previste per il 2100.
    \leavevmode\\
    \subsection{Cambiamento Demografico}
    \begin{center}
      \begin{minipage}{0.48\linewidth}
    	\includegraphics[width=\linewidth]{"Immagini/future-timeline-2050-2100"}
    	\captionof{figure}{previsioni stimate riguardo l'accrescimento della popolazione al 2100}
    	\label{fig:populationgrowth}
      \end{minipage}%
      \hfill
      \begin{minipage}{0.46\linewidth}
      \includegraphics[width=\linewidth]{"Immagini/countries_growth_1950_2100"}
      \captionof{figure}{distribuzione della popolazione stimata al 2100}
      \label{fig:countriesgrowth}
      \end{minipage}
    \end{center}
	\leavevmode\\
	Quando diciamo che la popolazione aumenta perché la qualità media della vita aumenta non intendiamo di certo per tutti. Ci sono punti del pianeta che hanno una qualità di vita nettamente superiore ad altri. Inoltre, questa condizione è strettamente legata alla fertilità generale.\\
	Vediamo dal grafico [\ref{fig:countriesgrowth}] che dal 1950 al 2010 l'Europa ha subito un cambiamento del'11\%   (22\% - 11\%), il che indica che altre zone del pianeta hanno avuto un aumento considerevole.
	\leavevmode\\
	\begin{center}
		\includegraphics[width=0.7\linewidth]{"Immagini/pyramid-demographic"}
		\captionof{figure}{esempi di piramidi demografiche}
	\end{center}
	\leavevmode\\
	Un'altro importante elemento da considerare è che nei paesi più ricchi la popolazione cambiando qualità di vita cambia anche le proprie statistiche demografiche.\\
	Se consideriamo ad esempio L'Italia dal secondo dopoguerra ad oggi, la piramide dell'età nel secondo dopoguerra aveva un aspetto uniforme. Oggi queste piramidi hanno invece un andamento a bottiglia o fuso, ciò implica che nascono meno bambino ed essendo l'aspettativa di vita più alta troveremo un numero più ampio di persone anziane.\\
	Tutto ciò porta una serie di problemi, quali la minoranza di bambini e persone in età lavorativa che potrebbero portare problemi riguardo il sostegno delle fasce di età più avanzate.\\
	\leavevmode\\
	 \begin{center}
		\includegraphics[width=0.7\linewidth]{"Immagini/popolazione urbana-rurale"}
		\captionof{figure}{comparativa tra popolazione urbana e rurale negli anni}
	 \end{center}
    \leavevmode\\
    Questo grafico è di grande interesse perché ci dice qual è la ripartizione a livello globale tra la popolazione abitativa delle zone urbane da quelle rurali.\\
    A partire dal 1975 c'è stato il sorpasso delle zone urbane su quelle rurali, inoltre da allora è sempre più crescente. Dal grafico vediamo infatti che nel 2010 il numero di abitanti per zona urbana era quasi 7 volte superiore a quello delle zone rurali. Ovviamente questo pone questioni per quanto riguarda la sostenibilità dei sistemi urbani.
    \leavevmode\\
    \begin{center}
    	\includegraphics[width=0.7\linewidth]{"Immagini/worldurbanpop"}
    	\captionof{figure}{percentuale di popolazione urbana rispetto ad ogni continente}
    \end{center}
    \leavevmode\\
    Abbiamo detto che le popolazioni che vivono con una maggiore ricchezza sono anche quelle con meno fertilità, tali popolazioni sono statisticamente più incentrate sulla vita urbana.\\
    Se consideriamo il Nord America vediamo che entro il 2050 si stima raggiunga una percentuale di urbanizzazione pari al 90\textdiscount. I paesi meno sviluppati come quelli dell'Africa però hanno lo stesso tipo di andamento. Tenendo conto proprio dell'Africa, nel 2010 vediamo una percentuale pari al 40\textdiscount, con una previsione di crescita fino al $\sim$62-63\% entro il 2050.
    \leavevmode\\
    \begin{center}
    	\begin{minipage}{0.48\linewidth}
    		\includegraphics[width=\linewidth]{"Immagini/Megacities"}
    		\captionof{figure}{proiezione megacities nel 2025}
    		\label{fig:megacities}
    	\end{minipage}%
    	\hfill
    	\begin{minipage}{0.46\linewidth}
    		\includegraphics[width=\linewidth]{"Immagini/alpha2004c"}
    		\captionof{figure}{gerarchie urbane a livello globale}
    		\label{fig:alphacities}
    	\end{minipage}
    \end{center}
    \leavevmode\\
    Nelle due immagini sopracitate, la prima ci mostra che le mega città dei prossimi anni saranno concentrate prevalentemente nelle regioni asiatiche, mentre la gerarchia delle città più potenti in quanto a economia globale rimane comunque incentrata ad ovest.\\
    Da queste grandi espansioni abbiamo capito quindi che il fenomeno urbano ha raggiunto rilevanza planetaria. L'espansione delle strutture però avviene in molti paesi senza un reale controllo o attenzione ai fattori ambientali (Solo in occidente si cerca di regolamentare in modo efficace).\\
    Le città quindi non sono uguali tra loro per dimensione, consapevolezza ambientale e soddisfacimento dei diritti primari. Per questo dobbiamo cercare di applicare il più possibile politiche di sostenibilità alle città.\\
  \chapter{Origine del Fenomeno Urbano}
    Nell'800 avviene un fenomeno che cambia completamente il modo di produrre i beni, e quindi le strutture sociologiche.\\
    Nasce per l'evoluzione dei processi produttivi, quindi la rivoluzione industriale ha come fenomeno lo sviluppo delle città.\\
    Questo processo di crescita delle città è durato per quasi più di 200 anni, infatti da poco le città hanno smesso di crescere in modo sostenuto. Dagli anni '80 invece abbiamo addirittura riscontrato un rallentamento di tali processi di crescita causati dal maggiore interessamento all'ecosistema globale.\\
    Londra nell'arco di 400 anni (1300-1700) è cresciuta di 7 volte, ma con percentuali di superficie di partenza molto piccole. Per raddoppiare ulteriormente la città ce ne vogliono altri 100. Via discorrendo l'espansione territoriale accelera sempre maggiormente, tanto che dal 1944 al 2008 passa da avere un 58\% di superficie urbana al 100\%.\\
    Se contiamo però la popolazione abitativa della città, con la crescita della superficie urbanizzata non c'è un uguale accrescimento della popolazione; alcune persone da un certo punto nel tempo vanno a vivere in aree limitrofe.\\
    \section{Rivoluzione Industriale}
    La rivoluzione industriale, come già detto in precedenza, è un fattore importante per la nascita dell'urbanizzazione.\\
    È stato un fenomeno rivoluzionario che ha cambiato la storia dell'epoca. Se prima un artigiano poteva produrre pochi prodotti al giorno, con le macchine si sono raggiunti livelli di produzione di massa.\\
    Massa che possiamo pensare anche come persone che vanno a lavorare per queste nuove fabbriche diventando appunto operai, e proprio queste fabbriche daranno il via a una espansione industriale incredibile, per produrre sempre di più.\\
    Con la nascita dell'industrializzazione però nascono anche nuove classi sociali e lotte interne per salvaguardare i diritti degli operai.\\
    Questo lato industriale ha luogo anche per il grande sviluppo scientifico dell'epoca. L'ingegneria prende forma così come anche lo sviluppo medico (scoperta dei vaccini) e di altre scienze, questo comporta la nascita di strutture in elevazione e lunghezza, nuovi mezzi di spostamento quali i primi treni e una idea di strutture fognarie atte anche a combattere le malattie con l'aiuto della già citata rivoluzione medica.\\
    \chapter{Pianificazione}
    Il grande sviluppo industriale, portando un miglioramento generale della vita, attira anche molte più persone verso una città industrializzata. Questo fenomeno di richiesta abitativa porta a una formulazione di piani urbanistici per permettere un collocamento dimensionato alla richiesta.\\
    Questo sviluppo possiamo considerarlo in 4 fasi principali : \\
    \begin{itemize}
      \item L'utopia
      \item Il piano
      \item Le tecniche di pianificazione
      \item I modelli urbani e territoriali
    \end{itemize}
    \leavevmode\\
    Abbiamo una fase di analisi di predisposizione di elemento utopico, quindi risolvere varie problematiche con modalità di tipo utopistico.\\
    C'è la modalità tecnica e quindi quella più ingegneristica, ovvero la costruzione del piano, da cui poi deriveranno le tecniche per la costruzione del piano stesso e lo sviluppo di modelli.\\
    Modelli urbani o territoriali che potessero essere usati appunto per la costruzione del piano.\\
    Questi 4 capisaldi sono quelli su cui si basa lo sviluppo complessivo dell'urbanistica.\\
    \section{Utopia}
    Il principale responsabile dei processi di urbanizzazione nell'800 è l'industrializzazione. Di solito quest'era industriale è vista in modo molto negativo (Fabbriche senza giornate lavorative, sfruttamento dei dipendenti). Per questo poi nasce il movimento operaio e i sindacata; per difendere i diritti degli operai.\\
    L'utopia però esiste nella cerchia degli imprenditori illuminati, infatti anziché impiantare le fabbriche sulla normale idea di fabbrica produttiva cercano di realizzare impianti che siano più sicuri e con un ambiente lavorativo migliore.
    	\leavevmode\\
    \begin{center}
    	\includegraphics[width=0.6\linewidth]{"immagini/Salina de Chaux"}
    	\\
    	\includegraphics[width=0.6\linewidth]{"immagini/Salina de Chaux top"}
    	\captionof{figure}{Le saline di Chaux, 1804}
    \end{center}
    \leavevmode\\
    Questo impianto ideato da Claude-Nicolas Ledoux contiene al suo interno nono solo l'impianto produttivo, ma anche altre attrezzature quali residenze per operai e altri spazi di svago.\\
    Tutto ciò in ordire di creare per gli operai un ambiente lavorativo più sereno.\\
    \begin{center}
    	\includegraphics[width=0.6\linewidth]{"immagini/Falansterio"}
    	\captionof{figure}{Falansterio, inizi del XIX secolo}
    \end{center}
    Un altro esempio classico è quello del Falansterio, idea nata da Charles Fourier che ipotizza questo edificio con al suo interno compartimenti per produzione, svago e vita giornaliera dei dipendenti.\\
    Queste strutture utopiche hanno creato realtà molto interessanti come New Lanark in Inghilterra. Fabbrica di tessuti di Robert Owen che come caratteristica principale ha sempre i due lati di produzione e residenza.\\
    Tutto questo serviva anche per ottenere controllo non solo sulla vita produttiva della fabbrica, ma anche su quella degli operai che, come in una bolla, conoscevano solo quella piccola comunità che si veniva a creare tra operai della stessa compagnia.\\
    New Lanark invece è un esempio interessante perché al suo fallimento i dipendenti assunsero i controlli di tutto il sistema e quindi avviarono una gestione cooperativa.\\
    I sistemi quindi assumono due tipi di gestione, quella classica dell'imprenditore e quella cooperativa che nasce dalla comunità operaia.\\
    \section{Il Piano Urbanistico}
    Dalla prima visione utopistica degli imprenditori, come nazione si cerca di trovare un modo per governare lo sviluppo delle città, o meglio regolamentare questi processi senza lasciare tutto in mano agli imprenditori.\\
    Da questa esigenza si afferma il piano urbanistico come strumento di regolazione della espansione delle città.\\
    \begin{center}
    	\includegraphics[width=0.6\linewidth]{"immagini/london 1829"}
    	\captionof{figure}{Piano per Londra, 1829}
    	\label{fig:londra1829}
    \end{center}
    John Claudius Loudon è stato probabilmente l'ideatore del primo piano urbanistico relativo a una città [\ref{fig:londra1829}].\\
    La città di Londra, fino ad allora si era sviluppato senza alcun piano per attuare risposte a problemi quotidiani quali ai tempi potevano essere ad esempio epidemie. Infatti nel periodo dell'800 è stata sì una grossa città, ma che nel tempo ha subito varie problematiche relative alla condizione di vita dei cittadini.\\
    Proprio per questo è stata una città che ha sviluppato per prima vari metodi risolutivi per arginare queste piaghe che affliggevano i popolani.\\
    Nel 1829 si è ideato questo piano urbanistico ad anelli concentrici. Il punto centrale della città è la cattedrale che per un cerco di quasi un miglio forma la prima "fascia" che racchiude il palazzo imperiale e le altri importanti strutture cittadine.\\
    A partire dalla striscia poco più distante dalla cattedrale c'è una zona verde che porta alla seconda fascia urbana in cui vengono realizzate le residenze.\\
    Dopo le residenze abbiamo ancora una zona verde e poi le fabbriche al centro più esterno.\\
    L'obbiettivo di questo piano dunque è che qualunque residenza si allontana non più di mezzo miglio da una fascia verde; avere a disposizione spazi non urbanizzati e aumentare la salubrità della stessa struttura urbana.\\
    \begin{center}
    	\includegraphics[width=0.6\linewidth]{"immagini/carta historica barcelona 1"}
    	\captionof{figure}{Barcellona, 150 a.C.}
    	\label{fig:barcelona1}
    \end{center}
    \leavevmode\\
    Un secondo esempio di piano è quello di Barcellona.\\
    Storicamente è sempre stata una città indipendente e non si è mai unita al regno spagnolo, nel corso del tempo viene conquistata dai romana e ne diventa uno dei loro primi possedimenti.\\
    L'impero sviluppa la città di Barcellona e viene costruita come se fosse appunto una città di fondazione romana; Cargo decumano e 4 quadranti su struttura a scacchiera.\\
    Nel 550 d.C. c'è la crisi dell'impero romano e la città viene conquistata dai visigoti.\\
    \begin{center}
    	\includegraphics[width=0.6\linewidth]{"immagini/carta historica barcelona 2"}
    	\captionof{figure}{Barcellona, 550 d.C.}
    	\label{fig:barcelona2}
    \end{center}
    \leavevmode\\
    Dal 1750 viene conquistata dagli spagnoli e costruiscono una piazzaforte per tenere sotto controllo la città, inoltre impongono il divieto di costruzione all'interno di una fascia di 10-15km intorno alle mura cittadine.\\
    Barcellona quindi diventa densamente popolata all'interno della fascia proprio per questa impossibilità di estendersi al di fuori di essa.\\
    \begin{center}
    	\includegraphics[width=0.6\linewidth]{"immagini/carta historica barcelona 3"}
    	\captionof{figure}{Barcellona, 1750}
    	\label{fig:barcelona3}
    \end{center}
    \leavevmode\\
    Nel 1855 viene eletto un governo progressista e abolisce la normativa anti-espansionista così che Barcellona possa utilizzarla per lo sviluppo.\\
    \begin{center}
    	\includegraphics[width=0.6\linewidth]{"immagini/carta historica barcelona 4"}
    	\captionof{figure}{Barcellona, 1855}
    	\label{fig:barcelona4}
    \end{center}
    \leavevmode\\
    Viene dato l'incarico di ideare un piano urbanistico a Ildefons Cerdà che costruisce una struttura dove il territorio di Barcellona espanso al di fuori delle mura è completamente diverso da quello storico.\\
    La struttura urbana che idealizza Cerdà è a maglia quadrata che viene interrotta da una serie di trasversali.\\
    Il vantaggio di questa struttura urbana è che Barcellona è situata su una pianura che ci favorisce questo tipo di soluzione.\\
    \begin{center}
    	\includegraphics[width=0.6\linewidth]{"immagini/ildefonscerdaspain"}
    	\includegraphics[width=0.6\linewidth]{"immagini/CruillesCerda"}
    	\captionof{figure}{Piano urbanistico di Barcellona ideato da Ildefons Cerdà}
    	\label{fig:cerda}
    \end{center}
    \leavevmode\\
    Barcellona si fonda quindi su una maglia quadrata che ha un modello organizzativo a isolati.\\
    Gli isolati si basano sulla costruzione di un quadrato di determinata dimensione. Con l'insieme di 100 isolati si forma una struttura di quartiere, e ciascun distretto avrà un isolato utilizzato per una qualsiasi tipologia di struttura o adibito al verde.\\
    Ogni distretto quindi sarà isolato e funzionale di per sè.\\
    \begin{center}
    	\includegraphics[width=0.6\linewidth]{"immagini/Barcelona-map-1890"}
    	\captionof{figure}{Mappa di Barcellona, 1890}
    	\label{fig:barcelona-map-1890}
    \end{center}
    \leavevmode\\
    Nell'arco di 50 anni il piano viene impostato e la città viene disegnata sul terreno.\\
    Il centro storico viene preservato mentre l'ambito della piazzaforte viene demolito per fare spazio a un giardino, i quartieri invece al di fuori delle mura iniziano ad essere sviluppati.\\
    \begin{center}
    	\includegraphics[width=0.6\linewidth]{"immagini/barcelona space components"}
    	\captionof{figure}{Isolati di Barcellona}
    	\label{fig:barcelona space components}
    \end{center}
    \leavevmode\\
    Quando Cerdà fece una prima collocazione di isolati, questi erano molto ampi perché al tempo ovviamente la popolazione di Barcellona era molto meno densa del giorno d'oggi.\\
    Troviamo nel primo piano di Cerdà una superficie edificabile pari al 28\%; una superficie viaria del 30\% e una superficie a verde del 42\%.\\
    Queste percentuali adesso sono completamente state stravolte, ma la disposizione delle aree è comunque rimasta molto simile.\\
    La prima misura su cui si basa il piano di Cerdà è la larghezza della strada, pari a 10m con 5m di marciapiede ad ambo i lati.\\
    \begin{center}
    	\includegraphics[width=0.6\linewidth]{"immagini/incrocio ottogonale"}
    	\captionof{figure}{Esempio di incrocio ottogonale}
    	\label{fig:incrocio ottogonale}
    \end{center}
    \leavevmode\\
    La seconda misura ideata è quella relativa all'incrocio tra 4 isolati, che non crea una croce ma bensì un ottagono. Da qui si ricava l'apotema dell'ottagono, ideato pari a 24m con la formula $B=A/2*(1+\sqrt{2})$\\
    Da queste due misure Cerdà ricava poi tutte le restanti, quali l'altezza degli edifici (20m); la profondità edificabile (24m); la larghezza del cortile (68m) e la dimensione dell'isolato (116m).\\
    Nel tempo, col crescente numero della popolazione a Barcellona, Cerdà introduce dei blocchi ad "U" che possono essere visti nella seconda fascia della figura [\ref{fig:barcelona space components}].\\
    Questo sistema aumenta del 33\% le volumetrie. Questi isolati poi vengono disposti in modo da avere strade con negozi o abitazioni ad ambo i lati, e strade più isolate dalla vita cittadina con molta presenza di aree verdi.\\
    Sorgono anche ordinanze comunali che incrementano di volta in volta la potenzialità di occupazione all'interno degli isolati.\\
    Con l'ordinanza edile del 1859-1889 si ha un aumento dell'occupazione massima al 50\% e la possibilità di costruire 4 piani (compreso il piano terra).\\
    Tra il 1890 e il 1932 vengono emanate nuove ordinanze comunali e l'occupazione massima passa al 75\% dell'isolato. L'altezza è pari al piano terra + 7 piani + attico e nello spazio al centro è possibile creare un edificio interno di 5m di altezza edito a magazzino o altri servizi.\\
    Altre ordinanze ci furono tra il 1933 e il 1975 con 73\% di occupazione massima, edificio interno di 5m + interrato ed edifici interni con interrati; piano terra; 7 piani; attico e superattico.\\
    A partire dal '76 invece si è cercato di ridurre tutte le specifiche degli isolati perché si raggiunse una densità edificata troppo grande. L'occupazione massima degli isolati scese al 70\% e si tolse la possibilità di avere attici e superattici. Gli interrati invece ebbero un cambiamento in positivo riguardo lo spazio occupato, infatti ci fu la possibilità di costruire doppi piani interrati.\\
    \begin{center}
    	\includegraphics[width=1.0\linewidth]{"immagini/isolati speciali barcellona"}
    	\captionof{figure}{Isolati speciali}
    	\label{fig:isolati speciali barcellona}
    \end{center}
    \leavevmode\\
    Cerdà ha anche ideato isolati con specifici casi d'uso, qui sotto la legenda relativa alla figura [\ref{fig:isolati speciali barcellona}]\\
    \begin{description}
    	\item[A.    ] \phantom{a|}Isolato doppio (Università, impianti industriali)
    	\item[B.    ] \phantom{a|}Edificio a croce di sant'Andrea
    	\item[C.    ] \phantom{a|}Decomposizione con paesaggi
    	\item[D.    ] \phantom{a|}Edifici nella banda centrale
    	\item[E, F.]  \phantom{..}Edificio su mezzo isolato
    	\item[G.   ] \phantom{a|}Chaflan (Edifici smussati con accesso su incrocio)
    \end{description}
  
    \section{Le Tecniche di Pianificazione}
    Il piano di Cerdà diventa anche l'occasione per l'approfondimento sulle tecniche di pianificazione.\\
    Si iniziano a scrivere manuali su questa nuova materia che è l'urbanistica, e approfondimenti sulle strutture urbane e la loro modalità di costruzione.\\
    Cerdà scrive nel 1876 un libro dal titolo "teoria general de la urbanizacion", che è la trasformazione del piano di Barcellona in teoria.\\ \\
    In questo libro Cerdà scrive : 
    \begin{displayquote}
    	Questi sono i motivi filologici che mi hanno indotto ad adottare il termine urbanizzazione. Tale termine indica l’insieme degli atti che tendono a creare un raggruppamento di costruzioni e a regolarizzare il loro funzionamento, così come designa l’insieme dei principi, dottrine e regole che si devono applicare perché le costruzioni e il loro raggruppamento, invece di reprimere, indebolire e corrompere le facoltà fisiche, morali e intellettuali dell’uomo che vive in una società, contribuiscano a favorire il suo sviluppo e ad accrescere il benessere sia individuale che pubblico.
   \end{displayquote}
   \leavevmode\\
   Questa frase ci fa capire il ruolo che ha l'urbanistica nella nostra vita e la propria importanza.\\
   Ci sono anche vari altri manuali scritti nel corso del tempo come ad esempio \textit{Stadt-Erweiterungen in Technischer Baupolizeilicher Und Wirthschaftlicher Beziehung} (1867) e \textit{Der Stadtebau} (1889).\\
   Come struttura urbana abbiamo già visto ad esempio la maglia quadratica di Cerdà, ce ne sono ovviamente anche altre come l'ipotizzazione di città lineari, ovvero città costruite lungo un asse centrale.\\
   Questa idea di Arturo Soria y Mata vedeva questo asse centrale organizzato con una strada passante proprio per quest'ultimo e poi degli spazi laterali a verde. Due corsie per il passaggio di tram elettrici e dei quartieri residenziali a destra e sinistra.\\
   \begin{center}
   	\includegraphics[width=0.8\linewidth]{"immagini/ciudad-lineal"}
   	\captionof{figure}{Ciudad Lineal, 1883}
   	\label{fig:Ciudad Lineal}
   \end{center}
   \leavevmode\\
   Questo modello non è stato molto utilizzato perché ha vari tipi di svantaggi, ad esempio l'asse centrale per città molto grandi presenta intasamenti considerevoli per la circolazione della mobilità.\\ \\
   Un altro modello di interesse è quello delle garden cities, messo appunto da Ebenezer Howard.\\
   L'ipotesi di Howard è costruire una città con un numero di abitanti stabilito (32000, di cui 2000 destinati alle aree agricole), inoltre con una superficie di 2400 ettari; una aerea agricola di 2000 ettari e una area urbana di 400 ettari.\\
   \begin{center}
   	\includegraphics[width=0.8\linewidth]{"immagini/garden cities"}
   	\captionof{figure}{Ciudad Lineal, 1883}
   	\label{fig:garden cities}
   \end{center}
   \leavevmode\\
   
   
    
    
\end{document}