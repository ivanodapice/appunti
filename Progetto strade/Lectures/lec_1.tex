\chapter{Modello digitale del terreno}

Prima di iniziare con la progettazione, bisogna creare un file DGN in cui il seed identifichi un foglio di lavoro 3D e dove il workflow utilizzato sia Openroads modeling. Dopo questa operazione è possibile creare un modello digitale del terreno, ovvero la ricostruzione di una parte di superficie terrestre a partire da un dato di origine.  Il modello digitale di terreno utilizzato è chiamato Digital Terrain Model (DTM) e comprende la superficie topografica (senza oggetti presenti sul terreno). Per inserire la cartografia come riferimento esterno bisogna inserire il comando Attach tools, e cliccando sulla vista fit view sarà possibile visualizzare il terreno nella sua completezza. Al fine di creare un DTM è necessario partire con la realizzazione di un filtro grafico, che è possibile creare nella sezione Terrain con il comando From Graphical Filter. Con lo stesso procedimento creiamo un filtro sia per le curve di livello che per i punti quotati. Successivamente bisogna creare un perimento (\ref{perimetro}) contenente i filtri creati e che permetta di attivare e visualizzare a video la vista triangolare del terreno (\ref{vista triangolare}) con il comando Max triangle lenght (all’aumentare dei triangoli aumenta la precisione). Grazie alla possibilità di ruotare il nostro DTM possiamo osservare il nostro modello 3D in varie angolazioni.

  \begin{figure}[H]
    \centering
    \begin{minipage}[b]{0.45\textwidth}
      \includegraphics[width=\textwidth]{Figures/perimetro.png}
      \caption{Perimetro}
      \label{perimetro}
    \end{minipage}
    \hfill
    \begin{minipage}[b]{0.45\textwidth}
      \includegraphics[width=\textwidth]{Figures/vista triangolare.png}
      \caption{Vista triangolare}
      \label{vista triangolare}
    \end{minipage}
  \end{figure}