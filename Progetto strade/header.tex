% basics
\usepackage[T1]{fontenc}
\usepackage{textcomp}
\usepackage[hyphens]{url}
%\usepackage[style=alphabetic,maxcitenames=1]{biblatex}
\usepackage[colorlinks=true,linkcolor=cyan,urlcolor=magenta,citecolor=violet]{hyperref}
\usepackage{graphicx}
\usepackage{float}
\usepackage{booktabs}
\usepackage{emptypage}
\usepackage{subcaption}
\usepackage[dvipsnames,table,xcdraw]{xcolor}

% quiver style
\usepackage{tikz-cd}
% `calc` is necessary to draw curved arrows.
\usetikzlibrary{calc}
% `pathmorphing` is necessary to draw squiggly arrows.
\usetikzlibrary{decorations.pathmorphing}

% A TikZ style for curved arrows of a fixed height, due to AndréC.
\tikzset{curve/.style={settings={#1},to path={(\tikztostart)
      .. controls
      ($(\tikztostart)!\pv{pos}!(\tikztotarget)!\pv{height}!270:(\tikztotarget)$)
      and
      ($(\tikztostart)!1-\pv{pos}!(\tikztotarget)!\pv{height}!270:(\tikztotarget)$)
  .. (\tikztotarget)\tikztonodes}},
  settings/.code={\tikzset{quiver/.cd,#1}
  \def\pv##1{\pgfkeysvalueof{/tikz/quiver/##1}}},
quiver/.cd,pos/.initial=0.35,height/.initial=0}

% TikZ arrowhead/tail styles.
\tikzset{tail reversed/.code={\pgfsetarrowsstart{tikzcd to}}}
\tikzset{2tail/.code={\pgfsetarrowsstart{Implies[reversed]}}}
\tikzset{2tail reversed/.code={\pgfsetarrowsstart{Implies}}}
% TikZ arrow styles.
\tikzset{no body/.style={/tikz/dash pattern=on 0 off 1mm}}

% useful macro for class
\newcommand{\at}[3]{\left.#1\right\vert_{#2}^{#3}}
\newcommand\quotient[2]{
  \mathchoice
  {% \displaystyle
    \text{\raise1ex\hbox{$#1$}\Big/\lower1ex\hbox{$#2$}}%
  }
  {% \textstyle
    #1\,/\,#2
  }
  {% \scriptstyle
    #1\,/\,#2
  }
  {% \scriptscriptstyle
    #1\,/\,#2
  }
}

\newcommand{\True}{\textsf{True}}
\newcommand{\T}{\textsf{T}}
\newcommand{\False}{\textsf{False}}
\newcommand{\F}{\textsf{F}}
\newcommand{\act}{\rotatebox[origin=c]{-180}{\(\,\circlearrowright\,\)}}

\usepackage{physics}
\usepackage{complexity}
\usepackage{gensymb}

\DeclareMathOperator{\id}{id}
\DeclareMathOperator{\Span}{span}
\DeclareMathOperator{\supp}{supp}
\DeclareMathOperator{\vol}{vol}
\DeclareMathOperator{\dist}{dist}
\DeclareMathOperator{\diam}{diam}
\DeclareMathOperator{\im}{Im}
\DeclareMathOperator{\sgn}{sgn}
\DeclareMathOperator{\Int}{Int}
\DeclareMathOperator{\diag}{diag}
\DeclareMathOperator{\dom}{dom}
\DeclareMathOperator*{\argmax}{arg\,max}
\DeclareMathOperator*{\argmin}{arg\,min}
\DeclareMathOperator{\Var}{Var}
\DeclareMathOperator{\Cov}{Cov}

\let\implies\Rightarrow
\let\impliedby\Leftarrow
\let\iff\Leftrightarrow

\usepackage{stmaryrd} % for \lightning
\newcommand\conta{\scalebox{1.1}{\(\lightning\)}}

\usepackage{bm}
\usepackage{bbm}

% figure support
\usepackage{import}
\usepackage{xifthen}
\usepackage{pdfpages}
\usepackage{transparent}
\newcommand{\incfig}[1]{%
  \def\svgwidth{\columnwidth}
  \import{./Figures/}{#1.pdf_tex}
}

\usetikzlibrary{arrows}% only needed for the arrow tip stealth'
\newcommand\tikznode[3][]{%
  \tikz[remember picture,baseline=(#2.base)]
  \node[minimum size=0pt,inner sep=0pt,#1](#2){#3};%
}

\def\env@cases{%
  \let\@ifnextchar\new@ifnextchar
  \left\lbrace
  \def\arraystretch{1.2}%
  \array{l@{\quad}l@{}}% Formerly @{}l@{\quad}l@{}
}


\usepackage{amsmath, amsfonts, mathtools, amsthm, amssymb}
\usepackage{geometry}
\usepackage{mathrsfs}
\usepackage{cancel}
\usepackage{systeme}

\usepackage[inline, shortlabels]{enumitem}
\usepackage{multicol}
\setlength\multicolsep{0pt}

\usepackage{caption}
\captionsetup{belowskip=0pt}
\geometry{a4paper,left=2.54cm,right=2.54cm,top=2.54cm,bottom=2.54cm}

% for the big braces
\usepackage{bigdelim}

% correct
\definecolor{correct}{HTML}{009900}
\newcommand\correct[2]{\ensuremath{\:}{\color{red}{#1}}\ensuremath{\to }{\color{correct}{#2}}\ensuremath{\:}}
\newcommand\green[1]{{\color{correct}{#1}}}

% hide parts
\newcommand\hide[1]{}


% tikz
\usepackage{tikz}
\usetikzlibrary{intersections, angles, quotes, positioning}
\usetikzlibrary{arrows.meta}
\usepackage{pgfplots}
\pgfplotsset{compat=1.13}


\tikzset{
    force/.style={thick, {Circle[length=2pt]}-stealth, shorten <=-1pt}
}

% Algorithm Env
\usepackage[linesnumbered,lined,vlined,ruled,commentsnumbered,resetcount,algochapter]{algorithm2e}
\SetKwComment{Comment}{// }{}
\SetArgSty{textsl}
\def\algocflineautorefname{Algorithm}
\counterwithin{algocfline}{chapter}

% Clear out the item autoref name
\makeatletter
\def\itemautorefname{\@gobble}
\makeatother

% theorems
\makeatother
\usepackage{thmtools}
\usepackage[framemethod=TikZ]{mdframed}

\mdfsetup{skipabove=1em,skipbelow=0em}

\theoremstyle{definition}

\declaretheoremstyle[
    headfont=\bfseries\sffamily\color{ForestGreen!70!black}, bodyfont=\normalfont,
    mdframed={
            linewidth=2pt,
            rightline=false, topline=false, bottomline=false,
            linecolor=ForestGreen, backgroundcolor=ForestGreen!5,
            nobreak=false
        }
]{thmgreenbox}

\declaretheoremstyle[
    headfont=\bfseries\sffamily\color{ForestGreen!70!black}, bodyfont=\normalfont,
    mdframed={
            linewidth=2pt,
            rightline=false, topline=false, bottomline=false,
            linecolor=ForestGreen, backgroundcolor=ForestGreen!8,
            nobreak=false
        }
]{thmgreen2box}

\declaretheoremstyle[
    headfont=\bfseries\sffamily\color{NavyBlue!70!black}, bodyfont=\normalfont,
    mdframed={
            linewidth=2pt,
            rightline=false, topline=false, bottomline=false,
            linecolor=NavyBlue, backgroundcolor=NavyBlue!5,
            nobreak=false
        }
]{thmbluebox}

\declaretheoremstyle[
    headfont=\bfseries\sffamily\color{TealBlue!70!black}, bodyfont=\normalfont,
    mdframed={
            linewidth=2pt,
            rightline=false, topline=false, bottomline=false,
            linecolor=TealBlue,
            nobreak=false
        }
]{thmblueline}

\declaretheoremstyle[
    headfont=\bfseries\sffamily\color{RawSienna!70!black}, bodyfont=\normalfont,
    mdframed={
            linewidth=2pt,
            rightline=false, topline=false, bottomline=false,
            linecolor=RawSienna, backgroundcolor=RawSienna!5,
            nobreak=false
        }
]{thmredbox}

\declaretheoremstyle[
    headfont=\bfseries\sffamily\color{RawSienna!70!black}, bodyfont=\normalfont,
    mdframed={
            linewidth=2pt,
            rightline=false, topline=false, bottomline=false,
            linecolor=RawSienna, backgroundcolor=RawSienna!8,
            nobreak=false
        }
]{thmred2box}

\declaretheoremstyle[
    headfont=\bfseries\sffamily\color{SeaGreen!70!black}, bodyfont=\normalfont,
    mdframed={
            linewidth=2pt,
            rightline=false, topline=false, bottomline=false,
            linecolor=SeaGreen, backgroundcolor=SeaGreen!2,
            nobreak=false
        }
]{thmgreen3box}

\declaretheoremstyle[
    headfont=\bfseries\sffamily\color{WildStrawberry!70!black}, bodyfont=\normalfont,
    mdframed={
            linewidth=2pt,
            rightline=false, topline=false, bottomline=false,
            linecolor=WildStrawberry, backgroundcolor=WildStrawberry!5,
            nobreak=false
        }
]{thmpinkbox}

\declaretheoremstyle[
    headfont=\bfseries\sffamily\color{MidnightBlue!70!black}, bodyfont=\normalfont,
    mdframed={
            linewidth=2pt,
            rightline=false, topline=false, bottomline=false,
            linecolor=MidnightBlue, backgroundcolor=MidnightBlue!5,
            nobreak=false
        }
]{thmblue2box}

\declaretheoremstyle[
    headfont=\bfseries\sffamily\color{Gray!70!black}, bodyfont=\normalfont,
    mdframed={
            linewidth=2pt,
            rightline=false, topline=false, bottomline=false,
            linecolor=Gray, backgroundcolor=Gray!5,
            nobreak=false
        }
]{notgraybox}

\declaretheoremstyle[
    headfont=\bfseries\sffamily\color{Gray!70!black}, bodyfont=\normalfont,
    mdframed={
            linewidth=2pt,
            rightline=false, topline=false, bottomline=false,
            linecolor=Gray,
            nobreak=false
        }
]{notgrayline}

% \declaretheoremstyle[
% 	headfont=\bfseries\sffamily\color{RawSienna!70!black}, bodyfont=\normalfont,
% 	numbered=no,
% 	mdframed={
% 			linewidth=2pt,
% 			rightline=false, topline=false, bottomline=false,
% 			linecolor=RawSienna, backgroundcolor=RawSienna!1,
% 		},
% 	qed=\qedsymbol
% ]{thmproofbox}

\declaretheoremstyle[
    headfont=\bfseries\sffamily\color{NavyBlue!70!black}, bodyfont=\normalfont,
    numbered=no,
    mdframed={
            linewidth=2pt,
            rightline=false, topline=false, bottomline=false,
            linecolor=NavyBlue, backgroundcolor=NavyBlue!1,
            nobreak=false
        }
]{thmexplanationbox}

\declaretheoremstyle[
    headfont=\bfseries\sffamily\color{WildStrawberry!70!black}, bodyfont=\normalfont,
    numbered=no,
    mdframed={
            linewidth=2pt,
            rightline=false, topline=false, bottomline=false,
            linecolor=WildStrawberry, backgroundcolor=WildStrawberry!1,
            nobreak=false
        }
]{thmanswerbox}

\declaretheoremstyle[
    headfont=\bfseries\sffamily\color{Violet!70!black}, bodyfont=\normalfont,
    mdframed={
            linewidth=2pt,
            rightline=false, topline=false, bottomline=false,
            linecolor=Violet, backgroundcolor=Violet!1,
            nobreak=false
        }
]{conjpurplebox}

\declaretheorem[style=thmgreenbox, name=Definizione, numberwithin=section]{definizione}
\declaretheorem[style=thmgreen2box, name=Definizione, numbered=no]{definizione*}
\declaretheorem[style=thmredbox, name=Teorema, numberwithin=section]{teorema}
\declaretheorem[style=thmred2box, name=Teorema, numbered=no]{teorema*}
\declaretheorem[style=thmredbox, name=Lemma, numberwithin=section]{lemma}
\declaretheorem[style=thmredbox, name=Proposizione, numberwithin=section]{proposizione}
\declaretheorem[style=thmredbox, name=Corollario, numberwithin=section]{corollario}
\declaretheorem[style=thmpinkbox, name=Problema, numberwithin=section]{problema}
\declaretheorem[style=thmpinkbox, name=Problema, numbered=no]{problema*}
\declaretheorem[style=thmblue2box, name=Affermazione, numbered=no]{affermazione}
\declaretheorem[style=conjpurplebox, name=Congettura, numberwithin=section]{congettura}

% Redefine proof environment to get a full control. 
\makeatletter
\renewenvironment{proof}[1][\proofname]{\par
    \pushQED{\qed}%
    \normalfont \topsep-2\p@\@plus6\p@\relax
    \trivlist
    \item[\hskip\labelsep
                \color{RawSienna!70!black}\sffamily\bfseries
                #1\@addpunct{.}]\ignorespaces
    \begin{mdframed}[linewidth=2pt,rightline=false, topline=false, bottomline=false,linecolor=RawSienna, backgroundcolor=RawSienna!1]
        }{%
        \popQED\endtrivlist\@endpefalse
    \end{mdframed}
}
\makeatother

\declaretheorem[style=thmbluebox, numbered=no, name=Esempio]{es}
\declaretheorem[style=thmexplanationbox, numbered=no, name=Dimostrazione]{dimostrazione}
\declaretheorem[style=thmexplanationbox, numbered=no, name=Spiegazione]{spiegazione}
\newenvironment{spiegaz}[1][]{\vspace{-10pt}\pushQED{\(\circledast\)}\begin{spiegazione}}{\null\hfill\popQED\end{spiegazione}}

\declaretheorem[style=thmblueline, numbered=no, name=Osservazione]{osservazione}
\declaretheorem[style=thmblueline, numbered=no, name=Nota]{nota}
\declaretheorem[style=thmpinkbox, numbered=no, name=Esercizio]{esercizio}
\declaretheorem[style=notgrayline, numbered=no, name=Come visto prima]{prima}
\declaretheorem[style=thmgreen3box, numbered=no, name=Intuizione]{intuizione}
\declaretheorem[style=notgraybox, numbered=no, name=Notazione]{notazione}
\declaretheorem[style=thmanswerbox, numbered=no, name=Risposta]{tmprisposta}
\newenvironment{risposta}[1][]{\vspace{-10pt}\pushQED{\(\circledast\)}\begin{tmprisposta}}{\null\hfill\popQED\end{tmprisposta}}


\usepackage{etoolbox}
\renewcommand{\qed}{\null\hfill\(\blacksquare\)}

\makeatletter

\def\testdateparts#1{\dateparts#1\relax}
\def\dateparts#1 #2 #3 #4 #5\relax{
    \marginpar{\small\textsf{\mbox{#1 #2 #3 #5}}}
}

\def\@lecture{}%
\newcommand{\lecture}[3]{
    \ifthenelse{\isempty{#3}}{%
        \def\@lecture{Lecture #1}%
    }{%
        \def\@lecture{Lecture #1: #3}%
    }%
    \section*{\@lecture}
    \marginpar{\small\textsf{\mbox{#2}}}
}
\usepackage{pgffor}%
\newcommand{\lec}[2]{%
    \foreach \c in {#1,...,#2}{%
            \IfFileExists{Lectures/lec_\c.tex} {%
                \input{Lectures/lec_\c.tex}%
            }{}%
        }%
}

\newenvironment{attenzione}{\par\small % smaller font size
	\begin{list}{}{
			\leftmargin=35pt % Indentation on the left
			\rightmargin=25pt}\item\ignorespaces % Indentation on the right
		\makebox[-2.5pt]{\begin{tikzpicture}[overlay, baseline=(current bounding box)]
			\node[draw=cPrim!85,line width=1pt,circle,fill=cPrim!25,font=\sffamily\bfseries,inner sep=2pt,outer sep=0pt] at (-15pt,4pt){\textcolor{cPrim}{\hspace{-2pt}\textbf{!}}};\end{tikzpicture}} % Orange R in a circle
		\advance\baselineskip -1pt}{\end{list}} % Tighter line spacing

% Style for sections
%\renewcommand{\thesection}{\normalsize\Roman{section}}
\definecolor{cPrim}{RGB}{15, 115, 145}
\usepackage{titlesec}
\titleformat{\section}%
{\sffamily\bfseries\Large\color{White!0!cPrim}}% format applied to label+text
{\llap{\colorbox{White!0!cPrim}{\parbox{1.25cm}{\hfill\color{white}\thesection}}}}% label
{1em}% horizontal separation between label and title body
{}% before the title body
[]% after the title body

% subsection format
\renewcommand{\thesubsection}{\thesection.\arabic{section}}
\titleformat{\subsection}%
{\sffamily\bfseries\large\color{White!37!cPrim}}% format applied to label+text
{\llap{\colorbox{White!37!cPrim}{\parbox{1.25cm}{\hfill\color{white}\thesubsection}}}}% label
{1em}% horizontal separation between label and title body
{}% before the title body
[]% after the title body

% fancy headers
%\usepackage{fancyhdr}
%\pagestyle{fancy}

% LE: left even
% RO: right odd
% CE, CO: center even, center odd
%\fancyhead[LE,RO]{Pingbang Hu}

%\fancyhead[RO,LE]{\@lecture} % Right odd,  Left even
%\fancyhead[RE,LO]{}          % Right even, Left odd
%\fancyfoot[RO,LE]{\thepage}  % Right odd,  Left even
%\fancyfoot[RE,LO]{}          % Right even, Left odd
%\fancyfoot[C]{\leftmark}     % Center

\makeatother

% notes
\usepackage[color=pink]{todonotes}
\usepackage{marginnote}
\let\marginpar\marginnote

% Fix some stuff
% %http://tex.stackexchange.com/questions/76273/multiple-pdfs-with-page-group-included-in-a-single-page-warning

% Appendix environment
\usepackage{appendix}
\def\chapterautorefname{Section}
\def\sectionautorefname{Section}
\def\appendixautorefname{Appendix}
\renewcommand\appendixname{Appendix}
\renewcommand\appendixtocname{Appendix}
\renewcommand\appendixpagename{Appendix}
% begin appendix autoref patch [\autoref subsections in appendix](https://tex.stackexchange.com/questions/149807/autoref-subsections-in-appendix)
\makeatletter
\patchcmd{\hyper@makecurrent}{%
    \ifx\Hy@param\Hy@chapterstring
        \let\Hy@param\Hy@chapapp
    \fi
}{%
    \iftoggle{inappendix}{%true-branch
        % list the names of all sectioning counters here
        \@checkappendixparam{chapter}%
        \@checkappendixparam{section}%
        \@checkappendixparam{subsection}%
        \@checkappendixparam{subsubsection}%
        \@checkappendixparam{paragraph}%
        \@checkappendixparam{subparagraph}%
    }{}%
}{}{\errmessage{failed to patch}}

\newcommand*{\@checkappendixparam}[1]{%
    \def\@checkappendixparamtmp{#1}%
    \ifx\Hy@param\@checkappendixparamtmp
        \let\Hy@param\Hy@appendixstring
    \fi
}
\makeatletter

\newtoggle{inappendix}
\togglefalse{inappendix}

\apptocmd{\appendix}{\toggletrue{inappendix}}{}{\errmessage{failed to patch}}
\apptocmd{\subappendices}{\toggletrue{inappendix}}{}{\errmessage{failed to patch}}
% end appendix autoref patch

\setcounter{tocdepth}{1}
