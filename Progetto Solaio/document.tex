\documentclass[a4paper,12pt, oneside]{book}
%\pdfminorversion=5 
%\pdfcompresslevel=9
%\pdfobjcompresslevel=2
%\pdfminorversion=5
% \usepackage{fullpage}
\usepackage[italian]{babel}
\usepackage[utf8]{inputenc}
\usepackage[document]{ragged2e}
\usepackage{float}
\usepackage{amssymb}
\usepackage{graphicx}
\usepackage[font=small,labelfont=bf]{caption}
\usepackage{csquotes}
\usepackage{amsthm}
\usepackage{graphics}
\usepackage{amsfonts}
\usepackage{amsmath}
\usepackage{amstext}
\usepackage{engrec}
\usepackage{rotating}
\usepackage[safe,extra]{tipa}
\usepackage{tikz,pgfplots}
\usetikzlibrary{positioning}
\usetikzlibrary{calc,through,backgrounds}
\usepackage{stanli}
\usepackage{multirow}
\usepackage{titlesec}
\usepackage{hyperref}
\usepackage{microtype}
\usepackage{enumerate}
\usepackage{braket}
\usepackage{marginnote}
\usepackage{pgfplots}
\usepackage{cancel}
\usepackage{polynom}
\usepackage{caption}
\usepackage{booktabs}
\usepackage{enumitem}
\usepackage{framed}
\usepackage{pdfpages}
\usepackage{pgfplots}
\usepackage{fancyhdr}
\fancyhead[LE,RO]{\slshape \rightmark}
\fancyhead[LO,RE]{\slshape \leftmark}
\fancyfoot[C]{\thepage}

\title{\textbf{Tecnica delle Costruzioni}\\ \textbf{Corso di laurea in ingegneria edile}\\ \textbf{Prof. Ing. Andrea Prota–a.a. 2022/2023}}
\author{Ivano D'Apice\\\\ N41002772}
\date{}

\pgfplotsset{compat=1.13}
\begin{document}
	\maketitle
	
	\definecolor{shadecolor}{gray}{0.80}
	
	\newtheorem{teorema}{Teorema}
	\newtheorem{definizione}{Definizione}
	\newtheorem{esempio}{Esempio}
	\newtheorem{corollario}{Corollario}
	\newtheorem{lemma}{Lemma}
	\newtheorem{osservazione}{Osservazione}
	\newtheorem{nota}{Nota}
	\newtheorem{esercizio}{Esercizio}
	\tableofcontents
	\renewcommand{\chaptermark}[1]{%
		\markboth{\chaptername
			\ \thechapter.\ #1}{}}
	\renewcommand{\sectionmark}[1]{\markright{\thesection.\ #1}}
	
	\chapter{Assegno Solaio}
	    
    \begin{tabbing}
	 Geometria \hspace{10em} \= \hspace{1em} \\
	 $L_1$=  $0,70+0,10\cdot n$              \> n=n.ro lettere del nome    \\
	 $L_2$=  $4,00+0,10\cdot c$              \> c=n.re lettere del cognome \\ 
	 $L_3$=  $4,80+0,10\cdot c-0,10\cdot n$  \>                             
    \end{tabbing}	    

	\begin{figure}[H]
		\centering
		\hspace*{-.5cm}
		\begin{tikzpicture}
			% here we construct our structure
			\scaling{.45};
			%\draw[help lines,step=.45]
			(0,-.9) grid (12.6,1.81);
			\point{a}{2}{1};                                
			\point{b}{26}{1};                               
			\point{c}{14}{1};                               
			\point{d}{0}{1};  
			\beam{2}{d}{b};                                          
			\support{1}{a}[0];                               
			\support{1}{b}[0];                               
			\support{1}{c}[0];        
			\dimensioning{1}{d}{a}{-1}[$L_1$];
			\dimensioning{1}{a}{c}{-1}[$L_2$];
			\dimensioning{1}{c}{b}{-1}[$L_3$];
			\notation{1}{d}{A}[below = 18mm];
			\notation{1}{a}{B}[below = 18mm];
			\notation{1}{c}{C}[below = 18mm];
			\notation{1}{b}{D}[below = 18mm];
		\end{tikzpicture}
		\caption{}
		\label{fig:traveesss1}
	\end{figure}
	
	\begin{figure}[H]
		\hspace*{-.4cm}
		\centering
		\includegraphics[width=0.7\linewidth]{"immagini/misure solaio flessione normale"}
		\caption{Dati numerici in 1 metro di solaio.}
		\label{fig:misure-solaio-flessione-normale}
	\end{figure}
	
	\begin{tabbing}
		Carichi Accidentali$^{[\textbf{1}]}$\label{1} \hspace{10em} \= Matricola pari \hspace{1em} \\
		Sullo Sbalzo $\longrightarrow$    \> $q=5,00kN/m^{2}$    \\
		In Campata   $\longrightarrow$    \> $q=3,50kN/m^{2}$                     
	\end{tabbing}	
	
	
	\chapter{Analisi dei carichi}
	
	Consideriamo due tipi di carico: $Q$ e $G$. I carichi di tipo $Q$ si dicono \textbf{variabili}, mentre quelli di tipo $G$ \textbf{permanenti}. Differenziamo poi i carichi $G$ in \textbf{permanenti strutturali} $G_1$ e \textbf{permanenti non strutturali} $G_2$.
    \leavevmode\newline
    \leavevmode\newline
	Si ricorda che verrà fatta una verifica rispetto allo \textbf{S.L.U} (Stati Limite Ultimo), tenendo conto dello \textbf{S.L.E} (Stato Limite di Esercizio) per quanto riguarda il dimensionamento del solaio.

	Dati:
	\begin{tabbing}
		\hspace{14em} \= \hspace{2em} \= \hspace{6em} \= \hspace{2em} \= \hspace{2em} \\ 
		$L_1$=  $0,70+0,10\cdot n$              \> = \> $0,70+0,50$ \> = \> $\textbf{1,20m}$  \\
		$L_2$=  $4,30+0,10\cdot c$              \> = \> $4,30+0,60$ \> = \> $\textbf{4,90m}$  \\
		$L_3$=  $4,80+0,10\cdot c-0,10\cdot n$  \> = \> $4,80+0,10$ \> = \> $\textbf{4,90m}$                     
	\end{tabbing}	
	\leavevmode\newline
	Utilizziamo la luce maggiore ($L_2=L_3$) per calcolare l'altezza del solaio grazie allo S.L.E. Avremo che $\textbf{H=}\frac{\textbf{L}}{\textbf{20}}$ e quindi $H=\frac{490cm}{20}=24,50cm\sim\textbf{25,00cm}$.$^{[\textbf{2}]}$\label{2}
    \leavevmode\newline
    \leavevmode\newline
	Come da progetto [\ref{fig:misure-solaio-flessione-normale}] avremo $\textbf{H}_{sbalzo}=H-4,00cm=25,00cm-4,00cm=\textbf{21,00cm}$.
	
	\section{Carichi strutturali permanenti $\textbf{G}_1$}
	
	\begin{tabbing}
		Materiale\hspace{2em} \= h (m)\hspace{3em} \= L (m)\hspace{3em} \= $G_1$ (kN/m$^3$)\hspace{3em} \= $G_1$ (kN/m$^2$)\hspace{1em}\\\\
		Soletta  \> 0,05 \> 1,00           \> 25,00           \> 1,25  \\
		Travetti \> 0,20 \> 0,10$\cdot$2   \> 25,00           \> 1,00  \\ 
		Laterizi$^{[\textbf{2.1}]}$\label{3} \> 0,20 \> 0,40$\cdot$2 \> \phantom{0}6,00 \> 	0,96                                    
	\end{tabbing}	    
	
	Totale $\textbf{G}_1=(1,25+1,00+0,96)kN/m^2=3,21kN/m^2$ 

    \section{Carichi permanenti non strutturali $\textbf{G}_2$}
    
    \begin{tabbing}
    	Materiale\hspace{2em} \= h (m)\hspace{3em} \= L (m)\hspace{3em} \= $G_1$ (kN/m$^3$)\hspace{3em} \= $G_1$ (kN/m$^2$)\hspace{1em}\\\\
    	Massetto  \> 0,60 \> 1,00 \> 16,00           \> 0,96  \\
    	Pavimento \> 0,01 \> 1,00 \> 16,00           \> 0,18  \\ 
    	Intonaco  \> 0,01 \> 1,00 \> 18,00           \> 0,18                                    
    \end{tabbing}	    
    
    Totale $\textbf{G}_2=(0,96+0,18+0,18)kN/m^2=1,32kN/m^2$ 
    
	\chapter{Note}
	[\ref{1}] I valori di carico accidentale in situazione normale sono $q=4.00kN/m^{2}$ e $q=2.00kN/m^{2}$ rispettivamente per lo sbalzo e campata. I valori usati in esercizio sono puramente didattici.
	\leavevmode\newline
	\leavevmode\newline
	[\ref{2}] Considerando che una pignatta non è alta meno di 12 cm, l’altezza minima del solaio è comunque di 17 cm.
	\leavevmode\newline
	\leavevmode\newline
	[\ref{3}] Il peso specifico dei blocchi di allegerimento in laterizio è stato ricavato dalle tabelle dei pesi specifici di normativa, considerando una percentuale di foratura pari al 67\textdiscount ( 18 $\cdot$ [ 1-0,67 ] ) = 5,94 -> 6,00 KN/m$^3$.
	
\end{document}