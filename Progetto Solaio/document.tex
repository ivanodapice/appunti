\documentclass[a4paper,12pt, oneside]{book}
%\pdfminorversion=5 
%\pdfcompresslevel=9
%\pdfobjcompresslevel=2
%\pdfminorversion=5
% \usepackage{fullpage}
\usepackage[italian]{babel}
\usepackage[utf8]{inputenc}
\usepackage{float}
\usepackage{amssymb}
\usepackage{graphicx}
\usepackage[font=small,labelfont=bf]{caption}
\usepackage{csquotes}
\usepackage{amsthm}
\usepackage{graphics}
\usepackage{amsfonts}
\usepackage{amsmath}
\usepackage{amstext}
\usepackage{engrec}
\usepackage{rotating}
\usepackage[safe,extra]{tipa}
\usepackage{tikz,pgfplots}
\usetikzlibrary{positioning}
\usetikzlibrary{calc,through,backgrounds}
\usepackage{stanli}
\usepackage{multirow}
\usepackage{titlesec}
\usepackage{hyperref}
\usepackage{microtype}
\usepackage{enumerate}
\usepackage{braket}
\usepackage{marginnote}
\usepackage{pgfplots}
\usepackage{cancel}
\usepackage{polynom}
\usepackage{caption}
\usepackage{booktabs}
\usepackage{enumitem}
\usepackage{framed}
\usepackage{pdfpages}
\usepackage{pgfplots}
\usepackage{fancyhdr}
\fancyhead[LE,RO]{\slshape \rightmark}
\fancyhead[LO,RE]{\slshape \leftmark}
\fancyfoot[C]{\thepage}

\title{\textbf{Tecnica delle Costruzioni}\\ \textbf{Corso di laurea in ingegneria edile}\\ \textbf{Prof. Ing. Andrea Prota– a.a. 2022/2023}}
\author{Ivano D'Apice\\\\ N41002772}
\date{}

\pgfplotsset{compat=1.13}
\begin{document}
	\maketitle
	
	\definecolor{shadecolor}{gray}{0.80}
	
	\newtheorem{teorema}{Teorema}
	\newtheorem{definizione}{Definizione}
	\newtheorem{esempio}{Esempio}
	\newtheorem{corollario}{Corollario}
	\newtheorem{lemma}{Lemma}
	\newtheorem{osservazione}{Osservazione}
	\newtheorem{nota}{Nota}
	\newtheorem{esercizio}{Esercizio}
	\tableofcontents
	\renewcommand{\chaptermark}[1]{%
		\markboth{\chaptername
			\ \thechapter.\ #1}{}}
	\renewcommand{\sectionmark}[1]{\markright{\thesection.\ #1}}
	
	\chapter{Assegno Solaio}
	    
    \begin{tabbing}
	 Geometria \hspace{10em} \= \hspace{1em} \\
	 $L_1$=  $0.70+0.10\cdot n$              \> n=n.ro lettere del nome    \\
	 $L_2$=  $4.30+0.10\cdot c$              \> c=n.re lettere del cognome \\ 
	 $L_3$=  $4.80+0.10\cdot c-0.10\cdot n$  \>                             
    \end{tabbing}	    

	\begin{figure}[H]
		\centering
		\hspace*{1cm}
		\begin{tikzpicture}
			% here we construct our structure
			\scaling{.45};
			%\draw[help lines,step=.45]
			(0,-.9) grid (12.6,1.81);
			\point{a}{2}{1};                                
			\point{b}{26}{1};                               
			\point{c}{14}{1};                               
			\point{d}{0}{1};  
			\beam{2}{d}{b};                                          
			\support{1}{a}[0];                               
			\support{1}{b}[0];                               
			\support{1}{c}[0];        
			\dimensioning{1}{d}{a}{-1}[$L_1$];
			\dimensioning{1}{a}{c}{-1}[$L_2$];
			\dimensioning{1}{c}{b}{-1}[$L_3$];
			\notation{1}{d}{A}[below = 18mm];
			\notation{1}{a}{B}[below = 18mm];
			\notation{1}{c}{C}[below = 18mm];
			\notation{1}{b}{D}[below = 18mm];
		\end{tikzpicture}
		\caption{}
		\label{fig:traveesss1}
	\end{figure}
	
\end{document}