	\chapter{\phantom{title}}

\lettrine{D}{al} leggio del refettorio, il lettore stava intonando gli annunci. La
luce delle candele sbiancava i visi dei legionari in tonaca ritti,
immobili, dietro gli sgabelli, in attesa dell\textquotesingle inizio del
pasto serale. La voce del lettore echeggiò nella sala da pranzo
dall\textquotesingle alta volta, il cui soffitto si perdeva nelle ombre
che ondeggiavano al di sopra delle chiazze di luce sulle tavole di
legno.

--- Il Reverendo Padre Abate mi ha comandato di annunciare --- esclamò
il lettore --- che la regola dell\textquotesingle astinenza è soppressa,
per il pasto di questa sera. Avremo ospiti, come forse avete udito.
Tutti i religiosi potranno prendere parte al banchetto di questa sera in
onore del Thon Taddeo e del suo seguito; potrete mangiare carne. Sarà
permessa la conversazione durante il pasto, se sarà una conversazione
tranquilla.

Rumori vocali soffocati, non molto diversi da applausi repressi, si
levarono dalle file dei novizi. Le tavole erano apparecchiate. Il cibo
non aveva ancora fatto la sua apparizione, ma grandi vassoi sostituivano
le solite tazze, aguzzando gli appetiti con allusioni a un festino. I
familiari bricchi per il latte rimasero nella dispensa, e il loro posto
fu preso, per quella sera, dalle migliori coppe di vino. Sulle tavole
erano state sparse rose.

L\textquotesingle abate si fermò nel corridoio, aspettando che il
lettore avesse finito. Guardò la tavola apparecchiata per lui, per padre
Gault, per l\textquotesingle onorato ospite e il suo seguito. In cucina
avevano ancora sbagliato a fare i conti, pensò. Era stato apparecchiato
per otto persone. Tre ufficiali, il Thon e il suo assistente, e i due
religiosi facevano in totale sette\ldots{} a meno che, per un caso
improbabile, padre Gault non avesse invitato frate Kornhoer a sedere con
loro. Il lettore concluse gli annunci, e don Paulo entrò nella sala.

--- \emph{Flectamus genua} --- intonò il lettore.

Le legioni in tonaca si inginocchiarono con precisione militare, mentre
l\textquotesingle abate benediceva il suo gregge.

--- \emph{Levate.}

Le legioni si alzarono. Don Paulo prese posto alla tavola speciale e
lanciò uno sguardo verso la porta. Gault avrebbe guidato gli altri. In
precedenza i pasti degli ospiti erano stati serviti nella foresteria
invece che nel refettorio, per evitare di assoggettarli
all\textquotesingle austerità della frugale dieta dei monaci.

Quando gli ospiti giunsero, don Paulo si guardò intorno per cercare
frate Kornhoer, ma il monaco non era fra loro.

--- Per chi è apparecchiato l\textquotesingle ottavo posto? --- mormorò
a padre Gault, quando tutti si furono seduti.

Gault lo guardò senza capire e alzò le spalle.

Lo studioso prese posto alla destra dell\textquotesingle abate e gli
altri si disposero in ordine tutto intorno, lasciando libero il posto
alla sua sinistra. Si voltò per fare cenno a Kornhoer di unirsi a loro,
ma il lettore cominciò a intonare il prefazio prima che egli potesse
attirare l\textquotesingle attenzione del monaco.

--- \emph{Oremus} --- rispose l\textquotesingle abate, e le legioni si
inchinarono.

Durante la benedizione, qualcuno si insinuò senza far rumore nel sedile
alla sinistra dell\textquotesingle abate. L\textquotesingle abate si
accigliò ma non alzò lo sguardo per identificare il colpevole prima che
la preghiera fosse conclusa.

---\ldots{} \emph{et Spiritus Sancti, Amen.}

--- Sedete --- disse il lettore, e i monaci cominciarono a sedersi.

L\textquotesingle abate lanciò un\textquotesingle occhiata tagliente
alla figura alla sua sinistra.

--- Poeta!

Il cane bastonato si inchinò in modo stravagante e sorrise. ---
Buonasera, signori, dotto Thon, onorevoli ospiti --- recitò. --- Cosa ci
sarà servito, questa sera? Pesce arrostito e favi di miele in onore
della resurrezione temporale che è imminente? O forse, Monsignor Abate,
avete finalmente fatto cucinare l\textquotesingle oca del podestà del
villaggio?

--- Mi piacerebbe cucinare\ldots{}

--- Ah! --- disse il Poeta, e si voltò affabilmente verso lo studioso.
--- Di una tale eccellenza culinaria si gode in questo posto, Thon
Taddeo! Dovreste unirvi più spesso a noi. Immagino che nella foresteria
vi cibino soltanto di fagiano arrostito e di bue privo di immaginazione.
Una vergogna! Qui si mangia meglio. Spero che il frate cuoco abbia avuto
il suo solito gusto, questa sera, la sua solita fiamma interiore, il suo
tocco incantato. Ah\ldots{} --- Il Poeta si fregò le mani e sorrise con
aria famelica. --- Forse ci serviranno Maiale Finto con Salsa alla frate
John, eh?

--- Sembra interessante --- disse lo studioso. --- Che
cos\textquotesingle è?

--- Armadillo grasso, con grano secco, bollito in latte
d\textquotesingle asina. Un piatto speciale in uso la domenica.

--- Poeta! --- insorse l\textquotesingle abate; e poi rivolto al Thon:
--- Chiedo scusa per la sua presenza. Non era stato invitato.

Lo studioso osservò il Poeta con distaccato divertimento. --- Anche
monsignor Hannegan tiene parecchi buffoni alla sua corte --- disse a don
Paulo. --- Conosco bene questa specie. Non avete bisogno di scusarvi per
lui.

Il Poeta schizzò dal suo sgabello e si inchinò profondamente al Thon..
--- Permettetemi invece di scusarmi per l\textquotesingle abate,
signore! --- gridò, sentitamente.

Rimase inchinato per un attimo. Gli altri attesero che ponesse fine alle
sue buffonerie. Invece scrollò improvvisamente le spalle, sedette e
infilzò un pollo fumante nel piatto posto davanti a loro da un
postulante. Ne strappò una coscia e l\textquotesingle addentò, con
gusto. Gli altri lo osservarono, perplessi.

--- Immagino che abbiate ragione, non accettando le mie scuse per lui
--- disse finalmente al Thon.

Lo studioso arrossì lievemente.

Prima che io vi butti fuori di qui, verme --- disse Gault --- sondiamo
la profondità di questa iniquità.

Il Poeta scosse il capo e masticò, pensieroso. --- È molto profonda,
veramente --- ammise.

--- ``Un giorno o l\textquotesingle altro, Gault si impiccherà'' pensò
don Paulo.

Ma l\textquotesingle ecclesiastico più giovane era visibilmente seccato,
e voleva portare l\textquotesingle incidente \emph{ad absurdum} per
trovare il terreno adatto per schiacciare quel pazzo. --- Scusatevi
completamente per il vostro ospite, Poeta --- ordinò. --- E spiegatevi
bene, mentre lo fate.

--- Lasciate perdere, padre, lasciate perdere --- disse in fretta don
Paulo.

Il Poeta sorrise garbatamente all\textquotesingle abate. --- Sta bene
così, monsignore --- disse. --- Non mi dispiace minimamente scusarmi per
voi. Voi vi siete scusato per me, io mi scuso per voi, non è questa una
adeguata manovra di carità e di buona volontà? Nessuno deve scusarsi per
se stesso\ldots{} il che è sempre così umiliante. Usando il mio sistema,
tuttavia, chiunque viene scusato, e nessuno deve formulare le proprie
scuse.

Soltanto gli ufficiali sembravano considerare divertenti le osservazioni
del Poeta. A quanto pareva, aspettarsi un divertimento era sufficiente
per produrre un\textquotesingle illusione di divertimento, e il
commediante poteva ottenere una risata con un gesto o
un\textquotesingle espressione, indipendentemente da ciò che diceva. Il
Thon Taddeo esibiva un sorriso asciutto, ma il suo sguardo era quello
che un uomo può dedicare a una goffa esibizione d\textquotesingle un
animale ammaestrato.

--- E così --- continuò il Poeta --- se volete permettermi di servirvi
come umile aiutante, monsignore, non dovrete mai mangiare il vostro
corvo. Come vostro Avvocato addetto alle Scuse, per esempio, potrei
essere da voi delegato a offrire contrizioni agli ospiti importanti per
la presenza delle cimici. E alle cimici per il brusco cambiamento di
dieta.

L\textquotesingle abate si corrucciò, e resistette a fatica
all\textquotesingle impulso di schiacciare il piede nudo del Poeta con
il tacco dei suo sandalo. Sferrò un calcio negli stinchi del pazzo, ma
quello insistette.

--- Io mi assumerei tutto il biasimo per voi naturalmente --- disse,
masticando rumorosamente la carne bianca. --- È uno splendido sistema,
che intendevo mettere anche a vostra disposizione, Eminentissimo
Studioso. Ero sicuro che lo avreste trovato conveniente. Mi è dato
comprendere che devono essere escogitati e perfezionati sistemi di
logica e di metodologia, prima che la scienza progredisca. E il mio
sistema di scuse negoziabili e trasferibili sarebbe stato
particolarmente prezioso per voi, Thon Taddeo.

--- Davvero?

--- Sì. È un peccato. Qualcuno mi ha rubato la mia capra dalla testa
azzurra.

--- Una capra dalla testa azzurra?

--- Aveva una testa calva come quella di Hannegan, Vostro Splendore, e
azzurra come la punta del naso di frate Armbruster. Volevo farvi un
presente di quell\textquotesingle animale, ma qualche malandrino me
l\textquotesingle ha rubata, prima che voi arrivaste.

L\textquotesingle abate serrò i denti e posò il tacco sul piede del
Poeta. Thon Taddeo aveva corrugato lievemente la fronte, ma sembrava
deciso a sbrogliare l\textquotesingle oscuro groviglio delle parole del
Poeta.

--- Abbiamo bisogno di una capra dalla testa azzurra? --- chiese al
segretario.

--- Non riesco a vederne alcuna pressante urgenza, signore --- disse il
segretario.

--- Ma questo bisogno è ovvio! --- disse il Poeta. --- Dicono che voi
scrivete equazioni che un giorno ricostruiranno il mondo. Dicono che una
nuova luce stia per sorgere. Se vi dovrà essere la luce, allora qualcuno
dovrà essere biasimato per l\textquotesingle oscurità che è passata.

--- Ah, e quindi è necessaria la capra. --- Il Thon Taddeo guardò
l\textquotesingle abate. --- Una battuta disgustosa. È il meglio che sa
fare?

--- Noterete che è disoccupato. Ma parliamo di qualcosa di più
sens\ldots{}

--- No, no, no, no! --- obiettò il Poeta. --- Voi avete frainteso ciò
che intendevo dire, Vostro Splendore. La capra dovrà essere accolta in
un tempio e onorata, non biasimata! Incoronatela con la corona che vi ha
mandato san Leibowitz, e ringraziatela per la luce che sorge. Poi
biasimate Leibowitz, e cacciate \emph{lui} nel deserto. In questo modo,
voi non dovrete portare la seconda corona. Quella di spine.
Responsabilità, è chiamata.

L\textquotesingle ostilità del Poeta aveva rotto gli argini; non cercava
più di sembrare divertente. Il Thon lo fissò, gelidamente. Il tacco
dell\textquotesingle abate ondeggiò sul piede del Poeta, e ancora ne
ebbe riluttante misericordia.

--- E quando --- disse il Poeta --- l\textquotesingle esercito del
vostro protettore verrà per impadronirsi di questa abbazia, la capra
potrà essere posta nel cortile e istruita a belare ``Qui non
c\textquotesingle è nessuno tranne me, qui non c\textquotesingle è
nessuno tranne me'' ogni volta che si presenti uno straniero.

Uno degli ufficiali si levò dallo sgabello con un grugnito collerico; la
sua mano si posò per riflesso sulla sciabola. Sollevò
l\textquotesingle impugnatura, e sei centimetri
d\textquotesingle acciaio scintillarono un avvertimento al Poeta. Il
Thon afferrò il polso dell\textquotesingle ufficiale e cercò di
ricacciare la lama nel fodero, ma era come spingere il braccio
d\textquotesingle una statua di marmo.

--- Ah! Uno spadaccino, non soltanto un disegnatore! --- schernì il
Poeta, che a quanto pareva non aveva paura di morire. --- I vostri
disegni delle difese dell\textquotesingle abbazia mostrano una tale
promessa di artistiche\ldots{}

L\textquotesingle ufficiale latrò una bestemmia e la lama uscì
completamente dal fodero. I suoi due compagni
l\textquotesingle afferrarono, tuttavia, prima che potesse scattare. Un
rombo attonito si levò dalla congregazione, mentre i monaci sbalorditi
si alzavano. Il Poeta continuava a sorridere, blandamente.

---\ldots{} di artistica evoluzione --- continuò. --- Io predico che un
giorno il vostro disegno delle gallerie che si aprono sotto le mura
verrà appeso in un museo di belle\ldots{}

Un tonfo sordo venne di sotto la tavola. Il Poeta si interruppe a metà
di un morso, abbassò l\textquotesingle osso dalla bocca, e impallidì,
lentamente. Masticò, deglutì, e continuò a impallidire. Guardò
distrattamente verso il soffitto.

--- Me lo state spiaccicando --- brontolò, con un angolo della bocca.

--- Avete finito di parlare? --- chiese l\textquotesingle abate, e
continuò a schiacciargli il piede.

--- Credo di avere un osso in gola --- ammise il Poeta.

--- Volete essere scusato?

--- Temo che sia necessario.

--- Che peccato. Ci mancherete molto. --- Paulo diede una ultima
schiacciata al piede, per buona misura. --- Potete andare, allora.

Il Poeta espirò, si asciugò la bocca, e si alzò. Vuotò la coppa di vino
e la rovesciò, al centro del vassoio. Qualcosa, nei suoi modi, costrinse
tutti a guardarlo. Abbassò le palpebre con un pollice; chinò la testa
sulla palma piegata in forma di coppa e premette.
L\textquotesingle occhio di vetro gli cadde nella palma, strappando un
suono soffocato ai texarkani, i quali, a quanto pareva, non sapevano che
il Poeta avesse un occhio artificiale.

--- Sorveglialo attentamente --- disse il Poeta
all\textquotesingle occhio di vetro, e poi lo depose nella base
rovesciata della coppa, dove rimase fissando severamente il Thon Taddeo.
--- Buonasera, signori miei --- disse allegramente al gruppo, e se ne
andò.

L\textquotesingle ufficiale mormorò furibondo una imprecazione e si
dibatté per liberarsi dalla stretta dei suoi compagni.

--- Riconducetelo al suo alloggio e stategli vicino fino a che si sarà
calmato --- disse il Thon. --- E badate bene che non si avvicini a quel
pazzo.

--- Sono mortificato --- disse poi all\textquotesingle abate, mentre
l\textquotesingle ufficiale, livido, veniva trascinato via. --- Non sono
miei servitori, e non posso dar loro ordini. Ma posso promettervi che si
pentirà di questo. E se rifiuta di scusarsi e di andarsene
immediatamente, dovrà incrociare quella spada frettolosa con la mia,
prima di domani a mezzogiorno.

--- Niente spargimento di sangue! --- implorò
l\textquotesingle ecclesiastico. --- Non è stato nulla di grave.
Dimentichiamolo. --- Le mani gli tremavano, ed era grigio in volto.

--- Si scuserà e se ne andrà --- insistette il Thon Taddeo --- o io
dovrò offrirmi di ucciderlo. Non preoccupatevi, non oserà battersi con
me perché, se vincesse, Hannegan lo farebbe impalare sulla pubblica
piazza e nel frattempo costringerebbe sua moglie a\ldots{} ma non
pensateci. Si scuserà e se ne andrà. Tuttavia, mi vergogno profondamente
che una cosa simile sia potuta accadere.

--- Avrei dovuto far buttare fuori il Poeta non appena si è presentato.
È stato lui a provocare l\textquotesingle incidente, e io non sono
riuscito a fermarlo. La provocazione era chiara.

--- Provocazione? Per le fantasiose menzogne di un buffone? Josard ha
reagito come se le accuse del Poeta fossero vere.

--- Allora voi non sapete che stanno preparando un rapporto completo sul
valore militare della nostra abbazia come fortezza?

Lo studioso spalancò la bocca. Guardò prima un ecclesiastico poi
l\textquotesingle altro, con evidente incredulità.

--- Ma allora è vero? --- chiese dopo un lungo silenzio.

L\textquotesingle abate annuì.

--- E voi ci avete permesso di rimanere.

--- Noi non abbiamo segreti. I vostri compagni possono fare questo
studio, se lo desiderano. Io non mi permetterei di chiedere perché
vogliono quelle informazioni. L\textquotesingle assunto del Poeta,
naturalmente, era una pura fantasia.

--- Naturalmente --- disse debolmente il Thon, senza guardare il suo
ospite.

--- Senza dubbio il vostro principe non ha mire aggressive su questa
regione, come insinuava invece il Poeta.

--- Senza dubbio no.

--- E anche se ne avesse, sono certo che egli avrebbe la saggezza di
comprendere\ldots{} o almeno avrebbe saggi consiglieri che glielo
farebbero capire\ldots{} che il valore della nostra abbazia come
magazzino dell\textquotesingle antico sapere è molto più grande di
quello che può avere come cittadella.

Il Thon colse la sfumatura di supplica, il significato sottinteso
d\textquotesingle una richiesta d\textquotesingle aiuto, nella voce del
religioso, e sembrò meditare, mentre giocherellava con i cibi e taceva,
per qualche tempo.

Riparleremo di questo argomento prima che io ritorni al collegio ---
promise, quietamente.

Sul banchetto era caduto un gelo improvviso, ma cominciò a disperdersi
durante i canti nel cortile, dopo il pasto, e svanì interamente quando
venne il momento della lezione dello studioso nell\textquotesingle Aula
Magna. L\textquotesingle imbarazzo sembrava quasi finito, e il gruppo
era ritornato a una superficiale cordialità.

Don Paulo condusse il Thon al leggio; li seguivano Gault e il segretario
del Thon, che si unirono a loro sul podio. Gli applausi risuonarono
cordiali, dopo che l\textquotesingle abate ebbe presentato il Thon; il
silenzio che seguì sembrava il silenzio in un tribunale, in attesa del
verdetto. Lo studioso non era un grande oratore, ma il verdetto si
rivelò soddisfacente per la folla dei monaci.

--- Sono sbalordito di ciò che abbiamo trovato qui --- disse. ---
Qualche settimana fa non avrei creduto, anzi non credevo che documenti
quali voi avete nei vostri Memorabilia potessero essere ancora rimasti,
superstiti del crollo dell\textquotesingle ultima grande civiltà. E
ancora difficile crederlo, ma l\textquotesingle evidenza ci forza ad
accettare l\textquotesingle ipotesi che i documenti siano autentici. La
loro sopravvivenza è già abbastanza incredibile: ma ancora più
fantastico, per me, è il fatto che siano rimasti inosservati durante
questo secolo, fino a ora. In questi ultimi tempi vi sono stati uomini
capaci di apprezzarne il valore potenziale\ldots{} e non io soltanto.
Cosa avrebbe potuto farne il Thon Kaschler, mentre era vivo\ldots{}
anche settant\textquotesingle anni or sono.

Il mare dei visi dei monaci era illuminato di sorrisi,
nell\textquotesingle udire una così favorevole reazione ai Memorabilia
da parte d\textquotesingle una persona dotata come il Thon. Paulo si
chiese se sfuggiva loro la vaga sfumatura di risentimento --- o forse
era sospetto? --- nel tono dell\textquotesingle oratore.

--- Se avessi conosciuto queste fonti dieci anni or sono --- stava
dicendo quello --- molto del mio lavoro nel campo
dell\textquotesingle ottica non sarebbe stato necessario.

``Ahah!'' pensò l\textquotesingle abate. ``Dunque è così.'' O almeno è
così, in parte. Si accorge che alcune delle sue scoperte sono soltanto
riscoperte, e questo gli lascia in bocca un sapore amaro. Ma senza
dubbio deve sapere che mai, durante tutta la sua vita, potrà essere
qualcosa di più di un riscopritore di teorie perdute; benché sia
geniale, potrà fare soltanto ciò che hanno fatto altri prima di lui. E
sarebbe stato così, inevitabilmente, finché il mondo non si fosse
evoluto fino al punto che aveva raggiunto prima del Diluvio di Fiamma.

Nonostante tutto, era evidente che il Thon Taddeo era molto
impressionato.

--- Il mio tempo, qui, è limitato --- proseguì. --- Da ciò che ho
veduto, sospetto che occorrerebbe il lavoro di venti specialisti durante
parecchi decenni, per mungere i Memorabilia e ottenerne informazioni
comprensibili. La scienza fisica di solito procede per ragionamenti
induttivi controllati per mezzo di esperimenti; ma qui il compito è
deduttivo. Da pochi frammenti spezzati di principi generali, noi
dobbiamo tentare di afferrare i particolari. In qualche caso, questo può
rivelarsi impossibile\ldots{} --- Si interruppe per un momento per
prendere un fascio di appunti e per sfogliarli brevemente. --- Qui
c\textquotesingle è una citazione che ho trovato sepolta nel
sotterraneo. Proviene da un frammento di quattro pagine
d\textquotesingle un libro che può essere stato un testo di fisica molto
progredita. Qualcuno di voi può averlo veduto.

``\ldots{} e se i termini spaziali predominano
nell\textquotesingle espressione per l\textquotesingle intervallo fra i
punti-eventi, l\textquotesingle intervallo è detto essere simile allo
spazio, poiché è possibile scegliere un sistema coordinato appartenente
a un osservatore con una velocità ammissibile in cui gli eventi appaiono
simultanei, e perciò separati solo spazialmente. Se, tuttavia,
l\textquotesingle intervallo è simile al tempo, gli eventi non possono
essere simultanei in alcun sistema coordinato, ma esiste un sistema
coordinato in cui i termini spaziali svaniranno completamente, così che
la separazione fra gli eventi sarà puramente temporale, id est, che essi
si verificano nello stesso luogo, ma in tempi diversi. Ora, esaminando
gli estremi dell\textquotesingle intervallo reale\ldots''

Alzò lo sguardo, con un sorriso capriccioso. --- Qualcuno, qui, ha
consultato recentemente questo frammento?

Il mare di facce rimase inespressivo.

--- Qualcuno ricorda di averlo veduto?

Kornhoer e altri due alzarono cautamente le mani.

--- Qualcuno sa che cosa significa?

Le mani si riabbassarono in fretta.

Il Thon ridacchiò. --- Questo passo è seguito da una pagina e mezzo di
matematica che non tenterò di leggere, ma tratta di alcuni dei nostri
concetti fondamentali come se non fossero affatto fondamentali, ma solo
apparenze evanescenti che cambiano secondo il punto di vista
d\textquotesingle una persona. Termina con la parola ``perciò'', ma il
resto della pagina è bruciato, e con esso la conclusione. Il
ragionamento è impeccabile, tuttavia, e la matematica molto elegante,
quindi io stesso posso scrivere la conclusione. Sembra la conclusione di
un pazzo. Cominciava con assunti, tuttavia, che sembravano egualmente
pazzeschi. È un falso? Se non lo è, qual è il suo posto
nell\textquotesingle intero schema della scienza degli antichi? Che cosa
lo precede, come prerequisito necessario alla comprensione? Cosa ne
segue, e come può essere provato? Domande cui non so rispondere. Questo
è soltanto un esempio dei molti enigmi proposti da questi documenti che
voi avete conservato per tanto tempo. Un ragionamento che non tocchi in
alcun punto una realtà esperienziale è affare che riguarda gli
angelologi e i teologi, non gli scienziati fisici. Eppure documenti come
questi descrivono sistemi che non toccano in alcun punto la nostra
esperienza. Erano forse alla portata sperimentale degli antichi? Alcuni
riferimenti sembrano indicarlo. Un documento si riferisce a una
trasmutazione di elementi\ldots{} che noi abbiamo proprio recentemente
stabilito essere impossibile teoricamente\ldots{} e poi qui dice
``l\textquotesingle esperimento lo prova''. Ma in che modo?

``Può darsi che occorrano generazioni intere per valutare e comprendere
alcuni di questi documenti. È una sfortuna che debbano rimanere qui, in
questo luogo inaccessibile, perché occorrerà lo sforzo concentrato di
parecchi studiosi per trarne un significato. Sono sicuro che voi
comprendete che le facilitazioni offerte attualmente da voi sono
inadeguate\ldots{} per non parlare della inaccessibilità al resto del
mondo.''

Seduto sul podio dietro all\textquotesingle oratore,
l\textquotesingle abate cominciò a corrucciarsi, ad aspettare il peggio.
Il Thon Taddeo, tuttavia, preferì non avanzare proposte. Ma le sue
osservazioni continuarono a chiarire la sua convinzione che quelle
reliquie avrebbero dovuto essere poste in mani più competenti di quelle
dei monaci dell\textquotesingle Ordine Albertiano di san Leibowitz, e
che la situazione, così com\textquotesingle era, era assurda.

Intuendo forse il crescente disagio dei presenti, incominciò a parlare
dei suoi studi attuali, che comprendevano una indagine sulla natura
della luce più approfondita di quante fossero state effettuate in
precedenza. Parecchi dei tesori dell\textquotesingle abbazia si
rivelavano di grande aiuto, e il Thon sperava di escogitare presto
metodi sperimentali per provare le sue teorie. Dopo una breve
discussione sul fenomeno della rifrazione, si interruppe poi disse, in
tono di scusa: --- Spero che questo non offenda le convinzioni religiose
di nessuno --- e si guardò intorno, ironicamente. Vedendo che i visi
rimanevano curiosi e blandi, continuò per qualche tempo, poi invitò la
congregazione a rivolgergli qualche domanda.

--- Vi dispiacerebbe una domanda proveniente dal podio? --- chiese
l\textquotesingle abate.

--- No, affatto --- disse lo studioso, assumendo
un\textquotesingle espressione un po\textquotesingle{} dubbiosa, come se
stesse pensando \emph{et tu, Brute}.

--- Mi chiedevo cosa può esservi nelle proprietà della luce che secondo
voi sarebbe offensivo per la religione.

--- Ecco\ldots{} --- Il Thon si interruppe, imbarazzato. --- monsignor
Apollo, che voi conoscete, si riscaldava molto su questo argomento.
Diceva che la luce non poteva venir rifratta prima del Diluvio, perché
l\textquotesingle arcobaleno sarebbe stato\ldots{}

Tutto il pubblico esplose in una ruggente risata, sommergendo il resto
dell\textquotesingle osservazione. Prima che l\textquotesingle abate
avesse ridotto tutti al silenzio con i suoi gesti, il Thon Taddeo era
diventato rosso come una barbabietola, e don Paulo faticava a mantenere
un\textquotesingle espressione solenne.

--- Monsignor Apollo è un ottimo uomo, un ottimo sacerdote, ma tutti gli
uomini possono essere incredibilmente somari, qualche volta,
specialmente al di fuori del loro campo specifico. Mi dispiace di avervi
rivolto questa domanda.

--- La vostra risposta è un sollievo, per me --- disse lo studioso. ---
Non cercavo di provocare un litigio.

Non vi furono altre domande, e il Thon procedette verso il suo secondo
argomento: l\textquotesingle evoluzione e le attuali attività del
\emph{collegium}. L\textquotesingle immagine che ne diede sembrava
incoraggiante. Il \emph{collegium} era sommerso da candidati che
volevano studiare in quell\textquotesingle istituto. Il \emph{collegium}
stava assumendo una funzione di educazione, non soltanto di ricerca.
L\textquotesingle interesse per la filosofia e la scienza naturale
cresceva fra il laicato più colto. L\textquotesingle istituto riceveva
liberali sovvenzioni. Sintomi di rinascita e di resurrezione.

--- Potrei citare alcune delle ricerche e delle indagini attualmente
condotte dai nostri --- proseguì. --- Seguendo il lavoro di Bret sul
comportamento del gas, il Thon Viche Mortoin sta indagando sulle
possibilità di una produzione artificiale del ghiaccio. Il Thon Friider
Halb sta cercando i mezzi pratici per trasmettere messaggi mediante
variazioni elettriche lungo un filo\ldots{} --- L\textquotesingle elenco
era lungo, e i monaci sembravano impressionati. Studi in molti campi,
medicina, astronomia, geologia, matematica, meccanica, venivano
intrapresi. Alcuni sembravano poco pratici e sconsiderati, ma moltissimi
promettevano una ricca ricompensa di conoscenza e di applicazioni
pratiche. Dalle ricerche di Jejene sul Nostrum Universale
all\textquotesingle instancabile attacco di Bodalk alle geometrie
ortodosse, le attività del \emph{collegium} mostravano un sano desiderio
di aprire gli scaffali segreti della Natura, chiusi da quando
l\textquotesingle umanità aveva bruciato i suoi ricordi istituzionali e
si era condannata all\textquotesingle amnesia culturale, più di un
millennio addietro.

--- Oltre a questi studi, il Thon Maho Mahh è a capo di un progetto che
cerca ulteriori informazioni sull\textquotesingle origine della specie
umana. Poiché questo è soprattutto un lavoro archeologico, mi ha chiesto
espressamente di frugare nella vostra biblioteca, ricca di tanti libri,
alla ricerca di ogni materiale sull\textquotesingle argomento, dopo che
avrò ultimato il mio studio, qui. Tuttavia, forse farei meglio a non
insistere molto su questo argomento, poiché di solito provoca
controversie con i teologi. Ma se qualcuno vuol fare domande\ldots{}

Un giovane monaco che studiava per diventare prete si alzò e ottenne un
cenno di approvazione del Thon.

--- Signore, mi chiedevo se voi conoscete il suggerimento di
sant\textquotesingle Agostino a questo proposito.

--- No.

--- Un vescovo e filosofo del Quarto secolo. Suggerì che, in principio,
Dio creò tutte le cose nelle loro cause germinali, includendo la
fisiologia dell\textquotesingle uomo, e che le cause germinali
inseminarono la materia informe\ldots{} che poi si evolvette
gradualmente nelle forme più complesse, fino a quella
dell\textquotesingle Uomo. Questa ipotesi è stata, considerata?

Il sorriso del Thon era condiscendente, sebbene egli non bollasse
apertamente di puerilità la proposta. --- Temo di no, ma controllerò ---
disse, in un tono che indicava che non l\textquotesingle avrebbe fatto.

--- Grazie --- disse il monaco, e sedette, umilmente.

--- Forse la ricerca più ardita, tuttavia --- continuò lo studioso --- è
quella condotta dal mio amico, il Thon Esser Shon. È un tentativo di
sintetizzare la materia vivente. Il Thon Esser spera di creare
protoplasma vivente, servendosi soltanto di sei ingredienti
fondamentali. Questo lavoro porterebbe a\ldots{} sì? Volete farmi una
domanda?

Un monaco della terza fila si era alzato e si stava inchinando
all\textquotesingle oratore. L\textquotesingle abate si piegò in avanti
per guardarlo e riconobbe, con orrore, che era frate Armbruster, il
bibliotecario.

--- Se volete fare una cortesia a un vecchio --- gracchiò il monaco,
strascicando le parole con voce monotona. --- Questo Thon Esser
Shon\ldots{} che si limita a sei soli ingredienti fondamentali\ldots{} è
molto interessante. Mi chiedevo\ldots{} gli permettono di usare entrambe
le mani?

--- Ecco, io\ldots{} --- Il Thon Taddeo si interruppe e corrugò la
fronte.

--- E posso chiedere, inoltre --- continuò la voce asciutta di
Armbruster --- se questa impresa straordinaria viene effettuata in
posizione seduta, eretta o prona? O forse a cavallo, mentre si suonano
due trombe?

I novizi sghignazzarono. L\textquotesingle abate balzò in piedi.

--- Frate Armbruster, eravate stato avvertito. Siete escluso dalla mensa
comune fino a che non darete soddisfazione. Potete aspettare nella
Cappella di Nostra Signora.

Il bibliotecario si inchinò di nuovo e si allontanò in silenzio dalla
sala, con umile portamento, ma con gli occhi trionfanti.
L\textquotesingle abate mormorò una scusa allo studioso, ma lo sguardo
del Thon era diventato improvvisamente gelido.

--- Per concludere --- disse --- un breve profilo di ciò che il mondo
può aspettarsi, secondo me, dalla rivoluzione intellettuale che è appena
incominciata. --- Con occhi ardenti, si guardò intorno, e la sua voce
passò da un tono distratto a un ritmo fervente. ---
L\textquotesingle ignoranza è stata la nostra regina. Dalla morte
dell\textquotesingle impero, siede incontrastata sul trono
dell\textquotesingle Uomo. La sua dinastia è vecchia
d\textquotesingle una intera epoca. Il suo diritto al dominio è ormai
considerato legittimo. I saggi del passato lo hanno confermato. Essi non
fecero nulla per detronizzarla. Domani, regnerà una nuova sovrana.
Uomini che comprendono, uomini di scienza staranno anno attorno al suo
trono, e l\textquotesingle universo conoscerà la sua potenza. Il suo
nome è Verità. Il suo impero comprenderà la Terra. E la dominazione
dell\textquotesingle Uomo sulla Terra si rinnoverà. Fra un secolo, gli
uomini voleranno nell\textquotesingle aria, a bordo di uccelli
meccanici. Carri di metallo correranno lungo strade di pietra fabbricata
dall\textquotesingle uomo. Vi saranno palazzi di trenta piani, navi che
scenderanno in fondo al mare, macchine che faranno ogni lavoro. E come
si realizzerà tutto questo? --- Fece una pausa e abbassò la voce. ---
Nello stesso modo in cui si verificano tutti i cambiamenti, temo. E mi
dispiace che sia così. Si verificherà nella violenza, nella fiamma e
nella furia, perché nessun cambiamento si verifica con calma, nel mondo.

Si guardò intorno, perché un sommesso mormorio si levò dalla comunità.

--- Sarà così anche se noi non vogliamo che sia così. Ma \emph{perché}?

``L\textquotesingle ignoranza è regina. Molti non trarrebbero più
profitto dalla sua abdicazione. Molti si arricchiscono grazie alla sua
buia monarchia. Sono la sua corte, e nel suo nome defraudano e
governano, si arricchiscono e perpetuano il loro potere. Temono persino
la sconfitta dell\textquotesingle analfabetismo, perché la parola
scritta è un altro canale di comunicazione che potrebbe portare
all\textquotesingle unificazione dei loro nemici. Le loro armi sono
affilate, e le usano con abilità. Porteranno la battaglia sul mondo
quando i loro, interessi saranno minacciati, e la violenza che ne
seguirà durerà fino a che la struttura della società come esiste
attualmente sarà ridotta a un cumulo di macerie, e fino a che non ne
emergerà una società nuova. Mi dispiace. Ma è così che io la vedo.''

Quelle parole portarono di nuovo il gelo nella sala. Le speranze di don
Paulo svanirono, perché la profezia dava forma al probabile punto di
vista dello studioso. Il Thon Taddeo conosceva le ambizioni militari del
suo monarca. Aveva una alternativa: approvarle, disapprovarle, o
considerarle come fenomeni impersonali al di fuori del suo controllo,
come un\textquotesingle inondazione, una carestia o un uragano.

Evidentemente, allora, le accettava come inevitabili\ldots{} per evitare
di dover formulare un giudizio morale. \emph{Vi sia sangue, ferro e
	pianto\ldots{}}

``Come può un uomo simile sfuggire alla propria coscienza e respingere
la propria responsabilità\ldots{} e così facilmente!'' tuonò fra sé
l\textquotesingle abate.

Ma poi gli tornarono alla mente quelle parole. ``Imperocché il Signore
Iddio aveva permesso che gli uomini sapienti di quei tempi conoscessero
i modi per cui il mondo medesimo poteva essere distrutto\ldots''

E permetteva anche che essi conoscessero come poteva essere salvato, e
come sempre, lasciava che fossero essi stessi a decidere. E forse
avevano scelto come sceglie il Thon Taddeo. Si lavano le mani davanti
alle moltitudini. Pensateci voi. Purché non venissero crocifissi essi
stessi.

Ed erano stati crocifissi, comunque. Senza dignità. Sempre perché
qualcuno, comunque, deve essere inchiodato e appeso alla croce, e se tu
ne cadi, essi batteranno\ldots{}

Vi fu un improvviso silenzio. Lo studioso aveva smesso di parlare.

L\textquotesingle abate batté le palpebre, guardò attraverso la sala.
Metà della comunità stava fissando l\textquotesingle ingresso. In
principio, i suoi occhi non riuscirono a distinguere nulla.

--- Che c\textquotesingle è? --- sussurrò a Gault.

--- Un vecchio con la barba e uno scialle --- sibilò Gault.

--- Sembra\ldots{} No, lui non\ldots{}

Don Paulo si alzò e avanzò verso il bordo del podio, per fissare la
forma, vagamente definita nelle ombre. Poi lo chiamò, sommessamente: ---
Benjamin?

La figura si agitò. Si strinse lo scialle attorno alle spalle magrissime
e avanzò, lentamente, nella luce. Si fermò di nuovo, mormorando fra sé
mentre si guardava intorno nella sala; poi il suo sguardo scopri lo
studioso dietro il leggio.

Appoggiandosi a un bastone nodoso, la vecchia apparizione avanzò
lentamente verso il podio, senza distogliere gli occhi
dall\textquotesingle uomo che vi stava dietro.

Il Thon Taddeo sembrò dapprima divertito e perplesso, ma quando si
accorse che nessuno si muoveva o parlava, sembrò perdere colore, man
mano che la decrepita visione gli si avvicinava. Il volto di quella
barbuta antichità splendeva della speranzosa ferocia di qualche passione
travolgente che gli bruciava dentro più furiosamente del principio della
vita, che da molto tempo ormai avrebbe dovuto smorzarsi.

Si avvicinò al leggio, si fermò. I suoi occhi ammiccarono verso
l\textquotesingle oratore sbalordito. La bocca gli tremò. Sorrise.. Tese
una mano tremante verso lo studioso. Il Thon si ritrasse con uno sbuffo
di repulsione.

L\textquotesingle eremita era agile. Balzò sul podio, schivò il leggio,
e afferrò il braccio dello studioso.

--- Che pazzia\ldots{}

Benjamin accarezzò quel braccio, mentre fissava, pieno di speranza, gli
occhi dello studioso\ldots{}

Il suo viso si rannuvolò. Il bagliore degli occhi si spense. Lasciò
cadere il braccio. Un grande sospiro uscì dai vecchi polmoni inariditi,
mentre la speranza svaniva. Il sorriso eternamente saputo del Vecchio
Ebreo della Montagna ritornò sul suo volto. Si rivolse alla comunità,
tese le magre braccia, scrollò eloquentemente le spalle.

--- Non è ancora \emph{Lui} --- disse, in tono acido, quindi si trascinò
via\ldots{}

Poi, vi fu ben scarsa formalità.
