	\chapter{\phantom{title}}

\lettrine{I}{l} sotterraneo fortificato era stato scavato durante i secoli delle
infiltrazioni nomadi dal Nord, quando l\textquotesingle Orda di Bayring
faceva scorrerie nelle Pianure e nel deserto, saccheggiando e
distruggendo i villaggi che trovava sul suo cammino.

I Memorabilia, il piccolo patrimonio di conoscenza
dell\textquotesingle abbazia conservato da secoli, erano stati nascosti
nei sotterranei per proteggere quegli scritti inestimabili dai nomadi e
dai sedicenti crociati degli Ordini scismatici, fondati per combattere
le orde, ma che si dedicavano ai saccheggi indiscriminati e alle lotte
settarie. Né i nomadi né l\textquotesingle Ordine Militare di san
Pancrazio avrebbero attribuito valore ai libri
dell\textquotesingle abbazia, ma i nomadi li avrebbero distrutti per la
gioia di distruggerli, e i frati-cavalieri ne avrebbero bruciati gran
parte come ``eretici'', secondo la teologia di Vissarion, il loro
antipapa.

Ora l\textquotesingle Età dell\textquotesingle Oscurantismo sembrava sul
punto di concludersi. Da dodici secoli, la fiammella della conoscenza
era stata tenuta accesa nei monasteri: soltanto ora vi erano menti
pronte a riceverla. Molto tempo addietro, durante
l\textquotesingle ultima Età della Ragione, certi orgogliosi pensatori
avevano sostenuto che la conoscenza valida era indistruttibile, che le
idee erano eterne e la verità immortale. Ma questo era vero soltanto in
un senso molto sottile, pensò l\textquotesingle abate, e non era vero
affatto, in superficie. C\textquotesingle era un significato oggettivo
nel mondo, senza dubbio: il \emph{logos} non morale, il disegno del
Creatore; ma questi significati appartenevano a Dio e non
all\textquotesingle Uomo, fino a che non trovavano una incarnazione
imperfetta, un riflesso oscuro, nella mente, nella parola, nella cultura
di una data società umana, che potesse ascrivere valore a quei
significati, in modo che divenissero validi, in senso umano,
all\textquotesingle interno della civiltà. Perché l\textquotesingle Uomo
è portatore di civiltà così come è portatore di
un\textquotesingle anima, ma le sue civiltà non sono immortali, e
potevano morire con una razza o con una età, e poi i riflessi umani del
significato e l\textquotesingle effigie umana della verità si
affievolivano, e la verità e il significato risiedevano, invisibili,
soltanto nel \emph{logos} obiettivo della Natura e
nell\textquotesingle ineffabile \emph{Logos} di Dio. La verità poteva
essere crocifissa; ma presto, forse, vi sarebbe stata la resurrezione.

I Memorabilia erano pieni di parole antiche, di antiche formule, di
antichi riflessi di significato, distaccati dalle menti che erano morte
tanto tempo prima, quando una società completamente diversa era caduta
nell\textquotesingle oblio. C\textquotesingle era ben poco, di essa, che
poteva essere ancora compreso. Certi documenti parevano insignificanti
quanto poteva sembrarlo un Breviario a uno sciamano
d\textquotesingle una tribù nomade. Altri conservavano una certa
bellezza ornamentale, e una certa apparenza ordinata che pareva
sottintendere un significato, come un rosario potrebbe indurre un nomade
a pensare a una collana. I primi frati dell\textquotesingle Ordine
Leibowitziano avevano cercato di trarre una specie di Sindone dal volto
d\textquotesingle una civiltà crocifissa: ne avevano ottenuto
un\textquotesingle immagine che conservava un riflesso
dell\textquotesingle antica grandezza, ma quell\textquotesingle immagine
era sbiadita, incompleta, difficile da comprendere. I monaci avevano
conservato l\textquotesingle immagine, che adesso era sopravvissuta
perché il mondo l\textquotesingle esaminasse e cercasse di
interpretarla, se lo desiderava. I Memorabilia non potevano, in se
stessi, provocare una rinascita dell\textquotesingle antica scienza o di
una grande civiltà, tuttavia, poiché le civiltà erano generate dalle
tribù dell\textquotesingle Uomo, non da torni polverosi; ma i libri
potevano aiutare, così sperava don Paulo\ldots{} i libri potevano
indicare una direzione e fornire una traccia a una scienza che
cominciava a evolversi. Era accaduto già una volta, come aveva affermato
il venerabile Boedullus nel suo \emph{De Vestigiis Antecessarum
	Civitatum}.

``E questa volta'' pensò don Paulo ``noi li indurremo a ricordare chi ha
mantenuto ardente quella scintilla mentre il mondo dormiva.'' Si fermò,
per guardarsi indietro; per un attimo aveva immaginato di avere udito il
belato atterrito della capra del Poeta.

Il clamore proveniente dal sotterraneo soverchiò ben presto il suo udito
mentre scendeva le scale, verso la sorgente di quel frastuono. Qualcuno
stava piantando chiodi d\textquotesingle acciaio nella pietra.
L\textquotesingle odore del sudore si mescolava a quello dei vecchi
libri.

Una febbrile esplosione di attività che si addiceva scarsamente agli
studiosi riempiva la biblioteca. Alcuni novizi correvano qua e là,
portando arnesi da lavoro. Altri novizi stavano in gruppo, e studiavano
dei piani. Altri spostavano scrivanie e tavoli per fare posto a una
strana macchina. C\textquotesingle era una grande confusione, al lume
delle lampade. Frate Armbruster, bibliotecario e custode dei
Memorabilia, se ne stava ritto a osservare la scena da una alcova
distante, fra gli scaffali, con le braccia conserte e il viso cupo.

Don Paulo evitò il suo sguardo accusatore.

Frate Kornhoer si avvicinò al suo superiore con un sorriso di
entusiasmo. --- Bene, Padre Abate, fra poco avremo una luce quale nessun
uomo vivente ha mai veduto.

--- Queste parole non sono esenti da una certa vanità, fratello ---
rispose Paulo.

--- Vanità, Domne? É vanità mettere a buon frutto ciò che noi abbiamo
imparato?

--- Pensavo alla nostra fretta di metterlo a frutto in tempo per fare
impressione a un certo studioso che verrà a visitarci. Vediamo questa
stregoneria, dunque.

Si avviarono verso la macchina costruita con materiale eterogeneo.
All\textquotesingle abate non sembrava affatto utile, a meno che si
considerasse utile uno strumento di tortura. Un asse era collegato da
pulegge e da cinghie a un tornichetto che arrivava fino alla cintura
dell\textquotesingle abate. Quattro ruote erano montate
sull\textquotesingle asse, a una distanza di pochi pollici
l\textquotesingle una dall\textquotesingle altra. I loro spessi
cerchioni di ferro erano segnati da scanalature, e le scanalature
sostenevano innumerevoli nidi di filo di rame, preparati nella locale
fucina di Sanly Bowitts. Le ruote erano libere di ruotare a
mezz\textquotesingle aria, notò don Paulo, perché i cerchioni non
toccavano alcuna superficie. Tuttavia, alcuni blocchi fissi di ferro
stavano di fronte ai cerchioni, a guisa di freni, senza però toccarli.
Anche quei blocchi erano avvolti da innumerevoli spire di filo\ldots{}
``bobine di campo'' come li chiamava Kornhoer.

Don Paulo scosse solennemente il capo.

--- Sarà certamente la più grande miglioria, nel campo della fisica, che
si sia avuta nell\textquotesingle abbazia da quando venne inventato il
torchio da stampa, cento anni or sono --- azzardò orgoglioso frate
Kornhoer.

--- Funzionerà? --- chiese don Paulo.

--- Ci scommetterei il lavoro straordinario di un mese, monsignore.

``Stai scommettendo molto di più'' pensò il religioso, ma non lo disse.

~

--- Da dove esce la luce? --- chiese, studiando di nuovo il bizzarro
meccanismo.

Il monaco rise. --- Oh, abbiamo una lampada speciale, per questo. Ciò
che vedete qui è soltanto la ``dinamo''. Produce
l\textquotesingle essenza elettrica che la lampada brucerà.

Melanconicamente, don Paulo contemplò lo spazio occupato dalla dinamo.

--- Questa essenza --- mormorò --- non può essere estratta dal grasso di
montone, vero?

--- No, no\ldots{} L\textquotesingle essenza elettrica è\ldots{}
bene\ldots{} Volete che ve lo spieghi?

È meglio di no. La scienza naturale non è la mia specialità. La lascio a
voi, che siete più giovani. --- Indietreggiò rapidamente per non essere
scotennato da una trave portata da due carpentieri frettolosi. ---
Ditemi --- chiese --- se studiando gli scritti dell\textquotesingle età
leibowitziana potete imparare a costruire queste cose, perché credete
che i nostri predecessori non abbiano ritenuto giusto costruirle?

Il monaco tacque per un momento. ---Non è facile spiegarlo --- disse
alla fine. --- In realtà, negli scritti che rimangono, non vi è alcuna
informazione specifica sul modo di costruire una dinamo. Si potrebbe
dire, piuttosto, che tale informazione è implicita in una intera
raccolta di scritti frammentari. Parzialmente implicita. E deve esserne
estratta per mezzo della deduzione. Ma per arrivare a questo è
necessario disporre di alcune teorie su cui lavorare\ldots{}
informazioni teoriche di cui i nostri predecessori non disponevano.

--- E noi?

--- Ebbene, si\ldots{} ora che vi sono alcuni uomini come\ldots{} il suo
tono divenne profondamente rispettoso; esitò prima di pronunciare il
nome ---\ldots{} come il Thon Taddeo\ldots{}

--- Era una frase completa? --- chiese l\textquotesingle abate, un
po\textquotesingle{} acido. --- Ecco, fino a tempi recenti, pochi
filosofi si sono occupati delle nuove teorie della fisica. In realtà, è
stato il lavoro di\ldots{} del Thon Taddeo\ldots{} --- di nuovo quel
tono rispettoso, notò don Paulo ---\ldots{} che ci ha dato gli assiomi
necessari. La sua opera sulla Mobilità delle Essenze Elettriche, per
esempio, e il suo Teorema della Conservazione\ldots{}

--- Allora dovrebbe essere compiaciuto nel vedere tradotta in realtà la
sua opera. Ma posso chiedere dov\textquotesingle è la lampada? Spero che
non sia più grande della dinamo.

--- È questa, Domne --- disse il monaco, prendendo dalla tavola un
piccolo oggetto. Sembrava soltanto una specie di supporto che reggeva un
paio di verghe nere e una vite per regolarne la distanza. --- Questi
sono carboni --- spiegò frate Kornhoer. --- Gli antichi
l\textquotesingle avrebbero chiamata ``lampada ad arco''. Ve ne erano di
altre specie, ma noi non abbiamo il necessario per fabbricarle.

--- Sbalorditivo. E da dove viene la luce?

--- Da qui. --- Il monaco indicò il varco fra i carboni.

--- Deve essere una fiamma molto piccola --- disse
l\textquotesingle abate.

--- Oh, ma è splendente! Più splendente, prevedo, di quella di cento
candele.

--- No!

--- Vi sembra impressionante?

--- Mi sembra assurdo\ldots{} --- notando l\textquotesingle espressione
improvvisamente offesa di frate Kornhoer, l\textquotesingle abate
aggiunse in fretta\ldots{} pensare per quanto tempo ci siamo serviti
della cera d\textquotesingle api e del grasso di montone.

--- Mi sono chiesto --- aggiunse timidamente il monaco se gli antichi
l\textquotesingle usavano sui loro altari, invece delle candele.

--- No --- disse l\textquotesingle abate, --- Decisamente no. Questo
posso dirvelo. Vi prego di abbandonare al più presto questa idea, e di
non pensarvi più.

--- Sì, Padre Abate.

--- Ora, dove avete intenzione di appendere questo oggetto?

--- Ecco\ldots{} --- Frate Kornhoer si interruppe per guardarsi intorno,
con aria speculativa, nel sotterraneo buio. --- Non vi avevo pensato.
Immagino che dovrebbe andare sopra la scrivania dove lavorerà\ldots{}
--- (``Perché fa una pausa ogni volta che deve pronunciare quel nome?'',
si chiese irritato don Paulo) ---\ldots{} il Thon Taddeo.

--- Faremmo meglio a chiederlo a frate Armbruster --- decise
l\textquotesingle abate; e poi, notando l\textquotesingle improvviso
disagio del monaco: --- Che succede? Forse voi e frate
Armbruster\ldots{}

Il viso di Kornhoer si alterò in una smorfia di scusa. --- Per la
verità, Padre Abate, io non ho mai perduto la calma, con lui, neppure
una volta. Oh, sono corse molte parole, fra noi, ma\ldots{} --- E alzò
le spalle. --- Non vuole spostare nulla. Continua a mormorare contro la
stregoneria e cose simili. Non è facile ragionare con lui. I suoi occhi
sono quasi ciechi, ormai, per aver letto sotto luci troppo
fioche\ldots{} eppure dice che stiamo lavorando a
un\textquotesingle opera del Demonio, Io non so che dire.

Don Paulo si accigliò lievemente mentre attraversavano la stanza,
dirigendosi verso l\textquotesingle alcova da cui frate Armbruster
osservava corrucciato gli eventi.

--- Bene, adesso l\textquotesingle avete spuntata --- disse il
bibliotecario a Kornhoer, mentre si avvicinavano. --- Quando metterete
qui un bibliotecario meccanico, fratello?

--- Abbiamo trovato alcuni accenni all\textquotesingle esistenza di
qualcosa di simile, nei tempi andati --- brontolò
l\textquotesingle inventore. --- Nelle descrizioni della \emph{Machina
	analytica}, troverete riferimenti a\ldots{}

--- Basta, basta --- si interpose l\textquotesingle abate. Poi, rivolto
al bibliotecario. --- Il Thon Taddeo avrà bisogno di un posto dove
lavorare. Quale suggerite?

Armbruster indicò con il pollice l\textquotesingle alcova riservata alle
Scienze Naturali. --- Fate che legga lì, alla luce della lanterna, come
tutti gli altri.

--- Cosa ne direste, invece, di preparargli uno studio, qui, dove
c\textquotesingle è più spazio, Padre Abate? --- suggerì Kornhoer, in
una pronta controproposta. --- Oltre a una scrivania gli occorrerà un
abaco, una lavagna e un tavolo da disegno. Potremmo isolarlo con
paraventi provvisori.

--- Credevo che avesse bisogno dei documenti leibowitziani e degli
scritti più antichi --- disse sospettoso il bibliotecario.

--- Infatti.

--- E allora dovrà andare avanti e indietro continuamente se lo mettete
in mezzo alla stanza. I volumi rari sono assicurati con catenelle, e le
catenelle non sono sufficientemente lunghe.

--- Non è un problema --- disse l\textquotesingle inventore. ---
Togliete le catene. Sono una sciocchezza, in ogni caso. I culti
scismatici si sono estinti, o sono diventati regionali. Sono cento anni
ormai che nessuno ha più sentito parlare dell\textquotesingle Ordine
Militare Pancraziano.

Armbruster arrossì, indignato. --- Oh, no --- insorse. --- Le catenelle
resteranno.

--- Ma perché?

--- Adesso non vi sono più gli incendiari di libri. Dobbiamo
preoccuparci degli abitanti dei villaggi, però. Le catenelle resteranno.

Kornhoer si rivolse all\textquotesingle abate e aprì le braccia. ---
Vedete, monsignore?

--- Ha ragione --- disse don Paulo. --- C\textquotesingle è troppa
agitazione nel villaggio. Il Consiglio cittadino ha espropriato la
nostra scuola, non dimenticatelo. Adesso hanno una biblioteca del
villaggio, e vogliono che siamo noi a riempire i loro scaffali.
Preferibilmente con volumi rari, naturalmente. E non c\textquotesingle è
solo questo; l\textquotesingle anno scorso abbiamo avuto dispiaceri dai
ladri. Frate Armbruster ha ragione. I volumi rari restano incatenati.

--- Benissimo --- sospirò Kornhoer. --- Quindi dovrà lavorare dentro
l\textquotesingle alcova.

--- E allora, dove appenderemo questa vostra lampada meravigliosa?

I monaci guardarono verso il cubicolo. Era uno dei quattordici stalli
identici, suddivisi per ordine di materia, che si aprivano nella sala
principale. Ogni alcova aveva la sua arcata, e da un gancio di ferro
infisso nella chiave di volta d\textquotesingle ogni arcata pendeva un
pesante crocifisso.

--- Bene, se dovrà lavorare nell\textquotesingle alcova disse Kornhoer,
--- dovremo togliere il crocifisso e appendervi la lampada,
provvisoriamente. Non mi sembra che vi sia altro\ldots{}

--- Ateo! --- sibilò il bibliotecario. --- Pagano! Sacrilego! ---
Armbruster levò al cielo le mani tremanti. --- Dio mi aiuti, perché io
non lo faccia a pezzi con queste mani! Dove si fermerà? Portatelo via,
via! --- Voltò le spalle ai due, mentre le mani continuavano a
tremargli.

Anche don Paulo aveva provato un brivido alla richiesta
dell\textquotesingle inventore, ma ora corrugò irritato la fronte
guardando il dorso di frate Armbruster. Non si era mai aspettato che
fingesse una mitezza che era estranea alla sua natura, ma
l\textquotesingle atteggiamento querulo del vecchio monaco era diventato
ben peggio.

--- Frate Armbruster, voltatevi, vi prego.

Il bibliotecario si voltò.

--- Adesso abbassate le mani, e parlate con più calma quando\ldots{}

--- Ma, Padre Abate, avete sentito che cosa ha\ldots{}

--- Frate Armbruster, fatemi la cortesia di prendere la scala e di
togliere quel crocifisso.

Il viso del bibliotecario impallidì. Fissò don Paulo, senza parlare.

--- Questa non è una chiesa --- disse l\textquotesingle abate. --- La
collocazione delle immagini è facoltativa. Per il momento, fatemi la
cortesia. di togliere il crocifisso. Pare che sia
l\textquotesingle unico posto adatto per la lampada. Più tardi potremo
cambiare. Ora, capisco che questa faccenda ha messo sottosopra la vostra
biblioteca, e forse anche la vostra digestione, ma noi speriamo che
questo sia nell\textquotesingle interesse del progresso. Se non lo è,
allora\ldots{}

--- Voi costringereste Nostro Signore a spostarsi per fare posto al
progresso!

--- Frate Armbruster!

--- Perché non Gli appendete al collo quella luce stregata? --- continuò
il bibliotecario.

Il viso dell\textquotesingle abate si gelò. --- Io non forzo la vostra
obbedienza, fratello. Venite nel mio studio, dopo Compieta --- rispose.

Il bibliotecario sussultò. --- Prenderò la scala, Padre Abate ---
sussurrò, e si allontanò strascicando i piedi.

Don Paulo levò lo sguardo verso il Cristo al sommo
dell\textquotesingle arcata. ``Ti dispiace?'' si chiese.

Aveva un groppo allo stomaco. Sapeva che quel groppo avrebbe reclamato
un prezzo da lui, più tardi. Lasciò il sotterraneo prima che qualcuno
potesse notare il suo imbarazzo. Non era bene permettere che la comunità
vedesse fino a che punto quelle spiacevoli piccolezze lo potevano
sopraffare, in quei giorni.

L\textquotesingle installazione fu completata il giorno seguente, ma don
Paulo rimase nel suo studio, durante la prova. Per due volte era stato
costretto ad ammonire privatamente frate Armbruster, e poi a
rimproverarlo pubblicamente, durante il Capitolo. Eppure, provava più
comprensione per l\textquotesingle atteggiamento del bibliotecario che
per quello di Kornhoer. Sedeva affranto dietro la scrivania e attendeva
che gli portassero notizie dal sotterraneo, ma era ben poco interessato
al successo o al fallimento della prova. Teneva una mano infilata
nell\textquotesingle abito, e la batteva sullo stomaco come se cercasse
di calmare un bambino isterico.

Ancora quei crampi interni. Sembravano presentarsi ogni volta che si
minacciava qualcosa di spiacevole, e qualche volta si allontanavano
quando l\textquotesingle evento spiacevole esplodeva chiaramente, in
modo che egli potesse affrontarlo. Ma ora non si allontanavano affatto.

Era un avvertimento, e lo sapeva. Venisse,
quell\textquotesingle avvertimento, da un angelo o da un demonio o dalla
sua coscienza, gli diceva di stare attento a se stesso e a qualche
realtà non ancora affrontata.

``E ora?'' si chiese, permettendosi un silenzioso singhiozzo e un
silenzioso \emph{Perdonami} rivolto alla statua di san Leibowitz, posta
nella nicchia a forma di cappelletta in un angolo dello studio.

Una mosca camminava sul naso di san Leibowitz. Gli occhi dei santo
sembravano storcersi per fissare la mosca, esortando
l\textquotesingle abate a scacciarla. L\textquotesingle abate si era
affezionato a quella scultura in legno del XXVI secolo; il suo volto
aveva un sorriso curioso, d\textquotesingle un tipo piuttosto insolito
in una immagine sacra. Il sorriso era obliquo; le sopracciglia erano
abbassate in un cipiglio vagamente dubbioso, sebbene vi fossero le
grinze d\textquotesingle un sorriso negli angoli degli occhi. Poiché
aveva su una spalla la corda del carnefice,
l\textquotesingle espressione del santo sembrava spesso enigmatica.
Probabilmente era il risultato di alcune lievi irregolarità nella grana
del legno, che avevano preso il sopravvento sulla mano dello scultore
mentre quello cercava di ottenere alcuni particolari più minuziosi,
possibili con quel tipo di legno. Don Paulo non era sicuro che
quell\textquotesingle immagine non fosse stata modellata
sull\textquotesingle albero vivo, prima di essere scolpita: qualche
volta i pazienti maestri scultori di quel periodo cominciavano a
scegliere alberelli giovani di quercia o di cedro e --- dedicando
tediosissimi anni a torcere e a scortecciare la pianta, a legame i rami
vivi nelle posizioni desiderate --- tormentavano il legno che cresceva
costringendolo ad assumere sbalorditive forme di driade, con le braccia
conserte o sollevate, prima di tagliare l\textquotesingle albero
cresciuto per prepararlo e scolpirlo. La statua che ne risultava era
insolitamente resistente a ogni tentativo di scheggiarla o di spezzarla,
poiché quasi tutte le linee della scultura seguivano la venatura
naturale.

Don Paulo si stupiva spesso che quel Leibowitz ligneo avesse resistito
ai suoi predecessori durante molti secoli, a causa del bizzarro sorriso
del santo. Un giorno o l\textquotesingle altro, quel lieve sogghigno
sarà la tua rovina, disse rivolto alla statua\ldots{} senza dubbio, i
santi devono ridere, in Paradiso; il salmista dice che anche Iddio ride,
ma l\textquotesingle abate Malmeddy doveva avere disapprovato\ldots{}
Dio conceda riposo alla sua anima. Quel solenne somaro! Come te la sei
cavata con \emph{lui}, mi domandò? Non sei abbastanza ipocrita, per
certa genia. Quel sorriso\ldots{} chi conosco, io, che sogghigni in quel
modo? Mi piace, ma\ldots{} Un giorno o l\textquotesingle altro, in
questa sedia siederà un altro individuo dall\textquotesingle umore
canino\ldots{} \emph{Cave canem}. Ti sostituirà con un Leibowitz di
gesso. Dall\textquotesingle aria di sopportazione. Che non guardi le
mosche con gli occhi storti. E tu verrai divorato dalle termiti, giù nel
magazzino. Per sopravvivere al lento mutamento di gusto della Chiesa in
materia di arte devi avere una superficie che piaccia a un virtuoso
semplicione; eppure devi avere una profondità, sotto quella superficie,
per piacere a un saggio che sa discernere le cose. Il mutamento è lento,
ma ogni tanto c\textquotesingle è qualche scossa\ldots{} quando qualche
nuovo prelato visita il suo appartamento episcopale e brontola: ``Un
po\textquotesingle{} di questa spazzatura deve sparire''. Di solito il
mutamento era carico di pappetta addolcita. Quando la vecchia pappetta
era esaurita, se ne aggiungeva una nuova. Ma ciò che non si esaurisce
era oro puro, e durava. Se una Chiesa aveva sopportato cinque secoli di
cattivo gusto, qualche sprazzo di buon gusto aveva di solito spazzato
via la maggioranza delle croste caduche, e ne aveva fatto un luogo di
maestà che intimoriva gli aspiranti innovatori.

L\textquotesingle abate si sventolò con un ventaglio di penne di poiana,
ma la brezza non lo rinfrescò. L\textquotesingle aria che entrava dalla
finestra era simile al soffio d\textquotesingle un forno, poiché
proveniva dal deserto arroventato, e si aggiungeva alla sofferenza
causatagli dal demonio o dall\textquotesingle angelo spietato che gli
tormentava lo stomaco. Era quell\textquotesingle afa che annuncia
pericoli in agguato da parte dei serpenti a sonagli inferociti dal sole,
dei temporali che turbinano sulle montagne, dei cani idrofobi e dei
temperamenti incattiviti dal calore. E peggiorava i suoi crampi.

``Per favore'' mormorò forte al santo, in una preghiera senza parole per
implorare un clima più fresco, uno spirito più acuto, e una maggiore
comprensione di quella vaga sensazione di inquietudine. ``Forse è stata
colpa di quel formaggio'' pensò. "E gommoso, in questa stagione, ed è
troppo fresco. Io potrei concedermi una dispensa\ldots{} e una dieta più
digeribile.

"Ma no, ci siamo di nuovo. Affronta la verità, Paulo: non è il cibo per
il ventre che provoca questo; è il cibo per il cervello.
C\textquotesingle è in giro qualcosa di indigesto.

``Ma cosa?''

Il santo di legno non gli diede una pronta risposta. Pappetta.
Mutamenti. Qualche volta la sua mente lavorava a tratti. Era meglio
lasciarla lavorare in quel modo, quando sopravvenivano i crampi e il
mondo lo opprimeva con il suo peso. Quanto pesava il mondo? Pesa, ma non
è pesato. Qualche volta le sue bilance sono alterate. Pesa la vita e la
fatica, contro argento e oro. Non vi sarà \emph{mai} equilibrio.

Ma, rapido e spietato, continua a pesare. Versa tutto intorno una
quantità di vita in questo modo, e qualche volta anche un
po\textquotesingle{} d\textquotesingle oro. E un re bendato viene a
cavallo attraverso il deserto, con una bilancia alterata e un paio di
dadi truccati. E sulle sue bandiere blasonate\ldots{} \emph{Vexilla
	regis\ldots{}}

--- No --- gemette l\textquotesingle abate, respingendo la visione.

\emph{Ma naturalmente!} sembrava insistere il sorriso legnoso del santo.

Don Paulo distolse lo sguardo dall\textquotesingle immagine con un lieve
brivido. Qualche volta aveva l\textquotesingle impressione che il santo
ridesse di lui. ``Ridono di noi, in Paradiso?'' si chiese. Anche santa
Maisie di York --- ricordati di lei, vecchio! --- morì di un accesso di
risa. Questo è diverso. Morì ridendo di se stessa. No, non è tanto
diverso. \emph{Ulp!} Di nuovo il singulto silenzioso. Giovedì è la festa
di santa Maisie. Il coro ride con reverenza
all\textquotesingle{}\emph{Alleluia} della sua messa: \emph{``Alleluia
	ah ah! Alleluia oh oh!''}.

\emph{``Sancta Maisie, interride pro me''.}

E il re veniva per pesare i libri nel sotterraneo con le sue bilance
alterate. Quanto alterate, Paulo? E cosa ti fa pensare che i Memorabilia
siano completamente esenti da pappette edulcorate? Persino il dotato e
venerabile Boedullus osservò ironicamente, una volta, che almeno metà
dei Memorabilia avrebbero dovuto essere chiamati gli Inscrutabilia.
Erano veramente frammenti tesaurizzati di una civiltà morta\ldots{} ma
quanto di essi era stato ridotto a balbettamenti incomprensibili,
abbelliti di rami d\textquotesingle ulivo e di cherubini, da quaranta
generazioni di monaci ignoranti, bambini dei secoli
dell\textquotesingle oscurantismo cui gli adulti affidarono un messaggio
incomprensibile, che doveva essere imparato a memoria e consegnato ad
altri adulti\ldots{}

``L\textquotesingle ho costretto a venire fin qui da Texarkana,
attraverso un cammino pericoloso'' pensò Paulo. ``E adesso mi preoccupo
perché ciò che noi abbiamo qui potrebbe rivelarsi privo di valore ai
suoi occhi, ecco tutto''.

Ma no, non era tutto. Guardò di nuovo il santo sorridente. E di nuovo:
\emph{Vexilla regis inferni prodeunt\ldots{}} Avanzano le bandiere del
Re dell\textquotesingle Inferno, sussurrò il ricordo di quel verso
alterato di una antica \emph{commedia}. Aleggiava nel suo pensiero come
una melodia indesiderata.

Strinse più forte il pugno. Lasciò cadere il ventaglio e respirò fra i
denti. Evitò di guardare di nuovo il santo. L\textquotesingle angelo
spietato gli inferì di nuovo un colpo ardente nel suo midollo corporeo.
Si piegò sulla scrivania. Era stato come se un filo ardente si fosse
spezzato. Il suo respiro pesante ripulì un angoletto del piano della
scrivania dal lieve strato di polvere del deserto che vi si era posata.
L\textquotesingle odore della polvere era soffocante. La stanza divenne
rosea, brulicante di moscerini neri. ``Non oso singultare, potrebbe
spezzarsi qualcosa\ldots{} ma per il Santo Patrono, devo farlo. È
dolore. Ergo sum. Cristo, Dio Signore, accetta questo pegno.''

Singultò; aveva un sapore salato, in bocca. Lasciò cadere il capo sulla
scrivania.

``Il calice è qui, in questo momento, Signore, o posso attendere ancora?
Ma la crocefissione è sempre in questo momento. Da prima di Abramo, è in
questo momento. Sempre, perché chiunque, in qualunque modo, viene
inchiodato alla croce e poi vi è sollevato, e se tu cedi ti batteranno a
morte con un badile, quindi comportati con dignità, vecchio. Se sai
vomitare con dignità potrai ascendere al cielo, se ti dispiace
abbastanza di aver rovinato il tappeto\ldots'' Si sentiva molto umile.

Attese, a lungo. Alcuni dei moscerini morirono e la stanza perdette il
suo riflesso rosato, divenne grigia e nebbiosa.

Ebbene, Paulo, avremo un\textquotesingle emorragia subito, o ci
limitiamo a immaginarla?

Filtrò lo sguardo attraverso la nebbia e ritrovò di nuovo il volto del
santo. Era un sogghigno così lieve.. triste, comprensivo\ldots{} ed era
anche qualcosa d\textquotesingle altro. Rideva del carnefice? No, rideva
per il carnefice. Rideva dello \emph{Stultus Maximus}, dello stesso
Satana. Era la prima volta che lo capiva chiaramente.
Nell\textquotesingle ultimo calice, poteva esservi una risata di
trionfo. \emph{Haec commixtio\ldots{}}

Improvvisamente sentì di avere molto sonno; il volto del santo ingrigì,
ma l\textquotesingle abate continuò a sogghignare lievemente, in
risposta.

Il priore Gault lo trovò abbandonato sulla scrivania, poco prima
dell\textquotesingle Ora Nona. Gli colava un filo di sangue, dai denti.

Il giovane prete gli tastò prontamente il polso. Don Paulo si svegliò
all\textquotesingle improvviso, si raddrizzò sulla sedia e, come se
stesse ancora sognando, pontificò imperiosamente: --- Io vi dico, è
tutto supremamente ridicolo! È assolutamente sciocco! Non potrebbe
esservi nulla di più assurdo.

--- Che cosa è assurdo, Donne?

L\textquotesingle abate scosse il capo, e batté parecchie volte le
palpebre. --- Cosa?

--- Andrò subito a cercare frate Andrew.

--- Oh? Questo è assurdo. Ritornate qui, immediatamente. Che cosa
volevate?

--- Niente, Padre Abate. Tornerò. non appena avrò trovato frate\ldots{}

--- Oh, disturbare il medico! Non eravate venuto qui senza ragione. La
mia porta era chiusa. Chiudetela di nuovo, sedetevi, ditemi che cosa
volevate.

--- L\textquotesingle esperimento ha avuto successo. La lampada di frate
Kornhoer, voglio dire.

--- Benissimo, sentiamo. Sedetevi, cominciate a parlare, ditemi tutto.
--- Si riassestò l\textquotesingle abito e si asciugò la bocca con un
pezzo di lino. Era ancora stordito, ma il pugno nel suo ventre si era
schiuso. Non avrebbe potuto provare un interesse minore per il racconto
dell\textquotesingle esperimento, ma fece del suo meglio per mostrarsi
attento. ``Devo tenerlo qui, finché sono ancora abbastanza sveglio per
pensare. Non posso lasciarlo andare dal medico\ldots{} non ancora. La
notizia si spargerebbe: Il vecchio è finito. Devo decidere se è un
momento sicuro o no per essere finito''.
