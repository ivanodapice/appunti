	\chapter{\phantom{text}}

\lettrine{E} poi, Padre, ho quasi preso il pane e il cacio.

--- Ma non li hai presi?

--- No.

--- Allora non è stato un peccato.

--- Ma li desideravo tanto, potevo sentirne il sapore.

--- Deliberatamente? Ti sei compiaciuto deliberatamente di quella
fantasia?

--- No.

--- Tu hai cercato di allontanarla.

--- Sì.

--- Quindi non hai peccato di gola neppure con il pensiero. Perché stai
confessando questo?

--- Perché allora ho perduto la calma e l\textquotesingle ho spruzzato
con l\textquotesingle acqua santa.

--- Cosa? Perché?

Padre Cheroki, con la stola sulle spalle, fissò il penitente che era
inginocchiato davanti a lui nella bruciante luce solare del deserto; il
prete continuava a chiedersi come fosse possibile che quel giovane (non
particolarmente intelligente, a quanto poteva stabilire) riuscisse a
trovare occasioni o quasi occasioni di peccato mentre era completamente
isolato nel deserto spoglio, lontano da ogni distrazione e da ogni
palese fonte di tentazione. C\textquotesingle erano ben pochi peccati
che un giovane poteva commettere in quel luogo, armato come era soltanto
d\textquotesingle un rosario, d\textquotesingle una pietra focaia,
d\textquotesingle un temperino e d\textquotesingle un libro di
preghiere. Così pareva a padre Cheroki. Ma questa confessione stava
richiedendo molto tempo; si augurò che il ragazzo si sbrigasse.
L\textquotesingle artrite aveva ripreso a tormentarlo, ma per la
presenza del Santissimo Sacramento sulla tavola portatile che recava con
sé nelle sue ronde, il prete preferiva rimanere in piedi, o
inginocchiarsi insieme al penitente. Aveva acceso una candela davanti
alla piccola pisside dorata che conteneva le Ostie, ma la fiamma era
invisibile nel bagliore del sole, e forse la brezza poteva averla già
spenta.

Ma l\textquotesingle esorcismo è permesso in questi tempi, anche senza
nessuna specifica autorizzazione superiore. Cosa stai
confessando\ldots{} di esserti adirato?

--- Anche.

--- E con chi ti sei adirato? Con il vecchio\ldots{} o con te stesso
perché avevi quasi accettato il cibo?

--- Io\ldots{} non ne sono sicuro.

--- Bene, deciditi --- disse impaziente padre Cheroki. --- O accusi te
stesso o non ti accusi.

--- Mi accuso.

--- Di che? --- sospirò Cheroki.

--- Di aver abusato d\textquotesingle un sacramentale in un accesso di
collera.

``Abusato?'' Non avevi una ragione logica per sospettare una influenza
diabolica? Ti sei limitato a adirarti e ad aspergerlo? Come se gli
avessi buttato l\textquotesingle inchiostro negli occhi?

Il novizio si agitò ed esitò, comprendendo il sarcasmo del prete, La
confessione era sempre difficile per frate Francis. Non riusciva mai a
trovare le parole adatte per i suoi misfatti, e quando cercava di
ricordare i suoi moventi, si confondeva irrimediabilmente. Il prete, dal
canto suo, non lo aiutava, partendo dal punto di vista
``l\textquotesingle hai-fatto-o-non-l\textquotesingle hai-fatto''\ldots{}
anche se, ovviamente, Francis aveva fatto una cosa o non
l\textquotesingle aveva fatta.

Credo di aver perduto il senno per un momento --- disse alla fine.

Cheroki aprì la bocca, come se intendesse discutere la faccenda, poi
cambiò idea. --- Capisco. E poi?

--- Pensieri di ghiottoneria --- disse Francis dopo un attimo.

Il prete sospirò. --- Credevo che avessimo finito, con questo. O è stata
un\textquotesingle altra volta?

--- Ieri. C\textquotesingle era quella lucertola, Padre. Era a strisce
azzurre e gialle, e delle cosce magnifiche\ldots{} grosse come il vostro
pollice, e grasse, e io continuavo a pensare che avrebbe avuto lo stesso
sapore del pollo, arrostita, tutta bruna e croccante di fuori e\ldots{}

--- Benissimo --- l\textquotesingle interruppe il prete. Soltanto una
sfumatura di repulsione alterò il suo vecchio viso. Dopotutto, quel
ragazzo passava molto tempo al sole. --- Hai preso piacere da questi
pensieri? Non hai cercato di allontanare la tentazione?

Francis arrossì. --- Io\ldots{} io ho cercato di prendere la lucertola.
É scappata.

--- Dunque non è stato soltanto un pensiero\ldots{} c\textquotesingle è
stata anche l\textquotesingle azione. Quella volta soltanto?

--- Ecco\ldots{} sì, solo quella volta.

--- Benissimo, pensiero e azione, con intenzione deliberata di mangiare
carne durante la Quaresima. Ti prego di essere più specifico che puoi,
dopo questo. Pensavo che avessi fatto un adeguato esame di coscienza.
C\textquotesingle è altro?

--- Oh, molte cose.

Il prete rabbrividì. Doveva visitare parecchi eremitaggi, era un cammino
lungo e afoso, e le ginocchia gli dolevano. --- Ti prego di sbrigartela
più presto che puoi.

--- Impurità, una volta.

--- Pensieri, parole o fatto?

--- Ecco, c\textquotesingle era quella succuba e lei\ldots{}

--- Succuba? Oh\ldots{} di notte. Dormivi?

--- Sì, ma\ldots{}

--- Allora perché lo confessi?

--- A causa di\ldots{} dopo.

--- Dopo che cosa? Quando ti sei svegliato?

--- Sì. Ho continuato a pensare a lei. Ho continuato a immaginarmi tutto
di nuovo.

--- Bene, pensiero concupiscente, deliberatamente intrattenuto. Sei
pentito? E poi?

Queste erano le solite cose che continuava a udire, interminabilmente,
da un postulante dopo l\textquotesingle altro, da un novizio dopo
l\textquotesingle altro, e a padre Cheroki sembrava che il meno che
frate Francis potesse fare era di, abbaiare le sue autoaccuse \emph{un,
	due, tre,} in modo del tutto ordinato, senza bisogno di essere pungolato
e sospinto perché gliele dicesse. Frate Francis sembrava trovare
difficoltà nel formulare tutto ciò che stava per dire; il prete
attendeva.

--- Io credo che la vocazione sia venuta a me, Padre, ma\ldots{} ---
Francis si inumidì le labbra screpolate e fissò un insetto su una
pietra.

--- Oh, davvero? --- La voce di Cheroki era incolore.

--- Sì, io credo\ldots{} ma sarebbe un peccato, padre, se quando
l\textquotesingle ho visto per la prima volta, ho pensato con ironia a
quella scrittura? Voglio dire\ldots{}

Cheroki batté le palpebre. Scrittura? Vocazione? Che domanda era
quella\ldots{} Studiò l\textquotesingle espressione seria del novizio
per pochi secondi, poi corrugò la fronte,

--- Tu e frate Alfred vi siete scambiati dei. biglietti? --- chiese in
tono minaccioso.

--- Oh, no, padre!

--- E allora di che scrittura stai parlando?

--- Di quella del beato Leibowitz.

Cheroki fece una pausa per riflettere. Esisteva o non esisteva, nella
collezione di documenti antichi dell\textquotesingle abate, un
manoscritto originale del fondatore dell\textquotesingle Ordine?\ldots{}
una copia originale? Dopo un attimo di riflessione, decise
affermativamente: sì, ne erano rimasti alcuni frammenti, accuratamente
custoditi sottochiave.

Stai parlando di qualcosa che è accaduto all\textquotesingle abbazia?
Prima che venissi qui?

--- No, padre. É accaduto là\ldots{} --- E accennò verso sinistra. ---
Sul terzo monticello, vicino ai cactus più alto.

Qualcosa che riguardava la tua vocazione, dici?

--- S-sì, ma\ldots{}

--- \emph{Naturalmente ---} disse Cheroki con voce tagliente. --- Non
vorrai farmi credere di aver ricevuto\ldots{} dal beato Leibowitz, morto
ormai da seicento anni\ldots{} un invito scritto a professare i voti
solenni? E che tu\ldots{} ehm\ldots{} hai deplorato la sua grafia?
Perdonami, ma è l\textquotesingle impressione che ne ho avuto io.

--- Ecco, è qualcosa del genere, padre.

Cheroki farfugliò qualcosa. Allarmato, frate Francis si tolse dalla
manica un pezzo di carta e lo porse al prete. Era macchiato, reso
fragile dal tempo. L\textquotesingle inchiostro era sbiadito. ---
\emph{Un etto pasticcini} --- pronunciò padre Cheroki, sbagliando
qualcuna delle parole poco familiari --- \emph{scatola crauti\ldots{}
	portare a casa per Emma. ---} Guardò fisso frate Francis per parecchi
secondi. --- E questo chi l\textquotesingle avrebbe scritto?

Francis glielo disse.

Cheroki rifletté. --- Non è possibile che tu faccia una buona
confessione finché sei in queste condizioni. E non sarebbe giusto che io
ti assolvessi, quando non sei del tutto lucido. --- Quando vide Francis
rabbrividire, il prete gli posò una mano sulla spalla, per rassicurarlo.
--- Non preoccuparti figliolo, ne parleremo quando ti sentirai meglio.
Allora ascolterò la tua confessione. Per il momento\ldots{} --- E lanciò
uno sguardo nervoso alla pisside che conteneva
l\textquotesingle Eucarestia. --- Per il momento voglio che tu raccolga
le tue cose e ritorni immediatamente all\textquotesingle abbazia.

--- Ma, padre, io\ldots{}

--- Io ti ordino --- disse il prete con voce incolore --- di ritornare
immediatamente all\textquotesingle abbazia.

--- S-sì, padre.

--- Ora, non ti assolverò, ma tu potresti fare un buon atto di
contrizione e offrire due decine del rosario come penitenza, in ogni
caso. Vuoi la mia benedizione?

Il novizio annuì, ricacciando le lacrime. Il prete lo benedisse, si
alzò, si genuflesse davanti al Sacramento, riprese la pisside dorata e
la riattaccò alla catena che portava al collo. Rimise in tasca la
candela, ripiegò la tavola e l\textquotesingle assicurò al suo posto
dietro la sella, poi rivolse a Francis un ultimo cenno solenne, salì in
groppa alla cavalla e si allontanò per completare la sua visita agli
eremitaggi quaresimali. Francis sedette sulla sabbia rovente e pianse.

Sarebbe stato semplice se avesse potuto condurre il prete nella cripta
per mostrargli l\textquotesingle antica stanza, se avesse mostrato la
cassetta e il suo contenuto e il segno che il pellegrino aveva tracciato
sulla pietra. Ma il prete portava l\textquotesingle Eucarestia, e non si
sarebbe lasciato convincere a scendere in una cantina piena di sassi
camminando sulle mani e sulle ginocchia, o a frugare nel contenuto della
vecchia cassetta e a addentrarsi in discussioni archeologiche. La visita
di Cheroki era necessariamente solenne, fino a che la pisside conteneva
anche una sola Ostia; tuttavia quando la pisside fosse stata vuota,
avrebbe potuto ascoltarlo, in via ufficiosa. Il novizio non poteva
biasimare padre Cheroki se aveva creduto che lui fosse uscito di senno.
Era veramente un po\textquotesingle{} stordito dal sole, e aveva
balbettato molto. Più di un novizio era ritornato sconvolto da una
vigilia di vocazione.

Non c\textquotesingle era altro da fare che obbedire
all\textquotesingle ordine di ritornare.

Si diresse verso il rifugio e vi lanciò ancora uno sguardo, per
assicurarsi che c\textquotesingle era veramente; poi andò a prendere la
cassetta. Quando l\textquotesingle ebbe richiusa e fu pronto per
andarsene, il vortice di polvere apparve verso sud-est, annunciando
l\textquotesingle arrivo del rifornimento d\textquotesingle acqua e di
grano dall\textquotesingle abbazia. Frate Francis decise di aspettare il
suo rifornimento prima di avviarsi per il lungo viaggio di ritorno.

Tre asinelli e un monaco comparvero, in testa alla scia di polvere. Il
primo asinello vacillava sotto il peso di frate Fingo. Nonostante il
cappuccio, Francis riconobbe l\textquotesingle aiutante del cuoco dalle
spalle aggobbite e dalle lunghe caviglie pelose che penzolavano dai
fianchi del ciuchino, così che i sandali di frate Fingo quasi sfioravano
il suolo. Gli animali che lo seguivano erano carichi di piccole bisacce
di grano e di otri d\textquotesingle acqua.

--- \emph{Suuuuuuu}, porco-porco-porco! \emph{Suuu} porco! ---chiamò
Fingo, portandosi le mani alla bocca e lanciando il grido attraverso le
rovine, come se non avesse veduto Francis che lo aspettava accanto alla
pista. --- Porco-porco-porco!\ldots{} Oh, sei là, Francisco! Ti avevo
scambiato per un mucchio d\textquotesingle ossa. Bene, dobbiamo
ingrassarti per i lupi. Ecco qua, serviti per i banchetti domenicali.
Come va l\textquotesingle eremitaggio? Credi di fartene una carriera?
Solo un otre d\textquotesingle acqua, ti dispiace? e un sacchetto di
grano. E sta\textquotesingle{} attento alle zampe posteriori di Malicia:
è in calore e ha voglia di scherzare\ldots{} ha dato un calcio ad
Alfred, laggiù\ldots{} \emph{pam}! proprio sul ginocchio. Stai attento!
--- Frate Fingo si spinse indietro il cappuccio e ridacchiò mentre il
novizio e Malicia si mettevano in posizione. Fingo era senza dubbio
l\textquotesingle uomo più brutto del mondo e quando rideva il vasto
spiegamento di gengive rosee e di grossi denti di vario colore
aggiungeva ben poco al suo fascino; era un anormale, ma difficilmente un
anormale poteva essere definito mostruoso; era una caratteristica
ereditaria piuttosto comune nel paese del Minnesota da cui proveniva;
produceva calvizie e una distribuzione molto ineguale di melanina, così
che la pelle del monaco era un mosaico di macchie color fegato e
cioccolata su uno sfondo albino. Tuttavia, il suo perpetuo buonumore
compensava il suo aspetto, tanto che la gente non lo notava più, dopo
pochi minuti; e, dopo una lunga consuetudine, le caratteristiche di
frate Fingo sembravano normali quanto quelle d\textquotesingle un pony
pezzato. Ciò che sarebbe sembrato orribile in un individuo imbronciato,
diventava quasi decorativo, come il trucco d\textquotesingle un
pagliaccio, se era accompagnato da un esuberante buonumore.
L\textquotesingle assegnazione di Fingo alla cucina era una punizione,
probabilmente temporanea. Era uno scultore in legno e di solito lavorava
nella carpenteria. Ma qualche episodio di presunzione, a proposito di
una figura del beato Leibowitz che aveva avuto il permesso di scolpire,
aveva indotto l\textquotesingle abate a trasferirlo in cucina fino a che
non mostrasse di far pratica di umiltà. Nel frattempo, la statua del
Beato aspettava nella carpenteria, scolpita a metà.

Il sogghigno di Fingo cominciò a svanire mentre studiava
l\textquotesingle espressione di Francis che scaricava il grano e
l\textquotesingle acqua dalla capricciosa somarella.

Mi sembri una pecora ammalata, ragazzo --- disse al penitente. --- Cosa
succede? Padre Cheroki ha ancora una delle sue crisi di rabbia lenta?

Frate Francis scosse il capo. --- No, che io sappia.

--- E allora cosa c\textquotesingle è? Sei veramente ammalato?

--- Mi ha ordinato di ritornare all\textquotesingle abbazia.

--- Co-o-o-sa? --- Fingo fece ruotare una caviglia pelosa al di sopra
dell\textquotesingle asino e piombò al suolo da
un\textquotesingle altezza di pochi centimetri. Torreggiò su frate
Francis, gli batté sulla spalla una mano carnosa, e lo guardò in faccia.
--- Cos\textquotesingle è, itterizia?

--- No. Crede che io sia\ldots{} --- Francis si batté un dito sulla
tempia e scrollò le spalle.

Fingo rise. --- Bene, è vero, mo lo sappiamo tutti. Perché ti rimanda
indietro?

Francis gettò uno sguardo sulla cassetta, vicino ai suoi piedi. --- Ho
trovato alcune cose appartenute al beato Leibowitz. Ho cominciato a
dirglielo, ma non mi ha creduto. Non ha lasciato che gli spiegassi.
Ha\ldots{}

--- Hai trovato \emph{che cosa?} --- Fingo sorrise incredulo, poi cadde
in ginocchio e aprì la cassetta mentre il novizio osservava nervoso. Il
monaco rimestò con un dito i cilindri baffuti negli scomparti e zufolò
sommessamente. --- Incantesimi dei pagani delle colline, no? È roba
antica, Francisco, veramente antica. Guardò il biglietto sul coperchio.

--- Cosa sono quelle sciocchezze? --- chiese, guardando lo sconsolato
novizio attraverso gli occhi socchiusi.

--- Inglese prediluviale.

--- Non l\textquotesingle ho mai studiato, tranne quello che cantiamo in
coro.

--- È stato scritto dal Beato in persona.

--- Questo? --- Frate Fingo spostò lo sguardo dal biglietto a frate
Francis e poi tornò a posarlo sul foglio. Scosse improvvisamente il
capo, richiuse la cassetta e si alzò. Il suo ghigno diventò artificiale.
--- Forse il padre ha ragione. Farai meglio a ritornare indietro e a
farti preparare dal frate farmacista qualcuna delle sue specialità a
base di funghi. Hai la febbre, fratello.

Francis alzò le spalle. --- Forse.

--- Dove hai trovato questa roba?

Il novizio glielo indicò. --- Da quella parte, dopo qualche monticello.
Ho smosso qualche pietra. C\textquotesingle è stata una frana, e ho
trovato un sotterraneo. Vai a vedere tu stesso.

Fingo scosse il capo. --- Mi aspetta un bel po\textquotesingle{} di
strada.

Francis raccolse la cassetta e si avviò verso l\textquotesingle abbazia
mentre Fingo ritornava ai suoi asinelli, ma dopo pochi passi il novizio
si fermò e lo chiamò.

--- Frate Macchie\ldots{} puoi perdere due minuti?

--- Forse --- rispose Fingo. Perché?

--- Allora vai là e guarda nella buca.

--- Perché?

--- Così potrai dire a padre Cheroki che c\textquotesingle è davvero.

Fingo si fermò, con una gamba già a cavalcioni del somaro. --- Ah! --- E
ritirò la gamba. --- Benissimo. E se non c\textquotesingle è, lo dirò a
te!

Francis osservò per un momento, mentre Fingo si allontanava a grandi
passi, scomparendo fra i monticelli; poi si voltò per percorrere, a
passi strascicati, la lunga pista polverosa verso
l\textquotesingle abbazia, mangiucchiando a intermittenza un
po\textquotesingle{} di grano e bevendo qualche sorso
dall\textquotesingle otre. Ogni tanto si voltava a guardarsi indietro.
Fingo era scomparso da più di due minuti. Frate Francis aveva smesso di
aspettarne la ricomparsa quando udì un grido lontano levarsi dalle
rovine, dietro di lui. Si voltò. Riuscì a distinguere la figura dello
scultore ritta su uno dei monticelli. Fingo agitava le braccia e annuiva
vigorosamente con il capo in segno affermativo. Francis agitò le braccia
a sua volta, poi proseguì fiaccamente il suo cammino.

Due settimane di inedia quasi totale avevano preteso il loro tributo.
Dopo due o tre miglia cominciò a barcollare. Quando distava ancora un
miglio dall\textquotesingle abbazia, svenne accanto alla strada. Era
pomeriggio inoltrato quando Cheroki, di ritorno dalle stie visite, lo
vide lì disteso, smontò in fretta e bagnò il viso del giovane fino a che
lo fece gradualmente rinvenire. Cheroki aveva incontrato gli asinelli
del rifornimento durante il cammino di ritorno, e si era, fermato ad
ascoltare il racconto di Fingo, che confermava la scoperta di frate
Francis. Sebbene non fosse disposto a credere che Francis avesse
scoperto qualcosa di veramente importante, il prete si pentì della sua
impazienza di poco prima nei confronti del giovane. Quando ebbe notato
la cassetta che giaceva, lì accanto, con il suo contenuto parzialmente
sparso al suolo, e quando ebbe lanciato un breve sguardo al foglietto
incollato sul coperchio, mentre Francis sedeva, stordito e confuso, sul
ciglio della pista, Cheroki cominciò a considerare i balbettamenti del
ragazzo come il risultato d\textquotesingle una immaginazione romantica
piuttosto che del delirio o della pazzia. Non aveva visitato la cripta e
non aveva esaminato attentamente il contenuto della cassetta, ma era
evidente, per lo meno, che il ragazzo aveva interpretato erroneamente
alcuni eventi reali, invece di confessare delle allucinazioni.

--- Puoi finire la tua confessione non appena saremo arrivati --- disse
sottovoce al novizio, aiutandolo a salire dietro la sella della
giumenta. --- Credo di poterti assolvere se non insisti
nell\textquotesingle affermare d\textquotesingle aver ricevuto messaggi
personali dai santi. Eh?

Per il momento, frate Francis era troppo debole per insistere su
qualsiasi cosa.
