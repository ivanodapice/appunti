	\chapter{\phantom{title}}

\lettrine{L}{}\textquotesingle ora era giunta.

Frate Francis, nel suo semplice abito da monaco, non si era mai sentito
meno importante che in quel momento, mentre si inginocchiava nella
maestosa basilica, prima che iniziasse la cerimonia. I movimenti
solenni, i vividi vortici di colore, i suoni che accompagnavano gli
scrupolosi preparativi della cerimonia sembravano già liturgici, in
ispirito, e rendevano difficile pensare che per il momento non stava
accadendo ancora qualcosa di importante. Vescovi, monsignori, cardinali,
preti e funzionari laici in abiti eleganti e antiquati andavano qua e là
nella grande chiesa, ma il loro andirivieni era un aggraziato movimento
a orologeria che non si fermava, non incespicava, non cambiava mai
direzione per dirigersi altrove. Un \emph{sampetrius} entrò nella
basilica: era così grandioso che Francis, dapprima, scambiò
l\textquotesingle operaio della cattedrale per un prelato. Il
\emph{sampetrius} reggeva uno sgabello poggiapiedi. Lo portava con tale
distratta pomposità che il monaco, se non fosse stato già inginocchiato,
si sarebbe genuflesso mentre l\textquotesingle oggetto gli passava
davanti. Il \emph{sampetrius} posò un ginocchio al suolo, davanti
all\textquotesingle altare, poi si avviò verso il trono papale dove mise
lo sgabello al posto di un altro, che sembrava avesse una gamba
allentata; poi si allontanò, facendo lo stesso percorso per cui era
venuto. Frate Francis si meravigliò della studiata eleganza di movimenti
che accompagnava persino i gesti più insignificanti. Nessuno aveva
fretta. Nessuno si muoveva a casaccio. Non si compiva alcun movimento
che non contribuisse quietamente alla dignità e alla bellezza
sopraffacente di questo luogo antico, come vi contribuivano le statue
immote e i dipinti. Persino il fruscio dei respiri sembrava echeggiare
debolmente nelle absidi lontane.

\emph{Terribilis est locus iste: hic domus Dei est, et porta coeli:}
questo è un luogo terribile, la Casa di Dio e la Porta del Cielo!

Alcune delle statue erano vive, notò Francis dopo qualche tempo. Una
armatura stava contro una parete, a sinistra, a pochi metri da lui. Il
suo pugno serrato in un guanto di maglia di ferro reggeva
l\textquotesingle impugnatura d\textquotesingle una splendente alabarda.
Neppure la piuma sull\textquotesingle elmo si era agitata, durante il
tempo che frate Francis aveva trascorso lì, in ginocchio. Una dozzina di
armature identiche era piazzata, a intervalli, lungo le pareti. Soltanto
dopo aver veduto una mosca cavallina che strisciava attraverso la
visiera della ``statua'' alla sua sinistra, cominciò a sospettare che
l\textquotesingle armatura contenesse un occupante. Il suo sguardo non
riusciva a distinguere alcun movimento, ma l\textquotesingle armatura
emise alcuni cigolii metallici, mentre ospitava la mosca. Quelle,
dunque, dovevano essere le guardie pontificie, così favolose per le loro
cavalleresche battaglie; il piccolo esercito privato del Vicario di
Cristo.

Un capitano delle guardie stava ispezionando maestosamente i suoi
uomini. Per la prima volta, la statua si mosse. Alzò la visiera in atto
di saluto. Il capitano si fermò, pensieroso, poi si servì del fazzoletto
per togliere la mosca dalla fronte di quel viso inespressivo chiuso
nell\textquotesingle elmo, prima di passare oltre. La statua riabbassò
la visiera e ritornò immobile.

Il maestoso scenario della basilica fu brevemente guastato
dall\textquotesingle ingresso delle folle di pellegrini. Quei gruppi
erano bene organizzati e guidati con efficienza, ma era chiaro che non
conoscevano la basilica. Quasi tutti parevano dirigersi in punta di
piedi verso i rispettivi posti, badando a non fare rumore e a muoversi
il meno possibile, a differenza dei \emph{sampetrii} e del clero di
Nuova Roma che rendevano eloquente ogni suono e ogni gesto. Qua e là,
tra i pellegrini, qualcuno tossiva o incespicava.

Improvvisamente la basilica assunse un aspetto guerresco, quando la
guardia venne rafforzata. Un nuovo drappello di guardie in giachi di
maglia entrò nel santuario; gli uomini posarono al suolo un ginocchio,
inclinarono le alabarde, salutando l\textquotesingle altare prima di
prendere posto. Due di essi si misero a fianco del trono papale. Un
terzo cadde in ginocchio alla destra del trono; e rimase lì, sorreggendo
la Spada di Piero sulle palme levate. Il quadro ritornò immobile, a
eccezione di qualche guizzo delle fiamme delle candele accese
sull\textquotesingle altare. Nel silenzio profondo esplose improvviso
uno squillo di trombe.

L\textquotesingle intensità del suono crebbe fino a che il pulsante
\emph{Ta-ra Ta-ra Ta-ra} batté sul volto dei presenti e diventò doloroso
alle orecchie. La voce delle trombe non era musicale, ma annunciatoria.
Le prime note cominciarono a metà della scala, poi salirono lentamente
di intensità, di tono e di imperiosità, fino a che il monaco si sentì
accapponare la pelle del cranio, fino a che sembrò che non vi fosse
altro, nella basilica, a eccezione dell\textquotesingle esplosione delle
trombe.

Poi, un silenzio mortale\ldots{} seguito dal grido d\textquotesingle un
tenore.
\leavevmode\\

PRIMO CANTORE:
\leavevmode\\

Appropinquat agnis pastor et ovibus pascendis.
\leavevmode\\

SECONDO CANTORE:
\leavevmode\\

Genua nunc flectantur omnia.
\leavevmode\\

PRIMO CANTORE:
\leavevmode\\

Jussit ohm Jesus Petrum pascere gregem Domini.
\leavevmode\\

SECONDO CANTORE:
\leavevmode\\

Ecce Petrus Pontifex Maximus.
\leavevmode\\

PRIMO CANTORE:
\leavevmode\\

Gaudeat igitur populus Christi, et gratias agar Domino.
\leavevmode\\

SECONDO CANTORE:
\leavevmode\\

Nam docebimur a Spiritu Sancto.
\leavevmode\\

CORO:
\leavevmode\\

Alleluia, alleluia\ldots{}
\leavevmode\\

La folla si levò e poi si inginocchiò, in una lenta ondata che seguiva
il movimento della sedia gestatoria su cui sedeva un fragile vecchio
vestito di bianco, che impartiva le sue benedizioni alla folla mentre la
processione dorata, nera, purpurea e rossa lo portava lentamente verso
il trono. Il respiro continuava a mozzarsi nella gola del piccolo monaco
venuto da una remota abbazia nel deserto lontano. Era impossibile vedere
tutto ciò che avveniva, tanto era soverchiante l\textquotesingle ondata
della musica e del movimento, che annegava i sensi e sospingeva la mente
verso ciò che stava per accadere.

La cerimonia fu breve. La sua intensità sarebbe diventata
insopportabile, se fosse durata più a lungo. Un monsignore --- Manfredo
Aguerra, l\textquotesingle avvocato del Santo, notò frate Francis --- si
avvicinò al trono e si inginocchiò. Dopo un breve silenzio, levò la sua
supplica in una calma cantilena. --- \emph{Sancte Pater, a Sapientia
	summa petimus ut ille Beatus Leibowitz, cuius miraculis mirati sunt
	multi\ldots{}}

L\textquotesingle invocazione chiedeva a Leone di illuminare il popolo
dei fedeli con una definizione solenne, relativa alla pia credenza che
il beato Leibowitz fosse in verità un santo, degno della dulia della
Chiesa come della venerazione dei fedeli.

--- \emph{Gratissima nobis causa, fili} --- cantò in risposta la voce
del vecchio vestito di bianco, il quale spiegò che era suo ardente
desiderio annunciare con proclamazione solenne che il Martire benedetto
era fra i Santi, ma anche che solamente per guida divina, \emph{sub
	ducatu Sancti Spiritus}, poteva esaudire la richiesta di Aguerra. E
chiese a tutti i fedeli di pregare per impetrare tale guida.

Di nuovo il tuono del coro riempì la basilica con le Litanie dei Santi:

``Padre Celeste, Dio, abbi misericordia di noi. Figlio, Salvatore del
Mondo, Dio, abbi misericordia di noi. Spirito Santissimo, Dio, abbi
misericordia di noi, O Santissima Trinità, Dio Uno e Trino, miserere
nobis! Santa Maria, prega per noi. Sancta Dei genitrix, ora pro nobis.
Sancta Virgo virginum, ora pro nobis\ldots''

Il tuono della litania continuò. Francis levò lo sguardo verso una
immagine del beato Leibowitz, appena scoperta.
L\textquotesingle affresco era di proporzioni gigantesche. Rappresentava
il processo del Beato davanti alla folla, ma il suo volto non sorrideva
ironicamente come sorrideva nella scultura di Fingo. Tuttavia era
maestoso, pensò Francis, e degno del resto della basilica.

\emph{``Omnes Sancti Martyres, orate pro nobis\ldots''}

Quando la litania fu finita, monsignor Manfredo Aguerra rivolse
nuovamente la sua supplica al papa, chiedendo che il nome di Isaac
Edward Leibowitz fosse ufficialmente iscritto nel Calendario dei Santi.
Di nuovo fu invocata la guida dello Spirito Santo, mentre il papa
intonava il \emph{Veni, Creator Spiritus.}

E per la terza volta Manfredo Aguerra chiese la proclamazione. ---
\emph{Surgat ergo Petrus ipse\ldots{}}

E giunse il momento. Leone XXI intonò la decisione della Chiesa, presa
sotto la guida dello Spirito Santo, proclamando il fatto che un antico e
oscuro tecnico, di nome Leibowitz, era veramente un Santo in Cielo, e
che la sua potente intercessione poteva, e legittimamente doveva, essere
reverentemente implorata. Fu stabilito un giorno per una messa in suo
onore.

--- San Leibowitz, \emph{intercedi per noi} --- mormorò frate Francis,
insieme agli altri.

Dopo una breve preghiera, il coro esplose nel \emph{Te Deum}. Dopo una
messa in onore del nuovo Santo, tutto fu finito.

Scortato da due sediari dalle livree scarlatte, il piccolo gruppo di
pellegrini passò per una sequenza di corridoi e di anticamere in
apparenza interminabile, fermandosi di tanto in tanto davanti
all\textquotesingle ornato tavolo di qualche funzionario che esaminava
le credenziali e apponeva la firma su un \emph{licet adire} perché uno
dei sediari lo consegnasse al funzionario seguente, il cui titolo
diventava progressivamente più lungo e meno pronunciabile man mano che
il corteo procedeva.

Frate Francis tremava. Fra i pellegrini c\textquotesingle erano due
vescovi, un uomo vestito di ermellino e d\textquotesingle oro, il capo
d\textquotesingle un clan della gente della foresta, convertito, che
tuttavia indossava ancora la tunica di pelle di pantera e il copricapo
di pantera del suo totem tribale, un semplicione vestito di cuoio che
portava sul polso un falco pellegrino incappucciato --- evidentemente un
dono per il Santo Padre --- e parecchie donne, che sembravano tutte
mogli o concubine --- da quanto Francis poteva giudicare dal loro
contegno --- del capo ``convertito'' del clan degli uomini-pantera; o
forse erano ex concubine messe in disparte secondo il canone ma non
secondo le usanze tribali.

Dopo aver salito la scala \emph{coelestis}, i pellegrini furono accolti
da un \emph{cameralis gestor} che indossava vesti funeree, e ammessi
nella piccola anticamera della grande sala concistoriale.

--- Il Santo Padre li riceverà qui --- disse sottovoce il lacchè di alto
rango al sediario che portava le-credenziali. Guardò i pellegrini con
aria di disapprovazione, pensò Francis, poi sussurrò qualcosa al
sediario. Il sediario arrossì e sussurrò qualcosa al capo del clan. Il
capo del clan si accigliò, si tolse dal capo l\textquotesingle ornamento
zannuto e ringhiante, e lo lasciò penzolare dalle spalle. Vi fu una
breve discussione sulle precedenze, mentre Sua Suprema Untuosità il
Lacchè, in toni così sommessi che sembravano di rimprovero, sistemava i
suoi pezzi degli scacchi nella stanza, secondo qualche arcano protocollo
apparentemente comprensibile soltanto ai sediari.

Il papa non tardò molto. Il piccolo uomo vestito di bianco, circondato
dal seguito, entrò nella sala delle udienze con passo spedito. Frate
Francis si sentì colto da vertigini. Ricordò che don Arkos aveva
minacciato di scuoiarlo vivo se fosse svenuto durante
l\textquotesingle udienza, e cercò di farsi animo.

La fila di pellegrini si inginocchiò. Il vecchio vestito di bianco li
fece alzare, con dolcezza. Finalmente frate Francis trovò il coraggio di
mettere a fuoco lo sguardo. Nella basilica, il papa era stato soltanto
un radiante punto bianco in un mare di colore. Gradualmente, qui nella
sala delle udienze, frate Francis osservò, a distanza ravvicinata, che
il papa non era, come i favolosi nomadi, alto tre metri. Con grande
sorpresa del monaco, il fragile vecchio, Padre dei Principi e dei Re,
Costruttore del Ponte sul Mondo, Vicario terreno di Cristo, sembrava
molto meno terribile di don Arkos, \emph{Abbas}.

Il papa avanzò lentamente lungo la fila dei pellegrini, salutandoli uno
per uno, abbracciando uno dei vescovi, conversando con ognuno nel suo
dialetto o attraverso un interprete, ridendo
dell\textquotesingle espressione del monsignore al quale diede
l\textquotesingle incarico di portare il rapace offerto dal falconiere,
e rivolgendosi al capo clan con un peculiare gesto della mano e una
parola che pareva un grugnito, tolta dal dialetto della foresta, che
ispirò al capo vestito da pantera un improvviso sogghigno di piacere. Il
papa notò il copricapo pendente sulle spalle dell\textquotesingle uomo e
si fermò per riaggiustarglielo sulla testa. Il petto del capo si gonfiò
d\textquotesingle orgoglio; lanciò uno sguardo fiammeggiante attraverso
la stanza, per guardare Sua Suprema Untuosità il Lacchè, ma il
funzionario sembrava scomparso nei pannelli di legno.

Il papa si avvicinò a frate Francis.

\emph{Ecce Petrus Poutitex\ldots{}} Ecco Pietro, il pontefice. Leone XXI
in persona, ``che, solo, Dio fece Principe su tutti i paesi e i regni,
con la facoltà di sradicare, di abbattere, di distruggere, di
annientare, di fondare e di costruire, affinché possa preservare un
popolo di fedeli\ldots''

Eppure sul viso di Leone il monaco vide una gentile mitezza che indicava
come egli fosse degno del titolo, molto più sommesso di quello concesso
a ogni principe o re, per cui egli era chiamato ``servo dei servi di
Dio''.

Francis si inginocchiò prontamente per baciare l\textquotesingle Anello
del Pescatore. Mentre si rialzava, si accorse di stringere dietro di sé
la reliquia del Santo, come se si vergognasse di mostrarla. Gli occhi
ambrati del pontefice lo esortavano, gentilmente. Leone parlava
sommessamente, secondo il tono curiale; una affettazione che gli pareva
sgradita, ma che praticava per amore della tradizione mentre parlava con
visitatori meno selvaggi del capo-pantera.

--- Il nostro cuore è stato profondamente afflitto quando abbiamo udito
della vostra sfortuna, diletto figlio. Un resoconto del vostro viaggio è
giunto alle nostre orecchie. Per nostra richiesta voi veniste sin qui,
ma mentre eravate in cammino foste aggredito dai ladroni. Non è vero?

--- Sì, Santo Padre. Ma non è stata una cosa importante. Voglio
dire\ldots{} \emph{era} importante, ma\ldots{} --- balbettò Francis.

Il vecchio vestito di bianco sorrise gentilmente. --- Sappiamo che ci
avevate portato un dono, e che vi fu rubato durante il viaggio. Non
siate turbato per questo. La vostra presenza è un dono sufficiente per
noi. Per lungo tempo abbiamo nutrito la speranza di incontrare in
persona lo scopritore dei resti di Emily Leibowitz. Noi sappiamo anche
del vostro lavoro all\textquotesingle abbazia. Per i Fratelli di san
Leibowitz noi abbiamo sempre provato un ferventissimo affetto. Senza il
vostro lavoro, l\textquotesingle amnesia del mondo sarebbe completa.
Poiché la Chiesa, \emph{Mysticum Christi Corpus}, è un Corpo, così il
vostro Ordine è servito come un organo della memoria, in quel Corpo. Noi
dobbiamo molto al vostro santo Patrono e Fondatore. Le età future,
forse, gli dovranno anche di più. Possiamo udire altri particolari del
vostro viaggio, diletto figlio?

Frate Francis mostrò la \emph{blueprint}. --- Il ladrone fu abbastanza
gentile da lasciarmi questo, Santo Padre. Egli\ldots{} egli la scambiò
per una copia del foglio alluminato che io intendevo portarvi in dono.

--- E voi non correggeste il suo errore?

Frate Francis arrossì. --- Mi vergogno di ammettere, Santo Padre\ldots{}

--- Dunque questa è la reliquia originale che trovaste nella cripta?

--- Sì\ldots{}

Il sorriso del papa divenne arguto. --- Quindi, allora\ldots{} il
bandito pensò che il tesoro fosse la vostra opera? Ah\ldots{} persino un
ladrone può avere un buon occhio per le opere d\textquotesingle arte,
no? Monsignor Aguerra ci parlò della bellezza della vostra copia
alluminata. E un peccato che sia stata rubata.

--- Non era nulla, Santo Padre. Mi dispiace soltanto di avere sprecato
quindici anni.

--- Sprecato? Perché ``sprecato''? Se il ladrone non fosse stato
ingannato dalla bellezza della vostra copia avrebbe potuto prendere
questa, non è vero?

Frate Francis ammise quella possibilità.

Leone XXI prese tra le mani avvizzite l\textquotesingle antica
\emph{blueprint} e la srotolò, cautamente. Studiò il disegno in silenzio
per un certo tempo, poi: --- Diteci, comprendete i simboli usati da
Leibowitz? Il significato del\ldots{} ehm\ldots{} della cosa
rappresentata?

--- No, Santo Padre, la mia ignoranza è completa.

Il papa si piegò verso di lui per sussurrare: --- Anche la nostra. ---
Rise sommessamente, posò le labbra sulla reliquia come se baciasse un
altare, poi tornò ad arrotolarla e la porse a un assistente. --- Vi
ringraziamo dal profondo del cuore per quei quindici anni, diletto
figlio --- aggiunse, rivolto a frate Francis. --- Quegli anni furono
spesi per salvare l\textquotesingle originale. Non pensate mai di averli
sprecati. Offriteli a Dio. Un giorno il significato
dell\textquotesingle originale potrà essere scoperto, e potrà rivelarsi
importante. --- Il vecchio batté le palpebre\ldots{} o ammiccò? Francis
era quasi convinto che il papa gli avesse strizzato
l\textquotesingle occhio. --- E dovremo ringraziare voi, di questo.

La strizzata d\textquotesingle occhio, o quel battito di ciglia, sembrò
mettere più chiaramente a fuoco lo sguardo del monaco. Per la prima
volta, notò che nella veste del papa c\textquotesingle era un buco fatto
da una tarma. La veste era quasi lisa. Il tappeto nella sala delle
udienze era logoro in molti punti; e in molti punti
l\textquotesingle intonaco era caduto dal soffitto. Ma la dignità
riusciva a adombrare la povertà. Solo per un attimo, dopo la strizzata
d\textquotesingle occhio, frate Francis notò quei segni di povertà. La
distrazione fu passeggera.

--- Per vostro mezzo, noi desideriamo mandare i nostri più calorosi
complimenti a tutti i membri della vostra comunità e al vostro abate ---
stava dicendo Leone. --- A essi, come a voi, noi desideriamo estendere
la nostra apostolica benedizione. Vi daremo una lettera per loro
annunciante la benedizione. --- Fece una pausa, poi di nuovo batté le
palpebre, o strizzò l\textquotesingle occhio. --- Incidentalmente, la
lettera sarà salvaguardata. Vi faremo affiggere il \emph{Noli
	molestare}, scomunicando chiunque molesti il latore.

Frate Francis mormorò il suo ringraziamento per quella garanzia contro i
banditi; non gli parve opportuno aggiungere che il ladrone poteva essere
incapace di leggere o di comprendere l\textquotesingle avvertimento. ---
Farò del mio meglio per consegnarla, Santo Padre.

Di nuovo, Leone si piegò verso di lui per sussurrare: --- E a voi, noi
daremo uno speciale pegno del nostro affetto. Prima di partire, fate
visita a monsignor Aguerra. Preferiremmo consegnarvelo con le nostre
mani, ma questo non è il momento opportuno. Il monsignore ve lo darà per
conto nostro. Fatene ciò che volete.

--- Vi ringrazio profondamente, Santo Padre.

--- E adesso addio, mio diletto figlio.

Il pontefice proseguì, parlando a tutti i pellegrini della fila, e
quando ebbe finito, impartì la benedizione solenne.
L\textquotesingle udienza era conclusa.

Monsignor Aguerra toccò il braccio di frate Francis mentre il gruppo dei
pellegrini varcava il portale. Abbracciò il monaco con affetto. Il
postulatore della causa del Santo era tanto invecchiato che Francis lo
riconobbe con difficoltà. Ma anche Francis si era fatto grigio alle
tempie, e gli erano venute le rughe attorno agli occhi, poiché li aveva
tenuti socchiusi per aguzzare la vista, al tavolo della copisteria.

Il monsignore gli porse un pacchetto e una lettera, mentre scendevano la
scala \emph{coelestis}.

Francis guardò l\textquotesingle indirizzo della lettera e annuì. Sul
pacchetto, che portava il sigillo diplomatico, c\textquotesingle era
scritto il suo nome. --- Per me, monsignore?

--- Sì, è un dono personale del Santo Padre. È meglio non aprirlo qui. E
adesso, posso fare qualcosa per te, prima che tu lasci Nuova Roma? Sarò
lieto di mostrarti ciò che può esserti sfuggito.

Frate Francis rifletté brevemente. Era già stata una visita faticosa.
--- Mi piacerebbe rivedere ancora una volta la basilica, monsignore ---
disse alla fine.

--- Sì, naturalmente. Ma questo è tutto?

Frate Francis fece un\textquotesingle altra pausa. Erano rimasti
indietro, rispetto agli altri pellegrini che se ne andavano. --- Vorrei
confessarmi --- aggiunse, sottovoce.

--- Niente di più facile --- disse Aguerra, aggiungendo con un risolino:
--- Sei nella città più adatta, sai. Ecco, puoi ottenere
l\textquotesingle assoluzione da tutto ciò che ti preoccupa.
C\textquotesingle è qualche peccato mortale che possa richiedere
l\textquotesingle attenzione del papa?

Francis arrossì e scosse il capo

--- E il Penitenziere Maggiore, allora? Non soltanto ti assolverà, se
sei pentito, ma ti toccherà anche la testa con la verga.

--- Volevo dire\ldots{} lo stavo chiedendo a voi, monsignore ---
balbettò il monaco.

\emph{---} \emph{Io?} Perché io? Non sono una persona importante. Sei in
una città piena di berretti rossi, e vuoi confessarti a Manfredo
Aguerra!

--- Perché\ldots{} perché voi siete stato l\textquotesingle avvocato del
nostro patrono --- spiegò il monaco.

--- Oh, capisco. Naturalmente, ascolterò la tua confessione. Ma non
posso assolverti in nome del tuo patrono, sai. Dovrà essere come al
solito in nome della Santissima Trinità. Ti andrà bene?

Francis aveva poco da confessare, ma il suo cuore era turbato da lungo
tempo, a causa di ciò che gli aveva detto don Arkos, dalla paura che la
sua scoperta del rifugio avesse intralciato la causa del Santo. Il
postulatore di Leibowitz lo ascoltò, lo consigliò, e
l\textquotesingle assolse nella basilica, poi gli fece da guida
nell\textquotesingle antica chiesa. Durante la cerimonia della
canonizzazione e la messa che ne era seguita, frate Francis aveva
osservato soltanto lo splendore maestoso dell\textquotesingle edificio.
Ora, il vecchio monsignore gli indicava i muri screpolati, i punti che
avevano bisogno di restauro, e le condizioni vergognose di alcuni
affreschi. Di nuovo vide uno spettacolo di povertà velato dalla dignità.
La Chiesa non era ricca, in quei tempi.

Finalmente, Francis fu libero di aprire il pacchetto: conteneva una
borsa. Nella borsa c\textquotesingle erano due heklos
d\textquotesingle oro. Guardò Manfredo Aguerra. Il monsignore sorrise.

--- Avevi detto che il ladrone ti aveva vinto la copia alluminata in una
lotta, non é vero? --- chiese Aguerra.

--- Sì, monsignore.

--- Bene, allora, anche se vi sei stato costretto, hai scelto di
batterti con lui per quella copia, non è così? Hai accettato la sua
sfida?

Il monaco annuì.

--- E allora non credo che faresti male se gliela ricomprassi. --- Batté
una mano sulla spalla del monaco e lo benedisse. Poi venne il momento di
partire.

Il piccolo custode della fiamma della conoscenza si avviò a piedi verso
l\textquotesingle abbazia. L\textquotesingle attendevano giorni e
settimane di cammino, ma il suo cuore cantava mentre si avvicinava alla
postazione del ladrone. \emph{Fatene ciò che volete}, aveva detto
dell\textquotesingle oro papa Leone. Non solo questo, pensava ora il
monaco; in aggiunta alla borsa, c\textquotesingle era una risposta alla
domanda sarcastica del ladrone. Pensava ai libri nella sala delle
udienze, che attendevano il risveglio.

Il ladrone, tuttavia, non era in attesa alla sua postazione come aveva
sperato frate Francis. In quel punto c\textquotesingle erano alcune orme
fresche, ma le orme andavano nella direzione opposta e non
c\textquotesingle era traccia del ladrone. Il sole filtrava fra gli
alberi, coprendo il suolo con l\textquotesingle ombra del fogliame. La
foresta non era fitta, ma offriva molta ombra. Sedette accanto al
sentiero, ad aspettare.

Una civetta ululò a mezzogiorno, dalla oscurità relativa del letto
prosciugato d\textquotesingle un fiume lontano. Le poiane tracciavano un
cerchio azzurro, al di sopra delle cime degli alberi. Tutto sembrava
pacifico, quel giorno, nella foresta. Mentre ascoltava assonnato i
passeri che svolazzavano negli arbusti vicini, si accorse che non gli
importava molto se il ladrone fosse giunto quel giorno o il giorno
seguente. Il suo viaggio era così lungo, che non gli sarebbe dispiaciuto
godere un giorno di riposo mentre aspettava. Rimase seduto, a osservare
le poiane. Ogni tanto riportava lo sguardo sul sentiero che conduceva
verso la sua casa lontana, nel deserto. Il ladrone aveva scelto un luogo
eccellente per i suoi agguati. Da quel punto, si poteva scorgere più di
un miglio di sentiero in ognuna delle due direzioni, pur rimanendo
inosservati nel folto della foresta.

Qualcosa si mosse in lontananza sul sentiero.

Frate Francis si schermò gli occhi e studiò quel movimento lontano.
C\textquotesingle era un\textquotesingle area soleggiata, lungo la
strada, dove un incendio aveva spazzato via parecchi acri di terra
attorno al sentiero che portava verso sud-ovest.

Non poteva vedere chiaramente a causa del riverbero splendente, ma in
mezzo a quel calore c\textquotesingle era un movimento. Era una tremante
iota nera. Qualche volta sembrava che avesse una testa. Qualche volta
era completamente oscurata nel riverbero del calore, ma nonostante tutto
riuscì a stabilire che si stava avvicinando gradualmente. Una volta,
quando l\textquotesingle orlo d\textquotesingle una nuvola passò sul
sole, il formicolio lucente del calore si quietò per pochi secondi; i
suoi stanchi occhi di miope stabilirono che la iota tremolante era
veramente un uomo, ma era troppo lontano per poterlo riconoscere.
Rabbrividì. Qualcosa, in quella iota, era troppo familiare.

Ma no, non poteva essere lo stesso.

Il monaco si segnò e cominciò a recitare il rosario mentre i suoi occhi
rimanevano fissi sulla cosa lontana, in mezzo al riverbero del calore.

Mentre era rimasto lì ad attendere il ladrone, c\textquotesingle era
stata una discussione, più in alto, sul fianco della collina. La
discussione era stata condotta in monosillabi appena sussurrati, ed era
durata quasi un\textquotesingle ora. Ora era finita. Due-Teste aveva
dato ragione a Una-Testa. Insieme i figli del papa si allontanarono
quietamente e cominciarono a strisciare, giù lungo il fianco della
collina. Giunsero a dieci metri da Francis prima che un ciottolo
rotolasse, rumoreggiando. Il monaco stava mormorando la terza \emph{Ave}
del Quarto Mistero Glorioso del rosario quando si voltò, per caso.

La freccia lo centrò in mezzo agli occhi.

--- Mangiare! Mangiare! Mangiare! --- gridò il figlio del papa.

Sul sentiero, a sud-ovest, il vecchio vagabondo sedette su un tronco e
chiuse gli occhi per riposarli dal sole. Si sventolò con un cappellaccio
sciupato e masticò una foglia aromatica. Aveva vagato per molto tempo.
La sua ricerca sembrava interminabile, ma c\textquotesingle era sempre
la promessa di trovare ciò che cercava al di là del prossimo dosso o
della prossima curva del sentiero. Quando ebbe finito di sventolarsi, si
rimise in testa il cappello e si grattò la barba irsuta, mentre si
volgeva intorno a guardare il paesaggio, battendo le palpebre.
C\textquotesingle era una striscia di foresta, indenne dal fuoco, ma il
vagabondo rimase lì, a osservare le poiane curiose. Si erano riunite, e
volavano piuttosto basse sulla fascia boschiva. Uno dei, rapaci si
azzardò a scendere fra gli alberi, ma svolazzò di nuovo in alto, volò di
forza fino a che non trovò una colonna d\textquotesingle aria
ascendente, poi si lanciò in una ripida salita. La cupa schiera di
becchini sembrava dedicare una quantità di energia superiore al normale
per sbattere le ali. Di solito planavano, risparmiando le forze. Adesso
sfrecciavano nell\textquotesingle aria, sulla collina, come se fossero
impazienti di atterrare.

Fino a che le poiane si mostrarono interessate ma riluttanti, il
vagabondo rimase dov\textquotesingle era. C\textquotesingle erano molti
puma, su quelle colline. Al di là del picco c\textquotesingle erano cose
peggiori dei puma, e qualche volta si spingevano fin li. Il vagabondo
attese. Finalmente le poiane scesero fra gli alberi. Il vagabondo attese
altri cinque minuti. Alla fine si alzò e si avviò verso la fascia
boschiva, dividendo il suo peso tra la gamba sana e il bastone.

Dopo un po\textquotesingle{} entrò nella foresta. Le poiane erano
indaffarate sui resti di un uomo. Il vagabondo scacciò gli uccelli con
il bastone ed esaminò quei resti umani. Ne mancavano alcune parti.
C\textquotesingle era una freccia infissa nel suo cranio, e spuntava
dalla nuca. Il vecchio si guardò intorno innervosito. Non
c\textquotesingle era nessuno, in vista, ma vicino al sentiero
c\textquotesingle erano molte orme. Non era prudente rimanere lì.

Prudente o no, doveva farlo. Il vecchio vagabondo trovò un punto in cui
la terra era abbastanza morbida per poterla scavare con le mani e con il
bastone. Mentre scavava, le poiane incollerite volavano in cerchio,
basse, sopra le cime degli alberi. Qualche volta sfrecciavano verso il
suolo, ma subito risalivano di nuovo verso il cielo, sbattendo le ali.
Per un\textquotesingle ora, poi per due, svolazzarono ansiose sulla
collina boscosa.

Finalmente, una di esse atterrò. Zampettò indignata su un mucchio di
terra smossa di fresco, su una estremità della quale era stata posta una
pietra. Delusa, riprese il volo. Lo stormo di neri becchini abbandonò
quel luogo e sali, sfruttando le correnti ascensionali, osservando
famelicamente la terra.

C\textquotesingle era un porco morto al di là della Valle dei Malnati.
Le poiane l\textquotesingle osservarono gaiamente e scivolarono per il
festino. Più tardi, su un lontano passo di montagna, un puma finì di
leccare i frammenti di carne e lasciò la sua preda. Le poiane sembrarono
grate per la possibilità di finire il suo pasto.

Le poiane deposero le uova, nella giusta stagione, e sfamarono
amorosamente i piccini: un serpente morto, pezzi di cane selvatico. La
generazione più giovane crebbe forte, volò alta e lontana sulle ali
nere, attendendo che la fruttifera terra cedesse loro la sua
misericordiosa carogna. Qualche volta, il pasto era soltanto un rospo.
Una volta era un messaggero proveniente da Nuova Roma.

Il loro volo le portò nelle pianure del Middle West. Erano felici per
l\textquotesingle abbondanza di buone cose che i nomadi si lasciavano
dietro, sulla terra, durante le loro migrazioni verso il Sud.

Per un poco le prede furono buone nella regione del Fiume Rosso; ma,
dalla carneficina, sorse una città-stato. Le poiane non avevano simpatia
per le città-stato che nascevano, sebbene ne approvassero la caduta
finale. Fuggirono da Texarkana e spaziarono lontano, sopra la pianura a
occidente. Come tutte le cose viventi, riempirono molte volte la Terra
della loro specie.

Finalmente venne l\textquotesingle anno del Signore 3174.

E si parlava di guerra.
