	\chapter{\phantom{title}}

\lettrine{E}{ra} spiacevolmente caldo, vicino al sentiero assolato che sfiorava la
collina boscosa, e il caldo aveva aggravato la sete del Poeta. Dopo
molto tempo sollevò il capo dal suolo, in preda alle vertigini, e cercò
di guardarsi intorno. La lotta era finita; tutte le cose se ne stavano
quiete, ora, tranne l\textquotesingle ufficiale di cavalleria. Le poiane
scendevano per atterrare, senza fare rumore.

C\textquotesingle erano parecchi profughi morti, un cavallo morto, e
l\textquotesingle ufficiale morente, bloccato sotto il peso del cavallo.
Ogni tanto,. il cavaliere rinveniva e gridava, con voce fioca. Qualche
volta invocava sua madre, qualche volta invocava un prete. Le sue grida
disturbavano le poiane e nauseavano ulteriormente il Poeta, che già si
sentiva bisbetico. Era un Poeta molto fuori di sé. Non si era mai
aspettato che il mondo si comportasse in modo cortese o sensato, e di
rado il mondo si era comportato così; spesso si era rincuorato della
consistenza della sua rozzezza e della sua stupidità. Ma mai, prima di
quella volta, il mondo aveva colpito il Poeta all\textquotesingle addome
con un moschetto. Questo non gli sembrava affatto incoraggiante.

E, cosa anche peggiore, questa volta non doveva biasimare la stupidità
del mondo, ma la sua propria. Era stato il Poeta a sbagliare. Stava
pensando solo alle proprie faccende e non dava fastidio a nessuno quando
aveva visto il gruppo di profughi galoppare da oriente verso la collina,
inseguito da un drappello di cavalieri. Per evitare di essere coinvolto,
si era nascosto dietro un cespuglio che cresceva
sull\textquotesingle orlo d\textquotesingle una delle scarpate che
fiancheggiavano il sentiero, un punto da cui avrebbe potuto assistere
allo spettacolo senza essere veduto. Non era un combattimento che
attraesse il Poeta, quello. Non gli importavano affatto i gusti politici
o religiosi dei profughi o dei cavalieri. Se il destino voleva un
massacro, non avrebbe potuto trovare un testimonio meno disinteressato
del Poeta. Quindi, perché aveva provato quel cieco impulso?

L\textquotesingle impulso l\textquotesingle aveva spinto a lanciarsi
dalla scarpata e a sbalzare di sella l\textquotesingle ufficiale e a
pugnalarlo tre volte con il coltello prima che entrambi cadessero al
suolo. Non riusciva a comprendere perché lo aveva fatto. Non aveva
ottenuto nulla. Gli uomini dell\textquotesingle ufficiale
l\textquotesingle avevano abbattuto prima che potesse rimettersi in
piedi. Il massacro dei profughi era continuato. Poi i cavalieri si erano
allontanati per inseguire altri fuggitivi, lasciandosi indietro i morti.

Sentiva il suo addome brontolare. L\textquotesingle inutilità, ahimè, di
tentare di digerire una palla di moschetto. Aveva compiuto un gesto
inutile, decise finalmente, per colpa di quella sciabola spuntata. Se
l\textquotesingle ufficiale si fosse limitato a uccidere la donna e a
buttarla di sella con un solo colpo netto, e avesse proseguito, il Poeta
avrebbe ignorato il suo gesto. Ma continuare a colpirla e a colpirla in
quel modo\ldots{}

Rifiutò di ripensarvi. Pensò all\textquotesingle acqua.

O Dio\ldots{} O Dio\ldots{} --- continuava a lamentarsi
l\textquotesingle ufficiale.

--- La prossima volta, affila la tua coltelleria --- gemette il Poeta.

Ma non vi sarebbe stata una prossima volta.

Il Poeta non riusciva a ricordare di aver mai temuto la morte, ma aveva
sospettato spesso la Provvidenza di tramare il peggio ai suoi danni,
quando sarebbe venuto il momento di morire. Si era aspettato di
imputridire. Lentamente, e non molto profumatamente. Qualche poetico
presentimento l\textquotesingle aveva avvertito che sarebbe morto
sicuramente di un bubbone lebbroso, penitente ma non pentito. Non aveva
mai pensato a nulla di così ottuso e definitivo come una pallottola
nello stomaco, senza neppure un pubblico che ascoltasse i suoi gemiti
morenti. L\textquotesingle ultima cosa che l\textquotesingle avevano
sentito dire quando gli avevano sparato era stato \emph{``Uf!''\ldots{}}
il suo testamento per la posterità. \emph{Uuf!\ldots{}} un Memorabile
per voi, Domnissime.

--- Padre? Padre? --- gemette l\textquotesingle ufficiale.

Dopo un po\textquotesingle, il Poeta raccolse le sue forze e alzò di
nuovo la testa, sbatté le palpebre per farne cadere il terriccio, e
studiò l\textquotesingle ufficiale per qualche secondo. Era certo che
fosse lo stesso che aveva assalito, anche se adesso era diventato
d\textquotesingle un verde gessoso. Sentirlo belare invocando un prete
in quel modo cominciava a infastidire il Poeta. C\textquotesingle erano
almeno tre religiosi che giacevano morti fra i profughi, eppure adesso
l\textquotesingle ufficiale non era molto schizzinoso nello specificare
la sua richiesta. Forse lo farò io, pensò il Poeta.

Cominciò a trascinarsi, lentamente, verso il cavaliere.
L\textquotesingle ufficiale lo vide arrivare e cercò di prendere una
pistola. Il Poeta si fermò; non aveva previsto di essere riconosciuto.
Si preparò per rotolare al coperto. La pistola puntava, ondeggiando,
nella sua direzione. La guardò ondeggiare per un momento, poi decise di
continuare la sua avanzata. L\textquotesingle ufficiale premette il
grilletto. Il colpo lo mancò di parecchi metri, sfortunatamente.

L\textquotesingle ufficiale stava cercando di ricaricare
l\textquotesingle arma quando il Poeta gliela tolse. Sembrava in
delirio, e continuava a cercare di farsi il segno della croce.

--- Continua --- grugnì il Poeta, prendendo il coltello.

--- Beneditemi, padre, perché ho peccato\ldots{}

--- \emph{Ego te absolvo}, figlio --- disse il Poeta, e gli affondò il
coltello nella gola.

Poi trovò la borraccia dell\textquotesingle ufficiale e bevve qualche
sorso. L\textquotesingle acqua era riscaldata dal sole, ma sembrava
deliziosa. Giacque, con la testa appoggiata sul cavallo
dell\textquotesingle ufficiale e attese che l\textquotesingle ombra
della collina avanzasse strisciando sulla strada. Gesù, come faceva
male! Quest\textquotesingle ultimo gesto non sarà facile da spiegare,
pensò; e non ho neanche il mio occhio di vetro. Eppure
c\textquotesingle è veramente qualcosa da spiegare. Guardò il cavaliere
morto.

--- È caldo come l\textquotesingle Inferno, laggiù, non è vero? ---
sussurrò, con voce rauca.

Il cavaliere non era disposto a dargli informazioni. Il Poeta bevve un
altro sorso dalla borraccia, poi un altro. Improvvisamente vi fu un
doloroso movimento delle budella. Per un attimo o due lo fece soffrire
molto.

Le poiane si avvicinarono; si allisciarono le penne, e litigarono sul
pranzo; non era ancora adeguatamente preparato. Attesero i lupi per
qualche giorno. Ce n\textquotesingle era per tutti. E finalmente
mangiarono il Poeta.

Come sempre, i neri becchini dei cieli deposero le uova nella stagione
adatta, e nutrirono amorosamente i piccini. Volarono alti sulle
praterie, sulle montagne e sulle pianure, adempiendo al destino di vita
che spettava loro, secondo il piano della Natura. I loro filosofi
dimostrarono, mediante la sola ragione e senza altri aiuti, che il
Supremo \emph{Cathartes aura regnans} aveva creato il mondo specialmente
per le poiane. Lo onorarono con cordiale appetito per molti secoli.

Poi, dopo le generazioni delle tenebre vennero le generazioni della
luce. E lo chiamarono l\textquotesingle Anno del Signore 3781\ldots{} un
anno della Sua pace, pregarono.
