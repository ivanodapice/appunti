	\chapter{\phantom{text}}

\lettrine{C}{antarono} mentre facevano salire i bambini sulla nave. Cantarono antichi
canti spaziali e aiutarono i bambini a salire la scaletta, uno alla
volta, fino alle mani delle suore. Cantarono di cuore, per disperdere la
paura dei piccini. Quando l\textquotesingle orizzonte eruttò, il canto
si interruppe. Fecero salire sulla nave l\textquotesingle ultimo
bambino.

L\textquotesingle orizzonte si accese di lampi mentre i monaci salivano
la scaletta. L\textquotesingle orizzonte diventò un bagliore rosso. Un
lontano banco di nubi era nato dove non c\textquotesingle era stata
alcuna nube. Due monaci, sulla scaletta, distolsero lo sguardo dai
lampi. Quando i lampi scomparvero, tornarono a guardare.

Il volto di Lucifero crebbe come un fungo orribile al di sopra del banco
di nubi, alzandosi lentamente come un titano che si levasse in piedi
dopo anni di prigionia nella Terra.

Qualcuno abbaiò un ordine. I monaci ripresero ad arrampicarsi. Presto
furono tutti dentro la nave.

L\textquotesingle ultimo monaco, nell\textquotesingle entrare, si fermò
nella camera stagna. Rimase ritto sul portello aperto e si tolse i
sandali.

--- \emph{Sic transit mundus} --- mormorò, guardando verso il bagliore.
Sbatté le suole dei sandali una contro l\textquotesingle altra, per
toglierne la polvere. Il bagliore abbracciava un terzo del cielo. Si
grattò la barba, gettò un ultimo sguardo all\textquotesingle oceano, poi
indietreggiò e chiuse il portello.

Vi fu un lampo, un riflesso di luce, un gemito alto e sottile che
vinceva il suono, e l\textquotesingle astronave si lanciò verso il
cielo.

I frangenti battevano monotoni la spiaggia, gettando a riva i detriti.
Un idrovolante abbandonato galleggiava oltre i frangenti. Dopo un
po\textquotesingle{} i frangenti catturarono
l\textquotesingle idrovolante e lo gettarono sulla spiaggia, insieme ai
detriti. Si inclinò, si spezzò un\textquotesingle ala.
C\textquotesingle erano gamberetti che facevano caroselli nei frangenti,
e il meriango che si nutriva di gamberetti, e il pescecane che mangiava
i merianghi e li trovava eccellenti, nella sportiva brutalità del mare.

Un vento spazzò l\textquotesingle oceano, portando con sé una cortina di
fine polvere bianca. La cenere cadde nel mare, nei frangenti. I
frangenti spinsero a riva i gamberetti morti, insieme ai detriti. Poi
spinsero a riva il meriango. Il pescecane nuotò verso le sue acque più
profonde, si crogiolò nelle fredde correnti pulite. Aveva molta fame, in
quella stagione.

\hfill\break

\hfill\break

\hfill\break

\begin{center}
	\textbf{FINE}
\end{center}