	\chapter{\phantom{text}}

\lettrine{L}{a} zona colpita dal fallout locale rimane relativamente stazionaria
disse l\textquotesingle annunciatore. ``E il pericolo di ulteriore
dispersione a causa del vento è quasi scomparso\ldots''

--- Bene, per il momento non è accaduto ancora il peggio --- osservò
l\textquotesingle ospite dell\textquotesingle abate. --- Fino a ora, qui
siamo stati al sicuro. E pare che lo saremo ancora, a meno che la
conferenza non sia un fallimento.

--- Lo saremo --- brontolò Zerchi. --- Ma ascoltate un momento.

``Il calcolo più recente delle perdite'' continuò
l\textquotesingle annunciatore "in questo nono giorno dopo la
distruzione della capitale, dà un totale di due milioni e ottocentomila
morti. Più di metà di questa cifra è costituita dalla popolazione della
città vera e propria. Il resto è una stima basata sulla percentuale
della popolazione dei dintorni e delle zone colpite dal fallout che
hanno ricevuto dosi critiche di radiazioni. Gli esperti prevedono che la
stima aumenterà via via che altri casi da radioattività verranno
denunciati.

``È stato chiesto a questa stazione di trasmettere due volte al giorno
questo annuncio, per tutta la durata della situazione di emergenza: I
provvedimenti della Legge Pubblica 10-WR-3E non autorizzano in alcun
modo i cittadini a praticare l\textquotesingle eutanasia alle vittime di
avvelenamento da radiazione. Le vittime che sono state esposte, o che
credono di essere state esposte a una radioattività molto superiore alla
dose critica devono presentarsi alla più vicina Stazione di Alleviamento
Stella Verde, dove un magistrato è autorizzato a emettere un certificato
di Mori Vult a chiunque venga giudicato come caso disperato, se il
paziente desidera l\textquotesingle eutanasia. Qualunque vittima delle
radiazioni che si tolga la vita in qualunque modo diverso da quello
prescritto dalla legge sarà considerato un suicida, e metterà a
repentaglio il diritto dei suoi eredi e dipendenti a reclamare
l\textquotesingle assicurazione e altri benefici per
l\textquotesingle assistenza anti-radiazione, secondo la legge. Inoltre,
ogni cittadino che assista un tale suicida può essere perseguito per
omicidio. La Legge sul Disastro da Radioattività autorizza
l\textquotesingle eutanasia soltanto dopo una regolare procedura legale.
I casi gravi di malattia da radiazione devono essere riferiti ai Centri
della Stella Verde\ldots''

Bruscamente Zerchi spense l\textquotesingle apparecchio. Balzò dalla
sedia e andò a fermarsi alla finestra, e guardò giù, nel cortile, dove
una folla di profughi si aggirava attorno a tavole di legno costruite in
fretta e furia. L\textquotesingle abbazia, vecchia e nuova, era invasa
da gente di tutte le età e condizioni, le cui case erano sorte nelle
regioni devastate. L\textquotesingle abate aveva temporaneamente
ristretto le zone ``clausura'' dell\textquotesingle abbazia per
concedere ai profughi accesso a tutto, a eccezione dei dormitori dei
monaci. La scritta davanti all\textquotesingle antica porta era stata
rimossa, perché c\textquotesingle erano donne e bambini che dovevano
essere sfamati, vestiti e ospitati.

Guardò due novizi che portavano un calderone fumante dalla cucina di
emergenza. L\textquotesingle issarono su di una tavola e cominciarono a
distribuire la minestra.

Il visitatore dell\textquotesingle abate si schiarì la gola e si agitò
irrequieto nella sedia. L\textquotesingle abate si voltò.

Procedura legale, la chiamano --- brontolò. --- Procedura legale di
massa, suicidio assistito dallo Stato. Con tutte le benedizioni della
società.

Già --- disse il visitatore --- ma è certamente meglio che lasciarli
morire orribilmente a poco a poco.

--- Davvero? Meglio per chi? Per gli spazzini? Meglio che i cadaveri
viventi camminino da soli fino a un centro di annientamento, finché
possono ancora camminare? Uno spettacolo meno pubblico? Meno orrore
sparso in giro? Meno disordine? Qualche milione di cadaveri che
giacciono sparsi qua e là potrebbe dare l\textquotesingle avvio a una
rivolta contro quelli che sono i responsabili. È questo che voi e il
vostro governo intendete per ``meglio'', non è così, dottore?

--- Non saprei come la pensa il governo --- disse il visitatore con una
lievissima sfumatura di stizza nella voce. --- Ciò che intendevo per
``meglio'' era ``più misericordioso''. Non intendo discutere con voi la
vostra teologia morale. Se credete di avere un\textquotesingle anima che
Dio manderebbe all\textquotesingle Inferno se sceglieste di morire senza
soffrire invece che in un modo orribile, allora continuate pure a
pensarla così. Ma voi siete una minoranza, lo sapete. Io non sono
d\textquotesingle accordo, ma non è il caso di discuterne.

--- Scusatemi --- disse l\textquotesingle abate Zerchi. --- Non mi
preparavo a discutere con voi di teologia morale. Stavo parlando
soltanto di questo spettacolo di eutanasia di massa, in termini di
movente umano. La stessa esistenza della Legge sul Disastro da
Radioattività, e di leggi corrispondenti in altri paesi, è la prova più
evidente che i governi erano completamente consci delle conseguenze di
un\textquotesingle altra guerra, ma invece di cercare di rendere
impossibile questo crimine, hanno cercato di provvedere in anticipo alle
conseguenze del crimine. I sottintesi di questo fatto non hanno
significato per voi, dottore?

--- Naturalmente no, padre. Personalmente, sono un pacifista. Ma
attualmente siamo legati al mondo, così come è. E se non riescono a
mettersi d\textquotesingle accordo sul modo di rendere impossibile un
atto di guerra, allora è meglio stabilire qualche provvedimento relativo
alle conseguenze, piuttosto che non prendere nessun provvedimento.

--- Sì e no. Sì, se è in previsione del crimine di qualcun altro. No, se
è in previsione di un crimine proprio. E specialmente no se i
provvedimenti che dovrebbero alleviare le conseguenze sono a loro volta
provvedimenti criminosi.

Il visitatore alzò le spalle. --- Come l\textquotesingle eutanasia? Mi
dispiace, padre, io penso che siano le leggi della società che rendono
qualcosa un crimine o no. So che non siete d\textquotesingle accordo. E
possono esservi leggi cattive, mal concepite, questo è vero. Ma in
questo caso penso che sia una buona legge. Se invece credessi di avere
un\textquotesingle anima e che in Cielo vi sia un Dio adirato, potrei
essere d\textquotesingle accordo con voi.

L\textquotesingle abate Zerchi sorrise, a labbra strette. --- Voi non
\emph{avete} un\textquotesingle anima, dottore. Voi \emph{siete}
un\textquotesingle anima. Voi avete un corpo, temporaneamente.

Il visitatore rise, con educazione. --- Una confusione semantica.

--- È vero. Ma chi di noi è confuso? Voi o io?

--- Non litighiamo, padre. Io non faccio parte delle Squadre della
Misericordia. Io lavoro nella Squadra Controllo Esposizione alle
Radiazioni. Noi non uccidiamo nessuno.

L\textquotesingle abate Zerchi lo fissò in silenzio per un momento. Il
visitatore era un uomo basso e muscoloso, con una simpatica faccia
rotonda e un cranio calvo bruciato dal sole. Indossava
un\textquotesingle uniforme di sala verde, e un berretto con il
distintivo della Stella Verde.

Già, perché litigare? Quell\textquotesingle uomo era un operatore
medico, non un carnefice. In parte, il lavoro di assistenza della Stella
Verde era ammirevole. Qualche volta era addirittura eroico. Il fatto che
in qualche caso fosse anche malvagio, secondo le convinzioni di Zerchi,
non era una ragione sufficiente per considerarne contaminate anche le
buone azioni. La società lo favoriva, e i suoi membri erano in buona
fede. Il dottore aveva cercato di essere amichevole. La sua richiesta
era sembrata abbastanza semplice. Non si era mostrato ne esigente né
burocratico. Eppure, l\textquotesingle abate esitava prima di dire di
sì.

--- Il lavoro che intendete svolgere qui\ldots{} richiederà molto tempo?

Il dottore scosse il capo. --- Due giorni al massimo, credo\ldots{}
Abbiamo due unità mobili. Possiamo portarle nel vostro cortile,
collegare i due furgoni, e cominciare il lavoro. Ci occuperemo dei casi
evidenti da radiazione, e dei feriti, in primo luogo. Noi ci occupiamo
solo dei casi più urgenti. Il nostro lavoro è un controllo clinico. I
malati verranno curati in un campo di emergenza.

--- E i più malati riceveranno qualcosa d\textquotesingle altro in un
``campo di misericordia''?

Il medico si accigliò. --- Soltanto se vogliono andarvi. Nessuno li
costringe.

--- Ma voi rilasciate il permesso che li autorizza ad andare.

--- Ho distribuito alcuni biglietti rossi, sì. Può darsi che debba farlo
anche questa volta. Ecco\ldots{} --- Si frugò nella tasca e ne tolse un
modulo di cartoncino rosso, qualcosa di simile a un cartellino per
spedizione munito di un cordoncino, per attaccarlo a
un\textquotesingle asola o a una cintura. Lo gettò sulla scrivania. ---
Un modulo ``dose critica'' in bianco. Ecco qui. Lo legga. Dice che
l\textquotesingle individuo è ammalato, molto ammalato. E questo\ldots{}
ecco un biglietto verde, anche. Dice che l\textquotesingle individuo sta
bene e non ha nulla di preoccupante. Guardate attentamente quello rosso!
``Esposizione calcolata in unità di radiazione.'' ``Esame del sangue.''
``Analisi delle urine.'' Su una facciata, è identico.. a quello verde.
Dall\textquotesingle altra parte, quello verde non reca nulla, ma
guardate quello rosso. Quella frase\ldots{} è citata direttamente dalla
Legge Pubblica 10-WR-3E. Deve esserci. La legge lo richiede. La si deve
leggere all\textquotesingle interessato, che deve conoscere i suoi
diritti. Ciò che ne fa è affar suo. Ora, se preferite che piazziamo le
unità mobili lungo l\textquotesingle autostrada, possiamo\ldots{}

--- Vi limitate a leggerglielo, vero? Nient\textquotesingle altro?

Il dottore fece una pausa. --- Glielo dobbiamo spiegare, se non lo
capisce. --- Fece un\textquotesingle altra pausa, dominando
l\textquotesingle irritazione. --- Buon Dio, padre, quando dite a un
uomo che è un caso disperato, ché cosa volete dirgli? Gli leggete
qualche paragrafo della legge, gli mostrate la porta, e gli dite
``Avanti un altro, prego''? ``Voi state per morire, buongiorno''?
Naturalmente non ci si limita a leggergli quelle frasi e basta, se si ha
qualche sentimento umano!

--- Lo capisco. Ciò che voglio sapere è qualcosa
d\textquotesingle altro. Voi, come medico, consigliate ai casi disperati
di andare a un campo di misericordia?

--- Io\ldots{} --- Il medico si interruppe e chiuse gli occhi. Appoggiò
la fronte sulla mano. Rabbrividì, leggermente. --- Sì, naturalmente ---
disse, alla fine. --- Se aveste visto ciò che ho visto io, lo fareste
anche voi. Naturalmente.

--- Ma qui non lo farete.

--- E allora noi\ldots{} --- Il dottore represse
un\textquotesingle esplosione d\textquotesingle ira. Si alzò, fece per
mettersi il berretto, poi si fermò. Buttò il berretto sulla sedia e si
avvicinò alla finestra. Guardò cupamente in cortile, poi
l\textquotesingle autostrada. E indicò qualcosa. --- C\textquotesingle è
il parco accanto alla strada. Possiamo impiantare bottega lì. Ma è a tre
chilometri. Quasi tutti dovranno venirci a piedi. --- Guardò
l\textquotesingle abate Zerchi, poi riabbassò pensieroso lo sguardo sul
cortile. --- Guardateli. Sono malati, feriti, fratturati, spaventati.
Anche i bambini. Stanchi, storpiati, miserabili. Voi permettereste che
fossero spinti sull\textquotesingle autostrada, per sedere nella polvere
e nel sole e\ldots{}

--- Non voglio che vada così --- disse l\textquotesingle abate. ---
Sentite\ldots{} mi stavate dicendo in che modo una legge fatta
dall\textquotesingle uomo abbia reso obbligatorio, per voi, leggere e
spiegare questo a un caso di radiazione critica. Io non ho fatto
obiezioni a questo. Diamo a Cesare ciò che gli spetta, fino a questo
punto, poiché è questo che la legge vuole da voi. Ma allora, non potete
comprendere che io sono soggetto a un\textquotesingle altra legge, la
quale mi proibisce di permettere a voi o a chiunque altro di
consigliare, a chiunque, qui, su questa proprietà affidata alle mie
cure, di fare qualcosa che la Chiesa considera un male?

--- Oh, lo comprendo abbastanza bene.

--- Ottimamente. Dovete farmi soltanto una promessa, e potrete servirvi
del cortile.

--- Quale promessa?

--- Semplicemente che non consiglierete a nessuno di andare a un ``campo
di misericordia''. Limitatevi a fare la diagnosi. Se troverete casi
disperati da radiazione, dite loro ciò che la legge vi costringe a dire,
consolateli come volete, ma non dite loro di andare a uccidersi.

Il medico esitò. --- Io credo che sarebbe giusto fare questa promessa
riguardo ai pazienti della vostra Fede.

L\textquotesingle abate Zerchi abbassò gli occhi. --- Mi dispiace ---
disse finalmente --- ma non è abbastanza.

--- Perché? Gli altri non sono legati dai vostri principi. Se un uomo
non appartiene alla vostra religione, perché dovrebbe rifiutare di
permettere\ldots{} --- Si interruppe, semi-soffocato, incollerito.

--- Volete una spiegazione?

--- Sì.

--- Perché se un uomo ignora il fatto che qualcosa è sbagliato, e agisce
nell\textquotesingle ignoranza, non incorre in una colpa, purché la
ragione naturale non sia sufficiente a mostrargli
l\textquotesingle errore. Ma, mentre l\textquotesingle ignoranza può
scusare l\textquotesingle uomo, non scusa \emph{l\textquotesingle atto}
che è errato in se stesso. Se io permettessi
\emph{l\textquotesingle atto} semplicemente perché
l\textquotesingle uomo ignora che esso è sbagliato, allora incorrerei
nella colpa, perché io so che è un errore. In realtà, come vedete è
dolorosamente semplice.

--- Ascoltatemi, padre. Stanno lì seduti, e vi guardano. Qualcuno grida.
Qualcuno piange. Qualcuno si limita a starsene lì seduto. E tutti
dicono: ``Dottore, cosa posso fare?''. E io, che cosa dovrei rispondere?
Non dovrei dire nulla? Devo dire: ``Puoi morire, ecco tutto''. Voi che
cosa direste?

--- ``Prega.''

--- Sì, voi lo direste, non è vero? Ascoltate, la sofferenza è
l\textquotesingle unico male che io conosco. È l\textquotesingle unico
che io sappia combattere.

--- E allora Dio vi aiuti.

--- Mi aiutano di più gli antibiotici.

L\textquotesingle abate Zerchi cercò un risposta tagliente, la trovò, ma
la ringoiò in fretta. Cercò un pezzo di carta bianca e una penna e li
spinse attraverso il piano della scrivania verso il dottore. --- E
allora scrivete: ``Non raccomanderò l\textquotesingle eutanasia ad alcun
paziente, finché sarò in questa abbazia''. E firmate. Poi potrete
servivi liberamente del cortile.

--- E se rifiutassi?

--- Allora, immagino che i malati dovranno trascinarsi per tre
chilometri lungo la strada.

--- Non ho mai sentito nulla di più spietato\ldots{}

--- Al contrario. Vi ho offerto una possibilità di fare il vostro
lavoro, come è richiesto dalla legge che voi riconoscete, senza
calpestare la legge che io riconosco. Spetta a voi decidere se dovranno
trascinarsi o no su quella strada.

Il medico fissò il foglio bianco. --- Cosa c\textquotesingle è di tanto
magico, se lo metto per iscritto?

--- Io preferisco così.

L\textquotesingle altro si chinò in silenzio sulla scrivania, e scrisse.
Guardò ciò che aveva scritto, poi tracciò in fretta la firma e si
raddrizzò. --- Benissimo, ecco la vostra promessa. Credete che valga di
più della mia parola?

--- No, no davvero. --- L\textquotesingle abate ripiegò il foglio e lo
nascose sotto la sua veste. --- Ma è qui, nella mia tasca, e voi sapete
che è qui, e che io posso guardarla di tanto in tanto, ecco tutto.
Mantenete le vostre promesse, fra parentesi, dottor Cors?

Il medico lo fissò, per un momento. --- La manterrò. --- Grugnì, poi
girò sui tacchi e uscì.

--- Frate Pat! --- chiamò l\textquotesingle abate Zerchi con voce
debole. --- Frate Pat, siete lì?

Il segretario si presentò sulla soglia. --- Sì, Reverendo Padre?

--- Avete sentito?

--- In parte. La porta era aperta, e non ho potuto fare a meno di
ascoltare. Non avevate attivato il silenziatore\ldots{}

--- Avete sentito cosa diceva? ``La sofferenza è l\textquotesingle unico
male che io conosco.'' L\textquotesingle avete sentito?

Il monaco annuì, solennemente.

--- E che la società è l\textquotesingle unica che stabilisce se un atto
è giusto o non è giusto? Avete sentito anche questo?

--- Sì.

--- Buon Dio, come hanno potuto ritornare nel mondo, queste due eresie,
dopo tutto questo tempo? L\textquotesingle Inferno ha una immaginazione
limitata. ``Il serpente mi ha ingannato, e io ne ho mangiato.'' Frate
Pat, farete meglio a uscire di qui, o comincerò a delirare.

--- Domne, io\ldots{}

--- Cos\textquotesingle è che vi trattiene? Cos\textquotesingle è, una
lettera? Benissimo, date qui.

Il monaco gliela porse e uscì. Zerchi non l\textquotesingle aprì, e
guardò di nuovo la dichiarazione del dottore. Non aveva valore, forse.
Eppure quell\textquotesingle uomo era sincero. E devoto al suo lavoro.
Doveva essere devoto al suo lavoro, con la paga che gli dava la Stella
Verde. Aveva l\textquotesingle aria di chi dorme troppo poco e lavora
troppo. Probabilmente viveva di benzedrina e di gallette, da quando
l\textquotesingle esplosione aveva assassinato la città. Vedere dovunque
la sofferenza e detestarla, e desiderare sinceramente di poter fare
qualcosa\ldots{} Sinceramente\ldots{} quello era
l\textquotesingle Inferno. In distanza, gli avversari sembravano
malvagi, ma quando li guardavi da vicino, ne vedevi la sincerità, che
era grande quanto la tua. Forse Satana era il più sincero di tutti.

Apri la lettera e la lesse. La lettera l\textquotesingle informava che
frate Joshua e gli altri erano partiti da Nuova Roma per una
destinazione imprecisata, nell\textquotesingle Ovest. La lettera
l\textquotesingle informava inoltre che qualche notizia sul \emph{Quo
	peregrinatur} era trapelata alla Difesa Interna di Zona, la quale aveva
mandato investigatori in Vaticano per indagare circa il supposto lancio
di una astronave non autorizzata\ldots{} Evidentemente
l\textquotesingle astronave non era ancora nello spazio.

Ben presto verranno a sapere del \emph{Quo peregrinatur}, ma con
l\textquotesingle aiuto del Cielo, lo scopriranno troppo tardi. E
allora? si chiese.

La situazione legale era complicata. La legge proibiva la partenza di
astronavi senza autorizzazione. L\textquotesingle autorizzazione era
difficile da ottenere e la procedura molto lenta. Zerchi era certo che
la Difesa Interna di Zona e la commissione avrebbero ritenuto che la
Chiesa aveva infranto la legge. Ma un concordato fra Stato e Chiesa
esisteva ormai da un secolo e mezzo: esentava chiaramente la Chiesa
dalle procedure di autorizzazione, e le assicurava il diritto di mandare
missioni in ``qualsiasi installazione spaziale e in qualsiasi avamposto
planetario che non saranno stati dichiarati dalla predetta Commissione
come ecologicamente critici o chiusi a spedizioni non autorizzate''.
Ogni installazione nel sistema solare era ``ecologicamente critica'' e
``chiusa'' al tempo del Concordato, ma più oltre il Concordato stabiliva
il diritto della Chiesa a ``possedere navi spaziali e a viaggiare, senza
restrizioni, alle installazioni e agli avamposti aperti''. Il Concordato
era molto amico. Era stato firmato nei giorni in cui il motore
interstellare Berkstrun era soltanto un sogno
nell\textquotesingle immaginazione di qualcuno che riteneva che i viaggi
interstellari avrebbero aperto l\textquotesingle universo a un flusso
illimitato di popolazione.

Ma le cose erano andate diversamente. Quando il progetto della prima
astronave vide la luce, fu chiaro che nessuna istituzione, a eccezione
del governo, disponeva dei mezzi o dei fondi per costruirle; che non
sarebbe derivato alcun profitto dal trasporto di colonie ai pianeti
extrasolari, a scopo di ``mercantilismo interstellare''. Tuttavia, i
dirigenti asiatici avevano fatto partire la prima astronave coloniale.
Poi, in Occidente, si era levato il grido: ``Dobbiamo permettere che le
razze inferiori ereditino le stelle''? C\textquotesingle era stata una
breve serie di lanci di astronavi coloniali cariche di gente nera,
bruna, bianca e gialla, mandate nei cieli, verso il Centauro, in nome
del razzismo. Poi, gli specialisti di genetica avevano maliziosamente
dimostrato che --- poiché ogni gruppo razziale era così piccolo che, se
i discendenti non avessero praticato il matrimonio misto, ciascuno di
essi avrebbe subito una degenerazione genetica a causa
dell\textquotesingle accoppiamento tra consanguinei nei pianeti
coloniali --- i razzisti avevano reso necessaria, per la sopravvivenza,
la mescolanza delle razze.

L\textquotesingle unico interesse che la Chiesa aveva mostrato per lo
spazio era stato per i coloni, i quali erano figli della Chiesa,
tagliati fuori dal gregge a causa delle distanze interstellari. Eppure
non aveva approfittato del concordato che permetteva
l\textquotesingle invio di missioni. Esistevano certe contraddizioni tra
il concordato e le leggi dello Stato che davano potere alla Commissione,
almeno nel senso che la legge più recente poteva, in teoria, influire
sull\textquotesingle invio di missioni. La contraddizione non era mai
stata portata dinanzi ai tribunali, poiché non vi era mai stato un
motivo di lite. Ma ora, se la Difesa Interna di Zona avesse intercettato
il gruppo di frate Joshua nell\textquotesingle atto di lanciare
un\textquotesingle astronave senza un permesso della Commissione, vi
sarebbe stato un motivo. Zerchi pregò che il gruppo potesse partire
senza bisogno di una discussione in tribunale, che avrebbe potuto
richiedere settimane o mesi. Naturalmente, dopo sarebbe scoppiato uno
scandalo. Molti avrebbero sostenuto non soltanto che la Chiesa aveva
violato le regole della Commissione ma anche quelle della carità,
mandando dignitari ecclesiastici e un gruppo di monaci, quando avrebbe
potuto usare la nave come strumento di salvezza per i poveri coloni,
affamati di terra. Il conflitto tra Marta e Maria si ripresentava
sempre.

L\textquotesingle abate Zerchi si rese conto,
all\textquotesingle improvviso, che il suo modo di pensare era cambiato,
in quegli ultimi giorni. Qualche giorno prima, tutti avevano aspettato
che il cielo esplodesse. Ma erano trascorsi nove giorni da quando
Lucifero era prevalso nello spazio e aveva ucciso una città con la sua
vampa. Nonostante i morti, gli storpiati e i morenti, erano stati nove
giorni di silenzio. Poiché l\textquotesingle ira si era fermata, fino a
quel momento, forse il peggio poteva essere evitato. Si sorprendeva a
pensare cose che potevano accadere la settimana prossima o il prossimo
mese, come se, dopotutto, potesse esservi, in realtà, una settimana
prossima o un prossimo mese. E perché no? Facendo un esame di coscienza,
scoprì che non aveva completamente perso la virtù della speranza.

Un monaco ritornò da una commissione in città, quel pomeriggio, e riferì
che nel parco, a tre chilometri dall\textquotesingle abbazia, veniva
preparato un campo profughi.

--- Credo che sia organizzato dalla Stella Verde, Domne --- aggiunse.

--- Bene! --- disse l\textquotesingle abate. --- Qui siamo anche in
troppi, e ho dovuto rimandare indietro tre camion carichi di profughi.

I profughi rumoreggiavano nel cortile, e quel rumore torturava i nervi
logori. La quiete perpetua della vecchia abbazia era infranta da suoni
estranei: la risata spavalda di uomini che raccontavano barzellette, il
pianto di un bambino, il tintinnio di pentole e tegami, singhiozzi
isterici, un medico della Stella Verde che gridava: ``Ehi, Raff, portami
un tubo per clistere!''. Parecchie volte l\textquotesingle abate
represse l\textquotesingle impulso di andare alla finestra e di ordinare
il silenzio.

Dopo aver sopportato più a lungo che poté, prese un binocolo, un vecchio
libro e un rosario, e salì su una delle antiche torri di guardia, dove
uno spesso muro di pietra tagliava fuori quasi tutti i suoni provenienti
dal cortile. Il libro era un volumetto di versi, anonimo, ma ascritto
dalla leggenda a un mitico ``santo'' la cui canonizzazione era stata
compiuta soltanto nella leggenda e nel folclore delle Pianure, ma non in
alcun atto della Santa Sede. Nessuno, in realtà, aveva mai trovato una
prova che fosse esistita una persona come il San Poeta
dell\textquotesingle Occhio Miracoloso; la favola era probabilmente nata
dal fatto che uno dei primi Hannegan aveva ricevuto in dono un occhio di
vetro da un geniale fisico teorico suo protetto --- Zerchi non ricordava
se lo scienziato fosse stato Esser Shon o Pfardentrott --- il quale
aveva detto al principe che l\textquotesingle oggetto era appartenuto a
un poeta morto per la Fede. Non aveva specificato per quale fede fosse
morto il poeta --- per quella di Pietro o per quella degli scismatici
texarkani --- ma evidentemente l\textquotesingle Hannegan
l\textquotesingle aveva tenuto caro, perché aveva fatto montare
l\textquotesingle occhio di vetro nella stretta d\textquotesingle una
minuscola mano d\textquotesingle oro che era ancora portata, in certe
occasioni ufficiali, dai principi della dinastia Harq-Hannegan.
L\textquotesingle occhio veniva chiamato \emph{Orbis Judicans
	Conscientiae} oppure \emph{Oculus Poetae Judicis}, e gli ultimi
scismatici texarkani lo veneravano come una reliquia.

Qualcuno, qualche anno prima, aveva avanzato l\textquotesingle ipotesi
piuttosto sciocca che San Poeta fosse lo stesso ``versificatore
scurrile'' menzionato una volta nelle Cronache del venerabile abate
Jerome, ma l\textquotesingle unica ``prova'' sostanziale in proposito
era il fatto che Pfardentrott --- o era Esser Shon? --- aveva visitato
l\textquotesingle abbazia durante il regno del venerabile Jerome,
all\textquotesingle incirca nell\textquotesingle epoca del riferimento
al ``versificatore scurrile'' nella Cronaca, e che il dono
dell\textquotesingle occhio a Hannegan era avvenuto qualche tempo dopo
la visita all\textquotesingle abbazia. Zerchi sospettava che il
libriccino di versi fosse stato scritto da uno degli scienziati secolari
che avevano visitato l\textquotesingle abbazia per studiare i
Memorabilia, all\textquotesingle incirca in quel tempo, e che uno di
essi potesse essere identificato con il ``versificatore scurrile'' e
probabilmente con il San Poeta del folclore e della favola. I versi
anonimi erano un po\textquotesingle{} troppo arditi, pensava Zerchi, per
essere stati scritti da un monaco dell\textquotesingle Ordine.

Il libro era un dialogo satirico in versi tra due agnostici che
tentavano di dichiarare, secondo la sola ragione naturale, ché
l\textquotesingle esistenza di Dio non poteva essere stabilita secondo
la sola ragione naturale.

I due riuscivano a dimostrare che il limite matematico di una sequenza
infinita di ``dubitando la certezza con cui qualcosa di dubitato è noto
come inconoscibile quando il ``qualcosa di dubitato'' è una
dichiarazione che precede la ``inconoscibilità\textquotesingle{} di
qualcosa di dubitato'', che il limite di questo processo
all\textquotesingle infinito poteva essere soltanto
l\textquotesingle equivalente di una dichiarazione di assoluta certezza,
sebbene espresso come una serie infinita di negazioni di certezza. Il
testo recava tracce del calcolo teologico di san Leslie, e, seppure
sotto forma di dialogo poetico fra un agnostico identificato solo come
''Poeta`` e un altro identificato soltanto come ''Thon", sembrava
suggerire una prova dell\textquotesingle esistenza di Dio attraverso un
metodo epistemologico, ma l\textquotesingle autore di quei versi era
stato un poeta satirico: né il Poeta né il Thon abbandonavano le loro
premesse di agnosticismo, dopo che era stata raggiunta la conclusione
dell\textquotesingle assoluta certezza, ma invece concludevano così il
loro dialogo: \emph{Non cogitamus, ergo nihil sumus}.

L\textquotesingle abate Zerchi si stancò presto dei suoi tentativi di
decidere se il libro era una commedia intellettuale o una buffonata
epigrammatica. Dalla torre poteva vedere l\textquotesingle autostrada e
la città, e più oltre la mesa. Mise a fuoco il binocolo sulla mesa e
osservò per qualche tempo le installazioni radar, ma non sembrava che lì
stesse accadendo qualcosa di insolito. Abbassò lentamente lo strumento,
per guardare il nuovo accampamento della Stella Verde, nel parco a
fianco della strada. La zona del parco era stata cintata. Venivano
rizzate le tende. Squadre di operai lavoravano alle deviazioni delle
linee del gas e dell\textquotesingle energia elettrica. Parecchi uomini
stavano rizzando un\textquotesingle insegna,
all\textquotesingle ingresso del parco, ma la reggevano trasversalmente
rispetto all\textquotesingle abate, che non riuscì a leggerla. In
qualche modo, quella frenetica attività gli ricordava un ``carnevale''
nomade che si avvicinasse alla città. C\textquotesingle era una grande
macchina rossa. Sembrava avesse un focolare e una caldaia, ma
l\textquotesingle abate non riuscì dapprima a indovinarne la funzione.
Uomini che indossavano l\textquotesingle uniforme della Stella Verde
stavano montando qualcosa che sembrava una piccola giostra. Almeno una
dozzina di camion erano fermi sulla strada laterale. Alcuni erano
carichi di legname, altri di tende e di lettini pieghevoli. Uno sembrava
stesse scaricando mattoni refrattari, e un altro era carico di paglia e
di vasellame.

Vasellame?

Studiò attentamente, attraverso il binocolo, il carico
dell\textquotesingle ultimo camion. Corrugò lentamente la fronte. Era un
carico di urne o di vasi, tutti eguali, avvolti in strati protettivi di
paglia. Aveva già veduto qualcosa del genere, ma non riusciva a
ricordare dove l\textquotesingle avesse visto.

Un altro camion non portava altro che una grande statua di
``pietra''\ldots{} fatta probabilmente di plastica rinforzata, e una
lastra quadrata, sulla quale evidentemente doveva venir montata la
statua. La statua era distesa sul dorso, sostenuta da una incastellatura
di legno e da un nido di materiale da imballaggio. Poteva vederne
soltanto le gambe e una mano protesa che sporgeva dalla paglia. La
statua era più lunga del pianale del camion: i suoi piedi nudi
sporgevano dal fondo. Qualcuno aveva legato una bandiera rossa a uno dei
suoi alluci. Zerchi rifletté. Perché sprecare un camion per una statua,
quando c\textquotesingle era probabilmente bisogno di un altro carico di
cibo?

Guardò gli uomini che stavano montando l\textquotesingle insegna.
Finalmente, uno di essi abbassò una estremità della tavola e salì su una
scaletta per sistemare i sostegni superiori. Adesso che una estremità
poggiava sul terreno, l\textquotesingle insegna si inclinò, e Zerchi,
allungando il collo, riuscì a leggerne la scritta:

\begin{center} 
	CAMPO DI MISERICORDIA NUMERO 18
\end{center}
\leavevmode\\
\begin{center} 
	STELLA VERDE
\end{center}
\leavevmode\\
\begin{center}
	ORGANIZZAZIONE D\textquotesingle EMERGENZA
\end{center}
\leavevmode\\

~

Guardò di nuovo il camion. Il vasellame! E ricordò. Una volta era
passato davanti a un crematorio e aveva veduto gli uomini che
scaricavano urne dello stesso tipo da un camion che portava il marchio
della stessa ditta. Spostò di nuovo il binocolo, cercando il camion
carico di mattoni refrattari. Si era spostato. Finalmente lo individuò;
adesso era fermo all\textquotesingle interno della zona cintata. I
mattoni venivano scaricati vicino alla grande macchina rossa. Esaminò di
nuovo quella macchina. Ciò che a prima vista era sembrata una caldaia,
adesso faceva pensare a una fornace. --- \emph{Evenit diabolus!} ---
grugnì l\textquotesingle abate, e si avviò verso la scala.

Trovò il dottor Cors nell\textquotesingle unità mobile, nel cortile. Il
dottore stava allacciando un cartellino giallo al bavero della giacca di
un vecchio, mentre gli diceva che doveva recarsi per qualche tempo in un
campo di riposo e obbedire alle infermiere, ma che si sarebbe rimesso
benissimo, se avesse avuto cura di sé.

Zerchi rimase ritto, a braccia conserte, mordicchiandosi le labbra e
osservando freddamente il medico. Quando il vecchio si fu allontanato,
Cors alzò lo sguardo, cautamente.

--- Sì? --- I suoi occhi notarono il binocolo e riesaminarono il volto
di Zerchi. --- Oh --- brontolò. --- Bene, io non ho niente a che fare
con quella roba, assolutamente niente.

L\textquotesingle abate lo fissò per qualche secondo, poi si voltò e se
ne andò. Si recò nel suo ufficio e disse a frate Pat di chiamare il più
alto funzionario della Stella Verde.

--- Voglio che venga allontanato dalla nostra zona.

--- Temo che la risposta sarà assolutamente no\ldots{}

--- Frate Pat, chiamate l\textquotesingle officina e fate salire subito
frate Lufter.

--- Non c\textquotesingle è, Domne.

--- E allora mi mandino un carpentiere e un pittore. Chiunque andrà
benissimo.

Qualche minuto dopo, arrivarono due monaci.

--- Voglio che facciate immediatamente cinque leggeri cartelli --- disse
loro l\textquotesingle abate. --- E voglio che abbiano aste lunghe e
solide. Devono essere abbastanza grandi perché sia possibile leggerli a
un isolato di distanza, ma abbastanza leggeri perché un uomo li possa
reggere per molte ore senza stancarsi. Siete in grado di farli?

--- Certo, monsignore. E cosa devono dire?

L\textquotesingle abate Zerchi lo scrisse. --- Deve essere grande e
chiaro disse. --- Fate che salti all\textquotesingle occhio. È tutto.

Quando se ne furono andati, richiamò frate Pat. --- Frate Pat, trovatemi
cinque bravi novizi, giovani e sani, preferibilmente con il complesso
del martire. Dite loro che potrebbero fare la fine di santo Stefano.

``E io posso fare una fine anche peggiore'' pensò ``quando lo saprà
Nuova Roma.''
