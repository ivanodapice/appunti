	\chapter{\phantom{title}}


\lettrine{I}{l} viaggio a Nuova Roma avrebbe richiesto almeno tre mesi, forse di più,
poiché la sua durata dipendeva in parte dalla distanza che Francis
sarebbe riuscito a coprire prima che l\textquotesingle inevitabile banda
di predoni gli togliesse l\textquotesingle asino. Avrebbe dovuto
viaggiare solo e disarmato, portando soltanto la sua bisaccia e la
ciotola delle elemosine, oltre alla reliquia e alla sua copia
alluminata. Pregava che i predoni ignoranti giudicassero inutile
quest\textquotesingle ultima; perché, invero, fra i banditi della strada
vi erano alcuni ladri di animo gentile che rubavano soltanto gli oggetti
di valore, e permettevano alle vittime di conservare la vita, la
carcassa e gli effetti personali. Altri erano meno rispettosi.

Per precauzione, Francis portava una benda nera
sull\textquotesingle occhio destro. I contadini erano superstiziosi e
spesso potevano essere messi in fuga anche dal semplice sospetto del
malocchio. Così equipaggiato, partì, per obbedire alla chiamata del
\emph{Sacerdos Magnus}, il Santo Sovrano, Papa Leone XXI.

Quasi due mesi dopo aver lasciato l\textquotesingle abbazia, il monaco
incontrò il suo ladrone sul sentiero di una montagna coperta di boschi,
lontano da ogni abitato umano, a eccezione della Valle dei Malnati, che
giaceva a poche miglia al di là di un picco, verso ovest, dove una
colonia di pochi esseri geneticamente mostruosi vivevano come lebbrosi,
isolati dal mondo. Alcune colonie di quel tipo venivano visitate dagli
Ospitalieri della Santa Chiesa, ma la Valle dei Malnati non era tra
queste.

Gli anormali che erano sfuggiti alla morte per mano dei membri delle
tribù delle foreste vi si erano raccolti, parecchi secoli prima. I loro
ranghi erano continuamente riempiti da esseri deformi e striscianti che
cercavano rifugio dal mondo, ma fra loro qualcuno era fertile e generava
nuove creature. Spesso quei figli ereditavano le mostruosità dei
genitori. Spesso nascevano morti o non raggiungevano mai la maturità. Ma
di tanto in tanto i tratti mostruosi erano recessivi, e
dall\textquotesingle unione di due anormali nasceva un figlio
apparentemente normale. Qualche volta, tuttavia, le creature
superficialmente ``normali'' erano oberate da qualche invisibile
deformità di cuore o di mente, che le orbava, a quanto pareva,
dell\textquotesingle essenza di umanità, mentre ne lasciava loro
l\textquotesingle aspetto. Anche nella Chiesa, qualcuno aveva osato
sostenere la convinzione che tali creature erano state in verità private
della \emph{Dei imago} fin dalla concezione, che le loro anime. erano
soltanto anime di bestie, e che potevano essere impunemente distrutte,
secondo la Legge Naturale, come animali e non come uomini, che Dio aveva
mandato nascite animali fra la specie umana come punizione per i peccati
che avevano quasi distrutto l\textquotesingle umanità. Pochi teologi,
che la credenza nell\textquotesingle Inferno non abbandonava mai,
preferivano affermare che Dio non avrebbe fatto mai ricorso ad alcuna
forma di punizione temporale, ma per gli uomini assumersi il diritto di
giudicare una creatura nata di donna come priva della divina immagine
era un\textquotesingle usurpazione dei privilegi del Cielo. Persino
l\textquotesingle idiota che pareva meno dotato d\textquotesingle un
maiale o d\textquotesingle una capra deve, se nato di donna, essere
chiamato anima immortale, tuonava il \emph{magisterium}, e continuava a
tuonare. Dopo che parecchi di questi pronunciamenti, miranti a reprimere
l\textquotesingle infanticidio, furono emessi da Nuova Roma, gli
infelici malnati erano chiamati ``nipoti del papa'' o ``figli del
papa'', da qualcun altro.

``Lasciate che colui che è nato vivo da genitori umani rimanga vivo''
aveva detto il precedente Leone, ``secondo la Legge Naturale e la Divina
Legge dell\textquotesingle Amore: sia esso allevato come Figlio e
nutrito, qualunque sia la sua forma e il suo comportamento, perché è un
fatto evidente alla ragione naturale, senza necessità di appoggio da
parte della Divina Rivelazione, che fra i Diritti Naturali
dell\textquotesingle Uomo, il diritto all\textquotesingle assistenza da
parte dei genitori nel tentativo di sopravvivere ha la precedenza su
qualsiasi altro diritto, e non può essere modificato legittimamente
dalla Società o dallo Stato, a eccezione dei casi in cui i principi
hanno il potere di rafforzare tale diritto. Neppure le bestie, sulla
Terra, agiscono altrimenti.''

Il ladrone che accostò frate Francis non era in modo evidente una
creatura deforme, ma fu chiaro che proveniva dalla Valle dei Malnati
quando due figure incappucciate si levarono dietro un groviglio di
arbusti sul pendio che incombeva sul sentiero e lanciarono grida
ironiche al monaco, mentre lo prendevano di mira con gli archi tesi. Da
quella distanza, Francis non ebbe la certezza che fosse esatta la sua
prima impressione, e cioè che una delle mani strette su un arco aveva
sei dita o un pollice in più: ma non v\textquotesingle era alcun dubbio
che una delle figure portasse una tonaca con due cappucci, sebbene non
riuscisse a distinguere i visi e non potesse stabilire se il cappuccio
in più contenesse o no una testa in più.

Il ladrone era ritto sul sentiero, davanti a Francis. Era basso, ma
forte e massiccio come un toro, con una calvizie lucente e una mascella
simile a un pezzo di granito. Stava ritto con le gambe divaricate e con
le braccia massicce conserte sul petto, mentre osservava
l\textquotesingle appressarsi della minuscola figura a cavalcioni
dell\textquotesingle asino. Il ladrone, per quanto poteva vedere frate
Francis, era armato soltanto dei suoi muscoli e di un coltello che non
si prese il disturbo di togliere dalla cintura. Fece cenno a Francis di
avanzare. Quando il monaco si fermò a 50 metri da lui, uno dei figli del
papa scagliò una freccia che si piantò nel sentiero dietro
l\textquotesingle asino, facendo sobbalzare l\textquotesingle animale.

--- Scendi --- ordinò il ladrone.

L\textquotesingle asino si fermò sul sentiero. Frate Francis gettò
indietro il cappuccio per mostrare la benda sull\textquotesingle occhio
e alzò un dito tremante fino a toccarla. Cominciò a sollevare lentamente
la benda sull\textquotesingle occhio.

Il ladrone rovesciò la testa e rise d\textquotesingle una risata che
avrebbe potuto sgorgare, pensò Francis, dalla gola di Satana; il monaco
mormorò un esorcismo, ma il ladrone non ne sembrò toccato.

--- Questo trucco di voi buffoni vestiti di nero è logoro ormai da anni
--- disse. --- Adesso scendi.

Frate Francis sorrise, alzò le spalle e smontò senza ulteriori proteste.
Il ladrone esaminò l\textquotesingle asino, gli batté sui fianchi, gli
osservò i denti.

--- Mangiare? Mangiare? --- gridò una delle figure incappucciate dalla
collina.

--- Questa volta no --- abbaiò il ladrone. --- Troppo magro. Frate
Francis non era completamente sicuro che stessero parlando
dell\textquotesingle asino.

--- Buongiorno a voi, signore --- disse cordialmente il monaco. ---
Potete prendere l\textquotesingle asino. Camminare migliorerà la mia
salute, credo. --- Sorrise di nuovo e fece per avviarsi.

Una freccia saettò sul sentiero e si infisse ai suoi piedi.

--- Finiscila! --- ululò il ladrone, e poi rivolto a Francis: --- Adesso
spogliati. E vediamo cosa c\textquotesingle è nel rotolo e nella
bisaccia.

Frate Francis toccò la ciotola delle elemosine e fece un gesto
impotente, che provocò soltanto un\textquotesingle altra risata
sarcastica del ladrone.

Ho già visto anche questo trucco --- disse. --- L\textquotesingle ultimo
uomo con la ciotola che ho visto aveva un heklo d\textquotesingle oro
nascosto nello stivale. E adesso spogliati.

Frate Francis, che non portava stivali, mostrò speranzoso i suoi
sandali, ma il ladrone fece un gesto impaziente. Il monaco slegò la
bisaccia, ne sparse il contenuto, e cominciò a svestirsi. Il ladrone gli
frugò gli abiti, non trovò nulla, e ributtò l\textquotesingle abito al
suo proprietario, che mormorò la sua gratitudine; aveva previsto di
essere lasciato nudo sul sentiero.

--- Adesso vediamo cosa c\textquotesingle è dentro
l\textquotesingle altro involto.

--- Contiene soltanto documenti, signore --- protestò il monaco --- che
non hanno alcun valore se non per il loro proprietario.

--- Apri.

In silenzio, frate Francis slegò l\textquotesingle involto e ne tolse la
\emph{blueprint} e la copia alluminata. Gli intarsi in foglia
d\textquotesingle oro e il disegno colorato lampeggiarono vivacemente
nella luce del sole che filtrava attraverso il fogliame. Il ladrone
spalancò la bocca e zufolò sommessamente.

--- Che bello! Alla donna piacerebbe, per appenderlo alla parete della
baracca! Francis si sentì male.

--- Oro! --- gridò il ladrone ai suoi complici incappucciati che erano
rimasti sulla collina.

--- Mangiare? Mangiare? --- venne la risposta gorgogliante.

--- Mangeremo, non abbiate paura! --- gridò il ladrone, poi spiegò a
Francis, in tono discorsivo: --- Hanno fame, dopo essere stati lì seduti
per due giorni. Gli affari vanno male. Il traffico è scarso, in questi
tempi.

Francis annuì. Il ladrone riprese ad ammirare la copia alluminata.

``Signore, se Tu lo hai mandato per mettermi alla prova, allora aiutami
a morire da uomo, fa\textquotesingle{} che possa prenderla soltanto
sopra il cadavere del Tuo servo. Beato Leibowitz, guardami e prega per
me\ldots''

--- Cos\textquotesingle è? --- chiese il ladrone. --- Un incantesimo?
--- Studiò i due documenti, uno accanto all\textquotesingle altro, per
qualche minuto. Oh! Uno è il fantasma dell\textquotesingle altro. Che
magia è questa? --- Fissò frate Francis con i sospettosi occhi grigi..
--- Come si chiama?

--- Oh\ldots{} Sistema di Controllo Transistorizzato per
l\textquotesingle Unità Sei-B --- balbettò il monaco.

Il ladrone, che aveva osservato i documenti a rovescio, aveva egualmente
compreso che un diagramma comportava l\textquotesingle inversione
fondo-disegno rispetto all\textquotesingle altro\ldots{} un effetto che
sembrava sbalordirlo quanto la foglia d\textquotesingle oro. Seguì le
somiglianze tra i due documenti con un dito tozzo e sudicio, lasciando
una lieve traccia sulla cartapecora alluminata. Francis ricacciò le
lacrime.

--- \emph{Vi prego!} --- ansimò il monaco. --- L\textquotesingle oro è
così sottile, non vale niente, in pratica. Soppesatelo nella mano. Pesa
ben poco più della carta. Non vi servirà a nulla. Vi prego, signore,
prendete il mio abito, invece. Prendete l\textquotesingle asino,
prendete la mia bisaccia. Prendete quello che volete, ma lasciatemi
questi. Per voi non significano nulla.

Lo sguardo grigio del ladrone era meditabondo. Osservò
l\textquotesingle agitazione dei monaco e si soffregò il mento. --- Ti
permetterò di tenere gli abiti, l\textquotesingle asino e tutto, tranne
questi! offrì. --- Prenderò soltanto gli incantesimi.

--- Per l\textquotesingle amore di Dio, signore, allora uccidetemi! ---
gemette frate Francis. Il ladrone rise cinicamente.

--- Vedremo. Dimmi a che cosa servono.

--- A niente. Uno è il ricordo di un uomo morto da molto tempo. Un
antico. L\textquotesingle altro è soltanto una copia.

--- E a te a che cosa servono?

Francis chiuse gli occhi per un attimo e cercò di pensare a una
spiegazione. --- Conoscete le tribù della foresta? Sapete come venerano
i loro antenati?

Gli occhi grigi del ladrone lampeggiarono d\textquotesingle ira, per un
momento. --- Noi \emph{disprezziamo} i nostri antenati --- latrò. ---
Maledetti siano coloro che ci hanno generati!

--- Maledetti, maledetti! --- fece eco uno dei due arcieri sulla
collina.

--- Sai chi siamo? Sai da dove veniamo?

Francis annui. --- Non intendevo offendervi. L\textquotesingle antico
cui appartiene questa reliquia\ldots{} non è un nostro antenato. Era il
nostro maestro. Noi veneriamo la sua memoria. Questo è soltanto un
ricordo, nient\textquotesingle altro.

--- E la copia?

--- L\textquotesingle ho fatta io stesso. Vi prego, signore, vi ho
impiegato \emph{quindici anni}. Per voi non significa nulla. Vi
prego\ldots{} non vorrete togliere \emph{quindici anni di vita} a un
uomo\ldots{} senza una ragione?

--- \emph{Quindici anni?} --- Il ladrone rovesciò la testa e ululò una
risata. --- Hai dedicato quindici anni a fare questo?

--- Oh, ma\ldots{} --- Francis si interruppe
all\textquotesingle improvviso. I suoi occhi puntarono verso il tozzo
dito del ladrone. Il dito stava battendo sulla \emph{blueprint}
originale.

--- Questo ti ha preso quindici anni? È quasi brutto, vicino
all\textquotesingle altro. --- Si batté una mano sulla pancia e fra le
risate continuò a indicare la reliquia. --- Ah, quindici anni! Dunque è
questo che fate, laggiù! Perché? A cosa serve questa immagine fantasma?
Quindici anni per farla! Oh, oh! Che lavoro da donna!

Francis lo osservava in un silenzio stordito. Il fatto che il ladrone
avesse scambiato la sacra reliquia per la sua copia lo aveva scosso
troppo profondamente perché potesse rispondere.

Continuando a ridere, il ladrone prese in mano entrambi i documenti e
fece il gesto di strapparli a metà.

--- Gesù, Maria, Giuseppe! --- gridò il monaco, inginocchiandosi sul
sentiero. --- Per l\textquotesingle amor di Dio, signore!

Il ladrone buttò in terra i fogli. --- Mi batterò con te per questi ---
offrì, sportivamente. --- Questi contro il mio coltello.

Ci sto --- disse impulsivamente Francis, pensando che una lotta avrebbe
per lo meno offerto al Cielo una possibilità di intervenire in modo
discreto. ``O Dio. Tu che desti forza a Giacobbe perché vincesse
l\textquotesingle angelo sulla montagna\ldots''

Si misero in posizione. Frate Francis si fece il segno della croce. Il
ladrone si tolse il coltello dalla cintura e lo buttò sui documenti.
Girarono uno attorno all\textquotesingle altro.

Tre secondi dopo, il monaco era riverso al suolo, sotto una piccola
montagna di muscoli. Un sasso aguzzo sembrava spezzargli la spina
dorsale.

--- \emph{Eh-eh} --- fece il ladrone, e si alzò per riprendere il
coltello e per arrotolare i documenti.

Con le mani giunte in preghiera, frate Francis lo seguì in ginocchio,
supplicando con tutto il fiato che aveva nei polmoni. --- Vi prego,
allora, prendetene soltanto uno, non tutti e due! Vi prego!

--- Adesso dovrai ricomprarlo --- ridacchiò il ladrone. --- Li ho vinti
in una lotta leale.

--- Ma io non ho nulla. Io sono \emph{povero}!

--- E va bene, se li desideri tanto, devi pagare in oro.

Due heklos d\textquotesingle oro è il prezzo del riscatto. Portali qui
quando vorrai. Io nasconderò questa roba nella mia tana. Se li rivuoi,
porta l\textquotesingle oro.

--- Ascoltate, sono importanti per altra gente, non per me. Io li stavo
portando al papa. Forse vi pagheranno per il documento più importante.
Ma lasciatemi l\textquotesingle altro, per mostrarlo. Non ha nessuna
importanza, quello.

Il ladrone si voltò, ridendo. --- Credo che mi baceresti gli stivali,
per riaverlo.

Frate Francis lo prese in parola e gli baciò con fervore lo stivale.

Questo fu troppo anche per un tipo come il ladrone. Respinse il monaco
con un piede, separò i due fogli, ne scagliò uno in faccia a Francis con
una maledizione. Salì in groppa all\textquotesingle asinello e lo spronò
su per il pendio. Frate Francis afferrò il prezioso documento e strisciò
dietro al ladrone, ringraziandolo a profusione e benedicendolo
ripetutamente mentre l\textquotesingle altro guidava
l\textquotesingle asino verso gli arcieri.

--- \emph{Quindici anni!} --- sbuffò il ladrone, e respinse di nuove
Francis con il piede. --- Vattene! --- E agitò alto nel sole quello
splendore alluminato. --- Ricordati\ldots{} due heklos
d\textquotesingle oro riscatteranno il tuo documento. E
di\textquotesingle{} al tuo papa che l\textquotesingle ho vinto
lealmente.

Francis smise di arrampicarsi. Tracciò un benedicente segno della croce
dietro il bandito che si allontanava e lodò quietamente Dio per
l\textquotesingle esistenza di ladroni tanto altruisti, che potevano
commettere simili errori di ignoranza. Si vezzeggiò la \emph{blueprint}
originale, teneramente, mentre percorreva il sentiero. Il ladrone stava
mostrando orgogliosamente la bellissima copia ai suoi compagni mutanti,
sulla collina.

--- Mangiare! Mangiare! --- disse uno di loro, accarezzando
l\textquotesingle asino.

--- Cavalcare, cavalcare --- corresse il ladrone. --- Mangiare, dopo.

Ma quando frate Francis li ebbe lasciati indietro, una grande amarezza
lo travolse, gradualmente. La voce sarcastica gli risuonava ancora nelle
orecchie. \emph{Ah, quindici anni! Dunque è questo che fate, laggiù!
	Quindici anni! Che lavoro da donna! Oh oh oh oh\ldots{}}

Il ladrone aveva commesso un errore. Ma quei quindici anni erano perduti
in ogni caso, e con essi tutto l\textquotesingle amore e il tormento che
aveva dedicato alla copia alluminata.

Adattatosi a vivere nel chiostro, Francis si era disabituato alle vie
del mondo esterno, alle sue rudi consuetudini e ai suoi modi bruschi. Si
accorse che il suo cuore era profondamente turbato dal sarcasmo del
ladrone. Pensò al più tenero sarcasmo di frate Jeris, nei primi anni..
Forse frate Jeris aveva avuto ragione.

Teneva il capo chino sotto il cappuccio, mentre proseguiva lentamente il
suo cammino.

Per lo meno, aveva la reliquia originale. Per lo meno.
