\mainmatter
{\chapterstyle{dowding} \chapter*{PARTE PRIMA\\\leavevmode\\\footnotesize{Fiat Homo}}}

\chapter{\phantom{text}}

\lettrine{F}{rate} Francis Gerard dello Utah non avrebbe mai scoperto, probabilmente,
i documenti benedetti, se non fosse stato per il pellegrino dai lombi
cinti che apparve al giovane novizio durante il digiuno quaresimale nel
deserto.

Mai, prima di allora, frate Francis aveva visto un pellegrino dai lombi
cinti, ma quella fu proprio una prova di buonafede che lo convinse non
appena si fu ripreso dall\textquotesingle agghiacciante effetto
dell\textquotesingle apparizione del pellegrino
all\textquotesingle orizzonte, come una tremolante iota nera nel
riverbero scintillante del calore. Privo di gambe e con un capo
minuscolo, la iota si materializzò dalla lucentezza di specchio della
strada dissestata e sembrò avvicinarsi vibrando più che camminando,
inducendo frate Francis ad afferrare stretto il crocifisso del suo
rosario e a mormorare un paio di \emph{Ave Maria}. La iota faceva
pensare a una minuscola apparizione evocata dai demoni del calore che
torturavano la terra a mezzogiorno, quando ogni essere vivente che si
trovava nel deserto --- a eccezione delle poiane e di pochi eremiti come
Francis --- giaceva immobile nel suo covo o si nascondeva dietro una
roccia per ripararsi dalla ferocia del sole. Soltanto una cosa
mostruosa, uria cosa preternaturale, o una cosa dallo spirito corrotto
poteva scendere deliberatamente lungo quella pista, a mezzogiorno, in
quel modo.

Frate Francis aggiunse una frettolosa preghiera a san Raul il Ciclopeo,
protettore dei malnati, invocando il suo aiuto contro gli infelici
protetti dal santo. Perché, chi non sapeva che in quei giorni
v\textquotesingle erano molti mostri sulla terra? Ciò che nasceva vivo e
vitale doveva, secondo la legge della Chiesa e della Natura, rimanere
vivo ed essere aiutato a raggiungere la maturità, se possibile, da
coloro che lo avevano generato. Non sempre la legge era rispettata, ma
lo era pur sempre in misura sufficiente da permettere
l\textquotesingle esistenza d\textquotesingle una popolazione sparsa di
mostri adulti, che spesso sceglievano per i loro vagabondaggi le più
remote tra le terre deserte, dove la notte si aggiravano attorno ai
fuochi dei viaggiatori della prateria. Ma alla fine la iota uscì dalle
colonne d\textquotesingle aria riscaldata nell\textquotesingle aria
limpida, dove diventò manifestamente un pellegrino lontano; frate
Francis lasciò andare il crocifisso con un piccolo \emph{amen}.

Il pellegrino era un vecchio magrissimo con un bastone, un cappellaccio,
una barba ispida, e un otre appeso alla spalla. Masticava e sputava con
eccessivo gusto per essere una apparizione, e sembrava troppo fragile e
sparuto per essere un bandito. Tuttavia Francis si scostò lentamente
dalla linea di visuale del pellegrino e si accosciò dietro un mucchio di
pietre, da dove poteva osservare senza essere visto. Gli incontri fra
estranei nel deserto, sebbene fossero rari, erano occasione di reciproco
sospetto, ed erano contraddistinti da preparazioni iniziali da ambo le
parti, in attesa d\textquotesingle un episodio che si dimostrasse
cordiale o ostile.

Era difficile che un laico, o uno straniero percorresse la vecchia
strada che passava accanto all\textquotesingle abbazia: questo non
accadeva più di tre volte all\textquotesingle anno, nonostante
l\textquotesingle oasi che consentiva l\textquotesingle esistenza di
quella abbazia e che avrebbe trasformato il monastero in un naturale
ospizio per i viandanti se quella strada non fosse stata una strada che
veniva dal nulla e puntava verso il nulla, secondo il concetto dei
viaggi di quei tempi. Forse, in età più antiche, quella strada era stata
una porzione della via più breve dal Grande. Lago Salato alla Vecchia El
Paso: a sud dell\textquotesingle abbazia intersecava una striscia molto
simile, di pietra spezzata, che puntava verso est e verso ovest. Il
crocicchio era consunto dal tempo\ldots{} ma non
dall\textquotesingle uomo, almeno in tempi recenti.

Il pellegrino avanzò fino a giungere a portata di voce, ma il novizio
rimase nascosto dietro il mucchio di macerie. I lombi del pellegrino
erano veramente cinti con un pezzo di canovaccio sudicio che era il suo
unico indumento, a eccezione del cappello e dei sandali. Avanzava
ostinato, con mosse meccaniche, aiutando la gamba invalida con il
pesante bastone. La sua andatura ritmica era quella d\textquotesingle un
uomo che aveva percorso molta strada e ne aveva ancora molta davanti a
sé. Ma, entrando nella zona coperta dalle antiche rovine, si fermò per
guardarsi intorno.

Francis si chinò.

Non c\textquotesingle era ombra fra i mucchi di macerie, là dove un
tempo sorgeva un gruppo di edifici antichissimi, ma qualcuna delle
pietre più grosse poteva offrire un po\textquotesingle{} di frescura ad
alcune parti dell\textquotesingle anatomia dei viaggiatori che
conoscevano la strada del deserto come aveva dimostrato di conoscerla il
pellegrino. Cercò per alcuni istanti una pietra di proporzioni adatte.
Frate Francis osservò, con approvazione, che il pellegrino non afferrava
la pietra per rovesciarla avventatamente: invece si fermò a qualche
passo e, usando il bastone come leva e una pietra più piccola come
fulcro, sollevò quella più grande fino a che
l\textquotesingle inevitabile creatura sibilante che era nascosta li
sotto uscì strisciando. Il viandante uccise spassionatamente il serpente
con il bastone e ne gettò da parte la carcassa che ancora si contorceva.
Dopo aver eliminato l\textquotesingle occupante della fresca fessura che
stava sotto la pietra, il pellegrino la rovesciò. Poi, sollevando la
parte posteriore della tela che gli fasciava i lombi, posò le natiche
avvizzite sulla parte inferiore, relativamente fresca, della pietra, si
tolse scalciando i sandali e premette le piante dei piedi contro ciò che
era stato il fondo sabbioso della fresca depressione. Così ristorato,
agitò le dita, sorrise con la bocca sdentata e cominciò a mormorare una
melodia. Dopo un po\textquotesingle, stava cantando una specie di
lamentosa cantilena in un dialetto che il novizio non conosceva. Stanco
di starsene acquattato, frate Francis si agitò, irrequieto.

Mentre cantava, il pellegrino aprì un involto che conteneva una galletta
e un pezzo di cacio. Poi smise di cantare e si alzò per un attimo a
dire, sommessamente, nel vernacolo della regione: ``Benedetto sia Adonai
Elohim, Re di Tutto, che fa crescere il pane dalla terra'' con una
strascicata voce nasale. Poi tornò a sedersi e cominciò a mangiare.

Il viandante veniva certo da molto lontano, pensò frate Francis, che non
conosceva alcun reame vicino governato da un monarca con un nome così
poco familiare e con pretese tanto strane. Il vecchio faceva un
pellegrinaggio di penitenza, azzardò frate Francis\ldots{} forse al
santuario dell\textquotesingle abbazia, sebbene il ``santuario'' non
fosse ancora ufficialmente un santuario, e il suo ``santo'' non fosse
ancora ufficialmente un santo. Frate Francis non riusciva a trovare
altra spiegazione per la presenza del vecchio viandante su quella strada
che non conduceva in alcun luogo.

Il pellegrino era occupato con il pane e il formaggio, e il novizio
diventava sempre più irrequieto, via via che la sua ansia svaniva. La
regola del silenzio per i giorni del digiuno quaresimale non gli
permetteva di conversare volontariamente con il vecchio, ma se avesse
lasciato il suo nascondiglio dietro il mucchio di macerie prima che il
vecchio si allontanasse, certamente sarebbe stato visto o sentito dal
pellegrino, poiché aveva ricevuto la proibizione di allontanarsi dal suo
eremitaggio prima della fine della Quaresima.

Ancora un po\textquotesingle{} esitante, frate Francis si schiarì forte
la gola, poi si alzò, mettendosi in vista.

--- Ehm!

Il pane e il cacio del pellegrino schizzarono via. Il vecchio afferrò il
bastone e scattò in piedi. --- Vuoi aggredirmi, eh?

Brandiva minacciosamente il bastone verso la figura incappucciata che
era apparsa dietro il mucchio di pietre. Frate Francis notò che
l\textquotesingle estremità del bastone era armata di uno sperone. Il
novizio si inchinò cortesemente per tre volte, ma il pellegrino non badò
a quel gesto gentile.

--- Adesso resta dove sei! --- gracchiò. --- Resta a distanza, amico.
Non ho nulla che possa interessarti\ldots{} a meno che non sia il cacio,
e questo posso dartelo. Se è carne che vuoi, non sono altro che
cartilagine, ma mi batterò per conservarla. Indietro, adesso! Indietro!

--- Aspetta\ldots{} --- Il novizio si interruppe. La carità, o anche la
semplice cortesia, poteva avere la precedenza sulla regola del silenzio
quaresimale, quando le circostanze lo richiedevano, ma spezzare il
silenzio di propria volontà lo rendeva sempre un po\textquotesingle{}
nervoso.

--- Non sono un malvagio, buon uomo --- continuò usando la formula più
educata. Gettò indietro il cappuccio per mostrare la sua tonsura
monastica, e sollevò la corona del rosario. --- Sai cosa è questo?

Per parecchi secondi il vecchio rimase all\textquotesingle erta come un
gatto pronto al combattimento, mentre studiava il volto
d\textquotesingle adolescente del novizio, bruciato dal sole. Il
pellegrino aveva commesso un errore naturale. Le creature grottesche che
infestavano il limitare del deserto portavano spesso cappucci, maschere,
o cappe voluminose per nascondere le loro deformità. Fra esse ve ne
erano di quelle la cui deformità non era limitata al corpo, e che
consideravano i viandanti come una sicura riserva di selvaggina.

Dopo un breve esame, il pellegrino si raddrizzò. --- Oh\ldots{} uno di
loro. --- Si appoggiò al bastone e fece una smorfia. --- Quella laggiù è
l\textquotesingle Abbazia di Leibowitz? --- chiese, indicando il lontano
gruppo di edifici, verso sud.

Frate Francis si inchinò educatamente e annuì.

--- Cosa stai facendo, qui fra le rovine?

Il novizio raccolse un frammento di pietra simile al gesso. Era
statisticamente improbabile che il viandante non fosse analfabeta, ma
frate Francis decise di tentare. Poiché il volgare del popolo non aveva
né alfabeto né ortografia, scrisse le parole latine per Penitenza,
Solitudine e Silenzio su una grande pietra piatta, e più sotto le
trascrisse in antico inglese, sperando che --- nonostante il suo segreto
desiderio d\textquotesingle aver qualcuno con cui parlare --- il vecchio
comprendesse e lo lasciasse alla sua solitaria vigilia quaresimale.

Il vecchio sorrise ironicamente, vedendo l\textquotesingle iscrizione.
La sua risata sembrò più un belato fatalistico che una risata. ---
\emph{Hmmmm-hmmm!} È ancora tutto scritto a rovescio!---disse; ma non
lasciò capire se avesse compreso l\textquotesingle iscrizione.

Depose il bastone, tornò a sedersi sulla pietra, riprese dalla sabbia il
pane e il cacio e cominciò a grattarli per ripulirli. Francis si inumidì
affamato le labbra, ma distolse lo sguardo. Non aveva mangiato altro che
fichi d\textquotesingle india e una manciata di grano secco fin dal
Mercoledì delle Ceneri: le regole del digiuno e
dell\textquotesingle astinenza erano piuttosto rigorose per le vigilie
di vocazione.

Notando il suo imbarazzo, il pellegrino spezzò il pane e il cacio e ne
offri una porzione a frate Francis. Nonostante la disidratazione,
causata dalla sua magrissima scorta d\textquotesingle acqua, la bocca
del novizio si inondò di saliva. Gli occhi rifiutarono di staccarsi
dalla mano che offriva il cibo. L\textquotesingle universo si contrasse
e il suo esatto centro geometrico fu quel pezzo sabbioso di pane nero e
di pallido formaggio. Un demone comandò ai muscoli della sua gamba
sinistra di portare il suo piede sinistro in avanti di mezzo metro. Poi
il demone si impossessò della sua gamba destra perché portasse il piede
destro davanti al sinistro, forzò i muscoli pettorali e i bicipiti
destri a tendere il braccio, finché la mano toccò la mano del
pellegrino. Le sue dita sentirono il cibo: sembrarono persino
assaggiarlo. Un brivido involontario percorse il suo corpo affamato.
Chiuse gli occhi e vide il Signor Abate che gli lanciava occhiate
folgoranti brandendo una sferza da toro. Ogni volta che il novizio
cercava di immaginare visivamente la Santissima Trinità,
l\textquotesingle aspetto di Dio Padre si confondeva sempre con il viso
dell\textquotesingle abate, che di solito era molto corrucciato, almeno
così pareva a Francis. Dietro l\textquotesingle abate infuriava un
fuoco, e in mezzo alle fiamme gli occhi del Beato Martire Leibowitz si
posavano, nella sofferenza della morte, sul suo protetto digiunante,
colto nell\textquotesingle atto di prendere il formaggio.

Il novizio rabbrividì ancora. --- \emph{Àpage Satanas!} --- sibilò,
mentre balzava all\textquotesingle indietro e lasciava cadere il cibo.
Senza preavviso, spruzzò il vecchio con acqua santa, da una minuscola
fiala che si tolse dalla manica. Il pellegrino era diventato per un
attimo indistinguibile dall\textquotesingle Arcinemico nella mente del
novizio stordito dal sole.

Quell\textquotesingle attacco di sorpresa contro le Potenze delle
Tenebre e della Tentazione non produsse alcun immediato risultato
soprannaturale, ma i risultati naturali si presentarono \emph{ex opere
	operato}. Il pellegrino-Belzebù non esplose in fumo sulfureo, ma emise
suoni gorgoglianti, arrossì violentemente e balzò verso Francis con uno
strillo raccapricciante. Il novizio incespicò nella tunica mentre
fuggiva per salvarsi dal bastone chiodato del pellegrino e riuscì a
scampare dalle unghiate soltanto perché il pellegrino aveva dimenticato
i sandali. La carica del vecchio si ridusse a una serie di sussulti
zoppicanti, Sembrò accorgersi all\textquotesingle improvviso dei sassi
taglienti sotto le sue piante nude. Si fermò preoccupato. Quando frate
Francis si voltò, vide il pellegrino che si ritirava verso il suo fresco
rifugio saltando sulla punta dell\textquotesingle alluce.

Vergognandosi dell\textquotesingle odore di formaggio che persisteva sui
suoi polpastrelli, e pentendosi del suo irrazionale esorcismo, il
novizio ritornò al suo lavoro tra le vecchie rovine, mentre il
pellegrino si rinfrescava i piedi e sfogava la sua ira scagliando di
tanto in tanto un sasso contro il giovane quando quello ricompariva fra
i mucchi di macerie. Quando, alla fine, il suo braccio fu troppo stanco,
si limitò a fingere di scagliare i sassi e quando Francis smise di
scostarsi alle sue finte si accontentò di brontolare sul pane e sul
cacio.

Il novizio si muoveva fra le rovine, e ogni tanto si dirigeva
barcollando verso qualche punto focale del suo lavoro con una pietra
grande quanto il suo torace, stretta in un abbraccio faticoso. Il
pellegrino lo osservò mentre sceglieva una pietra, ne calcolava le
dimensioni a spanne, la scartava, ne sceglieva un\textquotesingle altra,
la liberava dalle macerie, la sollevava e la trascinava via. Dopo pochi
passi la lasciò cadere e, sedendosi all\textquotesingle improvviso, si
posò la testa sulle ginocchia nello sforzo evidente di non svenire. Dopo
aver ansimato un poco, si alzò di nuovo e fece rotolare la pietra verso
la sua destinazione. Continuò il suo lavoro mentre il pellegrino, invece
di guardarlo corrucciato, cominciava a osservarlo con interesse.

Il sole scagliava le sue maledizioni meridiane sulla terra
incartapecorita, stendendo il suo anatema su tutte le cose umide.
Francis continuò a lavorare nonostante il caldo.

Quando il viandante ebbe inghiottito l\textquotesingle ultimo pezzo del
pane e del cacio sabbiosi con l\textquotesingle aiuto di pochi sorsi del
suo otre, infilò i piedi nei sandali, si alzò con un grugnito e avanzò
fra le rovine, verso il punto in cui il novizio lavorava. Notando
l\textquotesingle approssimarsi del vecchio, frate Francis si affrettò a
mettersi a distanza di sicurezza. Il vecchio brandì verso di lui il
bastone chiodato in un gesto irridente, ma sembrava più incuriosito dal
lavoro del giovane che ansioso di vendetta. Si fermò a osservare la tana
del novizio.

Lì, vicino al limitare orientale delle rovine, frate Francis aveva
scavato un trincea poco profonda, usando un bastone per zappa e le mani
per badile. Il primo giorno di Quaresima aveva coperto quel fossato con
un mucchio di frasche, e se ne era servito, di notte, come di un rifugio
contro i lupi del deserto. Ma, via via che i giorni del digiuno
passavano, la sua presenza aveva accresciuto le sue tracce nei dintorni,
e ora i lupi sembravano eccessivamente attratti
dall\textquotesingle area delle rovine e giungevano persino a raspare
con le zampe attorno al mucchio di frasche, dopo che il fuoco si era
spento.

Dapprima Francis aveva cercato di scoraggiare i loro scavi notturni
aumentando lo spessore del mucchio di arbusti sulla sua trincea, e
circondandola con un cerchio di pietre molto vicine le une alle altre,
Ma la notte precedente, qualcosa era balzato sul mucchio di arbusti e
aveva ululato mentre Francis se ne stava disteso lì sotto,
rabbrividendo; di conseguenza aveva deciso di fortificare il rifugio e,
usando il primo cerchio di pietre come fondamenta, aveva cominciato a
erigere un muro. Il muro si inclinava verso l\textquotesingle interno,
man mano che cresceva; ma poiché la sua pianta aveva approssimativamente
una rozza forma ovale, le pietre d\textquotesingle ogni nuovo strato si
appoggiavano alle pietre adiacenti, evitando un crollo verso
l\textquotesingle interno. Ora frate Francis sperava che, scegliendo con
cura le pietre e aiutandosi con terra e ciottoli per riempire gli
interstizi, sarebbe riuscito a completare una cupola. E
un\textquotesingle unica fila di pietre ad arco, sfidando in un certo
senso la gravità, se ne stava eretta sul suo rifugio, come simbolo di
questa ambizione. Frate Francis abbaiò come un cucciolo, quando il
pellegrino saggiò curiosamente la resistenza dell\textquotesingle arco
con il suo bastone.

Preoccupato per il suo rifugio, il novizio si era avvicinato durante
l\textquotesingle ispezione del pellegrino. Questi rispose al suo
strillo agitando il bastone e lanciando un ululato agghiacciante. Frate
Francis incespicò nell\textquotesingle orlo della tunica e cadde a
sedere. Il vecchio ridacchiò.

--- \emph{Hmmmm-hmmm!} Avrai bisogno d\textquotesingle una pietra dalla
forma strana per adattarla a questo buco --- disse, e batté il bastone
nell\textquotesingle interno d\textquotesingle uno spazio vuoto nella
fila di pietre più alta.

Il giovane annuì e distolse lo sguardo. Restò seduto sulla sabbia e, con
il suo silenzio e lo sguardo abbassato, sperò di far capire al vecchio
che non era libero di conversare né di accettare volentieri la presenza
di un estraneo nel luogo della sua solitudine quaresimale. Il novizio
cominciò a scrivere sulla sabbia con uno stecco: \emph{Et ne nos inducas
	in\ldots{}}

--- Non ti ho ancora offerto di cambiare in pane queste pietre, vero?
--- chiese di rimando il vecchio viaggiatore..

Frate Francis alzò subito lo sguardo. Dunque era così! Il vecchio sapeva
leggere, e sapeva leggere la Scrittura! Inoltre, la sua osservazione
sottintendeva che aveva compreso tanto l\textquotesingle uso impulsivo
dell\textquotesingle acqua santa da parte del novizio, quanto le ragioni
della sua presenza in quel luogo. Ormai conscio che il pellegrino
intendeva stuzzicarlo, frate Francis riabbassò lo sguardo e attese.

--- \emph{Hmmmm-hmmm!} Dunque bisogna lasciarti in pace, no? Bene,
allora, farò meglio a riprendere il cammino. Dimmi, i tuoi fratelli
dell\textquotesingle abbazia permetteranno a un vecchio di riposare un
po\textquotesingle{} alla loro ombra?

Frate Francis annuì. --- Ti daranno anche acqua e cibo --- aggiunse
sottovoce, in carità.

Il pellegrino ridacchiò. --- In cambio ti troverò una pietra adatta a
quella fessura, prima di andarmene. Dio sia con te.

\emph{Ma non c\textquotesingle è bisogno\ldots{}} La protesta morì,
prima ancora di essere pronunciata. Frate Francis osservò il vecchio che
si allontanava lentamente, vagando qua e là fra le macerie. Ogni tanto
si fermava per osservare una pietra e per toccarla con il bastone. Senza
dubbio la sua ricerca sarebbe stata infruttuosa pensò il novizio, perché
era la ripetizione d\textquotesingle una ricerca che egli stesso aveva
compiuto sin da metà mattina. Alla fine aveva deciso che sarebbe stato
più facile rimuovere e ricostruire una sezione della fila superiore, che
non trovare una chiave di volta dalla forma simile a quella della
fessura. Ma, senza dubbio, il pellegrino avrebbe esaurito presto la
propria pazienza e avrebbe proseguito il suo cammino.

Nel frattempo, frate Francis si riposò. Pregò per riottenere quella
intimità interiore richiesta dallo scopo della sua vigilia: una
pergamena dello spirito pulita su cui le parole d\textquotesingle una
chiamata potessero essere scritte nella sua solitudine.., se
quell\textquotesingle altra incommensurabile Solitudine che era Iddio
avesse teso la Sua mano e avesse toccato quella minuscola solitudine
umana per segnarvi la vocazione. Il \emph{Piccolo Libro} che il priore
Cheroki gli aveva lasciato la domenica precedente, serviva come guida
alla sua meditazione. Era vecchio di secoli ed era chiamato
\emph{Libellus Leibowitz}, sebbene soltanto una tradizione incerta
l\textquotesingle attribuisse allo stesso Beato.

\emph{``Parum equidem te diligebam, Domine, juventute mea; quarti doleo
	nimis\ldots{} Troppo poco, o} \emph{Signore, io Ti amai nel tempo della
	mia gioventù, e di questo molto mi dolgo nel tempo della mia vecchiaia.
	Invano fuggii da Te in quei giorni\ldots''}

--- Ehi! Ecco qua! --- fu il grido che si levò oltre i mucchi di
macerie.

Frate Francis levò lo sguardo per un attimo, ma il pellegrino non era
visibile. I suoi occhi si riabbassarono sulla pagina.

\emph{``Repugnans tibi, ausus sum quaerere quidquid doctius mihi fide,
	certius spe, aut dulcius caritate visum esset. Quis itaque stultior
	me\ldots''}

--- Ehi, ragazzo! --- risonò di nuovo il grido. --- Ti ho trovato una
pietra, che probabilmente si adatterà al buco.

Questa volta, quando frate Francis alzò lo sguardo, intravide il bastone
del pellegrino che si agitava facendogli segnali, dietro un mucchio di
macerie. Sospirando, il novizio riprese la lettura.

\emph{``O inscrutabilis Scrutator animarum, cui patet omne con, si me
	vocaveras, olim a tefugeram. Si autem nunc velis vocare me
	indignum\ldots''}

E la voce irritata, al di là del mucchio di macerie: --- E va bene,
allora, fai come vuoi. Farò un segno sulla pietra e vi pianterò vicino
un ramo. Tu provala o no, fai come vuoi.

--- Grazie --- sospirò il novizio, ma dubitò che il vecchio lo udisse.
Continuò a leggere: ``Libera me, Domine, a vitiis meis, ut solius tuae
voluntatis mihi cupidus sim, et vocationis\ldots''

--- Ecco --- gridò il pellegrino. --- Ho messo il ramo, e il segno. E ti
auguro di ritrovare presto la voce, ragazzo. \emph{Ullallà!}

Poco dopo che l\textquotesingle ultimo grido fu svanito, frate Francis
intravide il pellegrino avanzare sulla pista che conduceva
all\textquotesingle abbazia. Il novizio sussurrò una rapida benedizione
dietro di lui, e una preghiera perché il suo cammino fosse sicuro.

Ora che la sua intimità gli era stata resa, frate Francis rimise il
libro nel rifugio e ricominciò la sua azzardata attività edilizia, senza
prendersi ancora il disturbo di esaminare ciò che il pellegrino aveva
scoperto. Mentre il suo corpo famelico si tendeva e vacillava sotto il
peso delle pietre, la sua mente continuava a ripetere macchinalmente la
preghiera per la certezza della sua vocazione: \emph{"Libera me, Domine,
	a vitiis meis\ldots{} Liberami, o Signore, dai miei vizi, così che nel
	mio cuore io desideri soltanto la Tua volontà, e sia conscio della Tua
	chiamata se verrà\ldots{} ut solius tuae voluntatis mihi cupidus sim, et
	vocationis tuae conscius, si digneris me vocare. Amen.}

``Liberami, o Signore, dai miei vizi, così che nel mio cuore\ldots''

Un gregge di cumuli, diretti a impartire la benedizione della pioggia
alle montagne, dopo aver crudelmente deluso il deserto inaridito,
cominciò a nascondere il sole e a trascinare strisce
d\textquotesingle ombra sull\textquotesingle arida terra, offrendo
ristoro intermittente ma bene accetto dalla bruciante luce solare.
Quando una fuggevole ombra di nube passava sopra le rovine, il novizio
lavorava rapidamente fino a che l\textquotesingle ombra si allontanava,
poi riposava fino a che il cumulo seguente oscurava il sole.

Fu per puro caso che frate Francis scoprì, alla fine, la pietra del
pellegrino. Mentre vagava lì attorno, incespicò nel ramo che il vecchio
aveva infisso nel suolo come segnale. Si trovò a terra, sulle mani e
sulle ginocchia, a fissare un paio di segni scritti di fresco con il
gesso su una vecchia pietra:

\begin{center}
	{\Huge{\textcjheb{.sl}}}
\end{center}\

~

I segni erano tracciati con tanta cura che frate Francis intuì
immediatamente che doveva trattarsi di simboli, ma dopo alcuni minuti di
meditazione rimase egualmente perplesso. Forse erano simboli della
stregoneria? Ma no, il vecchio aveva esclamato ``Dio sia con te'' e uno
stregone non l\textquotesingle avrebbe fatto. Il novizio liberò la
pietra dalle macerie e la capovolse. Mentre lo faceva, il mucchio di
pietre rombò debolmente, dall\textquotesingle interno: una minuscola
pietra scese rumoreggiando lungo la china. Francis si scostò, temendo
una valanga, ma non successe altro. Tuttavia, nel punto in cui era stata
confitta la pietra del pellegrino adesso c\textquotesingle era un
piccolo buco nero.

I buchi sono spesso abitati.

Ma questo buco pareva essere stato tappato così saldamente dalla pietra
del pellegrino che difficilmente una pulce avrebbe potuto entrarvi,
prima che Francis lo scoperchiasse. Tuttavia prese uno stecco e lo
spinse, imbarazzato, nell\textquotesingle apertura. Il fuscello non
incontrò resistenza. Quando lo lasciò andare, scivolò nel buco e svanì,
come se sotto vi fosse una cavità più grande. Attese, innervosito: non
ne uscì nulla.

Si mise di nuovo in ginocchio e fiutò cautamente il buco. Poiché non
aveva sentito odore di animali o di zolfo, vi fece rotolare dentro un
sassolino e si piegò più vicino, per ascoltare. Il sassolino rimbalzò
una volta, qualche metro più sotto l\textquotesingle apertura, poi
continuò a rotolare verso il basso, colpì nel passare qualcosa di
metallico e finalmente si fermò in un punto imprecisabile, molto in
basso. Gli echi facevano pensare a una cavità sotterranea grande quanto
una stanza.

Frate Francis si rimise in piedi, faticosamente, e si guardò intorno.
Era solo, come al solito, a eccezione della sua amica poiana che,
incrociando in alto, lo aveva sorvegliato in quegli ultimi tempi con
tanto interesse che talvolta altre poiane avevano lasciato i loro
territori vicino all\textquotesingle orizzonte ed erano venute a
indagare.

Il novizio girò attorno al mucchio di macerie, ma non trovò alcuna
traccia di un secondo buco. Salì su un mucchio vicino e guardò verso la
pista, strizzando le palpebre.

Il pellegrino era scomparso ormai da molto tempo. Nulla si muoveva lungo
l\textquotesingle antica strada, ma intravide frate Alfred che
attraversava una collinetta, un paio di chilometri più a est, in cerca
di legna da ardere, vicino al suo eremitaggio quaresimale. Frate Alfred
era sordo come una campana. Non c\textquotesingle era nessun altro, in
vista. Francis non prevedeva di avere qualche motivo per invocare aiuto,
ma calcolare in anticipo i probabili risultati di una simile
invocazione, se mai si fosse resa necessaria, sembrava soltanto un
esercizio di prudenza. Dopo una attenta osservazione del terreno
circostante, scese dal monticello. Il fiato necessario per gridare
sarebbe stato meglio utilizzato per correre.

Pensò di rimettere la pietra del pellegrino al suo posto, per chiudere
il buco come prima, ma le pietre adiacenti si erano spostate
leggermente, così che quella, ora, non si riadattava più al suo posto
nel rompicapo. Inoltre, la lacuna nella fila superiore del suo rifugio
era ancora vuota, e il pellegrino aveva ragione: la forma e le
dimensioni di quella pietra lasciavano credere che sarebbe andata bene.
Dopo una breve apprensione, sollevò la pietra e tornò vacillando verso
il suo rifugio. La pietra si adattò perfettamente al foro. Provò il
nuovo cuneo con un calcio: la fila di pietre resse bene, anche se il
colpo provocò un piccolo crollo un po\textquotesingle{} più in là. I
segni tracciati dal pellegrino, sebbene un po\textquotesingle{} confusi
dal suo continuo maneggiare, erano ancora abbastanza chiari da poter
essere ricopiati. Frate Francis li tracciò attentamente su
un\textquotesingle altra pietra, usando come stilo uno stecco
carbonizzato. Quando il priore Cheroki avrebbe fatto il suo solito giro
del sabato per visitare gli eremitaggi, forse avrebbe potuto dire se
quei segni avevano un significato, come incantesimo o come maledizione,
forse. Era proibito temere le cabale pagane, ma il novizio era curioso
di sapere quale segno avrebbe coronato il suo rifugio, in considerazione
del peso dell\textquotesingle edificio su cui quel segno era tracciato.

Le sue fatiche continuarono durante l\textquotesingle afa del
pomeriggio. Un cantuccio della sua mente continuava a ricordargli del
buco --- il piccolo buco interessante eppure spaventevole --- e il modo
in cui il sassolino, rotolando, aveva destato deboli echi dalle
profondità sotterranee. Sapeva che le rovine che lo circondavano erano
molto antiche. Sapeva anche che, secondo la tradizione, quelle rovine
erano state gradualmente logorate, fino a ridursi ad anomali mucchi di
pietre, da generazioni di monaci e di visitatori occasionali, uomini che
cercavano un carico di pietre o i frammenti di acciaio arrugginito che
si potevano trovare fracassando le sezioni di colonne e le lastre più
grandi per estrarre le vecchie strisce di quel metallo, misteriosamente
piantato nella pietra da uomini d\textquotesingle una età quasi
dimenticata dal mondo. Questa erosione umana aveva quasi completamente
cancellato la somiglianza con gli edifici che la tradizione ascriveva
alle rovine in un periodo anteriore, benché l\textquotesingle attuale
mastro costruttore dell\textquotesingle abbazia fosse ancora orgoglioso
della sua abilità nel riconoscere e nell\textquotesingle indicare qua e
là i resti di un piano terreno. E c\textquotesingle era ancora metallo
da recuperare, se qualcuno ci teneva a spezzare una sufficiente quantità
di pietra per trovarlo. La stessa abbazia era stata costruita con quelle
pietre. Francis considerava molto improbabile che parecchi secoli di
costruzioni avessero lasciato ancora qualcosa di interessante da
scoprire fra le rovine. Eppure, non aveva mai sentito parlare di edifici
con cantine o stanze sotterranee. Il mastro costruttore, ricordò
finalmente, era stato molto chiaro nel precisare che gli edifici, in
quel luogo, dovevano essere stati costruiti in economia, senza
fondamenta profonde, e per la maggior parte posati su lastre di cemento.

Ora che il suo rifugio si avvicinava al completamento, frate Francis
tornò ad avventurarsi fino alla buca e rimase ritto a guardarla; era
incapace di accantonare la convinzione, tipica di un abitante del
deserto, che dovunque esista un posto per ripararsi dal sole,
c\textquotesingle è già dentro qualcosa che vi si ripara. Anche se la
buca era disabitata, adesso, senza dubbio qualcosa vi sarebbe rientrata,
scivolando, prima dell\textquotesingle alba seguente.
D\textquotesingle altra parte, se c\textquotesingle era già qualcosa che
viveva in quel buco, Francis pensava che sarebbe stato meno rischioso
farne la conoscenza di giorno, piuttosto che di notte. Non
c\textquotesingle erano altre orme, lì vicino, eccetto le sue, quelle
del pellegrino e quelle dei lupi.

Con improvvisa decisione, cominciò a togliere detriti e sabbia
dall\textquotesingle imboccatura del buco. Dopo mezz\textquotesingle ora
di lavoro, il buco non era più largo, ma la convinzione che si aprisse
su di una fossa sotterranea era diventata una certezza. Due piccoli
massi, semisepolti e vicini all\textquotesingle apertura,. erano
ovviamente incastrati dalla forza d\textquotesingle una massa eccessiva
che stringeva la bocca del pozzo: sembravano imprigionati in un collo di
bottiglia. Quando il novizio spinse una pietra verso destra, la sua
vicina rotolò a sinistra, fino a che non fu più possibile alcun
movimento. L\textquotesingle effetto contrario si verificò quando spinse
nella direzione opposta, ma continuò a spingere.

La pietra gli schizzò improvvisamente dalle mani, colpendolo di striscio
su un lato della testa, e sparì in una cavità. Il colpo lo fece
ondeggiare. Una pietra staccatasi dal pendio lo colpì sul dorso; cadde
cercando di aggrapparsi a qualcosa, senza capire se stava cadendo nel
buco o no, fino al momento in cui il suo ventre urtò contro il terreno
solido. Il rombo provocato dalla valanga di pietre fu assordante ma
breve.

Accecato dalla polvere, Francis giacque boccheggiando e chiedendosi se
doveva azzardarsi a muoversi, tanto era acuto il dolore che provava al
dorso. Dopo aver ripreso un po\textquotesingle{} di fiato, riuscì a
infilare una mano dentro l\textquotesingle abito e cercò a tentoni il
punto fra le spalle in cui dovevano esservi alcune ossa rotte. Il punto
pareva sbucciato, e pungeva. Quando ritrasse le dita, erano umide e
rosse. Si mosse, ma gemette e giacque di nuovo immobile.

Vi fu un lieve sbattere di ali. Frate Francis levò lo sguardo in tempo
per vedere la poiana che si accingeva a posarsi su un mucchio di
macerie, a pochi metri di distanza. Improvvisamente
l\textquotesingle uccello riprese quota, ma Francis ebbe
l\textquotesingle impressione che l\textquotesingle avesse guardato con
una specie di preoccupazione materna, come una gallina spaventata.
Rotolò in fretta su se stesso. Un intero stormo nero di rapaci si era
raccolto, e veleggiava in cerchio, a una quota curiosamente bassa.
Sfioravano i monticelli di macerie. Quando Francis si mosse, si levarono
a quota più alta. Dimenticando improvvisamente la possibilità di avere
qualche vertebra incrinata o una costola rotta, il novizio si rimise in
piedi, tremando. Delusa, la nera orda celeste risalì ad alta quota
sfruttando le invisibili correnti ascensionali d\textquotesingle aria
calda, poi si sciolse e si disperse verso più remote veglie aeree.
Oscure alternative al Paracleto di cui attendeva la discesa, i rapaci
sembravano talvolta ansiosi di scendere al posto della Colomba; il loro
sporadico interesse in quegli ultimi tempi era stato snervante, e
Francis decise prontamente, dopo qualche sperimentale scrollata di
spalle, che la pietra aguzza non aveva provocato altro che lividi e
abrasioni.

Una colonna di polvere che si era levata dal fianco della cavità si
stava disperdendo nella brezza. Sperò che qualcuno la vedesse dalle
torri di guardia dell\textquotesingle abbazia e venisse a indagare. Ai
suoi piedi, un buco di terra quadrata si apriva nella terra, proprio
dove un fianco del monticello era crollato nella fossa sottostante. Una
rampa di scale conduceva in basso, ma solo i primi gradini non erano
sepolti dalla valanga che si era fermata a mezza strada per sei secoli,
aspettando l\textquotesingle aiuto di frate Francis prima di completare
la sua ruggente discesa.

Su una parete una insegna semisepolta era ancora leggibile. Mettendo a
frutto la sua modesta conoscenza dell\textquotesingle inglese
prediluviale, sussurrò esitando le parole:
\begin{center}
	{\Large RIFUGIO SOPRAVVIVENZA FALLOUT }
\end{center}

\begin{center}
	{\large POSTI: 15}
\end{center}

\begin{center}
	\justify{Provviste per 180 giorni, per un solo occupante: dividere per il numero
		effettivo degli occupanti. Entrando nel rifugio, controllare che il
		Primo Portello sia ben chiuso e sigillato, che gli schermi anti-intrusi
		siano elettrificati per impedire l\textquotesingle accesso a persone
		contaminate che tentassero di entrare, che le luci di avvertimento siano
		ACCESE all\textquotesingle esterno della chiusura\ldots{}}
\end{center}
\leavevmode\\
Il resto era sepolto; ma le prime parole bastarono a Francis. Non aveva
mai visto un fallout e sperava di non doverlo vedere mai. Non era
rimasta alcuna consistente descrizione del mostro, ma Francis aveva
udito le leggende. Si fece il segno della croce e si allontanò dal
pertugio. La tradizione affermava che lo stesso beato Leibowitz si era
imbattuto in un fallout, e ne era stato posseduto per molti mesi, prima
che l\textquotesingle esorcismo che aveva accompagnato il suo battesimo
scacciasse il maligno.

Frate Francis immaginava un Fallout come un essere metà salamandra ---
perché secondo la tradizione, la Cosa era nata nel Diluvio di Fiamma ---
e metà incubo che contaminava le vergini nel sonno, perché i mostri del
mondo non erano forse tuttora chiamati ``figli del Fallout''? Che il
demone fosse capace di infliggere tutti i tormenti abbattutisi su Giobbe
era un fatto documentato, quasi un articolo di fede.

Il novizio fissò sbigottito la scritta. Il suo significato era
abbastanza chiaro. Aveva involontariamente fatto irruzione nel rifugio
(deserto, pregò) non soltanto di uno, ma di quindici di quegli esseri
temibili! Afferrò la fiala dell\textquotesingle acqua santa.

\newpage