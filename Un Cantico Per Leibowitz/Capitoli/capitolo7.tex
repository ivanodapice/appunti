	\chapter{\phantom{title}}

\lettrine{F}{rate} Francis trascorse sette anni di noviziato, sette vigilie
quaresimali nel deserto, e diventò abilissimo
nell\textquotesingle imitare i richiami dei lupi. Per divertire i
confratelli, chiamava l\textquotesingle intero branco nelle vicinanze
dell\textquotesingle abbazia ululando dall\textquotesingle alto delle
mura, quando era scesa la notte. Di giorno, serviva in cucina, fregava i
pavimenti di pietra e continuava il suo studio
dell\textquotesingle antichità.

Poi un giorno arrivò all\textquotesingle abbazia, cavalcando un asino,
un messaggero proveniente da un seminario di Nuova Roma. Dopo un lungo
colloquio con l\textquotesingle abate, il messaggero andò a cercare
frate Francis. Sembrò sorpreso nel vedere che il giovane, ormai
diventato uomo, indossava ancora l\textquotesingle abito di novizio e
puliva il pavimento della cucina.

--- Abbiamo studiato i documenti che tu hai scoperto alcuni anni or sono
--- disse al novizio. Alcuni di noi sono convinti della loro
autenticità.

Francis abbassò il capo. --- Non ho il permesso di trattare questo
argomento, padre --- disse.

--- Oh, già. --- Il messaggero sorrise e gli porse un pezzo di carta che
recava il sigillo dell\textquotesingle abate e lo scritto, di mano del
superiore: \emph{``Ecce Inquisitor Curiae. Ausculta et obsequere. Arkos
	AOL, Abbas''}.

Va tutto bene --- aggiunse, notando l\textquotesingle improvvisa
tensione del novizio. --- Non ti sto parlando ufficialmente: Qualche
altro incaricato del tribunale riceverà più tardi le tue dichiarazioni.
Tu sai, non è vero, che i tuoi documenti sono a Nuova Roma da qualche
tempo? Io ne ho riportato qualcuno.

Frate Francis scosse il capo. Forse ne sapeva meno di chiunque altro,
per ciò che riguardava le reazioni ad alto livello causate dalla sua
scoperta delle reliquie. Notò che il messaggero portava la veste bianca
dei Domenicani, e si chiese, un po\textquotesingle{} inquieto, quale
fosse la natura del ``tribunale'' di cui aveva parlato il frate. Nella
regione della Costa del Pacifico era in atto una inquisizione contro il
movimento dei Catari, ma non riusciva a immaginare in che modo
\emph{quel} tribunale potesse occuparsi delle reliquie del Beato.
\emph{Ecce Inquisitor Curiae}, diceva il biglietto. Probabilmente
l\textquotesingle abate intendeva ``investigatore''. Il Domenicano
pareva un uomo di animo mite, e non portava con sé alcun visibile
strumento di tortura.

Prevediamo che la causa per la canonizzazione del vostro fondatore sarà
presto riaperta --- spiegò il messaggero. --- L\textquotesingle abate
Arkos è un uomo molto saggio e prudente. E ridacchiò. --- Consegnando le
reliquie a un altro Ordine perché venissero esaminate, e facendo
chiudere il rifugio prima che fosse completamente esplorato\ldots{}
Bene, tu capisci, non è vero?

--- No, padre. Avevo creduto che considerasse l\textquotesingle intera
faccenda troppo trascurabile per sprecarvi altro tempo.

Il Frate Nero rise. --- Trascurabile! Credo di no. Ma se il tuo Ordine
produce prove, reliquie, miracoli o altre cose, il tribunale deve
considerarne la fonte. Ogni comunità religiosa è ansiosa di vedere
canonizzato il proprio fondatore. Quindi il vostro abate vi ha detto,
saggiamente: ``Giù le mani dal rifugio''. Sono sicuro che è stata una
delusione per tutti voi, ma\ldots{} è stato meglio per la causa del
vostro fondatore lasciare che il rifugio venisse esplorato alla presenza
di altri testimoni.

--- Lo riaprirete? --- chiese ansioso Francis.

--- No, non io. Ma quando il tribunale sarà pronto, manderà i suoi
osservatori. Così, tutto ciò che verrà trovato nel rifugio e che potrà
avere influenza sulla causa sarà sicuro, caso mai
l\textquotesingle opposizione ne contestasse
l\textquotesingle autenticità. Naturalmente, l\textquotesingle unica
ragione per sospettare che il contenuto del rifugio possa avere
influenza sulla causa è\ldots{} Bene, sono le cose che tu hai trovato.

--- Posso chiedere perché, padre?

Ecco, uno dei motivi d\textquotesingle imbarazzo,
all\textquotesingle epoca della beatificazione, fu la vita precedente
del beato Leibowitz\ldots{} prima che diventasse un prete.
L\textquotesingle avvocato della parte avversa continuò a cercare di
gettare un\textquotesingle ombra di dubbio sul periodo prediluviale.
Cercava di dimostrare che Leibowitz non fece mai una ricerca
scrupolosa\ldots{} che sua moglie poteva essere ancora viva al tempo
della sua ordinazione. Bene, non sarebbe la prima volta, naturalmente:
qualche volta sono state concesse dispense\ldots{} ma questo non
c\textquotesingle entra. L\textquotesingle{}\emph{advocatus diaboli}
stava cercando di gettare qualche dubbio sulla figura del vostro
fondatore. Cercava di suggerire che aveva accettato i Sacri Ordini e
aveva preso i voti prima di essere certo che le sue responsabilità
familiari erano finite. L\textquotesingle opposizione fu battuta, ma
potrebbe ritentare. E se quei resti umani che hai trovato sono
veramente\ldots{} --- Scrollò le spalle e sorrise.

Francis annuì. --- Questo determinerebbe con precisione la data della
morte della moglie.

--- Esattamente all\textquotesingle inizio della guerra che quasi pose
fine a tutto. E secondo me\ldots{} ecco, la scrittura nella cassetta, o
è di mano del Beato oppure si tratta di un\textquotesingle abilissima
contraffazione.

Francis arrossì.

--- Non sto affatto insinuando che tu sia implicato in una
contraffazione --- aggiunse in fretta il Domenicano, notando quel
rossore.

Il novizio, tuttavia, si era limitato a ricordare ciò che aveva pensato
di quegli scarabocchi.

--- Dimmi, come è accaduto?\ldots{} Come hai individuato quel luogo,
voglio dire. Ho bisogno di un resoconto completo.

--- Ecco, cominciò a causa dei lupi.

Il Domenicano si accinse a prendere appunti.

Qualche giorno dopo la partenza del messaggero, l\textquotesingle abate
Arkos mandò a chiamare frate Francis. --- Pensi ancora che la tua
vocazione sia con noi? --- chiese cordialmente Arkos.

--- Se Monsignor Abate vuole perdonare la mia esecrabile vanità\ldots{}

--- Oh, dimentichiamo la tua esecrabile vanità per un momento. Lo pensi
o non lo pensi?

--- Sì, \emph{Magister meus}.

L\textquotesingle abate si illuminò. --- Bene, figlio mio. Credo che
anche noi ne siamo convinti. Se sei pronto a prendere una decisione per
sempre, credo che sia venuto per te il tempo di professare i voti
solenni. --- Si interruppe per un attimo e, osservando la faccia del
novizio, sembrò deluso di non scorgervi alcun cambiamento di
espressione. --- Che c\textquotesingle è? Non sei lieto di sentirlo? Non
sei\ldots? Oh! Che succede?

Sebbene il viso di Francis fosse rimasto una maschera educatamente
attenta, quella maschera perdette gradualmente il suo colore. Le
ginocchia del novizio si piegarono all\textquotesingle improvviso.

Francis era svenuto.

Due settimane più tardi, il novizio Francis, dopo aver forse stabilito
un primato di sopravvivenza nelle vigilie nel deserto, lasciò i ranghi
del noviziato e, votando perpetua povertà, castità, obbedienza --- oltre
alle speciali promesse caratteristiche di quella comunità --- ricevette
le benedizioni e la bisaccia nell\textquotesingle abbazia, e diventò per
sempre monaco professo dell\textquotesingle Ordine Albertiano di
Leibowitz, legato da catene che egli stesso aveva forgiato ai piedi
della Croce e alla regola dell\textquotesingle Ordine. Per tre volte
secondo il rito, gli fu chiesto: --- Se Dio ti chiama a essere Suo
Contrabbandiere di Libri, preferirai morire piuttosto che tradire i tuoi
fratelli? E per tre volte Francis rispose: --- Sì, signore.

--- E allora alzati, Fratello Contrabbandiere di Libri e Fratello
Memorizzatore e ricevi il bacio della fratellanza. \emph{Ecce quam
	bonum, et quam jucundum\ldots{}}

Frate Francis fu trasferito dalla cucina e fu assegnato a un lavoro meno
umile. Divenne apprendista copista di un anziano monaco che si chiamava
Horner, e, se le cose fossero andate bene, avrebbe potuto con ragione
pensare a tutta una vita da trascorrere nella copisteria, dove avrebbe
dedicato il resto dei suoi giorni a copiare a mano testi di algebra e a
decorarne le pagine con fronde d\textquotesingle olivo e ridenti
cherubini che circondavano le tavole dei logaritmi.

Frate Horner era un vecchio gentile, e frate Francis gli si affezionò
subito.

--- Molti di noi lavorano meglio sulla copia assegnata gli disse Horner
--- se hanno anche un progetto personale. Molti copisti si interessano a
qualche particolare lavoro dei Memorabilia e amano spendere su di esso
un po\textquotesingle{} di tempo. Per esempio, frate Sarl,
laggiù\ldots{} il suo lavoro non procedeva bene, e faceva degli errori.
Cosi gli permettemmo di dedicare un\textquotesingle ora al giorno a un
progetto che si era scelto. Quando il lavoro lo annoia tanto che
comincia a commettere errori di copiatura, può metterlo da parte per un
po\textquotesingle{} e lavorare sul suo progetto. Io permetto a tutti di
fare lo stesso. Se finisci il lavoro assegnato prima che la giornata sia
terminata e non hai un progetto tuo, dovrai dedicare il tempo che ti
resta ai nostri perenni\ldots{}

--- Perenni?

--- Sì, e non intendo le piante perenni. C\textquotesingle è richiesta
di perenni da parte di tutto il clero, per vari libri\ldots{} Messali,
Sacra Scrittura, Breviari, la \emph{Summa}, enciclopedie e così via. Ne
vendiamo moltissimi. Così, se non hai un progetto personale, ti
assegneremo ai perenni, quando finirai presto. Hai tutto il tempo per
decidere. --- Il vecchio supervisore fece una pausa. --- Dubito che tu
lo capisca. Io no. Sembra che abbia trovato un metodo per ricostruire le
parole e le frasi mancanti in alcuni dei vecchi frammenti di testi
originali dei Memorabilia. Per esempio, la parte sinistra
d\textquotesingle un libro semi bruciato è leggibile, ma
l\textquotesingle orlo destro della pagina è bruciato, e in fondo a ogni
riga manca qualche parola. Ha escogitato un metodo matematico per
trovare le parole che mancano. Non è un metodo sicurissimo, ma funziona
discretamente. È riuscito a restaurare quattro pagine intere da quando
ha cominciato il tentativo.

Francis guardò frate Sarl, che era un ottuagenario quasi cieco. ---
Quanto tempo ha impiegato? --- chiese l\textquotesingle apprendista.

--- Quasi quarant\textquotesingle anni --- disse frate Horner. ---
Naturalmente vi ha dedicato soltanto cinque ore alla settimana, e il
metodo richiede considerevoli calcoli aritmetici.

Francis annui, pensieroso. --- Se potesse essere restaurata una pagina
ogni decennio, forse in pochi secoli\ldots{}

--- Anche meno --- gracchiò frate Sarl senza alzare lo sguardo dal suo
lavoro. --- Più si procede, più semplice diventa il resto. Finirò la
prossima pagina in un paio d\textquotesingle anni. Poi, a Dio piacendo,
forse\ldots{} --- La sua voce si smorzò in un mormorio. Francis notò che
di frequente frate Sarl parlava fra sé, mentre lavorava.

--- Accomodati --- disse frate Horner. --- Possiamo sempre utilizzare la
tua collaborazione per i perenni, ma quando vorrai potrai dedicarti a un
progetto tuo.

L\textquotesingle idea venne a frate Francis in un lampo inatteso. ---
Posso dedicare il tempo che mi avanza --- balbettò, --- per fare una
copia della \emph{blueprint} di Leibowitz che ho trovato?

Frate Horner si mostrò sbalordito, per un attimo. --- Ecco\ldots{} non
saprei, figliolo. Il nostro Signor Abate è\ldots{} ecco, un
po\textquotesingle{} sensibile a questo argomento. E può darsi che
quell\textquotesingle oggetto non appartenga ai Memorabilia. Per il
momento è nello scaffale dei sospesi.

--- Ma voi sapete che sbiadiscono, fratello. E quella è stata maneggiata
alla luce. I Domenicani l\textquotesingle hanno tenuta a Nuova Roma per
tanto tempo\ldots{}

--- Ecco\ldots{} immagino che sarebbe un lavoro piuttosto breve. Se
padre Arkos non ha obiezioni, ma\ldots{} --- E scrollò la testa,
dubbioso.

--- Forse potrei includerla in un mazzo --- si offrì frettoloso Francis.
--- Le poche \emph{blueprint} ricopiate che abbiamo sono così antiche da
essere fragili\ldots{} Se facessi parecchi duplicati\ldots{} di alcune
delle altre\ldots{}

Horner sorrise maliziosamente. --- Intendi dire che, se includessi nel
mazzo la \emph{blueprint} di Leibowitz, nessuno se ne accorgerebbe.

Francis arrossì.

--- Padre Arkos non lo noterebbe neppure, eh?\ldots{} se per caso vi
frugasse.

Francis si agitò.

--- Benissimo --- disse Horner, mentre gli occhi gli scintillavano
lievemente. --- Puoi usare il tuo tempo libero per fare duplicati di
qualsiasi disegno ricopiato che sia in cattive condizioni. Se per caso
nel mucchio ci finisce anche qualcosa d\textquotesingle altro, cercherò
di non notarlo.

Frate Francis dedicò per parecchi mesi il suo tempo libero a ricopiare
alcuni dei vecchi disegni tratti dagli scaffali dei Memorabilia prima di
osare toccare il disegno di Leibowitz. Se valeva la pena salvare i
vecchi disegni, essi dovevano venir comunque ricopiati ogni secolo o
due. Non solo gli originali sbiadivano, ma spesso anche le copie
diventavano quasi illeggibili dopo un certo tempo, a causa della
instabilità degli inchiostri impiegati. Non riusciva a comprendere
perché gli antichi avessero tracciato linee e lettere bianche su sfondo
scuro, invece del contrario. Quando ricopiò a carboncino uno dei
disegni, invertendo così il rapporto dei colori, il rosso schizzo sembrò
molto più realistico che in bianco-su-nero; ma gli antichi erano
immensamente più saggi di Francis: se si erano presi il disturbo di
mettere l\textquotesingle inchiostro dove di solito
c\textquotesingle era la carta bianca e lasciavano solo lievi strisce
bianche là dove una linea inchiostrata sarebbe dovuta apparire in un
disegno normale, dovevano avere le loro ragioni. Francis ricopiò i
documenti in modo che sembrassero simili il più possibile agli
originali\ldots{} anche se il compito di stendere
l\textquotesingle inchiostro azzurro attorno alle minuscole lettere
bianche era particolarmente noioso, e richiedeva un grande spreco di
inchiostro, un fatto che faceva brontolare frate Horner.

Copiò un antico progetto architettonico, poi un disegno per una parte di
una macchina, la cui geometria era evidente ma il cui uso era vago.
Ricopiò una bizzarra astrazione, intitolata STABILIZZATORE WNDG MOD.
73-A 3-PH 6-P 1800 RPM 5-HP CL-A GABBIA DA SCOIATTOLI, che si rivelò
completamente incomprensibile, e assolutamente incapace di imprigionare
uno scoiattolo. Gli antichi erano spesso molto sottili; forse era
necessaria una speciale serie di specchi per vedere lo scoiattolo.
Francis, comunque, lo ricopiò faticosamente.

Soltanto dopo che l\textquotesingle abate, il quale ogni tanto passava
per la copisteria, lo ebbe visto al lavoro su un altro disegno almeno
tre volte (e per due volte Arkos si era fermato per dare una rapida
occhiata al lavoro di Francis) riuscì a trovare il coraggio di
avventurarsi fino agli scaffali dei Memorabilia per prendere la
\emph{blueprint} di Leibowitz, quasi un anno dopo aver cominciato il
progetto cui dedicava il tempo libero.

Il documento originale era già stato sottoposto a un certo lavoro di
restauro. A eccezione del fatto che portava il nome del Beato, era
deludentemente simile a quasi tutti gli altri che aveva ricopiato.

Il disegno di Leibowitz, un\textquotesingle altra astrazione, non faceva
riferimento a nulla, né in particolare alla ragione. Lo studiò fino a
che poté vederne a occhi chiusi la sbalorditiva complessità, ma non ne
sapeva di più di quanto ne avesse saputo prima. Non pareva altro che una
rete di linee che collegava un tracciato di segni tortuosi, di sgorbi,
di segni incomprensibili e di minuscole lamelle. Le linee erano quasi
tutte orizzontali e verticali, e si incrociavano tra loro o con un
piccolo segno che indicava un salto o con un punto; facevano svolte ad
angolo retto per girare attorno ai segni più grandi, e non si fermavano
mai a metà strada ma terminavano sempre con uno sgorbio, un segno, una
macchia incomprensibile. Era così assurdo che osservarlo per un periodo
piuttosto lungo produceva un effetto ipnotico. Tuttavia, Francis
cominciò a riprodurre ogni particolare, ricopiando persino una macchia
centrale bruniccia che pensava potesse essere sangue del Beato Martire,
ma che secondo frate Jeris era soltanto una macchia lasciata da un
torsolo di mela marcio.

Frate Jeris, che era diventato copista avventizio insieme a frate
Francis, sembrava divertirsi a punzecchiarlo, per quanto riguardava il
suo progetto.

--- Cos\textquotesingle è, prego --- chiedeva, sbirciando al di sopra
della spalla di Francis --- un ``Sistema di Controllo Transistorizzato
per l\textquotesingle Unità Sei-B'', dotto fratello?

--- È evidente: il titolo del documento --- disse Francis sentendosi un
po\textquotesingle{} urtato.

--- È evidente. Ma che cosa significa?

--- È il \emph{nome} del diagramma che ti sta davanti agli occhi,
Fratello Semplicione. Cosa significa ``Jeris''?

--- Molto poco, ne sono sicuro --- disse frate Jeris con ironica umiltà.
--- Perdona la mia durezza di comprendonio, ti prego. Tu hai definito
benissimo il nome indicando la creatura che lo porta, e che in verità è
il significato del nome. Ma, ora, la creatura-diagramma in se stessa
rappresenta qualcosa, non è vero? Cosa rappresenta il diagramma?

--- Il sistema di controllo transistorizzato dell\textquotesingle Unità
Sei-B.

Jeris rise. --- Chiarissimo! Eloquente! Se la creatura è il nome, allora
il nome è la creatura. ``Gli eguali possono essere sostituiti da
eguali'', ovvero ``L\textquotesingle ordine di una equazione è
reversibile'', ma possiamo passare all\textquotesingle assioma seguente:
allora non c\textquotesingle è qualche ``stessa quantità'' rappresentata
tanto dal nome quanto dal diagramma? Oppure è un sistema chiuso?

Francis arrossì. --- Penso --- disse lentamente, dopo aver fatto una
pausa per reprimere la sua irritazione --- che il diagramma rappresenti
un concetto astratto, piuttosto che una \emph{cosa} concreta. Forse gli
antichi avevano un metodo sistematico per dipingere un pensiero puro. È
chiaro che non è una immagine riconoscibile d\textquotesingle un
oggetto.

--- Si, è chiaramente \emph{irriconoscibile}! --- ammise frate Jeris con
un risolino.

--- D\textquotesingle altronde, forse è l\textquotesingle immagine di un
oggetto, ma soltanto in un modo stilistico molto formale\ldots{} così
che sarebbe necessaria una speciale preparazione o\ldots{}

--- Una vista speciale?

--- Secondo la mia opinione, è un\textquotesingle altissima astrazione
di valore forse trascendentale che esprime un pensiero del beato
Leibowitz.

--- Bravo! E allora, a cosa stava pensando?

--- Ecco\ldots{} al Disegno del Circuito --- disse Francis, scegliendo
quella definizione dalle scritte nell\textquotesingle angolo inferiore
destro.

\emph{--- Uhmmmm}, a che disciplina appartiene questa arte, fratello?
Quali sono i suoi genere, specie, proprietà e differenza? O forse è
soltanto un ``accidente''?

Jeris stava diventando pretenzioso nel suo sarcasmo, pensò Francis, e
decise di rispondere sommessamente. --- Bene, osserva questa colonna di
numeri, e il suo titolo: ``Numerazione delle Parti Elettroniche''.
C\textquotesingle era un tempo un\textquotesingle arte o una scienza
chiamata Elettronica, che poteva appartenere tanto
all\textquotesingle Arte quanto alla Scienza.

--- Uh-uh! Questo regola il problema del ``genere'' e della ``specie''.
E in quanto alla ``differenza'', se posso continuare su questa linea,
qual era l\textquotesingle argomento dell\textquotesingle Elettronica?

Anche questo è scritto --- disse Francis, che aveva frugato i
Memorabilia da cima a fondo nel tentativo di trovare qualche indizio che
potesse rendere la \emph{blueprint} un po\textquotesingle{} più
comprensibile\ldots{} ma con scarso risultato. ---
L\textquotesingle argomento dell\textquotesingle Elettronica era
l\textquotesingle elettrone --- spiegò.

--- Così è scritto; in verità. Ne sono impressionato. So così poco di
queste cose. Cos\textquotesingle era, prego,
l\textquotesingle elettrone?

--- Ecco, c\textquotesingle è una fonte frammentaria che allude a esso
come a una ``Torsione Negativa del Nulla''.

--- Come! Come potevano negare un nulla? Questo non
l\textquotesingle avrebbe reso un qualche cosa?

--- Forse la negazione si applica a ``torsione''.

--- Ah! Allora noi avremmo un ``Nulla Non Distorto'', eh? Hai scoperto
come si fa a non distorcere un nulla?

--- Non ancora --- ammise Francis.

--- Attieniti a questo, fratello! Quanto devono essere stati abili gli
antichi\ldots{} sapevano in che modo non distorcere il nulla. Attieniti
a questo e potrai imparare come si fa. E allora avremo
\emph{l\textquotesingle elettrone} in mezzo a noi, no? E cosa ce ne
faremo? Lo metteremo sull\textquotesingle altare?

--- Bene, allora --- sospirò Francis. --- Non so. Ma sono sicuro che
\emph{l\textquotesingle elettrone} esistesse un tempo, anche se non so
come fosse costruito o per che cosa potesse venire usato.

--- Commovente! --- ridacchiò l\textquotesingle iconoclasta, e ritornò
al suo lavoro.

Le sporadiche punzecchiature di frate Jeris rattristarono Francis, ma
non diminuirono la sua devozione al progetto.

L\textquotesingle esatta duplicazione di ogni segno, macchia o chiazza
si rivelò impossibile, ma l\textquotesingle accuratezza del facsimile si
dimostrò sufficiente per ingannare l\textquotesingle occhio a due passi
di distanza, e di conseguenza adeguato perché la copia potesse venir
messa in mostra, e l\textquotesingle originale sigillato e riposto. Dopo
aver completato il facsimile, frate Francis scoprì di sentirsi deluso.
Il disegno era troppo spoglio. Non c\textquotesingle era nulla, in esso,
che suggerisse a prima vista che si trattava d\textquotesingle una sacra
reliquia. Lo stile era nitido e privo di pretese\ldots{} e questo si
addiceva, forse, al Beato, eppure\ldots{}

Una copia di quella reliquia non era sufficiente. I santi erano persone
umili che glorificavano non se stessi ma Dio, e toccava agli altri
ritrarre la gloria interiore della santità per mezzo di segni esteriori
e visibili. Quella copia così nuda non era abbastanza: era fredda e
priva di immaginazione, e non commemorava le qualità sante del Beato in
alcun modo visibile.

\emph{Glorificemus}, pensò Francis, mentre lavorava sui perenni. Stava
copiando alcune pagine dei Salmi, in quel momento, per rilegarle più
tardi. Si interruppe per ritrovare il segno nel testo, e per notare il
significato delle parole\ldots{} perché, dopo ore di copiatura, aveva
smesso di leggere, e si limitava a permettere alla sua mano di
ritracciare le lettere che i suoi occhi incontravano. Notò che stava
copiando la preghiera di David per invocare perdono, il quarto salmo
penitenziale. ``Miserere mei, Deus\ldots{} perché io conosco la mia
iniquità, e il mio peccato è sempre dinanzi a me''. Era una preghiera
umile, ma la pagina davanti ai suoi occhi non era scritta in modo
altrettanto umile. La \emph{M} di \emph{Miserere} era impressa in foglia
d\textquotesingle oro. Un fiorente arabesco di filamenti dorati e
purpurei intrecciati insieme riempiva i margini e formava nidi che
attorniavano le splendide maiuscole, all\textquotesingle inizio
d\textquotesingle ogni versetto. Per quanto la preghiera in se stessa
fosse umile, la pagina era magnifica. Frate Francis stava copiando
soltanto il corpo del testo, lasciando spazi liberi per le splendide
maiuscole e margini larghi quanto le linee del testo. Altri amanuensi
avrebbero riempito con orge di colori la sua copia scritta in semplice
inchiostro e avrebbero costruito le maiuscole pittoriche. Francis stava
imparando ad alluminare, ma non era ancora abbastanza abile da poter
miniare i perenni.

\emph{Glorificemus}. Stava pensando di nuovo alla \emph{blueprint}.

Senza rivelare a nessuno la sua idea, frate Francis cominciò a fare i
suoi piani. Trovò la più bella cartapecora disponibile e dedicò per
parecchie settimane tutto il suo tempo libero a curarla e a stenderla e
a pareggiarla a colpi di pietra fino a ottenerne una superficie
perfetta, che decolorò a una bianchezza nivea; poi la ripose con molta
cura. Quindi, per mesi interi, dedicò ogni minuto del suo tempo libero a
consultare i Memorabilia, cercando ancora qualche indicazione sul
significato del disegno di Leibowitz. Non trovò nulla che somigliasse
alle linee ramificate del disegno, né altre cose che
l\textquotesingle aiutassero a interpretarne il significato, ma dopo
molto tempo si imbatté in un frammento d\textquotesingle un libro che
conteneva una pagina, parzialmente distrutta, che si riferiva proprio
alla preparazione delle \emph{blueprints}. Sembrava un brano di una
enciclopedia. Il riferimento era breve e parte del brano mancava, ma
dopo averlo letto parecchie volte, cominciò a sospettare che lui stesso,
e molti altri copisti che l\textquotesingle avevano preceduto, avessero
sprecato tempo e inchiostro. L\textquotesingle effetto bianco-su-nero
non pareva essere stata una caratteristica particolarmente desiderabile,
ma risultante dalle peculiarità di un certo processo di riproduzione a
buon mercato. Il disegno originale dal quale era stata tratta una
\emph{blueprint} era nero su bianco. Francis dovette resistere
all\textquotesingle impulso improvviso di battere la testa sul
pavimento. Tutto quell\textquotesingle inchiostro e quella fatica per
copiare una riproduzione incidentale! Ecco, forse non era necessario
dirlo a frate Horner. Sarebbe stata opera di carità non parlargliene,
poiché frate Horner era malato di cuore.

La consapevolezza che lo schema dei colori d\textquotesingle una
\emph{blueprint} era una caratteristica incidentale di quegli antichi
disegni aggiunse nuovo impulso al suo piano. Una copia glorificata del
disegno di Leibowitz poteva essere realizzata senza incorporarvi la
caratteristica accidentale. Invertendo lo schema del colore, nessuno
avrebbe riconosciuto il disegno a prima vista. Certe altre
caratteristiche potevano essere ovviamente modificate. Non osò cambiare
nulla di ciò che non comprendeva, ma senza dubbio le tavole delle parti
e le informazioni in stampatello potevano essere sparse simmetricamente
tutto attorno al diagramma su rotoli e scudi. Poiché il significato del
diagramma in sé era oscuro, non osò alterarne minimamente la forma o la
disposizione; ma poiché la disposizione dei colori non era importante,
poteva farne una cosa bellissima. Pensò a insetti d\textquotesingle oro
per alcuni segni, ma altri sgorbi incomprensibili erano troppo
complicati per la lavorazione in oro, e una chiazza
d\textquotesingle oro a forma di cicca sarebbe stata una ostentazione. I
punti dovevano essere neri, ma questo significava che le linee dovevano
essere più nere ancora, per fare spiccare i punti. Mentre il disegno
asimmetrico doveva rimanere com\textquotesingle era, non riusciva a
capire perché il suo significato dovesse risultarne alterato se
l\textquotesingle avesse usato come sostegno per una vite rampicante, i
cui rami (che avrebbero attentamente evitato i punti) potevano essere
disegnati in modo da dare un\textquotesingle impressione di simmetria o
a rendere naturale l\textquotesingle asimmetria. Quando frate Horner
alluminava una M maiuscola, trasformandola in una meravigliosa giungla
di foglie, bacche, rami e forse addirittura in un serpente, pur
nondimeno rimaneva leggibile come una M. Frate Francis non vedeva una
ragione per supporre che lo stesso non potesse applicarsi al diagramma.

La forma generale, soprattutto, con un bordo a svolazzi poteva diventare
quella d\textquotesingle uno scudo, invece di rimanere lo spoglio
rettangolo che nell\textquotesingle originale racchiudeva il disegno.
Fece dozzine di schizzi preliminari. In cima alla pergamena vi sarebbe
stata una immagine della Trinità, e in fondo\ldots{} le armi
dell\textquotesingle Ordine Albertiano con l\textquotesingle immagine
del Beato.

Ma non c\textquotesingle era alcuna effigie fedele del Beato, per quanto
ne sapeva Francis. V\textquotesingle erano parecchie immagini di
fantasia, ma nessuna risaliva all\textquotesingle epoca della
Semplificazione. Non v\textquotesingle era ancora neppure una
rappresentazione convenzionale, sebbene la tradizione affermasse che
Leibowitz era stato alto e un po\textquotesingle{} curvo. Ma forse,
quando il rifugio. fosse stato riaperto\ldots{}

Il lavoro preliminare di frate Francis fu interrotto un pomeriggio dalla
improvvisa certezza che la presenza che incombeva dietro di lui e che
gettava un\textquotesingle ombra sul suo tavolo da copista era quella
di\ldots{} quella di\ldots{} "No! Ti prego! \emph{Beate Leibowitz, audi
	me!} Misericordia, o Signore! Fai che sia chiunque ma non\ldots»

--- Bene, cosa abbiamo qui? --- rombò l\textquotesingle abate, guardando
i disegni.

--- Un disegno, Monsignor Abate.

--- Me ne sono accorto. Ma che cos\textquotesingle è?

--- La \emph{blueprint} di Leibowitz.

--- Quella che hai trovato tu? Come? Non sembra più la stessa. Perché
quei cambiamenti?

--- Dovrà essere\ldots{}

--- Parla più forte!

--- UNA COPIA ALLUMINATA! --- strillò involontariamente frate Francis.

--- Oh.

L\textquotesingle abate Arkos scrollò le spalle e si allontanò.

Frate Horner, pochi secondi più tardi, mentre passava accanto allo
scrittoio dell\textquotesingle apprendista fu sorpreso di vedere che
Francis era svenuto.
