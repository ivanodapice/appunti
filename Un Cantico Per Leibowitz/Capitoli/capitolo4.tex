	\chapter{\phantom{title}}

\lettrine{A}{vete} fatto bene --- brontolò alla fine l\textquotesingle abate.
Aveva camminato lentamente avanti e indietro nel suo studio per circa
cinque minuti; la sua larga faccia da contadino aveva un serrato
cipiglio muscolare, mentre padre Cheroki se ne stava seduto nervosamente
sull\textquotesingle orlo della sedia. L\textquotesingle abate non aveva
pronunciato parola da quando Cheroki era entrato nella stanza, in
risposta al suo invito; Cheroki sussultò lievemente quando
l\textquotesingle abate Arkos brontolò finalmente quelle parole.

--- Avete fatto bene --- disse ancora l\textquotesingle abate,
fermandosi in mezzo alla stanza e guardando a occhi socchiusi il priore,
che finalmente cominciò a rilassarsi. Era quasi mezzanotte e Arkos era
stato sul punto di ritirarsi per dormire un paio d\textquotesingle ore
prima del Mattutino e delle Laudi. Ancora umido e spettinato dopo una
recente immersione nel barile che costituiva la sua vasca da bagno, a
Cheroki sembrava un orso mannaro solo parzialmente trasformato in uomo.
Indossava una veste di pelli di coyote, e l\textquotesingle unico segno
del suo ufficio era la croce pettorale che riposava sul suo petto tra il
pelo nero e lampeggiava, alla luce delle candele, ogni volta che
l\textquotesingle abate si voltava verso la scrivania. I capelli umidi
gli spiovevano sulla fronte; con la corta barbetta appuntita e le pelli
di coyote sembrava, in quel momento, non tanto un prete quanto un
comandante militare, pieno di repressa furia di battaglia dopo un
recente combattimento. Padre Cheroki, che veniva da una schiatta
baronale di Denver, aveva la tendenza a reagire formalmente alle facoltà
ufficiali dell\textquotesingle altro, a parlare con cortesia davanti al
simbolo del potere, senza permettersi di vedere l\textquotesingle uomo
che lo portava, seguendo in questo le usanze di corte in auge in molte
epoche. Così, padre Cheroki aveva sempre mantenuto rapporti formalmente
cordiali con l\textquotesingle anello e la croce pettorale, con
l\textquotesingle ufficio del suo abate, ma si permetteva di vedere il
meno possibile di Arkos in quanto uomo. Questo era piuttosto difficile
nelle circostanze attuali, poiché il Reverendo Padre Abate era uscito di
fresco dal bagno e zampettava nello studio a piedi nudi. A quanto
pareva, si era appena tagliato un callo, e aveva inciso troppo
profondamente: uno degli alluci sanguinava. Cheroki cercava di non
notarlo, ma si sentiva molto imbarazzato.

--- Sapete di che cosa sto parlando? --- grugni impaziente Arkos.

Cheroki esitò. --- Vi dispiacerebbe, Padre Abate, essere più
specifico\ldots. nel caso che sia connesso con qualcosa che io posso
avere udito soltanto in confessione?

--- Ah? Oh! Bene, sono veramente sconvolto! Voi avete udito la sua
confessione, l\textquotesingle avevo dimenticato. Bene, inducetelo a
raccontarvi tutto di nuovo, in modo che possiate parlare\ldots. sebbene,
lo sa il Cielo, ormai la voce si sia sparsa in tutta
l\textquotesingle abbazia. No, non andate subito da lui. Parlerò con
voi, e voi non rispondete se tocco un argomento. coperto dal segreto
della confessione. Avete visto quella roba?

L\textquotesingle abate Arkos fece un cenno in direzione della scrivania
su cui il contenuto della cassetta di frate Francis era stato rovesciato
per essere esaminato.

Cheroki annuì, lentamente. --- L\textquotesingle aveva lasciata cadere
vicino alla strada, quando è svenuto. Io l\textquotesingle ho aiutato a
raccogliere tutto, ma non l\textquotesingle ho guardata con molta
attenzione. --- Bene, sapete che cosa pretende che sia?

Padre Cheroki distolse lo sguardo e mostrò di non aver udito la domanda.

--- Sta bene, sta bene --- grugnì l\textquotesingle abate. --- Non
importa che cosa \emph{lui} sostiene che sia. Andate a guardare voi
stesso attentamente e decidete che cos\textquotesingle è, secondo
\emph{voi}.

Cheroki andò a curvarsi sulla scrivania ed esaminò con cura le carte,
una alla volta, mentre l\textquotesingle abate camminava avanti e
indietro e continuava a parlare, apparentemente al prete ma in realtà
quasi a se stesso.

È impossibile! Voi avete fatto bene a rimandarlo qui. prima che
scoprisse altra roba. Ma naturalmente questo non è il peggio. Il peggio
è il vecchio di cui va blaterando. È grave. Non c\textquotesingle è
niente che potrebbe danneggiare la causa più di un fiume di improbabili
``miracoli''. Qualche vera coincidenza, certamente! Si deve stabilire
che l\textquotesingle intercessione del Beato ha prodotto fatti
miracolosi\ldots{} prima che sia possibile la canonizzazione. Ma questo
può essere troppo! Pensate al Beato Chang, beatificato due secoli fa, e
mai canonizzato\ldots{} fino a ora. E perché? Il suo Ordine divenne
troppo impaziente, ecco perché. Ogni volta che qualcuno guariva da una
tosse, era un intervento miracoloso del Beato. Visioni in cantina,
evocazioni sul campanile: sembrava più una raccolta di storie di
fantasmi che un elenco dicasi miracolosi. Forse due o tre casi erano
veramente validi, ma quando c\textquotesingle è troppa paglia\ldots{}
ebbene?

Padre Cheroki alzò la testa. Le nocche delle sue mani erano divenute
bianche per la pressione esercitata sull\textquotesingle orlo della
scrivania, e il suo viso sembrava teso. Pareva non avesse ascoltato. ---
Scusatemi, Padre Abate.

--- Ebbene, la stessa cosa può capitare qui, ecco --- disse
l\textquotesingle abate, e ricominciò a camminare lentamente avanti e
indietro. --- L\textquotesingle anno scorso c\textquotesingle è stato
frate Noyon e il suo miracoloso cappio del carnefice. Ah! E
l\textquotesingle anno prima, frate Smirnov fu misteriosamente guarito
dalla gotta\ldots{} come? Toccando una probabile reliquia del nostro
beato Leibowitz, dicevano quei giovani zotici. E adesso Francis incontra
un pellegrino\ldots{} che indossa che cosa?\ldots{} indossa come
gonnellino la stessa tela di sacco con cui incappucciarono il beato
Leibowitz prima di impiccarlo. E cosa ha per cintura? Una corda. Che
corda? Ah, la stessa\ldots{} --- Si fermò, volgendosi a Cheroki. ---
Posso capire dalla vostra espressione sorpresa che questa non
l\textquotesingle avevate ancora saputa. No? Benissimo, non potete
dirlo. No, no, Francis non ha detto questo. Tutto quello che ha detto
è\ldots{} --- L\textquotesingle abate Arkos cercò di iniettare un lieve
tono di falsetto nella sua voce normalmente burbera. --- Tutto ciò che
ha detto frate Francis é: ``Ho incontrato un vecchietto, e ho pensato
che fosse un pellegrino diretto all\textquotesingle abbazia perché
andava da quella parte, e portava un vecchio sacco stretto attorno ai
fianchi da un pezzo di corda. Ha fatto un segno sulla pietra, e il segno
era così''.

Arkos tolse un pezzo di pergamena dalla tasca della veste di pelliccia e
lo tenne alto davanti al viso di Cheroki nella luce della candela. Poi
continuò, con poco successo, il tentativo di imitare frate Francis: ---
``Non sono riuscito a capire cosa significasse, voi lo sapete?''

Cheroki fissò i simboli e scosse il capo.

Non lo chiedevo a \emph{voi} --- brontolò Arkos con voce normale. --- È
quello che ha detto Francis. Non lo sapevo neanch\textquotesingle io.

E adesso lo sapete?

--- Adesso lo so. Qualcuno è andato a controllare. Questa è una
\emph{lamedh}, e quella è una \emph{sadhe}. Lettere ebraiche.

\emph{--- Sadhe lamedh?}

\emph{---} No. Da destra a sinistra, \emph{Lamedh sadhe}. Una
\emph{elle} e un suono tra la \emph{ti} e la \emph{esse}. Se vi fossero
segni di vocali, potrebbe essere ``loots'', ``lots'', ``lets'',
``latz'', ``litz''\ldots{} qualunque cosa di questo genere. Se vi fosse
qualche lettera in mezzo a queste due, potrebbe suonare come
Llll\ldots{} \emph{indovinate chi.}

--- Leibo\ldots{} Oh, no!

--- Oh, sì! Frate Francis non ci ha pensato, Ci ha pensato qualcun
altro. Frate Francis non ha pensato al cappuccio di tela di sacco e alla
corda del carnefice; ci ha pensato uno dei suoi confratelli.. Così, cosa
succede? Prima di notte, l\textquotesingle intero noviziato stava già
ronzando la dolce favoletta che Francis ha incontrato là fuori lo stesso
Beato, e il Beato ha accompagnato il nostro ragazzo fino al punto in cui
era questa roba e gli ha detto che avrebbe trovato la vocazione.

Un cipiglio di perplessità contrasse per un attimo il viso di Cheroki.
--- Frate Francis ha detto questo?

--- Noo! --- ruggì Arkos. --- Non avete ascoltato? Francis non ha detto
una cosa simile. Vorrei che l\textquotesingle avesse fatto, per la
miseria; allora l\textquotesingle avrei colto in fallo, il birbante! Ma
lui la racconta in modo dolce e semplice, piuttosto stupido, in realtà,
e lascia che siano gli altri a interpretarne il significato. Io non gli
ho parlato, personalmente. Ho mandato il Rettore dei Memorabilia a farsi
raccontare la sua versione.

--- Credo che farei meglio a parlare a frate Francis mormorò Cheroki.

--- Fatelo! Quando siete entrato, non sapevo ancora se dovevo arrostirvi
vivo o no. Per averlo fatto ritornare, voglio dire. Se
l\textquotesingle aveste lasciato fuori nel deserto, non ci troveremmo
alle prese con questa fantastica tiritera. Ma, d\textquotesingle altra
parte, se fosse rimasto là fuori, non si può sapere che altro avrebbe
tirato fuori da quel sotterraneo. Io credo che abbiate fatto bene a
mandarlo qui.

Cheroki, che aveva preso la decisione su basi molto diverse, giudicò che
la politica più appropriata fosse il silenzio. --- Parlategli ---
ringhiò l\textquotesingle abate. --- Poi mandatelo da me.

Erano circa le nove d\textquotesingle un luminoso lunedì mattina quando
frate Francis bussò timidamente alla porta dello studio
dell\textquotesingle abate. Una buona notte di sonno sul duro
pagliericcio, nella sua vecchia, solita cella, più una insolita
colazione non avevano forse fatto prodigi per i suoi tessuti esausti e
non avevano spazzato via completamente il riverbero del sole dal suo
cervello, ma quei lussi relativi lo avevano per lo meno restituito a una
chiarezza di mente sufficiente a consentirgli di intuire che aveva
motivo di essere spaventato. Infatti era terrorizzato, così che il suo
primo tocco alla porta dell\textquotesingle abate non si udì affatto.
Neppure Francis poté udirlo. Dopo parecchi minuti, riuscì a raccogliere
il coraggio necessario per bussare ancora.

--- \emph{Benedicamus Domino.}

--- \emph{Deo gratias?} --- chiese Francis.

--- Entra, figliolo entra! --- chiamò una voce affabile che Francis,
dopo qualche secondo di perplessità, riconobbe, sbalordito, per quella
del suo abate.

--- Gira la maniglia, figlio mio --- disse la stessa voce amichevole
dopo che frate Francis si era fermato irrigidito per parecchi secondi,
con le nocche ancora nella posizione di bussare.

--- S-s-sì\ldots{} --- Francis toccò appena la maniglia, ma pareva che
quella maledetta porta si aprisse comunque; aveva sperato che sarebbe
rimasta saldamente bloccata.

--- Monsignore l\textquotesingle Abate ha m-m-m-andato a
chiamare\ldots{} me? --- squittì il novizio.

L\textquotesingle abate Arkos sporse le labbra e annuì lentamente.

--- Uhm-sì, l\textquotesingle abate ha mandato a chiamare\ldots{}
\emph{te}. Entra e chiudi la porta.

Frate Francis chiuse la porta e rimase ritto, rabbrividendo, nel centro
della stanza. L\textquotesingle abate giocherellava con qualcuno degli
oggetti dai baffi di filo metallico tolti dall\textquotesingle antica
cassetta.

O forse sarebbe stato più conveniente --- disse l\textquotesingle abate
Arkos --- se il Reverendo Padre Abate fosse stato chiamato da \emph{te}.
Ora che tu sei stato così favorito dalla Provvidenza e sei diventato
così famoso, eh? --- E sorrise in modo accattivante.

--- Eh? Eh? --- Frate Francis rise con aria interrogativa. --- Oh,
n-n-no, monsignore.

--- Non contesti di avere acquisito fama molto rapidamente? Di essere
stato eletto dalla Provvidenza per scoprire QUESTO\ldots{} --- E indicò
con un gesto le reliquie sparse sulla scrivania ---\ldots{} questa
cassetta di CIANFRUSAGLIE come il suo precedente proprietario la
chiamava giustamente?

Il novizio balbettò, impotente, e in qualche modo riuscì a esibire una
specie di sogghigno.

--- Tu hai diciassette anni e sei evidentemente un idiota, non è così?

--- Questo è indubbiamente vero, Monsignor Abate.

--- Che scusa adduci per crederti chiamato alla Religione?

--- Nessuna scusa, \emph{magister meus.}

--- Ah? E così? Allora senti di non avere vocazione per
l\textquotesingle Ordine?

--- Oh, io l\textquotesingle ho! --- ansimò il novizio.

--- Ma non adduci alcuna giustificazione?

--- Nessuna.

--- Piccolo cretino, ti sto chiedendo quali ragioni hai. Poiché dichiari
di non averne, ne deduco che sei pronto a negare di aver incontrato
qualcuno nel deserto, l\textquotesingle altro giorno, che sei inciampato
in questa\ldots{} questa cassetta di CIANFRUSAGLIE senza alcun aiuto, e
che ciò che io ho udito dagli altri è soltanto\ldots{} un delirio
febbrile?

--- Oh, no, don Arkos!

--- Oh, no che cosa?

--- Non posso negare ciò che ho visto con i miei occhi, Reverendo Padre.

--- Quindi, tu \emph{hai} incontrato un angelo\ldots{} o era un
santo?\ldots{} O forse non ancora un santo?\ldots{} E ti ha indicato
dove cercare?

--- Non ho mai detto che era\ldots{}

--- E questa è la tua giustificazione per credere di avere una sincera
vocazione, non è così? Questa\ldots{} questa\ldots{} dobbiamo chiamarla
una ``creatura''?\ldots{} ti ha augurato di trovare una voce, e ha
segnato una pietra con le sue iniziali, e ti ha detto che era ciò che
cercavi, e quando tu hai guardato sotto la pietra\ldots{}
c\textquotesingle era QUESTO. Eh?

--- Sì, don Arkos.

--- Cosa ne pensi della tua esecrabile vanità?

--- La mia esecrabile vanità è imperdonabile, mio Signore e Maestro.

--- Immaginarti tanto importante da essere \emph{imperdonabile} è una
vanità ancora più grande --- ruggì il superiore
dell\textquotesingle abbazia.

--- Monsignore, io sono veramente un verme.

--- Benissimo, è solo necessario che tu neghi la parte relativa al
pellegrino. Nessun altro ha visto quella persona, sai. Mi pare di aver
capito che avrebbe dovuto venire in questa direzione. Ha detto persino
che si sarebbe fermato qui. E si è informato
sull\textquotesingle abbazia. Sì? E dove sarebbe sparito, se mai è
esistito? Nessuna persona di quel genere è passata di qui. Il fratello
che era di turno alla torre di guardia non l\textquotesingle ha visto.
Eh? Adesso sei disposto ad ammettere che te lo sei immaginato?

--- Se non vi fossero veramente quei due segni sulla pietra dove lui.,.
allora forse potrei\ldots{}

L\textquotesingle abate chiuse gli occhi e sospirò, stancamente. --- I
segni ci sono\ldots{} molto deboli --- ammise. --- Avresti potuto farli
tu.

--- No, monsignore.

--- Ammetti di avere immaginato quella vecchia creatura?

--- No, monsignore.

--- Benissimo, sai cosa ti capiterà, adesso?

--- Sì, Reverendo Padre.

--- Allora preparati a ricevere la punizione.

Tremando, il novizio si raccolse l\textquotesingle abito attorno alla
cintura e si piegò sulla scrivania. L\textquotesingle abate prese dal
cassetto una robusta riga di quercia, la provò sulla palma, poi diede a
Francis un abile colpo trasversale sulle natiche.

--- \emph{Deo gratias!} --- rispose doverosamente il novizio,
boccheggiando un po\textquotesingle.

--- Hai intenzione di cambiare idea, figlio mio?

--- Reverendo Padre, non posso negare\ldots{}

\emph{WHACK!}

--- \emph{Deo gratias!}

\emph{WHACK!}

--- \emph{Deo gratias!}

Dieci volte fu ripetuta la semplice ma dolorosa litania, mentre frate
Francis gemeva i suoi ringraziamenti al Cielo per ogni bruciante lezione
della virtù dell\textquotesingle umiltà, come era previsto che facesse.
L\textquotesingle abate si fermò dopo la decima sferzata. Frate Francis
stava in punta di piedi e vacillava leggermente. Le lacrime gli
spuntavano dagli angoli delle palpebre contratte.

Mio caro fratello Francis --- disse l\textquotesingle abate Arkos ---
sei \emph{assolutamente} sicuro di avere visto il vecchio?

--- Sicuro --- squittì il giovane, facendosi coraggio in attesa di altri
colpi.

L\textquotesingle abate Arkos sbirciò il giovane con aria clinica, poi
girò attorno alla scrivania e sedette con un brontolio. Fissò
accigliato, il pezzo di pergamena che recava le lettere:

\begin{center}
	{\Huge{\textcjheb{.sl}}}
\end{center}\

~

--- Chi credi che fosse? --- mormorò distrattamente
l\textquotesingle abate Arkos.

Frate Francis aprì gli occhi, provocando una breve doccia di lacrime.

--- Oh, mi hai convinto, figliolo, purtroppo per te.

Francis non disse nulla, ma pregò silenziosamente che la necessità di
convincere il superiore della propria veracità non si presentasse
spesso. In risposta a un gesto irritato dell\textquotesingle abate,
riabbassò la tunica.

--- Puoi sederti --- disse l\textquotesingle abate, assumendo un tono
distratto, se non cordiale.

Francis si mosse verso la sedia che gli era stata indicata, si abbassò a
metà, poi rabbrividì e si raddrizzò. --- Se per il Reverendo Padre Abate
è lo stesso\ldots{}

--- Benissimo, allora resta in piedi. Non ti tratterrò a lungo,
comunque. Dovrai uscire e finire la tua vigilia. --- Si interruppe,
notando che il viso del novizio si illuminava un poco. --- Oh, no, non
là! --- scattò. --- Non ritornerai nello stesso posto. Scambierai il tuo
eremitaggio con quello di frate Alfred, e non tornerai più vicino a
quelle rovine. Inoltre, ti comando di non discutere della cosa con
nessuno, eccetto il tuo confessore e me, sebbene, il Cielo lo sa, il
malanno sia già stato fatto. Sai a cosa hai dato
l\textquotesingle avvio?

Frate Francis scosse il capo. --- Poiché ieri era domenica, Reverendo
Padre, non ci era richiesto di tacere, e durante la ricreazione mi sono
limitato a rispondere alle domande dei confratelli. Pensavo\ldots{}

--- Bene, i tuoi confratelli hanno combinato una spiegazione molto
acuta, caro\_ figlio. Sapevi che era il beato Leibowitz in persona colui
che hai incontrato là fuori?

Francis lo guardò senza capire per un momento, poi scosse di nuovo il
capo. --- Oh, no, Monsignor Abate, sono sicuro che non poteva essere
lui. Il Beato Martire non farebbe una cosa simile.

--- Non farebbe che cosa?

--- Non inseguirebbe qualcuno cercando di colpirlo con un bastone
chiodato.

L\textquotesingle abate si passò una mano sulla bocca per nascondere un
sorriso involontario. Dopo un momento riuscì a mostrarsi pensieroso. ---
Oh, non so.. Eri tu quello che inseguiva, no? Sì, credo di sì. Hai
raccontato ai tuoi confratelli novizi anche questa parte? Sì, eh? Bene,
vedi, loro non credono che questo escluda la possibilità che si
trattasse del Beato. Ora, io dubito fortemente che vi siano molte
persone che il Beato inseguirebbe con un bastone chiodato, ma\ldots{}
--- Si interruppe, incapace di reprimere una risata davanti
all\textquotesingle espressione sul volto del novizio. --- Benissimo,
figliolo\ldots{} ma chi credi che potesse essere quel vecchio?

--- Pensavo che forse era un pellegrino diretto a visitare il nostro
santuario, Reverendo Padre.

--- Non è ancora un santuario, e non devi chiamarlo così. E comunque,
non era diretto qui, o per lo meno, qui non è venuto. E non è passato
oltre i nostri cancelli, a meno che la sentinella non fosse
addormentata. E il novizio di guardia nega di essersi addormentato,
sebbene abbia ammesso di aver avuto molto sonno, quel giorno. Dunque, tu
cosa suggerisci?

--- Se il Reverendo Padre vuole perdonarmi, anch\textquotesingle io sono
stato di guardia qualche volta.

--- E allora?

--- Bene, in una giornata luminosa, quando non c\textquotesingle è
niente che si muove, tranne le poiane, dopo qualche ora si comincia a
guardare le poiane.

--- Ah, tu lo fai, eh? Quando dovresti sorvegliare la pista!

--- E se si guarda il cielo troppo a lungo, ci si stordisce\ldots{} non
ci si addormenta veramente\ldots{} ma si resta\ldots{} come dire\ldots{}
intontiti.

--- Dunque è così che fai quando sei di guardia, vero? --- grugnì
l\textquotesingle abate.

--- Non necessariamente. Voglio dire, no, Reverendo Padre, non saprei.
Frate Je\ldots{} voglio dire, un fratello cui ho dato il cambio una
volta era proprio così. Non sapeva neppure che fosse
l\textquotesingle ora del cambio. Era là seduto sulla torre e fissava il
cielo a bocca aperta. Abbagliato.

--- Sì, e la prima volta che ti istupidisci in questo modo arriverà una
schiera di scorridori atei dallo Utah, ucciderà qualche giardiniere,
rovinerà il sistema di irrigazione, distruggerà il nostro raccolto, e
butterà pietre nel pozzo prima che noi cominciamo a difenderci. Perché
fai quella faccia\ldots{} oh, dimenticavo.., tu vivevi nello Utah, prima
di fuggire, no? Ma non importa, può darsi\ldots{} dico può darsi\ldots{}
che tu abbia ragione per quanto riguarda il fratello di guardia\ldots{}
che avrebbe potuto non vedere il vecchio, cioè. Tu sei sicuro che era
soltanto un comune vecchio\ldots{} nient\textquotesingle altro? Non un,
angelo? Non un beato?

Lo sguardo del novizio si levò verso il soffitto, pensierosamente, poi
ricadde in fretta sul viso dell\textquotesingle abate. --- Gli angeli e
i santi fanno ombra?

--- Sì\ldots{} voglio dire no. Voglio dire\ldots{} come posso saperlo?
Faceva ombra, non è vero?

--- Ecco\ldots{} era un\textquotesingle ombra così piccola che potevo
appena vederla.

--- \emph{Cosa?}

--- Perché era quasi mezzogiorno.

--- Imbecille! Non ti sto chiedendo che cosa era. So benissimo che
cos\textquotesingle era, se mai tu l\textquotesingle hai visto davvero.
L\textquotesingle abate Arkos batté ripetutamente sulla tavola, per
sottolineare la frase. --- Voglio sapere se \emph{tu\ldots{} tu\ldots{}
	sei sicuro oltre ogni dubbio} che fosse soltanto un vecchio come tutti
gli altri!

Questo genere di interrogatorio stupì frate Francis. Nella sua mente,
non c\textquotesingle era alcuna linea retta che separava il Naturale
dal Soprannaturale, ma c\textquotesingle era, piuttosto, una zona
crepuscolare intermedia. C\textquotesingle erano cose che erano
chiaramente naturali, e c\textquotesingle erano cose che erano
chiaramente soprannaturali, ma fra questi due estremi
c\textquotesingle era una zona di confusione --- la sua confusione ---
il preternaturale\ldots{} dove le cose fatte di terra, aria, fuoco o
acqua tendevano a comportarsi in modo inquietante come \emph{Cose}. Per
frate Francis, questa regione comprendeva tutto ciò che poteva vedere ma
non capire. E frate Francis non era mai ``sicuro al di là di ogni
dubbio'', come l\textquotesingle abate gli stava chiedendo di essere.
Così, sollevando il problema, l\textquotesingle abate Arkos stava
involontariamente lanciando il pellegrino del novizio nella regione
crepuscolare, nella stessa prospettiva che aveva avuto la prima
apparizione del vecchio, come una striscia nera, priva di gambe, che
fremeva nel mezzo d\textquotesingle un lago creato
dall\textquotesingle illusione del calore sulla pista, nella stessa
prospettiva che aveva occupato per un attimo quando il mondo del novizio
si era contratto fino a non contenere altro che una mano tesa per
offrirgli un pezzetto di cibo. Se qualche creatura super umana aveva
deciso di camuffarsi da creatura umana, come poteva, \emph{lui},
penetrare in quel travestimento, o sospettare che ve ne fosse uno? Se
una creatura del genere non voleva essere sospettata, non avrebbe
ricordato di gettare un\textquotesingle ombra, di lasciare orme di
passi, di mangiare pane e formaggio? Non poteva, forse, masticare foglie
aromatiche, sputare contro una lucertola, e ricordarsi di imitare la
reazione d\textquotesingle un mortale che dimenticava di infilare i
sandali prima di avventurarsi sul terreno scottante? Francis non era in
grado di valutare l\textquotesingle intelligenza o
l\textquotesingle ingegnosità di esseri infernali o celesti, o di
indovinare la portata delle loro abilità istrioniche, sebbene pensasse
che tali creature fossero infernalmente o celestialmente abili.
L\textquotesingle abate, sollevando la questione, aveva già formulato la
natura della risposta di frate Francis, che era questa: prendere in
esame la questione, sebbene prima non lo avesse fatto.

--- Ebbene, ragazzo mio?

--- Monsignor Abate, voi non supponete che potesse essere\ldots{}

--- Ti sto chiedendo di \emph{non} supporre. Ti sto chiedendo di essere
sicuro. Era o non era una normale persona di carne e di sangue?

La domanda era spaventosa. Il fatto che tale domanda fosse nobilitata
dal provenire dalle labbra d\textquotesingle una persona così illustre
come il suo abate la rendeva ancora più spaventosa, anche se Francis
capiva che il suo superiore l\textquotesingle aveva formulata
semplicemente perché voleva una particolare risposta. La voleva
intensamente. Se la voleva intensamente, la domanda doveva essere
importante. Se la domanda era abbastanza importante per un abate, era
troppo importante per frate Francis, che non osò sbagliare.

--- Io\ldots{} credo che fosse di carne e di sangue, Reverendo Padre, ma
non era precisamente ``normale''. In un certo senso, era straordinario.

--- \emph{In che senso?} --- chiese con voce tagliente
l\textquotesingle abate Arkos. --- Come\ldots{} come riusciva a sputare
diritto. E sapeva leggere, credo.

L\textquotesingle abate chiuse gli occhi e si soffregò le tempie, in
evidente segno di esasperazione. Quanto sarebbe stato semplice se avesse
potuto dire al ragazzo che il suo pellegrino era soltanto un vecchio
vagabondo, e se poi avesse potuto ordinargli di non pensare altrimenti.
Ma, permettendo al ragazzo di capire che era possibile una domanda,
aveva reso inefficiente l\textquotesingle ordine prima ancora di
pronunciarlo., Fino a che il pensiero poteva essere governato, gli si
poteva soltanto ordinare di seguire ciò che la ragione confermava; un
diverso comando non sarebbe stato obbedito. Come ogni saggio dominatore,
l\textquotesingle abate Arkos non emetteva ordini invano, quando era
possibile disobbedire e quando era impossibile imporli con la forza. Era
meglio distogliere lo sguardo, piuttosto che dare ordini ineseguibili.
Aveva formulato una domanda cui egli stesso non avrebbe saputo
rispondere secondo ragione, poiché non aveva mai visto il vecchio, e di
conseguenza aveva perduto il diritto di rendere obbligata la risposta.

--- Vattene --- disse alla fine, senza neppure aprire gli occhi.
