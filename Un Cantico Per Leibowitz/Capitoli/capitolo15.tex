	\chapter{\phantom{title}}

\lettrine{H}{ongan} Os era, essenzialmente, un uomo giusto e di animo gentile. Quando
vide un gruppo di suoi guerrieri che si divertivano con i prigionieri
laredani, si fermò a guardare; ma quando quelli legarono per le caviglie
tre laredani ai cavalli, e sferzarono le bestie per lanciarli al
galoppo, Hongan Os decise di intervenire. Ordinò che i guerrieri
venissero frustati, perché Hongan Os --- Orso Pazzo --- era conosciuto
come un capitano misericordioso. Non aveva mai maltrattato un cavallo.

--- Uccidere i prigionieri è un lavoro da femmina --- grugnì rivolto ai
colpevoli. --- Purificatevi, se non volete essere additati come femmine,
e ritiratevi dal campo fino alla Luna Nuova, perché siete banditi per
dodici giorni. --- E, in risposta ai gemiti di protesta: --- Immaginate
che i cavalli avessero trascinato uno dei prigionieri attraverso il
campo. I capi dei mangia-erba sono nostri ospiti, ed è noto che si
spaventano facilmente alla vista del sangue. Specialmente del sangue
della loro razza. Tenetelo a mente.

--- Ma questi sono mangia-erba del Sud --- obiettò un guerriero,
indicando i prigionieri mutilati. --- I nostri ospiti sono mangia-erba
dell\textquotesingle Est. Non c\textquotesingle è forse un patto fra
noi, il vero popolo, e l\textquotesingle Est, per fare guerra al Sud?

--- Se ne parli di nuovo, ti farò mozzare la lingua per darla in pasto
ai cani! --- l\textquotesingle ammonì Orso Pazzo. --- Dimentica di aver
mai sentito parlare di queste cose.

--- I mangiatori d\textquotesingle erba rimarranno tra noi per molti
giorni, o Figlio del Potente?

--- Chi può sapere cosa abbiano in mente quei contadini? --- chiese di
rimando Orso Pazzo. --- Il loro pensiero non è il nostro pensiero.
Dicono che alcuni di loro partiranno di qui per proseguire attraverso le
Terre Aride\ldots{} verso una località abitata da preti mangia-erba, da
individui vestiti di tonache scure. Gli altri resteranno qui per
parlare\ldots{} ma ciò non è per le vostre orecchie. Adesso andatevene,
e vergognatevi per dodici giorni.

Voltò loro le spalle, perché potessero scivolare via senza sentire il
suo terribile sguardo. Negli ultimi tempi la disciplina si era
allentata. I clan erano irrequieti. Si era saputo fra il popolo delle
Pianure che lui, Orso Pazzo, aveva incrociato le braccia, dalla parte
opposta del fuoco attorno al quale si negoziava, di fronte a un
messaggero di Texarkana, e che uno sciamano aveva tagliato a entrambi le
unghie e ciuffi di capelli per farne un feticcio della buonafede come
difesa contro i possibili tradimenti da ambo le parti. Si era risaputo
che era stato concluso un accordo, e qualunque accordo fra il vero
popolo e i mangia-erba era considerato motivo di vergogna dalle tribù.
Orso Pazzo aveva sentito il sarcasmo velato dei guerrieri più giovani,
ma non avrebbe dato loro alcuna spiegazione fino a che non fosse giunto
il momento opportuno.

Orso Pazzo era disposto ad ascoltare ogni buon pensiero, anche se veniva
da un cane. Il pensiero dei mangia-erba era buono solo raramente, ma era
stato impressionato dai messaggi del re mangia-erba
dell\textquotesingle Est, che aveva esposto i pregi della segretezza e
aveva deplorato le inutili vanterie. Se i laredani avessero saputo che
le tribù venivano armate da Hannegan, il piano sarebbe indubbiamente
fallito. Orso Pazzo aveva meditato su quel pensiero; gli
ripugnava\ldots{} perché certamente era più soddisfacente e più virile
dire al nemico che cosa avevi intenzione di fargli, prima di mettere in
atto i tuoi propositi; eppure, più vi rifletteva, e più comprendeva la
saggezza di quel pensiero. O il re mangia-erba era un codardo, oppure
era saggio più di qualsiasi altro uomo: Orso Pazzo non aveva deciso
quale delle due ipotesi fosse vera\ldots{} ma giudicava saggio quel
pensiero. La segretezza era importantissima, anche se per un certo tempo
sembrava piuttosto femminile. Se il popolo di Orso Pazzo avesse saputo
che le armi che gli giungevano erano i doni di Hannegan, e non le
spoglie di scorrerie di confine, allora si sarebbe presentata la
possibilità che anche Laredo venisse a conoscenza del piano dai
prigionieri catturati durante le scorrerie. Di conseguenza, era
necessario lasciare che le tribù brontolassero contro la vergogna di
parlare di pace con gli agricoltori dell\textquotesingle Est.

Ma i discorsi non erano di pace. I discorsi erano buoni, promettevano
bottino.

Poche settimane prima, lo stesso Orso Pazzo aveva guidato una
``scorreria'' all\textquotesingle Est, e ne era ritornato con cento
cavalli, quattro dozzine di lunghi fucili, parecchi barili di polvere
nera, molte munizioni e un prigioniero. Ma neppure i guerrieri che
l\textquotesingle avevano accompagnato sapevano che il deposito di armi
era stato messo lì apposta per lui dagli uomini di Hannegan, e che il
prigioniero era in realtà un ufficiale della cavalleria di Texarkana che
in avvenire avrebbe consigliato Orso Pazzo circa le possibili tattiche
usate dai laredani durante i futuri combattimenti. Il pensiero dei
mangia-erba era privo di vergogna, ma il pensiero
dell\textquotesingle ufficiale poteva sondare quello dei mangia-erba del
Sud; non poteva sondare invece quello di Hongan Os.

Orso Pazzo era comprensibilmente orgoglioso delle sue capacità di
negoziatore. Non aveva promesso niente, se non di non fare guerra a
Texarkana e di smettere di rubare il bestiame ai confini orientali, ma
soltanto finché Hannegan lo riforniva di armi e di munizioni.
L\textquotesingle accordo di guerra contro Laredo era un accordo tacito,
ma corrispondeva alle naturali inclinazioni di Orso Pazzo, e non
c\textquotesingle era bisogno, quindi, di un patto formale.
L\textquotesingle alleanza con uno dei suoi nemici gli avrebbe permesso
di liquidare un avversario alla volta, e alla fine avrebbe potuto
riconquistare le terre che gli agricoltori avevano occupato e
colonizzato durante il secolo precedente.

Era caduta la notte quando il capo dei clan entrò a cavallo
nell\textquotesingle accampamento: il freddo era sceso sulle Pianure. I
suoi ospiti venuti dall\textquotesingle Est sedevano raggomitolati nelle
loro coperte attorno al fuoco del Consiglio insieme a tre anziani,
mentre la solita cerchia di bambini curiosi si stringeva
nell\textquotesingle ombra circostante e sbirciava gli stranieri al di
sotto dei lembi delle tende. C\textquotesingle erano in tutto dodici
stranieri, separati in due gruppi distinti, che avevano viaggiato
insieme ma che mostravano di tenere ben poco l\textquotesingle uno alla
compagnia dell\textquotesingle altro.

Il capo di uno dei due gruppi era evidentemente un pazzo. Hongan Os non
aveva obiezioni contro la pazzia (in realtà, era considerato dai suoi
sciamani come l\textquotesingle uomo che era visitato dal soprannaturale
più frequentemente d\textquotesingle ogni altro), ma non aveva mai
saputo che anche gli agricoltori considerassero la follia come una
virtù, in un loro capo. Quest\textquotesingle uomo, però, trascorreva
metà del tempo a scavare la terra lungo il letto del fiume disseccato, e
l\textquotesingle altra metà a scribacchiare misteriosamente su un
libriccino. Era ovvio che fosse uno stregone, e forse era meglio non
fidarsi di lui.

Orso Pazzo si fermò soltanto per il tempo necessario a indossare le sue
pelli di lupo da cerimonia e a permettere a uno sciamano di dipingergli
sulla fronte il segno del totem, prima di unirsi al gruppo attorno al
fuoco.

--- \emph{Tremate!} --- gemette ritualmente un vecchio guerriero, mentre
il capo dei clan si mostrava nel riverbero del fuoco. --- Tremate,
perché il Potente cammina fra i suoi figli. Tremate, o clan, perché il
suo nome è Orso Pazzo\ldots{} un nome ben meritato, perché da giovane
egli vinse senza armi un orso impazzito, e con le mani nude lo
strangolò, lassù nelle terre del Nord\ldots{}

Hongan Os ignorò gli elogi e accettò una coppa di sangue dalla vecchia
donna che serviva il fuoco del Consiglio. Era stato raccolto di fresco
da un giovane bue appena macellato e ancora caldo. Vuotò la coppa prima
di voltarsi per fare un cenno con il capo agli orientali, che
osservavano la breve cerimonia con evidente inquietudine.

--- \emph{Aaah!} --- fece il capo dei clan.

--- \emph{Aaah!} --- risposero i tre anziani, insieme a un mangia-erba
che osò fare eco. I membri dei clan guardarono il mangia-erba per un
attimo, disgustati.

Il pazzo tentò di coprire la goffaggine del suo compagno. --- Dimmi ---
disse il pazzo, quando il capitano si fu seduto --- come mai il tuo
popolo non beve acqua? I vostri dei lo proibiscono?

--- Chi sa che cosa bevono gli dei? --- ruggì Orso Pazzo. Si dice che
l\textquotesingle acqua è per il bestiame e per gli agricoltori, il
latte è per i bambini e il sangue è per gli uomini. Dovrebbe essere
altrimenti?

II pazzo non si sentì insultato. Osservò per un attimo il capo con gli
intenti occhi grigi, poi fece un cenno a uno dei suoi compagni. ---
Quella frase, ``l\textquotesingle acqua è per il bestiame'', spiega
tutto --- disse. --- L\textquotesingle eterna siccità, in queste zone.
Un popolo di allevatori conserva per il bestiame quel
po\textquotesingle{} d\textquotesingle acqua che c\textquotesingle è. Mi
domando se hanno rafforzato questa convinzione con un tabù religioso.

Il suo compagno fece una smorfia e parlò in lingua texarkana. --- Acqua!
Per gli dei, perché noi non possiamo bere acqua, Thon Taddeo? Mi sembra
che stiamo diventando troppo conformisti! --- E sputò. --- Sangue! Bah!
Si attacca in gola. Perché non possiamo bere un sorso\ldots{}

--- No, fino a che non partiamo!

--- Ma, Thon\ldots{}

--- No! --- scattò lo studioso; poi, osservando che gli uomini dei clan
li fissavano corrucciati, parlò di nuovo a Orso Pazzo nella lingua delle
Pianure. --- Il mio compagno stava parlando della vigoria e della salute
del tuo popolo --- disse. --- Forse è merito della vostra dieta.

--- Ah! --- latrò il capo, ma poi gridò quasi allegramente alla vecchia:
--- Dai a quel forestiero una coppa di rosso.

Il compagno di Thon Taddeo rabbrividì, ma non protestò.

--- O capo, io ho una richiesta da rivolgere alla tua grandezza ---
disse lo studioso. --- Domani noi proseguiremo il viaggio verso
occidente. Se qualcuno dei tuoi guerrieri potesse accompagnare la nostra
comitiva, ci sentiremmo onorati.

--- Perché?

Il Thon Taddeo fece una pausa. --- Ecco\ldots{} potrebbero
guidarci\ldots{} --- Si interruppe, e improvvisamente sorrise. --- No,
sarò del tutto sincero. Alcuni, tra la tua gente, disapprovano la nostra
presenza qui. Mentre la tua ospitalità è stata\ldots{}

Hongan Os rovesciò la testa all\textquotesingle indietro e ruggì una
risata. --- Hanno paura dei clan minori disse agli anziani. --- Temono
di cadere in un\textquotesingle imboscata non appena lasceranno le mie
tende. Mangiano erba e hanno paura di un combattimento!

Lo studioso arrossì lievemente,

--- Non aver paura, forestiero! --- ridacchiò il capo dei clan. --- Vi
accompagneranno dei veri uomini.

Il Thon Taddeo chinò il capo, con ironica gratitudine.

--- Dimmi --- fece Orso Pazzo --- cos\textquotesingle è che vai a
cercare nelle Terre Aride occidentali? Nuovi posti da trasformare in
campi? Posso dirti che non ve ne sono. A eccezione delle zone vicino ai
pochi pozzi d\textquotesingle acqua, non cresce niente che il bestiame
voglia mangiare.

--- Non cerchiamo nuove terre --- rispose il visitatore. --- Non tutti
siamo agricoltori, lo sai. Noi andiamo alla ricerca\ldots{} --- Si
interruppe. Nel linguaggio dei nomadi, non c\textquotesingle era modo di
spiegare lo scopo del viaggio all\textquotesingle Abbazia di san
Leibowitz. ---\ldots{} alla ricerca del segreto d\textquotesingle un
antico sortilegio. Uno degli anziani, uno sciamano, drizzò le orecchie.
--- Un antico sortilegio, nell\textquotesingle Ovest? Non mi risulta che
vi siano maghi, laggiù. O forse vuoi parlare degli uomini vestiti di
tonache scure?

Infatti.

--- Ah! Quale magia possiedono, che sia degna di essere cercata? I loro
messaggeri possono essere catturati così facilmente che non è neppure
divertente\ldots{} sebbene sopportino bene la tortura. Che sortilegio
puoi imparare da loro?

--- Ecco, da parte mia sono d\textquotesingle accordo con te --- disse
il Thon Taddeo. --- Ma si dice che alcuni scritti, ehm, alcuni
incantesimi di grande potenza siano custoditi in uno dei loro
nascondigli. Se questo è vero, allora è evidente che gli uomini dalle
tuniche scure non sanno come usarli, ma noi speriamo di potercene
impadronire e di usarli a nostro beneficio.

--- E le tuniche scure ti permetteranno di osservare i loro segreti?

Il Thon Taddeo sorrise. Credo di sì. Non osano più nasconderli. E noi
potremmo prenderli, se fosse necessario.

--- Parole coraggiose --- ringhiò Orso Pazzo. --- È chiaro che gli
agricoltori sono più coraggiosi, con quelli della loro specie\ldots.
anche se sono molto mansueti fra i veri uomini.

Lo studioso, che aveva trangugiato la sua parte di insulti del nomade,
decise di ritirarsi presto.

I guerrieri rimasero attorno al fuoco del Consiglio per discutere con
Hongan Os la guerra che era certo imminente; ma la guerra, dopotutto,
non riguardava il Thon Taddeo. Le aspirazioni politiche del suo
ignorante cugino erano ben lontane dal suo interesse per una rinascita
del sapere in un mondo buio, salvo quando la protezione del monarca si
rivelava utile, come era già avvenuto in diverse occasioni.
