	\chapter{\phantom{title}}

\lettrine{F}{u} durante la decima settimana della permanenza del Thon Taddeo che il
messaggero portò le nere notizie. Il capo della dinastia regnante di
Laredo aveva chiesto che le truppe texarkane venissero evacuate dal suo
reame. Il Re era morto di veleno quella notte stessa, e fra gli Stati di
Laredo e di Texarkana era stata proclamata la guerra. Sarebbe stata una
guerra breve. Si poteva affermare con sicurezza che la guerra era finita
il giorno dopo il suo inizio, e che adesso Hannegan controllava tutte le
terre e tutti i popoli, dal Fiume Rosso al Rio Grande.

Fin qui, era tutto previsto: ma non erano state previste le notizie che
l\textquotesingle accompagnavano.

Hannegan II, per Grazia di Dio Podestà, Viceré di Texarkana, Difensore
della Fede, e Vaquero Supremo delle Pianure, dopo aver giudicato
monsignor Marcus Apollo colpevole di ``tradimento'' e di spionaggio,
aveva ordinato che il Nunzio papale venisse impiccato e poi, quando era
ancora vivo, tolto dal patibolo per essere squartato e spellato vivo,
come esempio a chiunque altro volesse tentare di minare il potere del
podestà. Ridotta in pezzi, la carcassa
dell\textquotesingle ecclesiastico era stata gettata ai cani.

Il messaggero ebbe appena bisogno di aggiungere che Texarkana era stata
colpita da interdizione assoluta da un decreto papale che conteneva
vaghe ma minacciose allusioni alla \emph{Regnans in Excelsis}, una bolla
del Sedicesimo secolo che ordinava la deposizione di un monarca. Non vi
era ancora notizia di contromisure da parte di Hannegan.

Sulle Pianure, le forze laredane avrebbero dovuto aprirsi la strada
verso casa combattendo fra le tribù nomadi, soltanto per deporre le armi
ai propri confini, perché la loro nazione e i loro parenti erano tenuti
come ostaggi.

--- Una vera tragedia! --- disse il Thon Taddeo, con evidente sincerità.
--- A causa della mia nazionalità, mi offro di partire immediatamente.

--- Perché? --- chiese don Paulo. --- Voi non approvate le azioni di
Hannegan, non è vero?

Lo studioso esitò, poi scosse il capo. Si guardò intorno, per essere
certo che nessuno li ascoltasse. --- Personalmente, le condanno. Ma in
pubblico\ldots{} --- E scrollò le spalle. --- C\textquotesingle è il
\emph{collegium} cui devo pensare. Se si trattasse soltanto del mio
collo, allora\ldots{}

--- Comprendo.

--- Posso dirvi in confidenza una mia opinione?

--- Naturalmente.

--- Allora qualcuno dovrebbe distogliere Nuova Roma dal pronunciare
minacce oziose. Hannegan è capacissimo di crocifiggere parecchie dozzine
di Marcus Apollo.

--- E allora nuovi martiri saliranno in Cielo; Nuova Roma non pronuncia
minacce oziose.

Il Thon sospirò. --- Immaginavo che l\textquotesingle avreste presa in
questo modo. Ma rinnovo la mia offerta di andarmene.

--- Sciocchezze. Qualunque sia la vostra nazionalità, la vostra umanità
fa di voi il benvenuto.

Ma c\textquotesingle era stata una frattura. Lo studioso stette molto
sulle sue, in seguito, e conversava soltanto raramente con i monaci. I
suoi rapporti con frate Kornhoer divennero notevolmente formali, sebbene
l\textquotesingle inventore trascorresse un\textquotesingle ora o due,
ogni giorno, nella manutenzione e nell\textquotesingle ispezione della
dinamo e della lampada e si tenesse informato dei progressi del lavoro
del Thon, che procedeva con una rapidità insolita. Gli ufficiali si
avventuravano solo raramente fuori della foresteria.

C\textquotesingle erano sintomi di un esodo dalla regione. Voci
inquietanti continuavano a giungere dalle Pianure. Nel villaggio di
Sanly Bowitts, la gente cominciava a trovare improvvisamente buone
ragioni per partire tutto d\textquotesingle un tratto per qualche
pellegrinaggio o per visitare altre terre. Persino i mendicanti e i
vagabondi se ne andavano. Come sempre, i mercanti e gli artigiani si
trovarono di fronte alla spiacevole alternativa di abbandonare le loro
proprietà ai ladri e ai saccheggiatori o di rimanere sul posto per
vederle saccheggiare. Un comitato di cittadini, guidato dal podestà del
villaggio, visitò l\textquotesingle abbazia per chiedere rifugio per la
cittadinanza, in caso di invasione.

--- La mia offerta definitiva --- disse l\textquotesingle abate, dopo
parecchie ore di discussione --- è questa: accoglieremo tutte le donne,
i bambini, gli invalidi e i vecchi, senza fare domande. Ma per quanto
riguarda gli uomini capaci di maneggiare armi, considereremo ogni caso
individualmente, e può darsi che dobbiamo respingerne molti.

--- Perché? --- chiese il podestà.

--- Dovrebbe essere chiaro persino per voi! --- disse don Paulo con voce
tagliente. --- Può darsi che noi stessi veniamo attaccati, ma a meno che
non ci attacchino direttamente, noi cercheremo di restarne fuori. Non
permetterò che questo luogo sia usato da chicchessia come una
guarnigione da cui sferrare un contrattacco, se sarà solo il villaggio a
venire investito. Quindi, nel caso dei maschi abili a portare armi,
dovremo insistere per ottenere una promessa\ldots{} difendere
l\textquotesingle abbazia ai \emph{nostri} ordini. E decideremo caso per
caso se la promessa è degna di fede o no.

--- È ingiusto! --- ululò un membro del comitato. --- Voi volete
discriminare\ldots{}

--- La discriminazione verrà praticata soltanto nei confronti di chi non
meriterà fiducia. Perché? Speravate di nascondere qui un esercito di
riserva? Ebbene, non vi sarà permesso. Non piazzerete una milizia
cittadina, qui fuori. Questa decisione è definitiva.

Considerate le circostanze, il comitato non poteva rifiutare qualunque
aiuto venisse offerto. Non vi furono ulteriori discussioni.

Don Paulo aveva intenzione di accogliere tutti, quando fosse venuto il
momento, ma intanto intendeva sventare tutti i progetti del villaggio
che comprendessero l\textquotesingle abbazia in un piano militare. Più
tardi sarebbero venuti degli ufficiali da Denver, con richieste eguali;
avrebbero pensato meno a salvare delle vite che a salvare il loro regime
politico. E voleva dare anche a loro una risposta simile.
L\textquotesingle abbazia era stata costruita per essere una fortezza di
fede e di dottrina, e intendeva conservarla tale.

Il deserto cominciò a brulicare di viaggiatori che venivano
dall\textquotesingle Est. Commercianti, cacciatori e mandriani che si
spostavano verso occidente, portavano notizie dalle Pianure. La
pestilenza dei bestiame si spargeva come un incendio fra le mandrie dei
nomadi; sembrava imminente la carestia. Le forze di Laredo avevano
subito continui sfaldamenti e ammutinamenti, dopo la caduta della
dinastia laredana. Parte dei soldati erano ritornati in patria, come era
stato loro ordinato, ma altri si erano accinti, per un terribile voto, a
marciare su Texarkana e a non fermarsi fino a che non si fossero
impadroniti della testa di Hannegan II o fossero morti nel tentativo.
Indeboliti da quella divisione, i laredani venivano gradualmente
spazzati via dalle continue scaramucce dei guerrieri di Orso Pazzo che
erano assetati di vendetta contro coloro che avevano portato la moria
fra le loro mandrie.

Si vociferava che Hannegan avesse generosamente offerto di accogliere i
sudditi di Orso Pazzo come dipendenti e protetti, se avessero giurato
fedeltà alla legge ``civile'', se avessero accettato i suoi ufficiali
nei loro consigli e se avessero abbracciato la Fede Cristiana.
``Sottomettersi o morire di fame'' era la scelta che il fato e Hannegan
offrivano ai popoli nomadi. Molti preferivano morire di fame piuttosto
di giurare lealtà a uno Stato di agricoltori e di mercanti. Si diceva
che Hongan Os andasse ruggendo la sua sfida verso sud, verso est e verso
il Cielo; e per sottolineare quest\textquotesingle ultima bruciava uno
sciamano al giorno per punire gli dei tribali del loro tradimento.
Minacciava di diventare cristiano se gli dei cristiani lo avessero
aiutato a massacrare i suoi nemici.

Fu durante una breve visita d\textquotesingle un gruppo di pastori che
il Poeta scomparve dall\textquotesingle abbazia. Il Thon Taddeo fu il
primo a notare l\textquotesingle assenza del Poeta dalla foresteria e a
chiedere notizie del vagabondo verseggiatore.

Il viso di don Paulo si contrasse per la sorpresa. --- Siete certo che
se ne sia andato? --- chiese. --- Spesso trascorre qualche giorno nel
villaggio, o va alla mesa per discutere con Benjamin.

--- Manca tutta la sua roba --- disse il Thon. --- Dalla sua stanza è
scomparsa ogni cosa.

L\textquotesingle abate storse la bocca. --- Quando il Poeta se ne va, è
un brutto segno. Fra l\textquotesingle altro, se è veramente scomparso,
vi consiglierei di fare un immediato inventario delle vostre proprietà.

Il Thon assunse un\textquotesingle espressione pensierosa. --- Dunque i
miei stivali\ldots{}

--- Senza dubbio.

--- Li avevo messi fuori dalla stanza perché venissero lucidati. Ma non
mi sono stati resi. Fu lo stesso giorno in cui tentò di abbattere la mia
porta.

--- Di abbattere\ldots{} chi, il Poeta?

Thon Taddeo ridacchiò. --- Temo di essermi divertito un
po\textquotesingle{} alle sue spalle. Io ho il suo occhio di vetro.
Ricordate la sera in cui lo abbandonò sulla tavola del refettorio?

--- Sì.

--- Lo presi io.

Il Thon aprì la borsa, vi frugò per un attimo, poi posò sulla scrivania
l\textquotesingle occhio di vetro del Poeta. --- Sapeva che
l\textquotesingle avevo io, ma ho continuato a negarlo. Però da allora
ci siamo divertiti alle sue spalle, inventando addirittura la voce che
fosse in realtà l\textquotesingle occhio di vetro, perduto da molto
tempo, dell\textquotesingle idolo Bayring, e che avrebbe dovuto essere
riportato al museo. Il Poeta è diventato frenetico, dopo un poco.
Naturalmente avevo intenzione di restituirglielo prima di ripartire.
Pensate che ritornerà, dopo che ce ne saremo andati?

--- Ne dubito --- disse l\textquotesingle abate, rabbrividendo
leggermente mentre fissava il globo oculare. --- Ma lo conserverò per
lui, se volete. Sebbene sia probabile che il Poeta compaia a Texarkana
per cercarlo lì. Sostiene che è un talismano potente.

--- E in che senso?

Don Paulo sorrise. --- Dice che ci vede molto meglio quando lo porta.

--- Che assurdità! --- Il Thon si interruppe, sempre pronto,
evidentemente, a considerare almeno per un attimo qualsiasi insolita
premessa e aggiunse: --- È un\textquotesingle assurdità\ldots{} a meno
che, riempiendo l\textquotesingle orbita vuota, non influenzi in qualche
modo i muscoli di entrambe le orbite. E questo ciò che afferma?

--- Si limita a giurare che non può vedere altrettanto bene, senza di
esso. Sostiene che deve averlo per la percezione dei ``veri
significati''\ldots{} sebbene gli procuri accecanti mal di testa quando
lo porta. Ma non si sa mai quando il Poeta racconta fatti, fantasie o
allegorie. Se la fantasia è abbastanza intelligente, dubito che il Poeta
ammetta una differenza tra fantasia e realtà.

Il Thon sorrise ironicamente. --- Fuori della mia porta,
l\textquotesingle altro giorno, gridava che io ne avevo più bisogno di
lui. Questo mi sembra indicare che egli lo consideri, in se stesso, come
un potente feticcio\ldots{} utile a chiunque. Mi domando perché.

--- Ha detto che voi ne avevate bisogno? \emph{Oh-oh!}

--- Questo vi diverte?

--- Scusatemi. Probabilmente intendeva insultarvi. Sarà meglio che non
tenti di spiegarvi l\textquotesingle insulto del Poeta; potrebbe
sembrare che io lo condivida.

--- Affatto. Io sono curioso.

L\textquotesingle abate guardò l\textquotesingle immagine di san
Leibowitz nell\textquotesingle angolo della stanza. --- Il Poeta usava
il suo occhio di vetro come una buffoneria corrente --- spiegò. ---
Quando voleva prendere una decisione, o riflettere su qualcosa, o
discutere un argomento, metteva l\textquotesingle occhio di vetro
nell\textquotesingle orbita. Lo toglieva di nuovo quando vedeva qualcosa
che gli spiaceva, quando fingeva di ignorare qualcosa, o quando voleva
fare la parte dello stupido. Quando lo portava, il suo contegno
cambiava. I frati cominciarono a chiamarlo ``la coscienza del Poeta'' e
lui stava allo scherzo. Dava piccole lezioni e dimostrazioni sui
vantaggi di una coscienza removibile. Fingeva che qualche impulso
frenetico lo possedesse\ldots{} qualcosa di scarsa importanza, di
solito\ldots{} come un impulso diretto verso una bottiglia di vino.
Quando portava l\textquotesingle occhio, accarezzava la bottiglia di
vino, si leccava le labbra, ansimava e gemeva, poi staccava la mano.
Finalmente, l\textquotesingle impulso lo riprendeva. Afferrava la
bottiglia, ne versava un poco in una coppa e vi deglutiva sopra per un
secondo. Ma poi la coscienza aveva il sopravvento, e gettava la coppa
attraverso la stanza. Quindi ricominciava a guardare avidamente la
bottiglia di vino, e cominciava a gemere e a perdere saliva dalla bocca,
ma continuava a combattere l\textquotesingle impulso\ldots{}
L\textquotesingle abate ridacchiò, controvoglia. --- Era uno spettacolo
terribile, comunque. Finalmente, quando era sfinito, si toglieva
l\textquotesingle occhio di vetro. Una volta tolto
l\textquotesingle occhio, si calmava improvvisamente.
L\textquotesingle impulso smetteva di agire su di lui. E allora, freddo
e arrogante, prendeva la bottiglia, si guardava intorno e rideva. ``Lo
farò in ogni caso'' diceva. Poi, mentre tutti si aspettavano che bevesse
il vino, sfoggiava un sorriso beato e si versava il contenuto della
bottiglia sulla testa. Il vantaggio di una coscienza removibile, vedete.

--- Quindi crede che io ne abbia più bisogno di lui.

Don Paulo scrollò le spalle. --- Ma è soltanto il Poeta!

Lo studioso sbuffò divertito. Toccò la sfera vitrea e la fece rotolare
attraverso la tavola, con il pollice. Improvvisamente, scoppiò a ridere.
--- Mi piace. Credo di sapere chi ne ha bisogno più del Poeta, Forse la
terrò, dopotutto. --- La raccolse, la lanciò, l\textquotesingle afferrò
al volo, e guardò dubbioso l\textquotesingle abate.

Paulo si limitò a scrollare di nuovo le spalle.

Il Thon Taddeo lasciò cadere di nuovo l\textquotesingle occhio nella
borsa. --- Potrà riaverlo, se mai verrà a reclamarlo. Ma, fra
l\textquotesingle altro, avevo intenzione di dirvelo: il mio lavoro,
qui, è quasi finito. Partiremo fra pochissimi giorni.

--- Non siete preoccupato per i combattimenti nelle Pianure?

Il Thon Taddeo guardò corrucciato la parete. --- Ci accamperemo su una
collina isolata, a circa una settimana di cammino da qui, verso oriente.
Un gruppo di\ldots{} ehm\ldots{} la nostra scorta ci incontrerà li.

--- Io spero --- disse l\textquotesingle abate, assaporando
quell\textquotesingle educato saggio di cattiveria --- che la vostra
scorta non abbia cambiato le sue alleanze politiche, da quando avete
concluso l\textquotesingle accordo. Sta diventando difficile distinguere
i nemici dagli alleati, di questi tempi.

Il Thon arrossì. --- Specialmente se vengono da Texarkana, intendete
dire?

--- Non ho detto questo.

--- Siamo franchi l\textquotesingle uno con l\textquotesingle altro,
padre. Io non posso combattere il principe che rende possibile il mio
lavoro\ldots{} qualunque cosa io pensi dei suoi metodi o della sua
politica. Io mostro di appoggiarlo, superficialmente, o per lo meno di
ignorarlo\ldots{} per il bene del \emph{collegium}. Se egli allarga i
suoi domini, il \emph{collegium} può trarne profitto. Se il
\emph{collegium} prospera, l\textquotesingle umanità trarrà profitto dal
nostro lavoro. --- Quelli che sopravviveranno, forse.

--- È vero\ldots{} ma è sempre stato vero, in ogni circostanza..

--- No, no\ldots{} Dodici secoli or sono, neppure i sopravvissuti ne
trassero profitto. Dobbiamo ricominciare di nuovo per quella via?

Il Thon Taddeo alzò le spalle. --- E io che posso farci? --- chiese, di
rimando. --- Il principe è Hannegan, non sono io.

--- Ma voi promettete di cominciare a restaurare il dominio
dell\textquotesingle Uomo sulla Natura. Però chi governerà
l\textquotesingle uso della potenza per dominare le forze naturali? Chi
l\textquotesingle userà? A quale fine? Come lo terrete in iscacco?
Queste decisioni devono ancora essere prese. Ma se voi e la vostra
fazione non le prendete adesso, altri le prenderanno, presto, al vostro
posto. L\textquotesingle umanità ne trarrà profitto, voi dite. Con il
consenso di chi? Con il consenso di un principe che firma le sue lettere
con una X? Oppure credete veramente che il vostro collegio sarà al
sicuro dalle sue ambizioni, quando comincerà a scoprire che voi siete
preziosi, per lui?

Don Paulo non aveva preteso di convincerlo. Ma fu con il cuore pesante
che l\textquotesingle abate notò la paziente condiscendenza con cui il
Thon lo ascoltava: era la pazienza di un uomo che ascolta un argomento
che ha da molto tempo confutato con propria soddisfazione.

--- Ciò che consigliereste in realtà --- disse lo studioso --- è che noi
aspettiamo ancora un poco. Che sciogliamo il \emph{collegium}, o che lo
trasferiamo nel deserto, e in un modo o in un altro\ldots{} senza
possedere oro o argento\ldots{} facciamo rivivere una scienza
sperimentale e teorica, in un modo lento e difficile, senza dirlo a
nessuno. Che noi salviamo tutto per il giorno in cui
l\textquotesingle Uomo sarà buono, puro, santo e saggio.

--- Non è questo che intendevo\ldots{}

--- Non è questo che intendevate dire, ma è ciò che significa quello che
avete detto. Tenere la scienza chiusa in un chiostro, non tentare di
applicarla, non tentare di far nulla fino a che gli uomini non saranno
santi. Ebbene, non funzionerebbe. Voi lo avete fatto qui, in questa
abbazia, e per intere generazioni.

--- Noi non abbiamo nascosto nulla.

--- No, non l\textquotesingle avete nascosto; ma vi ci siete seduti
sopra, così quietamente, e nessuno sapeva che era qui, e voi non ne
avete fatto nulla.

Una breve collera lampeggiò negli occhi del vecchio ecclesiastico. --- È
tempo che voi conosciate il nostro fondatore, mi pare --- brontolò,
indicando la scultura lignea nell\textquotesingle angolo. --- Era uno
scienziato come voi, prima che il mondo impazzisse e che egli corresse
in cerca d\textquotesingle un rifugio. Fondò
quest\textquotesingle Ordine per salvare ciò che poteva essere salvato
dei documenti dell\textquotesingle ultima civiltà. ``Salvato'' da che
cosa, e per quale scopo? Vedete su che cosa è ritto\ldots{} vedete i
fuscelli e la legna? I libri? Ecco quanto poco il mondo voleva la vostra
scienza, allora, e per parecchi secoli, poi. Così egli morì per noi.
Quando lo aspersero d\textquotesingle olio combustibile, la leggenda
dice che egli ne chiese una tazza. Pensarono che
l\textquotesingle avesse scambiato per acqua, quindi risero e gliene
diedero una coppa. Egli lo benedisse e\ldots{} e qualcuno afferma che
l\textquotesingle olio si cambiò in vino quando lo benedisse\ldots{} e
poi esclamò \emph{``Hic est enim calix Sanguinis Mei''} e lo bevve prima
che l\textquotesingle impiccassero e lo ardessero vivo. Devo leggervi un
elenco dei nostri martiri? Devo citarvi tutte le battaglie che abbiamo
combattuto per serbare intatti questi documenti? Tutti i monaci
diventati ciechi nella copisteria? Per il vostro bene? Eppure voi dite
che non ne abbiamo fatto nulla, li abbiamo nascosti nel silenzio.

--- Non intenzionalmente --- disse lo studioso --- ma in effetti voi
l\textquotesingle avete fatto\ldots{} per gli stessi motivi che, come
voi sottintendete, dovrebbero essere i miei. Se voi tentate di salvare
la saggezza fino a che il mondo diventerà saggio, padre, il mondo non
l\textquotesingle avrà mai.

--- Vedo che l\textquotesingle incomprensione è radicale! --- disse
burberamente l\textquotesingle abate. --- Servire prima Dio o servire
prima Hannegan\ldots{} questa scelta spetta a voi.

--- Ho poca scelta, allora --- rispose il Thon. --- Vorreste forse che
lavorassi per la Chiesa? --- Il sarcasmo nella sua voce era
inconfondibile.
